\begin{otherlanguage}{german}%
\addchap*{Zusammenfassung}
%
Diese Arbeit präsentiert neue Visualisierungsmethoden für Simulationen aus zwei
verschiedenen Ingenieurwissenschaften: Turbulente Verbrennung und
Festkörpermechanik.
%

%
Der Flaschenhals in großen Simulationen turbulenter Verbrennungsvorgänge ist
inzwischen nicht mehr die Berechnungszeit, sondern das Speichern der
Ausgabedaten.
%
Wir präsentieren zwei neue Techniken für die Visualisierung und Analyse der
Flammenobefläche in solchen Simulationen vor dem Hintergrund dieses
Flaschenhalses.
%
Die erste ist eine platzsparende, ausgedünnte Darstellung für einen bestimmten
Typ von Flammen.
%
Diese ermöglicht die Analyse einer größeren Anzahl von Zeitschritten der
Simulation und ist die Basis für eine neue Art von Flammenvisualisierung.
%
Die zweite ist ein Algorithmus zur Verfolgung der Flammenoberfläche
\emph{in-situ} während der Simulation selbst.
%
Dadurch wird der Flaschenhals des Speichervorganges umgangen und nur die
wesentlich kleineren Ergebnisse müssen geschrieben werden.
%
Beide Verfahren tragen dazu bei, dass Verbrennungswissenschaftler auch in
Zukunft die Daten Analysieren können, die ihre immer größer werdenden
Simulationen produzieren.
%

%
Tensorfelder gehören wegen ihrer vielen Freiheitsgrade zu den herausforderndsten
Daten für die Visualisierung.
%
Eine Möglichkeit, diese Komplexität zu reduzieren ist Feature-basierte
Visualisierung.
%
Diese reduziert die Daten zu einer Menge von geometrischen Primitiven die
interessantes Verhalten markieren.
%
Der \emph{parallel vectors operator}, der alle Orte bestimmt an denen zwei
Vektorfelder parallel sind, ist die Basis für eine Menge von Linien- Features
für Skalar- und Vektorfelder.
%
Wir übertragen diesen Operator auf Tensorfelder und definieren dort den
\emph{parallel eigenvectors operator}, der alle Orte bestimmt, an denen zwei
Tensorfelder parallele reelle Eigenvektoren haben.
%
Diese Idee nutzen wir anschließend zur Definition von \emph{tensor core lines},
die die Zentren von ``wirbelndem'' Verhalten der Eigenvektoren markieren und
auf Wirbelkernlinien in Vektorfeldern basieren.
%
Mit diesem neuen Feature können wir Verwindungen in Stresstensorfeldern aus
Strukturmeachaniksimulationen erkennen.
%
\end{otherlanguage}%