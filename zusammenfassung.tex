\begin{otherlanguage}{ngerman}%
\addchap*{Zusammenfassung}
%
Diese Arbeit pr\"asentiert neue Visualisierungsmethoden f\"ur Simulationen aus
zwei verschiedenen Ingenieurdisziplinen: Turbulente Verbrennung und
Festk\"orpermechanik.
%

%
Direkte numerische Simulationen turbulenter Verbrennung sind eine Basis f\"ur
die Entwicklung und Validierung h\"oherer Verbrennungsmodelle.
%
Ein besonderes Augenmerk bei der Analyse solcher Simulationen liegt auf der
Flammenoberfl\"ache, wo der Gro\ss{}teil aller chemischen Reaktionen
stattfindet.
%
Die Rechenleistung von Supercomputern w\"achst inzwischen wesentlich schneller
als die Leistung ihrer Speicherinfrastruktur.
%
Infolgedessen ist heute das Speichern der Ausgabedaten der Flaschenhals in
gro\ss{}en Simulationen.
%
Wir pr\"asentieren zwei neue Techniken f\"ur die Visualisierung und Analyse der
Flammenobefl\"ache in gro\ss{}en Simulationen turbulenter Verbrennungsvorg\"ange
vor dem Hintergrund dieses Flaschenhalses.
%
Die erste ist eine platzsparende, ausged\"unnte Darstellung f\"ur einen
bestimmten Typ von Flammen.
%
Diese erm\"oglicht die Analyse einer gr\"o\ss{}eren Anzahl von Zeitschritten der
Simulation und ist die Basis f\"ur eine neue Art von Flammenvisualisierung.
%
Die zweite ist ein Algorithmus zur Verfolgung der Flammenoberfl\"ache
\emph{in-situ} w\"ahrend der Simulation selbst.
%
Der Flaschenhals des Speichervorganges wird umgangen indem nur die wesentlich
kleineren Ergebnisse geschrieben werden.
%
Beide Verfahren tragen dazu bei, dass Verbrennungswissenschaftler auch in
Zukunft die Daten analysieren k\"onnen, die ihre immer gr\"o\ss{}er werdenden
Simulationen produzieren.
%

%
Tensorfelder geh\"oren wegen ihrer vielen Freiheitsgrade zu den
herausforderndsten Daten f\"ur die Visualisierung.
%
Eine M\"oglichkeit, diese Komplexit\"at zu reduzieren ist die Extraktion von
Features.
%
Diese reduziert die Daten auf geometrische Primitive, die interessantes
Verhalten markieren.
%
Der \emph{parallel vectors operator}, der alle Orte bestimmt an denen zwei
Vektorfelder parallel sind, ist die Basis f\"ur eine Menge von Linien-Features
f\"ur Skalar- und Vektorfelder.
%
Wir übertragen diesen Operator auf Tensorfelder und definieren dort den
\emph{parallel eigenvectors operator}, der alle Orte bestimmt, an denen zwei
Tensorfelder parallele reelle Eigenvektoren haben.
%
Diese Idee nutzen wir anschlie\ss{}end zur Definition von \emph{tensor core
lines}, die die Zentren von ``wirbelndem'' Verhalten der Eigenvektoren markieren
und auf Wirbelkernlinien in Vektorfeldern basieren.
%
Mit diesem neuen Feature k\"onnen wir Verwindungen in Stresstensorfeldern aus
Strukturmechaniksimulationen erkennen.
%
\end{otherlanguage}%