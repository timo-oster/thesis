\addchap*{Abstract}
\pdfbookmark{Abstract}{abstract}%
%
This thesis presents new visualization techniques for engineering simulations in
two different disciplines: Turbulent combustion and solid mechanics.
%

%
Direct numerical simulations of turbulent combustion are used as a basis to
develop and validate higher-level combustion models.
%
A focus of interest in the analysis of such simulations is the flame surface,
where most of the chemical reactions take place.
%
The computational power of supercomputers is increasing much faster than the
performance of storage infrastructures.
%
This has caused the output and storage of simulation data to become the
bottleneck in large-scale simulation runs.
%
We introduce two new techniques for the visualization and analysis of the flame
surface in large-scale simulations of turbulent combustion before the background
of this storage bottleneck.
%
The first is a space-saving sparse representation for certain types of flames.
%
It allows for the analysis of a larger number of simulation time steps and is
the basis for a new flame visualization technique.
%
The second is an algorithm for tracking the flame surface in-situ during the
simulation.
%
The storage bottleneck is circumvented by only writing to disk the much smaller
results.
%
Both contribute to the continued ability of combustion researchers to analyze
the data produced by their increasingly large simulations.
%

%
Due to their many degrees of freedom, tensor fields are some of the most
challenging types of data to visualize.
%
One possibility to break down their complexity is feature-based visualization,
which reduces the data to a set of geometric primitives that represent the
occurrence of some kind of interesting behavior.
%
The parallel vectors operator, which yields locations where two vector fields
are parallel, is the basis of a number of line-type features in scalar and
vector fields.
%
We translate this operator to tensor fields by introducing the parallel
eigenvectors operator, which yields locations where two tensor fields have
parallel real eigenvectors.
%
We then use this idea to introduce tensor core lines, which mark the centers of
``swirling'' behavior of the eigenvectors, and are based on vortex core lines in
vector fields.
%
Using this new feature, we can detect twist in stress tensor fields from solid
mechanics simulations.
%