\chapter{Fundamentals of Scientific Visualization} % (fold)
\label{cha:fundamentals}
%
\lettrine[lines=3, findent=-2pt, nindent=5pt, loversize=0.02]{S}{cientific
visualization} is the visualization of data from the natural sciences.
%
Although such data can come in a variety of forms, the term is most often used
to describe the techniques used for displaying spatial, possibly time-varying
data from physics, chemistry, engineering, or biomedical applications.
%
Examples for such data are volumetric scans of patients from medicine, wind
speed and pressure measurements from meteorology, and simulations of the air
flow around a car.
%

%
This thesis presents several new scientific visualization techniques.
%
The techniques are demonstrated on simulation data from two different
engineering domains: turbulent combustion and solid mechanics.
%
To set the scene for these contributions, we will therefore use this chapter to
briefly recall the fundamental methods for visualizing scientific data.
%
Such data usually comes in the form of scalar-, vector-, or tensor fields.
%
After listing the notation used throughout the text, we will therefore take a
look at the definition and basic visualization techniques for each of these
types of fields in the following sections.
%

%
\section{Notation} % (fold)
\label{sec:notation}
%
We will use the following mathematical notation.
%
Additional notation is introduced wherever it is needed.
%

% %
% Scalars are typeset in italics: $s$, $f$, $a$, $\lambda$.
% %
% They usually have lowercase names.
% %

% %
% Vectors have lowercase names in bold roman type: $\vv$, $\vr$, $\vx$.
% %
% The vector of all zeros is written as $\vNull$.
% %
% All vectors are by convention column vectors, and their components are written
% as scalars with a subscript index: $v_i$, $r_i$, $x_i$.
% %

% %
% Matrices are named with uppercase letters and typeset like vectors in bold roman
% font: $\mA$, $\mS$, $\mT$.
% %
% The identitiy matrix is written as $\mI$.
% %
% The transpose of a matrix (or vector) is indicated by a superscript T: $\T\mA$,
% $\T\vv$.
% %

% %
% A block matrix $\begin{pmatrix}\va & \vb & \vc\end{pmatrix}$ is a matrix
% composed of several other matrices or vectors.
% %

% %
% The inner (dot-) product of two vectors $\T{\vv}\vw$ is written as $\vv \cdot
% \vw$.
% %

% %
% The Nabla-Operator $\nabla$ is a vector of partial derivatives
% $\T{\left({\partial}/{\partial x_1}, \dots, {\partial}/{\partial x_n}\right)}$.
% %
% It is thought of as a regular vector, but multiplication with one of its
% components is regarded as taking the partial derivative in the corresponding
% coordinate direction.
% %
% It is useful for writing operations such as the gradient of a scalar field
% ($\nabla s$), or the divergence of a vector field ($\nabla \cdot \vv$).
% %

\vspace*{\baselineskip}
%
\begin{tabularx}{\textwidth}{lX}
$s$, $a$, $\lambda$ & Scalars \\
$\vv$, $\vr$, $\vx$ & Column vectors \\
$v_i$, $r_i$, $x_i$ & Components of the respective vectors \\
$\vNull$ & The vector of all zeros \\
$\mA$, $\mS$, $\mSigma$ & Matrices \\
$\mI$ & The identity matrix \\
$\T{\mA}$, $\T{\vv}$ & Transpose of a matrix/vector \\
$\begin{pmatrix}\va & \vb & \vc\end{pmatrix}$ & Block matrix composed of several
    matrices/vectors \\
$\vv \cdot \vw$ & Inner (dot-) product of two vectors \\
$\nabla$ & Nabla-Operator
    $\T{\left({\partial}/{\partial x_1}, \dots, {\partial}/{\partial x_n}\right)}$
\end{tabularx}
%
% section notation (end)

\section{Scalar Fields} % (fold)
\label{sec:scalar_fields}

% section scalar_fields (end)

\section{Vector Fields} % (fold)
\label{sec:vector_fields}

% section vector_fields (end)

\section{Second-Order Tensor Fields} % (fold)
\label{sec:tensor_fields}

% section tensor_fields (end)

% chapter fundamentals (end)