\chapter{Fundamentals of Scientific Visualization} % (fold)
\label{cha:fundamentals}
%
\lettrine[lines=3, findent=-2pt, nindent=5pt, loversize=0.02]{S}{cientific
visualization} is the visualization of data from the natural sciences.
%
Although such data can come in a variety of forms, the term is most often used
to describe the techniques used for displaying spatial, possibly time-varying
data from physics, chemistry, engineering, or biomedical applications.
%
Examples for such data are volumetric scans of patients from medicine, wind
speed and pressure measurements from meteorology, and simulations of the air
flow around a car.
%

%
This thesis presents several new scientific visualization techniques.
%
The techniques are demonstrated on simulation data from two different
engineering domains: turbulent combustion and solid mechanics.
%
To set the scene for these contributions, we will therefore use this chapter to
briefly recall the fundamental methods for visualizing scientific data.
%
Such data usually comes in the form of scalar-, vector-, or (second-order)
tensor fields.
%

%
After listing the notation used throughout the text, we will take a look at the
definition and basic visualization techniques for each of these types of fields
in the following sections.
%

%
\section{Notation} % (fold)
\label{sec:notation}
%
We will use the following mathematical notation.
%
Additional notation is introduced wherever it is needed.
%

% %
% Scalars are typeset in italics: $s$, $f$, $a$, $\lambda$.
% %
% They usually have lowercase names.
% %

% %
% Vectors have lowercase names in bold roman type: $\vv$, $\vr$, $\vx$.
% %
% The vector of all zeros is written as $\vNull$.
% %
% All vectors are by convention column vectors, and their components are written
% as scalars with a subscript index: $v_i$, $r_i$, $x_i$.
% %

% %
% Matrices are named with uppercase letters and typeset like vectors in bold roman
% font: $\mA$, $\mS$, $\mT$.
% %
% The identitiy matrix is written as $\mI$.
% %
% The transpose of a matrix (or vector) is indicated by a superscript T: $\T\mA$,
% $\T\vv$.
% %

% %
% A block matrix $\begin{pmatrix}\va & \vb & \vc\end{pmatrix}$ is a matrix
% composed of several other matrices or vectors.
% %

% %
% The inner (dot-) product of two vectors $\T{\vv}\vw$ is written as $\vv \cdot
% \vw$.
% %

% %
% The Nabla-Operator $\nabla$ is a vector of partial derivatives
% $\T{\left({\partial}/{\partial x_1}, \dots, {\partial}/{\partial x_n}\right)}$.
% %
% It is thought of as a regular vector, but multiplication with one of its
% components is regarded as taking the partial derivative in the corresponding
% coordinate direction.
% %
% It is useful for writing operations such as the gradient of a scalar field
% ($\nabla s$), or the divergence of a vector field ($\nabla \cdot \vv$).
% %

\vspace*{\baselineskip}
%
\begin{tabularx}{\textwidth}{lX}
$s$, $a$, $\lambda$ & Scalars \\
$\vv$, $\vr$, $\vx$ & Column vectors \\
$v_i$, $r_i$, $x_i$ & Components of the respective vectors \\
$\vNull$ & The vector of all zeros \\
$\mA$, $\mS$, $\mSigma$ & Matrices \\
$\mI$ & The identity matrix \\
$\T{\mA}$, $\T{\vv}$ & Transpose of a matrix/vector \\
$\begin{pmatrix}\va & \vb & \vc\end{pmatrix}$ & Block matrix composed of several
    matrices/vectors \\
$\vv \cdot \vw$ & Inner (dot-) product of two vectors \\
$\nabla$ & Nabla-Operator
    $\T{\left({\partial}/{\partial x_1}, \dots, {\partial}/{\partial x_n}\right)}$
\end{tabularx}
%
% section notation (end)

\section{Scalar Fields} % (fold)
\label{sec:scalar_fields}
%
A scalar field is a map $s(\vx, t): D \times T \mapsto \RRSet$ that assigns a
scalar value to each position $\vx$ in a spatial domain $D$ and time $t$ in a
temporal domain $T$.
%
The spatial domain $D$ is a subset of the (two- or three-dimensional) Euclidean
space $\EESet^n$.
%
The temporal domain $T$ is usually an interval of $\RRSet$.
%
Examples of scalar fields are the temperature in a solid object, the population
density on a map, and the attenuation coefficient in a \ac{CT} scan.
%
If the scalar field does not change with time, or we are only interested in a
single instant, we often just write $s(\vx)$ and omit the time parameter.
%

% \subsection{Visualization Methods for Scalar Fields} % (fold)
% \label{sub:visualization_methods_for_scalar_fields}
%
% \begin{itemize}
%     \item Space-filling
%     \begin{itemize}
%         \item Colormapping (2D)
%         \item Height-mapping (2D)
%         \item Direct Volume rendering (3D)
%     \end{itemize}
%     \item Geometry-based
%     \begin{itemize}
%         \item Isocontours/surfaces (cite marching cubes)
%         \item maxima/minima/saddles
%         \item Morse-Smale-complex
%         \item Ridge lines/surfaces
%     \end{itemize}
% \end{itemize}
%
Methods for visualizing scalar fields can be roughly separated into two groups:
image-based and geometry-based.
%
We will briefly cover the most important methods in the following.
%
\Todo{add citations}
%

\subsection{Image-Based Methods} % (fold)
\label{sub:image_based_methods}
%
Image-based methods map the scalar at each position in space to a property and
display it directly.
%
Among such methods are \emph{color-mapping} and \emph{height-mapping} for
\ac{2D} scalar fields as well as \emph{direct volume rendering} for \ac{3D}
scalar fields.
%

%
\paragraph{Color-mapping} means assigning each scalar value a color from a
predefined colormap and displaying the result as an image or texture.
%
Due to its simplicity, it is probably the most widespread method presented here.
%
This technique is only applicable to \ac{2D} scalar fields, but is commonly used
on \ac{3D} data by selecting slices or surfaces in a volume, or by displaying
the scalar value on the outside surface of the volume.
%

%
\paragraph{Height-mapping} only works on \ac{2D} data.
%
It essentially means interpreting the scalar value at each point as a height
value and displaying the resulting three-dimensional surface.
%
Because it transforms \ac{2D} information into \ac{3D} information, it is best
suited for interactive settings, where the resulting surface can be rotated and
viewed from all directions.
%

%
\paragraph{Direct volume rendering} is the extension of color-mapping to \ac{3D}
scalar fields.
%
It involves two steps: Applying a transfer function, and accumulating the
resulting colors and opacities along viewing rays to produce an image.
%
The transfer function has to be chosen carefully to reveal the structures in the
data that are interesting, and make the uninteresting parts transparent.
%
Color and opacity values are then accumulated along viewing rays to simulate the
transport of light through a semitransparent medium.
%
% subsection image_based_methods (end)

\subsection{Geometry-Based Methods} % (fold)
\label{sub:geometry_based_methods}
%
Geometry-based methods extract some geometrical structures from the data and
display these structures.
%
\emph{Isocontours and -surfaces} belong in this category together with
topological features such as \emph{extremal- and saddle points}, as well as
\emph{ridge- and valley lines and -surfaces}.
%

%
\paragraph{Isocontours and -surfaces} are subsets of the scalar field where the
scalar is equal to a constant (iso-) value.
%
In the literature these are also often referred to as \emph{level sets}.
%
They form lines or surfaces that are always closed or end at the domain
boundary, and that never intersect each other.
%
For \ac{2D} scalar fields, it is common to plot contours for several isovalues
at once, which resemble the height lines we know from maps.
%
In \ac{3D}, it is rare to display more than two or three different isosurfaces
at the same time due to occlusion problems.
%

%
\paragraph{Minima, maxima, and saddle points} are points in a scalar field where
the gradient becomes zero.
%
As such, they are features of the scalar field's topology.
%

%
\paragraph{Ridge- and valley lines and -surfaces} also belong to the category
of topological features.
%
There is no universal agreed-upon definition for these types of features.
%
The two most common, competing definitions are \emph{watersheds} and
\emph{height ridges} \cite{Peikert2008,Eberly2012}.
%
Watersheds are global features that separate the scalar field into ``areas of
influence'' of the different maxima (or minima).
%
Together with the critical points, they form the topological skeleton of the
scalar field.
%
Height ridges are defined by a local differential analysis of the scalar field.
%
As such, their computation is less costly, but they do not provide a
space-filling segmentation of the data.
%
% subsection geometry_based_methods (end)
%
% subsection visualization_methods_for_scalar_fields (end)
%
% section scalar_fields (end)

\section{Vector Fields} % (fold)
\label{sec:vector_fields}
%
A vector field $\vv(\vx,t): D \times T \mapsto \RRSet^n$ is a map from spatial
and temporal domains $D \subset \EESet^n$ and $T \subset \RRSet$ to the
$n$-dimensional vector space $\RRSet^n$.
%
Just like a scalar field, it assigns a value to each position and time in the
domains, but this time the value is a vector.
%
Examples for vector fields are the velocity of a fluid flow, the displacement
field of a deformed object, and the magnetic field around an electromagnet.
%
If the vector field does not change with time, or we are only interested in the
field at a single instant, we say that we have a \emph{steady} vector field
$\vv(\vx)$.
%
If the vector field changes with time, we call it an \emph{unsteady} vector
field.
%

%
Vector field visualization methods can be sorted into roughly five different
classes: Direct methods, image-based methods, integral lines and -surfaces,
topological features, and other feature-based methods.
%
We will again visit the most important methods in the following sections.
%

%
\subsection{Direct Methods} % (fold)
\label{sub:direct_methods}
%
Direct methods display the vector data in a very basic way, much like
image-based methods for scalar fields.
%
They encompass techniques such as \emph{arrow plots}, or \emph{color-mapping}
the velocity magnitude, vorticity and other derived quantities directly.
%

%
\paragraph{Arrow plots} are the simplest way to visualize a vector field.
%
In such a plot, arrow glyphs are placed at multiple locations throughout the
domain.
%
The arrows are aligned with the direction of the local vector, and their
length is typically scaled based on its magnitude.
%
Such plots can very accurately show the vectors at a limited number of
locations, but they quickly become cluttered once too many arrows are plotted,
or the arrows become too long and occlude each other.
%

%
\paragraph{Color-mapping} can be applied to vector fields in different ways.
%
The most common one is simply displaying the magnitude of the vector field as a
scalar.
%
Other scalars derived from the vector field, such as the vorticity magnitude and
divergence, can be visualized in the same way.
%
All of this obviously goes along with a loss of information.
%
Since color has three degrees of freedom, a vector field can also be visualized
without information loss by directly mapping the vector values to colors.
%
The most naive way is to simply interpret the three components of a \ac{3D}
vector as RGB values.
%
Such images theoretically contain they full information of the original vector
field.
%
However, they are very hard to interpret, as there is no inherent meaningful
connection between the direction of the vector and the color it is mapped to.
%
This can be slightly improved for \ac{2D} vector fields by mapping the angle
and magnitude of the vector to hue and value of the HSV color space.
%
% subsection direct_methods (end)
%
\subsection{Image-Based Methods} % (fold)
\label{sub:image_based_methods}
%
Image-based methods visualize the vector data by generating a space-filling
texture.
%
Among such methods are \emph{line integral convolution}, \emph{spot noise},
and \emph{texture advection}.
%

%
\paragraph{\Ac{LIC}} is the most popular
image-based vector field visualization method.
%
It is based on ``smearing'' a random noise texture along stream lines of a
\ac{2D} vector field.
%
More specifically, the color at a certain position is determined as a weighted
integral of the color values encountered along a stream line passing through
that position\ToCite{LIC paper}.
%
The result is a space-filling image where the direction of the vector field is
visible at each location.
%
The basic \ac{LIC} technique is only applicable to \ac{2D} steady vector fields,
but extensions have been developed for \ac{3D} and unsteady datasets.
%

%
\paragraph{Spot noise} is also a method designed for \ac{2D} data and produces
results similar to \ac{LIC}.
%
It works by blending noise sprites that have been stretched and rotated
according to the local vector direction.
%
In contrast to \ac{LIC}, spot noise better represents the local vector
magnitude, but in regions with low magnitude, the vector direction is not well
visible.
%

%
\paragraph{Texture advection} works on \ac{2D} steady and unsteady vector
fields.
%
It is best suited for vector fields representing a flow, as it simulates the
transport(advection) of a texture with the flow.
%
A texture is placed in the flow at some point in time, and each point on the
texture is moved with the flow over time.
%
As time progresses, the texture is warped, and the viewer can follow where each
part of the texture is transported.
%
This method has a lot in common with the integration-based methods presented in
the next section.
%
In fact, texture advection simply displays a time surface with a mapped texture
in a \ac{2D} vector field.
%

%
% subsection image_based_methods (end)
%
\subsection{Integral Lines and -Surfaces} % (fold)
\label{sub:integral_lines_and_surfaces}
%
Integral lines and -surfaces are generated by integrating the vector field
starting from different kinds of seed structures.
%
Depending on the seeding strategy and the kind of vector field, we can obtain
\emph{streamlines}, \emph{pathlines}, \emph{timelines}, \emph{streaklines},
and the accompanying surfaces.
%
Since integral structures play an important role in several parts of this
thesis, we will cover them here in more detail.
%
% - vector field integration
% - algorithms:
%   - euler direct/indirect
%   - runge-kutta
%

%
% subsection integral_lines_and_surfaces (end)
%
\subsection{Vector Field Topology} % (fold)
\label{sub:vector_field_topology}
%
The topology of a vector field is defined by \emph{critical points},
\emph{boundary switch points}, \emph{attachment- and detachment points}, the
accompanying \emph{separatrices} connecting these points, and \emph{isolated
closed streamlines}
%

%
% subsection vector_field_topology (end)
%
\subsection{Other Features} % (fold)
\label{sub:other_features}
%
Other important features, particularly of unsteady fluid flow fields, are
vortices and \ac{LCS}.
%
Vortices can be visualized indirectly by the value of certain derived scalar
fields, such as the \emph{vorticity}, \emph{$\lambda_2$}- and
\emph{Q-criteria}, or their \emph{vortex core lines} can be computed explicitly.
%
\Ac{LCS} are volumes of a fluid that stay spatially intact for long periods of
time, \eg, points that start within such a volume at some point in time will
remain near each other.
%
Popular methods for visualizing such structures are the \ac{FTLE} and
\ac{FSLE}.
%

%
% subsection other_features (end)
%
% Direct
% =====
% Reduction to scalar field (e.g. velocity magnitude)
% Arrow plot
%
% Image based
% ===========
% Spot noise??
% LIC
% Texture advection
%
% Integration based
% =================
% Streamlines
% Pathlines
% Timelines
% Streaklines
% -surfaces !!!
%
% Topology
% ========
% Critical Points
% Boundary Switch points
% Separatrices
% Attachment-/Detachment Points
% Isolated closed streamlines
% Tracking critical points (fff) !!! (tracking isosurface similar)
% Collapse/bifurcation of critical points
%
% Feature based
% =============
% \lambda_2, Q-criterion, vorticity
% Vortex core lines !!! (Tensor core lines)
% - integration based
% - local (PV, Sujudi/Haimes)
% Lagrangian coherent structures/FTLE/FSLE !!! (deformation of surface)
%
% section vector_fields (end)

\section{Second-Order Tensor Fields} % (fold)
\label{sec:tensor_fields}

% section tensor_fields (end)

% chapter fundamentals (end)