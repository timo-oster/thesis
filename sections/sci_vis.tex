\chapter{Fundamentals of Scientific Visualization} % (fold)
\label{cha:fundamentals}
%
\lettrine[lines=3, findent=-2pt, nindent=5pt, loversize=0.02]{S}{cientific
visualization} is the visualization of data from the natural sciences.
%
Although such data can come in a variety of forms, the term is most often used
to describe the techniques used for displaying spatial, possibly time-varying
data from physics, chemistry, engineering, or biomedical applications.
%
Examples for such data are volumetric scans of patients from medicine, wind
speed and pressure measurements from meteorology, and simulations of the air
flow around a car.
%

%
This thesis presents several new scientific visualization techniques.
%
The techniques are demonstrated on simulation data from two different
engineering domains: turbulent combustion and solid mechanics.
%
To set the scene for these contributions, we will therefore use this chapter to
briefly recall the fundamental methods for visualizing scientific data.
%
Such data usually comes in the form of scalar-, vector-, or (second-order)
tensor fields.
%

%
After listing the notation used throughout the text, we will take a look at the
definition and basic visualization techniques for each of these types of fields
in the following sections.
%

%
\section{Notation} % (fold)
\label{sec:notation}
%
We will use the following mathematical notation throughout the thesis.
%
Additional notation is introduced wherever it is needed.
%

% %
% Scalars are typeset in italics: $s$, $f$, $a$, $\lambda$.
% %
% They usually have lowercase names.
% %

% %
% Vectors have lowercase names in bold roman type: $\vv$, $\vr$, $\vx$.
% %
% The vector of all zeros is written as $\vNull$.
% %
% All vectors are by convention column vectors, and their components are written
% as scalars with a subscript index: $v_i$, $r_i$, $x_i$.
% %

% %
% Matrices are named with uppercase letters and typeset like vectors in bold roman
% font: $\mA$, $\mS$, $\mT$.
% %
% The identitiy matrix is written as $\mI$.
% %
% The transpose of a matrix (or vector) is indicated by a superscript T: $\T\mA$,
% $\T\vv$.
% %

% %
% A block matrix $\begin{pmatrix}\va & \vb & \vc\end{pmatrix}$ is a matrix
% composed of several other matrices or vectors.
% %

% %
% The inner (dot-) product of two vectors $\T{\vv}\vw$ is written as $\vv \cdot
% \vw$.
% %

% %
% The Nabla-Operator $\nabla$ is a vector of partial derivatives
% $\T{\left({\partial}/{\partial x_1}, \dots, {\partial}/{\partial x_n}\right)}$.
% %
% It is thought of as a regular vector, but multiplication with one of its
% components is regarded as taking the partial derivative in the corresponding
% coordinate direction.
% %
% It is useful for writing operations such as the gradient of a scalar field
% ($\nabla s$), or the divergence of a vector field ($\nabla \cdot \vv$).
% %

\vspace*{\baselineskip}
%
\begin{tabularx}{\textwidth}{lX}
$s$, $a$, $\lambda$ & Scalars \\
$\vv$, $\vr$, $\vx$ & Column vectors \\
$v_i$, $r_i$, $x_i$ & Components of the respective vectors \\
$\vNull$ & The vector of all zeros \\
$\mA$, $\mS$, $\mSigma$ & Matrices \\
$\mI$ & The identity matrix \\
$\T{\mA}$, $\T{\vv}$ & Transpose of a matrix/vector \\
$\begin{pmatrix}\va & \vb & \vc\end{pmatrix}$ & Block matrix composed of several
    matrices/vectors \\
$\vv \cdot \vw$ & Inner (dot-) product of two vectors \\
$\nabla$ & Nabla-Operator
    $\T{\left({\partial}/{\partial x_1}, \dots, {\partial}/{\partial x_n}\right)}$
\end{tabularx}
%
% section notation (end)

\section{Scalar Fields} % (fold)
\label{sec:scalar_fields}
%
A scalar field is a map $s(\vx, t): D \times T \mapsto \RRSet$ that assigns a
scalar value to each position $\vx$ and time $t$ in spatial and temporal domains
$D$ and $T$.
%
The spatial domain $D$ is a subset of the (two- or three-dimensional) Euclidean
space $\EESet^n$.
%
The temporal domain $T$ is usually an interval of $\RRSet$.
%
Examples of scalar fields are the temperature in a solid object, the population
density on a map, and the attenuation coefficient in a \ac{CT} scan.
%
If the scalar field does not change with time, or we are only interested in a
single instant, we often just write $s(\vx)$ and omit the time parameter.
%

% \subsection{Visualization Methods for Scalar Fields} % (fold)
% \label{sub:visualization_methods_for_scalar_fields}
%
% \begin{itemize}
%     \item Space-filling
%     \begin{itemize}
%         \item Colormapping (2D)
%         \item Height-mapping (2D)
%         \item Direct Volume rendering (3D)
%     \end{itemize}
%     \item Geometry-based
%     \begin{itemize}
%         \item Isocontours/surfaces (cite marching cubes)
%         \item maxima/minima/saddles
%         \item Morse-Smale-complex
%         \item Ridge lines/surfaces
%     \end{itemize}
% \end{itemize}
%
Methods for visualizing scalar fields can be roughly separated into two groups:
image-based and geometry-based.
%
We will briefly cover the most important methods in the following.
%
\Todo{add citations}
%

\subsection{Image-Based Methods} % (fold)
\label{sub:image_based_methods}
%
Image-based methods map the scalar at each position in space to a property and
display it directly.
%
Among such methods are \emph{color-mapping} and \emph{height-mapping} for
\ac{2D} scalar fields as well as \emph{direct volume rendering} for \ac{3D}
scalar fields.
%

%
\paragraph{Color-mapping} means assigning each scalar value a color from a
predefined color map and displaying the result as an image or texture.
%
Due to its simplicity, it is probably the most widespread method presented here.
%
This technique is only applicable to \ac{2D} scalar fields, but is commonly used
on \ac{3D} data by selecting slices or surfaces in a volume, or by displaying
the scalar value on the outside surface of the volume.
%

%
\paragraph{Height-mapping} only works on \ac{2D} data.
%
It essentially means interpreting the scalar value at each point as a height
value and displaying the resulting three-dimensional surface.
%
Because it transforms \ac{2D} information into \ac{3D} information, it is best
suited for interactive settings, where the resulting surface can be rotated and
viewed from all directions.
%

%
\paragraph{Direct volume rendering} is the extension of color-mapping to \ac{3D}
scalar fields.
%
It involves two steps: Applying a transfer function, and accumulating the
resulting colors and opacities along viewing rays to produce an image.
%
The transfer function has to be chosen carefully to reveal the structures in the
data that are interesting, and make the uninteresting parts transparent.
%
Color and opacity values are then accumulated along viewing rays to simulate the
transport of light through a semitransparent medium.
%
% subsection image_based_methods (end)

\subsection{Geometry-Based Methods} % (fold)
\label{sub:geometry_based_methods}
%
Geometry-based methods extract some geometrical structures from the data and
display these structures.
%
\emph{Isocontours and -surfaces} belong in this category together with
topological features such as \emph{extremal- and saddle points}, as well as
\emph{ridge- and valley lines and -surfaces}.
%

%
\paragraph{Isocontours and -surfaces} are subsets of the scalar field where the
scalar is equal to a constant (iso-) value.
%
In the literature these are also often referred to as \emph{level sets}.
%
They form lines or surfaces that are always closed or end at the domain
boundary, and that never intersect each other.
%
For \ac{2D} scalar fields, it is common to plot contours for several isovalues
at once, which resemble the height lines we know from maps.
%
In \ac{3D}, it is rare to display more than two or three different isosurfaces
at the same time due to occlusion problems.
%

%
\paragraph{Minima, maxima, and saddle points} are points in a scalar field where
the gradient becomes zero.
%
As such, they are features of the scalar field's topology.
%

%
\paragraph{Ridge- and valley lines and -surfaces} also belong to the category
of topological features.
%
There is no universal agreed-upon definition for these types of features.
%
The two most common, competing definitions are \emph{watersheds} and
\emph{height ridges} \cite{Peikert2008,Eberly2012}.
%
Watersheds are global features that separate the scalar field into ``areas of
influence'' of the different maxima (or minima).
%
Together with the critical points, they form the topological skeleton of the
scalar field.
%
Height ridges are defined by a local differential analysis of the scalar field.
%
As such, their computation is less costly, but they do not provide a
space-filling segmentation of the data.
%
% subsection geometry_based_methods (end)
%
% subsection visualization_methods_for_scalar_fields (end)
%
% section scalar_fields (end)

\section{Vector Fields} % (fold)
\label{sec:vector_fields}
%
A vector field $\vv(\vx,t): D \times T \mapsto \RRSet^n$ is a map from spatial
and temporal domains $D \subset \EESet^n$ and $T \subset \RRSet$ to the
$n$-dimensional vector space $\RRSet^n$.
%
Just like a scalar field, it assigns a value to each position and time in the
domains, but this time the value is a vector.
%
Examples for vector fields are the velocity of a fluid flow, the displacement
field of a deformed object, and the magnetic field around an electromagnet.
%
If the vector field does not change with time, or we are only interested in the
field at a single instant, we say that we have a \emph{steady} vector field
$\vv(\vx)$.
%
If the vector field changes with time, we call it an \emph{unsteady} vector
field.
%
If we are talking about a vector field representing the velocity of a fluid
flow, we sometimes call it a (steady or unsteady) \emph{flow field}.
%

%
Vector field visualization methods can be sorted into roughly five different
classes: Direct methods, image-based methods, integral curves and -surfaces,
topological features, and other feature-based methods.
%
We will again visit the most important methods in the following sections.
%

%
\subsection{Direct Methods} % (fold)
\label{sub:direct_methods}
%
Direct methods display the vector data in a very basic way, much like
image-based methods for scalar fields.
%
They encompass techniques such as \emph{arrow plots}, or \emph{color-mapping}
the velocity magnitude, vorticity and other derived quantities directly.
%

%
\paragraph{Arrow plots} are the simplest way to visualize a vector field.
%
In such a plot, arrow glyphs are placed at multiple locations throughout the
domain.
%
The arrows are aligned with the direction of the local vector, and their
length is typically scaled based on its magnitude.
%
Such plots can very accurately show the vectors at a limited number of
locations, but they quickly become cluttered once too many arrows are plotted,
or the arrows become too long and occlude each other.
%

%
\paragraph{Color-mapping} can be applied to vector fields in different ways.
%
The most common one is simply displaying the magnitude of the vector field as a
scalar.
%
Other scalars derived from the vector field, such as the vorticity magnitude and
divergence, can be visualized in the same way.
%
All of this obviously goes along with a loss of information.
%
Since color has three degrees of freedom, a vector field can also be visualized
without information loss by directly mapping the vector values to colors.
%
The most naive way is to simply interpret the three components of a \ac{3D}
vector as RGB values.
%
Such images theoretically contain they full information of the original vector
field.
%
However, they are very hard to interpret, as there is no inherent meaningful
connection between the direction of the vector and the color it is mapped to.
%
This can be slightly improved for \ac{2D} vector fields by mapping the angle
and magnitude of the vector to hue and value of the HSV color space.
%
% subsection direct_methods (end)
%
\subsection{Image-Based Methods} % (fold)
\label{sub:image_based_methods}
%
Image-based methods visualize the vector data by generating a space-filling
texture.
%
Among such methods are \emph{line integral convolution}, \emph{spot noise},
and \emph{texture advection}.
%

%
\paragraph{Line integral convolution (\acs{LIC})}\acused{LIC} is the most
popular image-based vector field visualization method.
%
It is based on ``smearing'' a random noise texture along stream lines of a
\ac{2D} vector field.
%
More specifically, the color at a certain position is determined as a weighted
integral of the color values encountered along a stream line passing through
that position\ToCite{LIC paper}.
%
The result is a space-filling image where the direction of the vector field is
visible at each location.
%
The basic \ac{LIC} technique is only applicable to \ac{2D} steady vector fields,
but extensions have been developed for \ac{3D} and unsteady datasets.
%

%
\paragraph{Spot noise} is also a method designed for \ac{2D} data and produces
results similar to \ac{LIC}.
%
It works by blending noise sprites that have been stretched and rotated
according to the local vector direction.
%
In contrast to \ac{LIC}, spot noise better represents the local vector
magnitude, but in regions with low magnitude, the vector direction is not well
visible. \ToCite{Spot Noise}
%

%
\paragraph{Texture advection} works on \ac{2D} steady and unsteady vector
fields.
%
It is best suited for vector fields representing a flow, as it simulates the
transport(advection) of a texture with the flow.
%
A texture is placed in the flow at some point in time, and each point on the
texture is moved with the flow over time.
%
As time progresses, the texture is warped, and the viewer can follow where each
part of the texture is transported.
%
This method has a lot in common with the integration-based methods presented in
the next section.
%
In fact, texture advection simply displays a time surface with a mapped texture
in a \ac{2D} vector field.
%

%
% subsection image_based_methods (end)
%
\subsection{Integral Curves and -Surfaces} % (fold)
\label{sub:integral_curves_and_surfaces}
%
Integral curves and -surfaces are generated by integrating the vector field
starting from different kinds of seed structures.
%
Depending on the seeding strategy and the kind of vector field, we can obtain
\emph{streamlines}, \emph{pathlines}, \emph{streaklines}, \emph{timelines},
and the accompanying surfaces.
%
Since integral structures play an important role in several parts of this
thesis, we will cover them here in more detail.
%

\subsubsection*{Streamlines} % (fold)
\label{ssub:streamlines}
%
Streamlines are the simplest form of integral curve in a vector
field.
%
They are defined for steady vector fields.
%
Depending on the application area, they are also sometimes called \emph{field
lines}.
%
A streamline is a curve that is tangent to the vector field everywhere along its
path.
%
Given a streamline $\vc(s)$ of the steady vector field $\vv(\vx)$, this means
that $\vc(s) \times \vv(\vc(s))$ is $\vNull$ for all $s$.
%
This criterion is valid for any parameterization of the curve, but we can only
use it to check if a given curve is a streamline.
%
To compute streamlines, we typically solve the ordinary differential equation
%
\begin{equation}
    \frac{\partial \vc(s)}{\partial s} = \vv(\vc(s))
    \text{, with } \vc(0) = \vx_0 \, \text{.}
\end{equation}
%
This yields a streamline with a particular parameterization: an integral curve
of the vector field.
%
Due to their definition, two stream lines never intersect at single points.
%
They are either completely disjoint, or they coincide.
%

%
Visualizing a steady vector field with stream lines shows the direction of the
flow much like a \ac{LIC} image does, but not in a space-filling manner.
%
This allows for limited use of streamlines also in \ac{3D} data.
%
Streamline visualization can be deceiving when used on single time slices of
an unsteady flow field.
%
The connected lines mistakenly suggest paths of fluid elements.
%
However, if the flow field changes significantly with time, the actual paths of
fluid elements can deviate significantly from the streamlines in a single time
slice.
%
In this case, the more appropriate visualization tools are pathlines,
streaklines, and timelines, which incorporate the temporal information of the
flow.
%
% subsubsection streamlines (end)

\subsubsection*{Pathlines} % (fold)
\label{ssub:pathlines}
%
Pathlines describe the paths of massless particles moving with a
flow.
%
They are defined as the solution to the ordinary differential equation
%
\begin{equation}
    \frac{\partial \vc(t)}{\partial t} = \vv(\vc(t), t)
    \text{, with } \vc(t_0) = \vx_0 \, \text{,}
\end{equation}
%
where $\vc(t)$ is the curve of the pathline, $t$ is time, and $\vv(\vx, t)$ is
an unsteady vector field.
%
Looking at this definition, it becomes apparent that for a steady vector field,
which does not change with time, streamlines and pathlines are identical.
%

%
The set of all pathlines for all possible combinations of start position
$\vx$, start time $t_0$ and end time $t$ forms the \emph{flow map}.
%
This function, which we write as $\bPhi(\vx, t_0, t)$, determines where a
massless particle starting at position $\vx$ and time $t_0$ ends up after
advecting with the flow until time $t$.
%

%
Pathlines allow the visualization of the dynamic behavior of an unsteady flow
in a static image.
%
To show the temporal information, the time is often color-mapped on the
curve.
%
Unlike streamlines, pathlines can and do intersect each other.
%
For very complex flows, showing a lot of pathlines can therefore quickly become
confusing.
%
In such cases it can be better to show animated streaklines instead.
%
% subsubsection pathlines (end)

\subsubsection*{Streaklines} % (fold)
\label{ssub:streaklines}
%
Streaklines are the connected locations of a continuously injected
stream of massless particles into a flow.
%
They approximate the behavior of a thin stream of dye injected at a certain
position that is often applied in experimental settings to visualize the flow.
%
Formally, a streakline is formed by the connected endpoints of a set of
pathlines with the same start position and end time, but continuously increasing
start time.
%
Using the flow map $\bPhi$, which we introduced earlier, we can formally define
a streakline as
%
\begin{equation}
    \vc(s) = \bPhi(\vx, s, t)\,\text{,}
\end{equation}
%
where $t$ is the current time, $\vx$ is the injection point, and $s$ is the
continuously increasing start time that runs along the curve.
%
Like pathlines, streaklines also become identical to streamlines if the flow
does not change with time.
%

%
While a pathline shows the behavior of the flow over a period of time, a
streakline only ever shows the position of the injected particles at a
single time instant.
%
Streaklines are therefore often animated by continuously increasing the end
time $t$ and injecting more particles.
%
This again mirrors the behavior of injected dye observed in an experiment.
%
% subsubsection streaklines (end)

\subsubsection*{Timelines} % (fold)
\label{ssub:timelines}
%
Timelines are different from all the previous integral lines in that
their seeding structure is not a single point, but a whole line.
%
A timeline is formed by placing a line somewhere in the flow, treating it as a
set of massless particles, and letting the whole line advect with the flow at
once.
%
Formally, a time line is formed by the connected endpoints of a set of pathlines
with the same start and end time, but continuously changing start position.
%
Given a seed curve $\vs(s)$ and start and end times $t_0$ and $t$, we can
formally define a timeline in terms of the flow map as
%
\begin{equation}
    \vc(s) = \bPhi(\vs(s), t_0, t)\,\text{.}
\end{equation}
%
Much like streaklines, timelines are often animated to show the progressive
effect of the flow.
%
As the end time $t$ advances, the line is transported and warped by the flow and
visualizes the way the flow mixes and perturbs a region of the fluid.
%
% subsubsection timelines (end)

\subsubsection*{Integral Surfaces} % (fold)
\label{ssub:integral_surfaces}
%
Integral surfaces can be formed from any of the integral lines by
using a higher-dimensional seeding structure.
%
For stream-, path-, and streaklines this means using a line as the seeding
structure.
%
Timesurfaces are formed by using a surface as the seed.
%
Integral surfaces can be helpful for visualization because they have a better
visual coherency than a number of single lines.
%
The curvature, wrinkling and folding of structures induced by the flow become
easier to grasp when using integral surfaces.
%
On the flip side, integral surfaces have more of a problem with occlusion
compared to lines, as they are more massive.
%
% subsubsection integral_surfaces (end)

%
% subsection integral_lines_and_surfaces (end)
%
\subsection{Vector Field Topology} % (fold)
\label{sub:vector_field_topology}
%
% The topology of a vector field is defined by \emph{critical points},
% \emph{boundary switch points/lines}, \emph{attachment- and detachment
% points/lines}, the accompanying \emph{separatrices} connecting these points,
% and \emph{isolated closed streamlines}.
%
The topology of a vector field is defined by \emph{critical points},
\emph{separation- and attachment points} and the accompanying
\emph{separatrices} connecting them.
%

\subsubsection*{Critical Points} % (fold)
\label{ssub:critical_points}
%
Critical points of a vector field are locations where the magnitude
of the vector becomes zero.
%
They are interesting because they are at the centers of interesting structures
in the vector field.
%
Critical points in vector fields can be sorted into different categories.
%
Which category a critical point belongs to depends on the behavior of the vector
field in its vicinity, which is encoded in its derivative.
%
The Jacobian matrix $\mJ(\vv) = \vv \T{\nabla}$ gathers the partial derivatives
of all components of the vector field.
%
The signs of the real parts of its eigenvalues indicate if the vector field is
attracting or repelling in the vicinity of the critical point.
%
Depending on these signs, the critical point can be categorized into
\emph{sinks} (negative), \emph{sources} (positive), and \emph{saddles} (mixed
signs).
%
The presence of an imaginary part of the eigenvalues/eigenvectors indicates
swirling behavior.
%
For a more in-depth discussion of critical points in vector fields see
\cite{Helman1991}.
%
% subsubsection critical_points (end)

%
% \paragraph{Boundary switch points/lines} are locations where the vector field
% is parallel to the boundary of the domain.
% %
% They separate inflow and outflow boundary regions.
% %

\subsubsection{Separation- and Attachment Lines/Surfaces} % (fold)
\label{ssub:subsubsection_name}
%
Separation- and attachment lines/surfaces occur on no-slip boundaries.
%
They are locations where the flow separates from/attaches to a surface.
%
In a \ac{2D} flow, these are points.
%
In \ac{3D}, they can be point or line structures.
%
Separation- and attachment structures are similar to critical points,
specifically to saddles, in that they are end points of streamlines of the
vector field.
%
However, they are not exactly critical points, as the velocity is zero
everywhere on a no-slip boundary.
%
Instead, they are points where in the limit, the vector is orthogonal to the
boundary, whereas everywhere else, the limit is a vector that is parallel to the
boundary.
%
\cite{Surana2006}
%
% subsubsection subsubsection_name (end)

\subsubsection{Separatrices} % (fold)
\label{ssub:separatrices}
%
Separatrices connect critical points with each other.
%
They are lines in \ac{2D} vector fields and can be lines or surfaces in \ac{3D}
vector fields \cite{Helman1989,Helman1991}.
%
Separatrices start at saddle points or separation-/attachment points.
%
They are streamlines with a special property: Streamlines passing through two
points on opposite sides of the separatrix will diverge from each other near a
saddle or separation-/attachment point in forward or backward flow direction.
%
As such, they separate the flow into distinct regions that streamlines starting
in these regions will never leave.
%
Because streamlines and separatrices are an instantaneous observation (they
exist in steady vector fields or at single points in time in an unsteady vector
field), this does not imply that pathlines will never leave these regions in
unsteady flow.
%
For unsteady flows, the equivalent of separatrices are \acl{LCS}, which we will
cover in the next section.
%
% subsubsection separatrices (end)

%
% subsection vector_field_topology (end)
%
\subsection{Other Features} % (fold)
\label{sub:other_features}
%
Other important features, particularly of unsteady flow fields, are vortices and
\acf{LCS}.
%

%
Vortices are structures in a flow that show a swirling motion around a common
center.
%
Even though this concept seems rather simple, hundreds of years of fluid
dynamics research still has not resulted in a universally accepted formal
definition.
%
This is why there is a plethora of methods for detecting and quantifying
vortices in the literature.
%
Most methods can be sorted into two categories: \emph{region-based} and
\emph{line-based}.
%
We will cover some significant representatives for both categories here.
%
An extensive survey of vortex extraction methods can be found in
\cite{Guenther2018}.
%

%
\ac{LCS} are volumes of a fluid that stay spatially intact for long periods of
time, \eg, massless particles that are placed within such a volume at some point
in time will remain near each other when advecting with the flow.
%
Popular methods for visualizing such structures are the
\emph{\acl{FTLE}} (\acs{FTLE}) and \emph{\acl{FSLE}} (\acs{FTLE}).
%

\subsubsection{Vortices} % (fold)
\label{ssub:vortices}
%
The vorticity is commonly used in fluid dynamics literature to characterize the
rotational behavior of the flow.
%
It is defined as the curl of the flow field $\nabla \times \vv$.
%
The result is a vector field that points in the direction of the local axis of
rotation and whose magnitude indicates the rotation strength.
%
Vorticity magnitude is sometimes used to detect vortices.
%
However, in highly turbulent regions vortices can not be reliably separated from
each other by this method, because the vorticity might be high everywhere.
%
Also, slowly rotating vortices with a relatively low vorticity magnitude might
be missed.
%
This is why other methods often perform better at detecting vortices.
%

%
The $\lambda_2$-criterion \cite{Jeong1995} identifies vortex core regions by
investigating the eigenvalues of the tensor $\mS^2+\bOmega^2$, where $\mS$ and
$\bOmega$ are the symmetric and antisymmetric parts of the Jacobian.
%
If the local tensor has at least two negative eigenvalues (\ie, if the middle
eigenvalue $\lambda_2$ is smaller than zero), a point is considered to belong to
a vortex core region.
%
This tends to generate a lot of small regions that are not considered to be
significant and have to be filtered out based on their size.
%
% subsubsection vortices (end)
%
\subsubsection*{Lagrangian Coherent Structures} % (fold)
\label{ssub:lagrangian_coherent_structures}
%

%
% subsubsection lagrangian_coherent_structures (end)
%
% subsection other_features (end)
%
% Direct
% =====
% Reduction to scalar field (e.g. velocity magnitude)
% Arrow plot
%
% Image based
% ===========
% Spot noise??
% LIC
% Texture advection
%
% Integration based
% =================
% Streamlines
% Pathlines
% Timelines
% Streaklines
% -surfaces !!!
%
% Topology
% ========
% Critical Points
% Boundary Switch points
% Separatrices
% Attachment-/Detachment Points
% Isolated closed streamlines
% Tracking critical points (fff) !!! (tracking isosurface similar)
% Collapse/bifurcation of critical points
%
% Feature based
% =============
% \lambda_2, Q-criterion, vorticity
% Vortex core lines !!! (Tensor core lines)
% - integration based
% - local (PV, Sujudi/Haimes)
% Lagrangian coherent structures/FTLE/FSLE !!! (deformation of surface)
%
% section vector_fields (end)

\section{Second-Order Tensor Fields} % (fold)
\label{sec:tensor_fields}

% section tensor_fields (end)

% chapter fundamentals (end)