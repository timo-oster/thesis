\chapter{Turbulent Combustion Simulation and -Visualization} % (fold)
\label{cha:turbulent_combustion}
%
\lettrine[lines=3, lhang=0.05]{C}{ombustion} is one of the cornerstones of our
civilization.
%
Its uses range from providing light to carrying objects into space.
%
The majority of high-tech combustion processes occur under turbulent conditions.
%
The nature of turbulent flow, and therefore also turbulent combustion, is still
being actively studied by the scientific community.
%
Reducing the pollutant emissions and increasing the efficiency of combustion
processes is essential in todays world where we are confronted more and more
often with the realization that our resources are finite and the damage we do to
our environment can not easily be undone.
%

%
Simulations are an invaluable tool in the design of improved combustion
processes.
%
They allow to test different setups quickly and for a relatively low cost.
%
Efficient simulations need models of the relevant physical and chemical
processes.
%
As the demands on the accuracy of such models rises, more detailed insight into
the low-level phenomena of turbulent combustion is needed.
%
Obtaining this insight via experiments is challenging.
%
Often only a small number of variables can be observed at the same time and
observations are frequently limited to a \ac{2D} slice.
%

%
Another approach is \acf{DNS}, which is sometimes referred to as a ``numerical
experiment''.
%
In \ac{DNS}, the Navier-Stokes equation is directly solved on a very fine grid,
without using any higher-level modeling assumptions.
%
This is computationally very expensive, which is why \ac{DNS} is typically
performed on supercomputers using hundreds or thousands of cores.
%
In the simulation, all variables are available in full spatial and temporal
resolution.
%

%
Analyzing the data from such simulations poses a different challenge.
%
Due to the high spatial and temporal resolution, the raw data produced by a
single \ac{DNS} run can range from Terabytes to Petabytes.
%
Data of this size can not be written to disk or transferred over a network in a
reasonable amount of time, even if the enormous storage space that is required
was available.
%
This limits the post-analysis of the data to spatially or temporally downsampled
versions which lose a lot of important information.
%
In recent years, the subject of \emph{in-situ} analysis and visualization has
therefore gained popularity.
%
The idea is to process the data while it is still in memory during the
simulation.
%
The raw data is then discarded and only the results, which are typically much
smaller in size, are stored on disk.
%

%
This thesis contains two new in-situ focused approaches for analysis and
visualization of turbulent combustion \ac{DNS}.
%
To provide some important context, this chapter provides a short introduction
into the field of turbulent combustion research, the basics of turbulent
combustion modeling and simulation, and an overview of relevant research
concerning in-situ and post-processing of turbulent combustion data in
particular and large-scale simulations in general.
%

\section{Combustion} % (fold)
\label{sec:combustion}
%
Combustion is the exothermic chemical reaction of a fuel and an oxidizer into
oxidized products and heat.
%
We will only concern ourselves with the combustion of gases, which is the
most common case used in industrial applications.
%
The oxidizer is usually oxygen, while fuels can range from simple ones such as
hydrogen or methane to complex organic fuels.
%

%
A combustion process is a complex system of elementary chemical reactions
transforming various chemical species into each other while absorbing or
releasing heat.
%
It involves reactants and products but also various intermediate species which
can be more or less stable and are often radicals with unpaired electrons.
%
The process can additionally be influenced by the diluting presence of inert
gases that do not participate in the reaction.
%

%
The reaction rate of each elementary reaction in an infinitesimal volume is
dependent on the amounts (\ie, mass) of the different chemical species as well
as the current temperature.
%
These characteristic quantities are intrinsically linked to the fluid motion of
the gas:
%
As the gas is deformed and transported by the flow, the local concentration and
temperature gradients change, which in turn control the diffusion of chemical
species and heat.
%
Through diffusion, the local mixture and temperature changes, which in turn
influences the reaction rates.
%
As the chemical reactions produce or consume heat, the gas expands or contracts,
which in turn influences the fluid's velocity (the expanded gas has to go
somewhere) and viscosity (molecules that are farther away from each other
interact less).
%
This again influences the mixing and transport of the gas, and so the cycle goes
on and on.
%
\subsection{Laminar Flames} % (fold)
\label{sub:laminar_flames}
%

%
\subsubsection{Laminar Premixed Flames} % (fold)
\label{ssub:laminar_premixed_flames}
%

%
% subsubsection laminar_premixed_flames (end)
%
\subsubsection{Laminar Diffusion Flames} % (fold)
\label{ssub:laminar_diffusion_flames}
%

%
% subsubsection laminar_diffusion_flames (end)
%
% subsection laminar_flames (end)
%
\subsection{Turbulent Flames} % (fold)
\label{sub:turbulent_flames}
%
\subsubsection{Turbulent Premixed Flames} % (fold)
\label{ssub:turbulent_premixed_flames}
%

%
% subsubsection turbulent_premixed_flames (end)
%
\subsubsection{Turbulent Diffusion Flames} % (fold)
\label{ssub:turbulent_diffusion_flames}
%

%
% subsubsection turbulent_diffusion_flames (end)
%
% subsection turbulent_combustion (end)
%
% section combustion (end)
%
\section{Turbulent Combustion Modeling and Simulation} % (fold)
\label{sec:simulation_of_turbulent_combustion}
%
\subsection{Chemical Schemes} % (fold)
\label{sub:chemical_schemes}
%

%
% subsection chemical_schemes (end)
%
\subsection{The Flamelet Assumption} % (fold)
\label{sub:the_flamelet_assumption}
%

%
% subsection the_flamelet_assumption (end)
%
\subsection{High-Level Models: \acs{RANS} and \acs{LES}} % (fold)
\label{sub:high_level_models}
%

%
% subsection high_level_models_for_turbulent_combustion_ac (end)
%
\subsection{Direct Numerical Simulations} % (fold)
\label{sub:direct_numerical_simulations}
%

%
% subsection direct_numerical_simulations_for_turbulent_combustion (end)
%
% section modeling_and_simulation_of_turbulent_combustion (end)
%
% chapter turbulent_combustion (end)