\chapter{Interpolating the Transformation for New Surface Patches} % (fold)
\label{sec:transform_interpolation}
%
\Todo[inline]{Add introductory paragraph}
%
If the deformation across the parent patch was completely uniform, we could just
pass this deformation on to the child patches after a split.
%
However, in general there will be slight differences in the transformations
each of the four ghost particles have experienced.
%
At the time of the split, we still have this information for the time interval
since the parent patch was last reset (or first created).
%

%
Let $\vx_i(t_{k-1})$ be the positions of the ghost particles (relative to the
central point) at the last reset of the parent patch, or at the start time $t_s$
if the patch was not reset since that time.
%
Let $\vx_i(t_k)$ be the ghost particle positions at the time of the split.
%
Then $\mE_{t_{k-1}}^{t_k}$ obtained via \eqref{eqn:deformation_reconstruction}
can be thought of as the solution of the system
%
{\small
\begin{gather}
    \begin{pmatrix}
        \T{\vb_1(t_{k-1})} \\
        \T{\vb_2(t_{k-1})} \\
        \T{\vn(t_{k-1})}
    \end{pmatrix}
    \T{\left(\mE_{\,t_{k-1}}^{\,t_k}\right)}
    =
    \begin{pmatrix}
        \T{\vb_1(t_k) }\\
        \T{\vb_2(t_k)} \\
        \T{\vn(t_k)}
    \end{pmatrix}\\
    \vb_1(t) = (\vx_1(t) - \vx_2(t))/2 \\
    \vb_2(t) = (\vx_3(t) - \vx_4(t))/2\, \text{.}
\end{gather}
}
%
In other words, $\mE_{\,t_{k-1}}^{\,t_k}$ is the average of the transformations
that map the corresponding ghost particles to each other exactly.
%
\begin{figure}[t]
    \centering
    \setlength{\figurewidth}{0.9\linewidth}
    %
%
\begin{tikzpicture}[
	font=\small,
	thick,
	scale=\figurewidth/1cm*0.13,
	line join=round]
\tikzstyle{ghostparticle} = [fill=white, arrows=-o]
\tikzstyle{b1style} = [mycolor4, arrows=-latex]
\tikzstyle{b2style} = [mycolor1, arrows=-latex]
\tikzstyle{b1dstyle} = [very thick, mycolor4!50, arrows=-latex]
\tikzstyle{b2dstyle} = [very thick, mycolor1!50, arrows=-latex]
\tikzstyle{ellipse1style} = [mycolor3!70]
\tikzstyle{ellipse2style} = [mycolor2!70]

\coordinate (center1) at (0, 0);
\coordinate (x1) at (1.5, 0.15);
\coordinate (x2) at (-1, -0.1);
\coordinate (x3) at (0.15, 1.5);
\coordinate (x4) at (-0.1, -1);
\coordinate (b1) at (1.25, 0.125);
\coordinate (b2) at (0.125, 1.25);
\coordinate (vx) at ($ (center1)!2/3!(b1) + (center1)!2/3!(b2)$);

% normal
% \draw [fill=black, arrows=-latex]
% 	(center1) -- (0, 0, 1) node [below left] {$\vn$};

\fill [ellipse1style] (center1) circle [radius=0.4];
\fill [ellipse2style] (vx) circle [radius=0.4];

\node [below right] at (center1) {$\vx$};

\draw [ghostparticle] (center1) -- (x1) node [above] {$\vx_1$};
\draw [ghostparticle] (center1) -- (x2) node [left] {$\vx_2$};
\draw [ghostparticle] (center1) -- (x3) node [above] {$\vx_3$};
\draw [ghostparticle] (center1) -- (x4) node [below] {$\vx_4$};

\draw [b1style] (center1) -- (b1) node [below left] {$\vb_1$};
\draw [b2style] (center1) -- (b2) node [below left] {$\vb_2$};


\draw [b1dstyle] (center1) -- ($ (center1)!2/3!(b1) $)
		node [above, pos=0.6] {$a$};
\draw [b2dstyle, shorten >=0.1cm] ($ (center1)!2/3!(b1) $) --
		($ (center1)!2/3!(b1) + (center1)!2/3!(b2)$)
		node [left, pos=0.4] {$b$};

\draw [fill=black] (vx) circle [radius=0.08]
		node [right, xshift=0.05cm] {$\vx'$};

\begin{scope}[shift={(3.5, 0)}]
	\coordinate (center2) at (0, 0);
	\coordinate (x1) at (2, -1);
	\coordinate (x2) at (-1, 0.5);
	\coordinate (x3) at (2, 1);
	\coordinate (x4) at (-1, -0.5);
	\coordinate (b1) at (1.5, -0.75);
	\coordinate (b2) at (1.5, 0.75);
	\coordinate (vx2) at ($ (center2)!2/3!(b1) + (center2)!2/3!(b2)$);

	% normal
	% \draw [fill=black, arrows=-latex]
	% 	(center2) -- (0, 0, 1) node [below] {$\vn$};
	\fill [ellipse1style] (center2) circle [x radius=0.5, y radius=0.4];
	\fill [ellipse2style] (vx2) circle [x radius=0.6, y radius=0.4];

	\node [below] at (center2) {$\vx$};

	\draw [ghostparticle] (center2) -- (x1) node [right] {$\vx_1$};
	\draw [ghostparticle] (center2) -- (x2) node [left] {$\vx_2$};
	\draw [ghostparticle] (center2) -- (x3) node [right] {$\vx_3$};
	\draw [ghostparticle] (center2) -- (x4) node [left] {$\vx_4$};

	\draw [b1style] (center2) -- (b1) node [below left] {$\vb_1$};
	\draw [b2style] (center2) -- (b2) node [above left] {$\vb_2$};

	\draw [b1dstyle] (center2) -- ($ (center2)!2/3!(b1) $)
			node [above right, pos=0.6] {$a$};
	\draw [b2dstyle, shorten >=0.1cm] ($ (center2)!2/3!(b1) $) -- (vx2)
			node [above left, pos=0.4] {$b$};

	\draw [fill=black] (vx2) circle [radius=0.08]
		node [right, xshift=0.05cm] {$\vx'$};
\end{scope}

\draw [very thick, ellipse1style, shorten <=0.6cm, shorten >=0.6cm, arrows=-latex]
		(center1) to[out=-45, in=225]
			node [below, midway, text=black, yshift=-0.1cm] {$\mE$}
		(center2);

\draw [very thick, ellipse2style, shorten <=0.6cm, shorten >=0.7cm, arrows=-latex]
		(vx) to[out=25, in=145]
			node [above, midway, text=black, yshift=0.1cm] {$\mE'$}
		(vx2);

\end{tikzpicture}
    \tikzset{external/export=false}
    \caption{
        Interpolating the transformation at an offset position $\vx'$.
        The ghost particles in the direction of $\vx'$ have been stretched more
        than their counterparts during the time interval. Therefore the
        interpolated $\mE'$ has a stronger stretching effect on the local
        neighborhood of $\vx'$
        (\protect\tikz[baseline=-0.5ex]{
        \protect\fill[mycolor2!70] circle [radius=1ex];})
        than $\mE$ stretches the
        local neighborhood of $\vx$
        (\protect\tikz[baseline=-0.5ex]{
        \protect\fill[mycolor3!70] circle [radius=1ex];}).
        }
    \label{fig:interpolate_base_vectors}
    \tikzset{external/export=true}
\end{figure}
%

%
For the central point, it makes sense to weight those transformations equally,
but if we want to initialize a new patch whose center is slightly offset, we
get a more accurate result if we adjust the weights depending on where the
new center is located.
%
For this purpose, we parameterize the tangent space of the micro-patch by
expressing it as a linear combination of the two base vectors $\vb_1(t)$ and
$\vb_2(t)$.
%
We can now express the location of the new center $\vx'$ in this new basis:
%
\begin{equation*}
    \vx' = \lambda_1 \vb_1(t_k) + \lambda_2 \vb_2(t_k)\, \text{.}
\end{equation*}
%
These coordinates correspond directly to the coordinates of
$(\mE_{\,t_{k-1}}^{\,t_k})^{-1} \vx'$ at time $t_{k-1}$:
%
\begin{equation*}
    (\mE_{\,t_{k-1}}^{\,t_k})^{-1} \, \vx'
      = \lambda_1 \vb_1(t_{k-1}) + \lambda_2 \vb_2(t_{k-1})\, \text{.}
\end{equation*}
%
If $\lambda_1$ is positive, \ie, if $\vx'$ is located more towards $\vx_1(t_k)$
than towards $\vx_2(t_k)$, we want $\vx_1(t_k)$ to have a stronger influence on
the result.
%
The same applies to the direction of $\vb_2$.
%
We therefore compute new interpolated base vectors $\vb'_{1,2}(t)$ by weighting
the ghost particle positions with $\lambda_1$ and $\lambda_2$:
%
{\small
\begin{align}
    \vb'_1(t) =& (1+2\lambda_1) \, \vx_1(t) - (1-2\lambda_1) \, \vx_2(t) \\
    \vb'_2(t) =& (1+2\lambda_2) \, \vx_3(t) - (1-2\lambda_2) \, \vx_4(t)
        \, \text{.}
\end{align}
}
%
The adjusted transformation ${\mE_{\,t_{k-1}}^{\,t_k}}'$ is then the solution to
the system
%
{\small
\begin{equation}
    \begin{pmatrix}
        \T{\vb'_1(t_{k-1})} \\
        \T{\vb'_2(t_{k-1})} \\
        \T{\vn(t_{k-1})}
    \end{pmatrix}
    \T{\left({\mE_{\,t_{k-1}}^{\,t_k}}'\right)}
    =
    \begin{pmatrix}
        \T{\vb'_1(t_k)} \\
        \T{\vb'_2(t_k)} \\
        \T{\vn(t_k)}
    \end{pmatrix}\, \text{.}
\end{equation}
}
%
The complete transformation of a micro-patch that has been split off from a
parent at some intermediate time $t_k$ is then obtained by
%
\begin{equation}
    \mE_{\,t_s}^{\,t_e} = \mE_{\,t_n}^{\,t_e} \cdot
                      \mE_{\,t_{n-1}}^{\,t_n} \cdot \; \cdots \; \cdot
                      \mE_{\,t_k}^{\,t_{k+1}} \cdot
                      {\mE_{\,t_{k-1}}^{\,t_k}}' \cdot
                      \mE_{\,t_s}^{\,t_{k-1}}\, \text{.}
\end{equation}
%
Here, $\mE_{\,t_s}^{\,t_{k-1}}$ is the transformation of the parent patch from
the start of the interval up to the time when its ghost particles were last
reset before the split operation at $t_k$.
%
If $t_s > t_{k-1}$, it is omitted.
%
Here, $\mE_{\,t_s}^{\,t_{k-1}}$ is the transformation of the parent patch from
the start of the interval up to the time when its ghost particles were last
reset before the split operation at $t_k$.
%
${\mE_{\,t_{k-1}}^{\,t_k}}'$ is the estimated transformation at a point with a
slight offset from the center of the parent patch in the interval before the
split.
%
${\mE_{\,t_{k-1}}^{\,t_k}}'$ is the estimated transformation at a point with a
slight offset from the center of the parent patch in the interval before the
split.
%
% section interpolating_the_transformation_for_new_patches (end)
