
%
\section{Summary and Conclusion} % (fold)
\label{sec:fst_summary_and_conclusion}
%
We have introduced a novel method for tracking the flame surface on-the-fly
during a massively parallel distributed combustion simulation.
%
Because of this parallelism, our method uses independent micro-patches to
represent the flame surface.
%
Patches are split and merged adaptively depending on their deformation in order
to ensure an adequate sampling.
%
Based on the relative movement of surface points, we compute the tangential
deformation experienced by a micro-patch over an arbitrary time interval.
%

%
We validate the accuracy of our method on an analytic ground truth function
and demonstrate its applicability on a real-world turbulent combustion
simulation.
%

%
Our method is the first to track the complete flame surface over the course of a
simulation, making an important step towards the investigation of
turbulence-chemistry interactions on a new scale.
%
With the tangential deformation gradient, we introduce a new time-integrated
quantity describing the effect of turbulent flow on the flame that is a direct
consequence of the advantages of an on-the-fly analysis compared to the
traditional snapshot-based postprocessing.
%
% section summary_and_conclusion (end)