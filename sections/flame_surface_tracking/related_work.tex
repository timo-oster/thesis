%
\section{Related Work} % (fold)
\label{sec:fst_background_and_related_work}
%
The evolution of simulation variables on the path of single points on the flame
surface has been used in multiple works in combustion literature
\cite{Yeung1990,Sripakagorn2004,Scholtissek2017}.
%
In these works only a relatively small number of points on the surface are
tracked and they do not consider the relative tangential movement of points.
%
Stretching of a flow restricted to a surface has been investigated similarly to
our approach by Garth \etal{}~\cite{Garth2008}.
%

%
Tracking different kinds of features has been the subject of numerous works in
the field of flow visualization.
%
A lot of research deals with tracking volumetric features over time
\cite{Silver1997,Sauer2014,Clyne2013,Mascarenhas2009,Duque2012,Muelder2009},
while some methods focus on point-\cite{Garth2004,Theisel2003a} or line-type
features \cite{Bremer2010}.
%
Most of these methods are designed for datasets that fit into the main memory of
a consumer-grade computer, although some are explicitly designed to work in
distributed-memory settings \cite{Wang2013}.
%

%
Surface extraction and tracking is another large field of research.
%
In the context of this work, particle-based isosurface extraction methods, like
the one proposed by Crossno \etal{}~\cite{Crossno1997}, as well as methods for
tracking evolving surfaces over time \cite{Krishnan2009,Buerger2009,Berres2015}
are relevant.
%
Of particular interest for our application is the work by Camp
\etal{}~\cite{Camp2012}, which deals with stream surface integration in a
distributed-memory environment.
%

%
To the best of our knowledge, there is no existing approach that solves the
problem of tracking a time surface in a large distributed-memory simulation
while maintaining temporal correspondence between surface points.
%
% section background_and_related_work (end)