
%
\section{Background and Related Work} % (fold)
\label{sec:fst_background_and_related_work}
%
The flame surface is the location of a flame where chemical reactions take
place.
%
In premixed flames, where fuel and oxidizer are mixed before ignition, it is the
interface between burnt and unburnt gases.
%
In non-premixed combustion, it is the burning part of the interface between
fuel and oxidizer.
%
The flame surface is defined as an isosurface of a scalar variable such as the
temperature or the mixture fraction of fuel and oxidizer.
%
Over the course of the simulation, the flame surface is influenced by multiple
factors.
%
Chemical reactions move the interface between gases by transforming them,
turbulent flow transports the gases, moving and deforming the flame surface,
and molecular and thermal diffusion have a smoothing effect on the surface
shape.
%
An example of the effects of tangential stretching on the burning behavior of
the flame is given by Renard \etal in \cite{Renard1999}.
%
For a more detailed introduction to turbulent combustion, see ``Theoretical and
Numerical Combustion'' by Poinsot and Veynante \cite{Poinsot2012}.
%
A modern direct numerical simulation code is presented by Abdelsamie \etal in
\cite{Abdelsamie2016}.
%

%
The evolution of simulation variables on the path of single points on the flame
surface has been used in multiple works in combustion literature
\cite{Sripakagorn2004,Scholtissek2017}.
%
In these works only a small number of points on the surface are tracked and they
do not consider the relative tangential movement of points.
%
Our approach allows these kinds of studies on a much larger scale, providing
data for the whole surface, which enables statistical evaluation and visual
analysis.
%
Stretching of a flow restricted to a surface has been investigated similarly to
our approach by Garth \etal~\cite{Garth2008}.
%

%
Tracking different kinds of features has been the subject of numerous works in
the field of flow visualization.
%
A lot of research deals with tracking volumetric features over time
\cite{Silver1997,Sauer2014,Clyne2013,Mascarenhas2009,Duque2012,Muelder2009},
while some methods focus on point-\cite{Garth2004,Theisel2003a} or line-type
features \cite{Bremer2010}.
%
Most of these methods are designed for datasets that fit into the main memory of
a consumer-grade computer, although some are explicitly designed to work in
distributed-memory settings \cite{Wang2013}.
%

%
Surface extraction and tracking is another large field of research.
%
In the context of this work, particle-based isosurface extraction methods, like
the one proposed by Crossno \etal \cite{Crossno1997}, as well as methods for
tracking evolving surfaces over time \cite{Krishnan2009,Buerger2009,Berres2015}
are relevant.
%
Of particular interest in the context of this work is the work by Camp \etal
\cite{Camp2012}, wich deals with stream surface integration in a
distributed-memory environment.
%

%
Visualization methods designed for an on-the-fly setting in large scale parallel
simulations include the rendering of volume, surface and point data
\cite{Akiba2007,Yu2010}.
%
These provide an overview of the state of a simulation while it is running and
allow a visual analysis of the data for all time steps without saving them to
disk.
%
Agranovsky \etal~\cite{Agranovsky2014} proposed storing flow fields computed
during a simulation as a collection of path lines.
%
This Lagrangian form allows a compressed storage of the flow without relying on
snapshots with large temporal gaps.
%

%
To the best of our knowledge, there is no existing approach that solves the
problem of tracking a time surface in a large distributed-memory simulation
while maintaining temporal correspondence between surface points.
%
% section background_and_related_work (end)