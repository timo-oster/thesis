\lettrine[lines=3, findent=0.3pt, nindent=0pt, loversize=-0.015,
lhang=0.12]{U}{nsteady effects} and their incorporation into turbulent combustion
models are still an area of active research in the combustion community.
%
In order to observe and analyze such effects, researchers need access to data
with high temporal resolution.
%
Storing raw \ac{DNS} data in this resolution is only practical for very short
time intervals.
%
To observe behavior that stretches over longer time spans, or whose occurrence
cannot be easily predicted, in-situ processing becomes an absolute necessity.
%

%
The flame surface is central to many combustion models and -studies.
%
Quantities such as surface speed, -stretch, -curvature, -area and their
relation to the structure and behavior of the flame are frequently discussed in
the literature.
%
In order to observe the detailed unsteady behavior of the flame surface, we need
an algorithm to track it during the simulation.
%
This chapter presents an approach for such an in-situ surface tracking.
%
Our goal is to capture the shape of the surface, the history of individual
surface points, and the local tangential deformation of the surface over
extended periods of time.
%
This also allows studies of the history of single surface points over extended
time periods, such as described by Yeung \etal~\cite{Yeung1990} and Sripakagorn
\etal~\cite{Sripakagorn2004}, to be extended to the complete flame surface.
%
Additionally, we extend the notion of instantaneous surface stretch (such as
described by Poinsot and Veynante~\cite{Poinsot2012}) to the tangential
deformation of the surface over arbitrary time intervals.
%

%
Our algorithm must overcome the following challenges:
%
\begin{enumerate}
    \item \label{it:parallel}
        The massively parallel simulation, distributes its domain across a large
        number of processes. The surface tracking must be similarly
        parallelizeable.
    \item \label{it:in-situ}
        Performing analysis in-situ during a simulation means that at any
        point in time, only data for the current time step is available in
        memory and there is no way of going back in time to retrieve previous
        information.
    \item \label{it:refinement}
        The surface is expected to undergo significant deformation over time,
        making an adaptive refinement and coarsening necessary.
    \item \label{it:deformation}
        The tangential deformation must be reconstructed for areas of the
        surface that have been refined and/or coarsened multiple times over an
        arbitrary time interval.
\end{enumerate}
%

%
\Cref{it:parallel} precludes the use of a mesh or space partition data
structure with neighborhood information, as the global nature of operations in
such a structure is not well suited for a massively parallel algorithm.
%
Instead, we represent the surface as a number of independent micro-patches
consisting of a central point and four ghost particles measuring the local
surface deformation.
%
This is described in \cref{sub:surface_patch_representation}.
%
\Cref{it:in-situ,it:refinement} mean that we need to refine
the micro-patches adaptively before they are significantly distorted.
%
We therefore introduce a way of monitoring the distortion of a patch and split
it before the distortion becomes too large in
\cref{sub:splitting_surface_patches}.
%
This method of tracking and refining a surface without explicit neighbor
information is a major contribution of this work.
%
Based on the behavior of the micro-patches over time, we introduce the
computation of the tangential deformation gradient (\cref{it:deformation})
for arbitrary time intervals in
\cref{sub:reconstructing_tangential_surface_deformation}.
%
% section introduction (end)