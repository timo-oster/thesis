%
\lettrine[lines=3, findent=-12pt, nindent=7pt, slope=7pt]{A}{ central topic} of today's combustion research are the interactions between
turbulent flow and chemical reactions in flames.
%
In this context, the flame surface, \ie, the thin region where reaction happens,
is of central importance.
%
Temperature, concentrations of chemical species and other variables on the flame
surface show the state of the reaction.
%
The transport and deformation of the flame surface over time show the joint
influence of chemistry and turbulence on the combustion process.
%

%
To study the detailed mechanisms of turbulent combustion, \acp{DNS} are used.
%
These simulations directly compute the fundamental physical and chemical
equations on a high-resolution grid without any high-level modeling assumptions.
%
As a consequence, all physical and chemical quantities can be analyzed in detail.
%
However, the computational costs are very high and the data sets produced can
easily reach into the tera- or petabyte range.
%
This makes them practically impossible to store completely or transfer over a
network connection.
%
For this reason, only single snapshots of the simulation are commonly stored
on disk and analyzed in post-processing.
%
This approach only allows investigating the state of the flame surface at single
points in time.
%
The temporal distance between subsequent snapshots is generally too large to
interpolate between them in any meaningful way or make any statement about
the correspondence of surface points.
%

%
In this work, we introduce an algorithm for tracking the flame surface
on-the-fly during the simulation, while the necessary data is still in memory.
%
Our goal is to capture the shape of the surface, the history of individual
surface points, and the local tangential deformation of the surface over
extended periods of time.
%
This allows us to expand studies of the history of single points on the surface
over extended time periods, such as described by Sripakagorn
\etal~\cite{Sripakagorn2004}, to the complete flame surface.
%
Additionally, we extend the notion of instantaneous surface stretch (such as
described by Poinsot and Veynante\cite{Poinsot2012}) to the tangential
deformation of the surface over arbitrary time intervals.
%

%
Our algorithm must overcome the following challenges:
%
\begin{enumerate}
    \item \label{it:parallel}
        The massively parallel simulation, distributes its domain across a large
        number of processes. The surface tracking must be similarly
        parallelizeable.
    \item \label{it:on-the-fly}
        Performing analysis on-the-fly during a simulation means that at any
        point in time, only data for the current time step is available in
        memory and there is no way of going back in time to retrieve previous
        information.
    \item \label{it:refinement}
        The surface is expected to undergo significant deformation over time,
        making an adaptive refinement and coarsening necessary.
    \item \label{it:deformation}
        The tangential deformation must be reconstructed for areas of the
        surface that have been refined and/or coarsened multiple times over an
        arbitrary time interval.
\end{enumerate}
%
\Cref{it:parallel} precludes the use of a mesh or space partition data
structure with neighborhood information, as the global nature of operations in
such a structure is not well suited for a massively parallel algorithm.
%
Instead, we represent the surface as a number of independent micro-patches
consisting of a central point and four ghost particles measuring the local
surface deformation.
%
This is described in \cref{sub:surface_patch_representation}.
%
\Cref{it:on-the-fly,it:refinement} mean that we need to refine
the micro-patches adaptively before they are significantly distorted.
%
We therefore introduce a way of monitoring the distortion of a patch and split
it before the distortion becomes too large in
\cref{sub:splitting_surface_patches}.
%
This method of tracking and refining a surface without explicit neighbor
information is a major contribution of this work.
%
Based on the behavior of the micro-patches over time, we introduce the
computation of the tangential deformation gradient (\cref{it:deformation})
for arbitrary time intervals in
\cref{sub:reconstructing_tangential_surface_deformation}.
%
% section introduction (end)