
%
\section{Mathematical Basis} % (fold)
\label{sec:fst_mathematical_basis}
%
In this section, we give a brief introduction into the mathematical basis of the
problems we want to solve.
%
First, we derive the equation for the movement of the flame surface over time.
%
Then, we introduce the tangential deformation gradient, which describes the
distortion an infinitesimally small section of the surface experiences in a
certain time interval.
%
\subsection{Tracking the Flame Surface} % (fold)
\label{sub:tracking_the_flame_surface}
%
We view the flame surface as an implicit surface $s(\vx, t) = \Gamma$.
%
This scalar function is transported by the fluid velocity $\vv(\vx, t)$ and
influenced by diffusion and chemical reaction processes.
%
A point $\vx$ with velocity $\vu$ tracking the surface has a zero Lagrangian
derivative, \ie, the value of $s$ at the point does not change over time.
Therefore
%
\begin{equation*}
    \frac{\mathrm{D}s}{\mathrm{D}t} =
    \frac{\partial s}{\partial t} + \nabla s \cdot \vu = 0 \, \text{.}
\end{equation*}
%
This constrains the component of $\vu$ that is normal to the surface.
%
The tangential component of $\vu$ is constrained by the fluid velocity through
%
\begin{equation*}
    \Norm{\vu - \vv}^2 \rightarrow \text{min}\,\text{.}
\end{equation*}
%
Combining both constraints using Lagrange multipliers, the solution for
the velocity of a point on the implicit surface is given by
%
\begin{equation}
    \vu = \vv - \frac{\frac{\partial s}{\partial t} + \nabla s \cdot \vv}
                     {\Norm{\nabla s}^2}
                \nabla s \,\text{.}
\end{equation}
%
In combustion literature, this equation is expressed as
%
\begin{equation}
\begin{gathered}
    \vu = \vv + s_d\,\vn \,\text{,} \\
    \text{with}\quad s_d = \frac{\partial s}{\partial t} \frac{1}{\Norm{\nabla s}}
                       + \frac{\nabla s}{\Norm{\nabla s}} \vv \quad
    \text{and}\quad  \vn = -\frac{\nabla s}{\Norm{\nabla s}}\,\text{.}
\end{gathered}
\end{equation}
%
Here, $s_d$ is the speed of flame propagation normal to the surface and relative
to the fluid velocity and $\vn$ is the unit surface normal.
%
\subsection{Tangential Deformation of an Implicit Surface in a Flow} % (fold)
\label{ssb:tangential_surface_deformation}
%
In the previous section, we showed how a point on the surface moves over time.
%
We now derive the tangential deformation gradient, which encodes the relative
movement of surface points in an infinitesimal neighborhood.
%

%
We consider the starting position of a point on the surface at some time $t_0$.
%
\Wlog we assume that this point is at the origin $\vNull$.
%
Let $\bchi(\vx,t)$ be the mapping function that maps a point $\vx$ at $t_0$ to
its position at time $t > t_0$ after moving with the surface.
%
\Wlog we assume that $\bchi(\vNull,t) = \vNull$.
%
The spatial deformation gradient $\mF = \nabla \bchi$, which encodes the behavior
of a point $\vx$ in an infinitesimally small neighborhood around $\vNull$, can
then be expressed as
%
\begin{equation*}
    \mF\, \vx = \bchi(\vx, t)\quad
        \text{for $\vx \rightarrow \vNull$\,.}
\end{equation*}
%

%
We are only interested in the tangential part of this deformation, \ie, the
behavior in a tangential coordinate system that moves and rotates with the
surface.
%
In order to isolate the tangential component, we first need to eliminate this
rotation.
%
A (right) polar decomposition $\mF = \mU \, \mP$ separates the rotational
part $\mU$ from the rest of the deformation.
%
The tangential component is now obtained by projecting $\mP = \T\mU \, \mF$ into
the local tangent space at time $t_0$.
%
Let $\mB$ be a matrix with orthonormal basis vectors of this tangent space as
its columns.
%
Then the tangential deformation gradient $\hat\mF$ is
%
\begin{equation} \label{eqn:tangent_deformation_tensor}
    \hat\mF = \T\mB \, \T\mU \, \mF \, \mB\,\text{.}
\end{equation}
%
Note that $\mB \in \RRSet^{3\times2}$; and thus $\hat\mF \in \RRSet^{2\times2}$.
%
See \cref{fig:tangential_deformation} for a visual interpretation of this
transformation.
%
$\hat\mF$ now encodes the behavior of a point $\hat\vx$ in an infinitesimally
small neighborhood around $\hat\vNull$ in tangent space:
%
\begin{equation*}
    \hat\mF \cdot \hat\vx = \T\mB \, \bchi(\mB \, \hat\vx, t)\quad
    \text{for $\vx \rightarrow \vNull$\,.}
\end{equation*}
%
\Todo[inline]{closing sentence. Don't end with equation.}
%
\begin{figure}[tb]
    \centering
    \setlength{\figurewidth}{0.8\textwidth}
    %
%
\begin{tikzpicture}[scale=\figurewidth/1cm*0.14,
                    line join=round]
\tikzstyle{facefill} = [fill=mycolor1,fill opacity=0.0]
\tikzstyle{planefill} = [fill=mycolor2,fill opacity=1.0]

\coordinate (lll) at (-0.5, -0.5, -0.5);
\coordinate (ull) at ( 0.5, -0.5, -0.5);
\coordinate (lul) at (-0.5,  0.5, -0.5);
\coordinate (uul) at ( 0.5,  0.5, -0.5);
\coordinate (llu) at (-0.5, -0.5,  0.5);
\coordinate (ulu) at ( 0.5, -0.5,  0.5);
\coordinate (luu) at (-0.5,  0.5,  0.5);
\coordinate (uuu) at ( 0.5,  0.5,  0.5);

\coordinate (ll) at (-0.5, 0.0, -0.5);
\coordinate (lu) at (-0.5, 0.0, 0.5);
\coordinate (ul) at (0.5, 0.0, -0.5);
\coordinate (uu) at (0.5, 0.0, 0.5);


\draw [thick, facefill] (lul) -- (uul) -- (ull) -- (lll) -- cycle;

\draw [thick, facefill] (luu) -- (lul) -- (lll) -- (llu) -- cycle;

\draw [thick, facefill] (llu) -- (ulu) -- (ull) -- (lll) -- cycle;

\draw [thick, planefill] (ll) -- (lu) -- (uu) -- (ul) -- cycle;

% axes inside
\draw [thick] (0, 0, 0) -- (0, 0.5, 0);
\draw [thick] (0, 0, 0) -- (0.5, 0, 0);
\draw [thick] (0, 0, 0) -- (0, 0, 0.5);

\draw [thick, facefill] (luu) -- (uuu) -- (ulu) -- (llu) -- cycle;

\draw [thick, facefill] (luu) -- (uuu) -- (uul) -- (lul) -- cycle;

\draw [thick, facefill] (uuu) -- (uul) -- (ull) -- (ulu) -- cycle;

% axes outside
\draw [thick, arrows=-latex] (0, 0.5, 0) -- (0, 1, 0) node [right] {$\vn$};
\draw [thick, arrows=-latex] (0.5, 0, 0) -- (1, 0, 0); % node  [below right] {$\ve_1$};
\draw [thick, arrows=-latex] (0, 0, 0.5) -- (0, 0, 1); % node [below right] {$\ve_2$};

% transformation
\draw [very thick, arrows=-latex] (1.5, 0) -- (2.5, 0) node [above, midway] {$\mF$};

\begin{scope}[shift={(4, 0)}]
    \coordinate (lll) at (-0.31287012,  -0.84534744,  -0.5);
    \coordinate (ull) at ( 1.30516387,   0.33022307, -0.5);
    \coordinate (lul) at (-1.30516387,  -0.33022307, -0.5);
    \coordinate (uul) at ( 0.31287012,   0.84534744,  -0.5);
    \coordinate (llu) at (-0.31287012,  -0.84534744,   0.5);
    \coordinate (ulu) at ( 1.30516387,   0.33022307,  0.5);
    \coordinate (luu) at (-1.30516387,  -0.33022307,  0.5);
    \coordinate (uuu) at ( 0.31287012,   0.84534744,   0.5);

    \coordinate (ll) at (-0.80901699, -0.58778525, -0.5);
    \coordinate (lu) at (-0.80901699, -0.58778525, 0.5);
    \coordinate (ul) at (0.80901699, 0.58778525, -0.5);
    \coordinate (uu) at (0.80901699, 0.58778525, 0.5);


    \draw [thick, facefill] (lul) -- (uul) -- (ull) -- (lll) -- cycle;

    \draw [thick, facefill] (luu) -- (lul) -- (lll) -- (llu) -- cycle;

    \draw [thick, facefill] (llu) -- (ulu) -- (ull) -- (lll) -- cycle;

    \draw [thick, planefill] (ll) -- (lu) -- (uu) -- (ul) -- cycle;

    \draw [thick] (0, 0, 0) -- (-0.29389263, 0.4045085, 0);
    \draw [thick, arrows=-latex] (0, 0, 0) -- (0.80901699, 0.58778525, 0) node (e1) {};;
    \draw [thick] (0, 0, 0) -- (0, 0, 0.5);

    \draw [thick, facefill] (luu) -- (uuu) -- (ulu) -- (llu) -- cycle;

    \draw [thick, facefill] (luu) -- (uuu) -- (uul) -- (lul) -- cycle;

    \draw [thick, facefill] (uuu) -- (uul) -- (ull) -- (ulu) -- cycle;

    \draw [thick, arrows=-latex] (-0.29389263, 0.4045085, 0) -- (-0.58778525, 0.80901699, 0) node [above right] {$\mU \cdot \vn$};
    % \draw [thick, arrows=-latex] (0.80901699, 0.58778525, 0) -- (0.80901699, 0.58778525, 0) node [right] {$\ve_1$};
    %\node [below right] at (e1) {$\ve_1$};
    \draw [thick, arrows=-latex] (0, 0, 0.5) -- (0, 0, 1); % node [below right] {$\ve_2$};

    % transformation
    \draw [very thick, arrows=-latex] (-1.5, -0.8) -- (-2.5, -1.4) node [above, midway] {$\T\mU$};
\end{scope}

\begin{scope}[shift={(0, -2)}]
    \coordinate (lll) at (-0.75,  -0.5,  -0.5);
    \coordinate (ull) at ( 1.25,  -0.5, -0.5);
    \coordinate (lul) at (-1.25,   0.5, -0.5);
    \coordinate (uul) at ( 0.75,   0.5,  -0.5);
    \coordinate (llu) at (-0.75,  -0.5,   0.5);
    \coordinate (ulu) at ( 1.25,  -0.5,  0.5);
    \coordinate (luu) at (-1.25,   0.5,  0.5);
    \coordinate (uuu) at ( 0.75,   0.5,   0.5);

    \coordinate (ll) at (-1, 0.0, -0.5);
    \coordinate (lu) at (-1, 0.0, 0.5);
    \coordinate (ul) at (1, 0.0, -0.5);
    \coordinate (uu) at (1, 0.0, 0.5);

    \draw [thick, facefill] (lul) -- (uul) -- (ull) -- (lll) -- cycle;

    \draw [thick, facefill] (luu) -- (lul) -- (lll) -- (llu) -- cycle;

    \draw [thick, facefill] (llu) -- (ulu) -- (ull) -- (lll) -- cycle;

    \draw [thick, planefill] (ll) -- (lu) -- (uu) -- (ul) -- cycle;

    \draw [thick] (0, 0, 0) -- (0, 0.5, 0);
    \draw [thick, arrows=-latex] (0, 0, 0) -- (1, 0, 0) node (e1) {};;
    \draw [thick] (0, 0, 0) -- (0, 0, 0.5);

    \draw [thick, facefill] (luu) -- (uuu) -- (ulu) -- (llu) -- cycle;

    \draw [thick, facefill] (luu) -- (uuu) -- (uul) -- (lul) -- cycle;

    \draw [thick, facefill] (uuu) -- (uul) -- (ull) -- (ulu) -- cycle;

    \draw [thick, arrows=-latex] (0, 0.5, 0) -- (0, 1, 0) node [below right] {$\vn$};
    %\node [below right] at (e1) {$\ve_1$};
    \draw [thick, arrows=-latex] (0, 0, 0.5) -- (0, 0, 1); % node [below right] {$\ve_2$};

    % transformation
    \draw [very thick, arrows=-latex] (1.5, 0) -- (2.5, 0) node [above, midway] {$\T\mB \bigcirc \mB$};
\end{scope}

\begin{scope}[shift={(4, -2)}]
    \coordinate (ll) at (-1.0, 0, -0.5);
    \coordinate (ul) at ( 1.0, 0, -0.5);
    \coordinate (lu) at (-1.0, 0,  0.5);
    \coordinate (uu) at ( 1.0, 0,  0.5);

    \draw [thick, planefill] (ll) -- (ul) -- (uu) -- (lu) -- cycle;

    % \draw [thick, arrows=-latex] (0, 0, 0) -- (0, 1, 0) node [right] {$\vn$};
    \draw [thick, arrows=-latex] (0, 0, 0) -- (1, 0, 0); % node [below right] {$\ve_2$};
    \draw [thick, arrows=-latex] (0, 0, 0) -- (0, 0, 1); % node [below right] {$\ve_2$};
\end{scope}

\end{tikzpicture}
    \caption{Extracting the tangential component of the deformation
    gradient. A local neighborhood is transformed by the deformation gradient
    $\mF$. In the process, the local coordinate system is rotated by
    $\mU$, which must be reverted before isolating the tangential component
    of $\mF$ by multiplying with the tangential basis $\mB$ from both sides.}
    \label{fig:tangential_deformation}
\end{figure}
%
% section mathematical_basis (end)
%
\section{Discretization} % (fold)
\label{sec:fst_discretization}
%
Our objective is an algorithm for tracking the flame surface in-situ during a
simulation run.
%
It must produce the paths of single points on the surface over time, as well as
the tangential deformation of the surface for arbitrary time intervals.
%
Additionally, the algorithm needs to be highly parallelizeable in order to work
well in the environment of a massively parallel simulation.
%

%
A naive approach might be performing an isosurface extraction in each time step
of the simulation.
%
However, subsequent isosurfaces do not contain information about surface point
correspondence between time steps.
%
This correspondence is necessary if we want to provide point paths and compute
the surface deformation.
%

%
Alternatively, one could advect the vertices of an explicit surface mesh,
adaptively remeshing it as the surface deforms over time.
%
Due to the global nature of such remeshing operations, this is infeasible in a
massively parallel environment, where irregular communication between
neighboring processors is known to be a major source of bottlenecks.
%

%
We therefore choose to represent the surface as a cloud of micro-patches
consisting of a central point and four \textit{ghost particles} (see
\cref{fig:point_group}, left).
%
The ghost particles sample the shape and deformation of the micro-patch in the
local neighborhood around the central point.
%
Each group of points is tracked completely independently.
%
We monitor the deviation of the ghost particles from the tangent plane at the
central point and the deviation of their relative movement from linear behavior.
%
Once this deviation exceeds a user-defined threshold, we split the patch into
three independent new patches to ensure sufficient sampling.
%
To avoid oversampling, we merge patches when they become too small, \eg, because
the surface flattens over time or tangential movement bunches up
many patches in a small area.
%
The tangential surface deformation is then reconstructed from the relative
behavior of the points in a group between split/merge events.
%

%
\subsection{Micro-Patches for Surface Tracking} % (fold)
\label{sub:surface_patch_representation}
%
\begin{figure}[tb]
    \centering
    \setlength{\figurewidth}{\linewidth}
    %
%
%
%
\begin{tikzpicture}[
    scale=\figurewidth/1cm*0.08,
    thick, line join=round,
    font=\small]
\tikzset{xzplane/.style={canvas is xz plane at y=#1,very thin}}
\tikzstyle{ghostparticle} = [fill=white]
\tikzstyle{central} = [fill=black]
\tikzstyle{patch} = [dashed]
\tikzstyle{lin} = [gray]
\tikzstyle{real} = [thick]
\tikzstyle{surface} = [real]
\tikzstyle{tanplane} = [lin]
\tikzstyle{normalerror} = [mycolor4, very thick]
\tikzstyle{tanerror} = [mycolor1, very thick]

\begin{scope}
    % normal
    \draw [central, arrows=-latex]
        (0, 0, 0) -- (0, 1.5, 0) node [right] {$\vn$};
    \draw [central] (0, 0, 0) circle [radius=0.12] node [below right] {$\vx$};

    \begin{scope}[canvas is xz plane at y=0]
        \draw [surface] (1.8, 2) -- (-1.8, 2)
                -- (-1.8, -2) -- (1.8, -2) -- cycle;
        \draw [patch] (0, 0) circle [radius=1.5];
        % ghost particles
        \draw (0, 0) -- (1.5, 0);
        \draw (0, 0) -- (-1.5, 0);
        \draw (0, 0) -- (0, -1.5);
        \draw (0, 0) -- (0, 1.5);
    \end{scope}

    \draw [ghostparticle, real] (1.5, 0, 0) circle [radius=0.12]
            node [below right] {$\vx_1$};
    \draw [ghostparticle, real] (-1.5, 0, 0) circle [radius=0.12]
            node [above left] {$\vx_2$};
    \draw [ghostparticle, real] (0, 0, -1.5) circle [radius=0.12]
            node [above, yshift=0.1cm] {$\vx_3$};
    \draw [ghostparticle, real] (0, 0, 1.5) circle [radius=0.12]
            node [below, yshift=-0.1cm] {$\vx_4$};

\end{scope}

\draw [very thick, arrows=-latex]
        (1.5, 1) to[out=35, in=150]
            node [above, midway] {time}
        (4, 1);

\begin{scope}[shift={(6, 0)}]

    \coordinate (x1n) at (1.7, 0, 0.3);
    \coordinate (x2n) at (-1.7, 0, -0.3);
    \coordinate (x3n) at (0.6, 0, -1.5);
    \coordinate (x4n) at (-0.6, 0, 1.5);

    \coordinate (x1en) at (0, -0.4, 0);
    \coordinate (x2en) at (0, -0.4, 0);
    \coordinate (x3en) at (0, -0.4, 0);
    \coordinate (x4en) at (0, -0.4, 0);

    \coordinate (x12te) at (-0.3, -0.2, 0);
    \coordinate (x34te) at (0.5, 0, 0);

    % surface
    \draw [surface] (-2.5, -0.75, 2.5)
            to[out=25, in=180] (0, -0.2, 2.5)
            to[out=0, in=155] (2.5, -0.75, 2.5)
            to[out=70, in=225] (2.5, -0.2, 0)
            to[out=45, in=185] (2.5, -0.75, -2.5)
            to[out=155, in=0] (0, -0.2, -2.5)
            to[out=180, in=45] (-2.5, -0.2, 0)
            to[out=225, in=70] (-2.5, -0.75, 2.5);


    % errors
    \draw [normalerror] ($(x1n) + (x12te)$) -- ($(x1n) + (x12te) + (x1en)$);
    \draw [normalerror] ($(x2n) + (x12te)$) -- ($(x2n) + (x12te) + (x2en)$);
    \draw [normalerror] ($(x3n) + (x34te)$) -- ($(x3n) + (x34te) + (x3en)$);
    \draw [normalerror] ($(x4n) + (x34te)$) -- ($(x4n) + (x34te) + (x4en)$);

    \draw [real] (0, 0, 0) to[out=-10, in=135] ($(x1n) + (x12te) + (x1en)$);
    \draw [real] (0, 0, 0) to[out=180, in=25] ($(x2n) + (x12te) + (x2en)$);
    \draw [real] (0, 0, 0) to[out=15, in=175] ($(x3n) + (x34te) + (x3en)$);
    \draw [real] (0, 0, 0) to[out=215, in=80] ($(x4n) + (x34te) + (x4en)$);

    \draw [ghostparticle, real] ($(x1n) + (x1en) + (x12te)$)
            circle [radius=0.12]
            node [below, yshift=-0.1cm] {$\vx_1$};
    \draw [ghostparticle, real] ($(x2n) + (x2en) + (x12te)$)
            circle [radius=0.12]
            node [left] {$\vx_2$};
    \draw [ghostparticle, real] ($(x3n) + (x3en) + (x34te)$)
            circle [radius=0.12]
            node [right] {$\vx_3$};
    \draw [ghostparticle, real] ($(x4n) + (x4en) + (x34te)$)
            circle [radius=0.12]
            node [below, yshift=-0.1cm] {$\vx_4$};


    \begin{scope}[canvas is xz plane at y=0]
        \draw [tanplane] (2.5, 2.5) -- (-2.5, 2.5)
                -- (-2.5, -2.5) -- (2.5, -2.5) -- cycle;
        \draw [patch] (0, 0)
                circle [x radius=1.791, y radius=1.51, rotate=-150.5];
        \foreach \num in {1, 2, 3, 4}
            \draw [lin] (0, 0) -- (x\num n);
        % \draw [linparticles] (0, 0) -- (x2n);
        % \draw [linparticles] (0, 0) -- (x3n);
        % \draw [linparticles] (0, 0) -- (x4n);
    \end{scope}

    \draw [tanerror] (x1n) -- ($(x1n) + (x12te)$);
    \draw [tanerror] (x2n) -- ($(x2n) + (x12te)$);
    \draw [tanerror] (x3n) -- ($(x3n) + (x34te)$);
    \draw [tanerror] (x4n) -- ($(x4n) + (x34te)$);

    % ghost particles
    \foreach \num in {1, 2, 3, 4}
        \draw [lin, ghostparticle] (x\num n) circle [radius=0.12];
    % normal
    \draw [central, arrows=-latex]
        (0, 0, 0) -- (0, 1.5, 0) node [right] {$\vn$};
    \draw [central] (0, 0, 0) circle [radius=0.12]
    		node [below right, yshift=-0.05cm] {$\vx$};

    % surface
    % \draw [surface] (-1.5, -0.8) to[out=15, in=150] (1.2, -1.2)
    %                            to[out=95, in=-130] (1.5, 0.1)
    %                            to[out=150, in=25] (-1, 0.4)
    %                            to[out=205, in=90] (-1.5, -0.8);
\end{scope}

\end{tikzpicture}
    \tikzset{external/export=false}
    \caption{
            Ghost particles
            (\protect\tikz
                \protect\draw[thick, fill=white] circle [radius=0.08cm];)
            are initialized in an orthogonal configuration around the central
            point
            (\protect\tikz\protect\draw[fill=black] circle [radius=0.08cm];)
            in the tangent plane. While integrating the points over time, they
            deviate from a planar, linear behavior. The difference between the
            real positions and the nearest linear configuration
            (\protect\tikz\protect\draw[thick, gray] circle [radius=0.08cm];)
            in the tangent plane is measured by the normal error $e_\perp$
            (\protect\tikz[baseline=-0.5ex]
                \protect\draw[very thick, mycolor4] (0, 0) -- (2ex, 0);)
            and the tangential error $e_\shortparallel$
            (\protect\tikz[baseline=-0.5ex]
                \protect\draw[very thick, mycolor1] (0, 0) -- (2ex, 0);).
            }
    \label{fig:point_group}
    \tikzset{external/export=true}
\end{figure}
%
A micro-patch is represented by a group of five points: A center and four
\textit{ghost particles}.
%
On initialization, the points are arranged in an orthogonal cross shape in the
tangent plane defined by the surface normal $\vn$ at the central point $\vx$
(see \cref{fig:point_group}).
%
In this way, they sample the shape and deformation in all directions as they
follow the surface over time.
%

%
We describe the configuration (\eg, the state excluding the absolute position)
of a micro-patch at time $t$ as a matrix $\mC(t) \in \RRSet^{3\times5}$
consisting of the surface normal $\vn(t)$ at the central point as well as the
relative positions of the ghost particles $\vx_i(t)$, $i \in \{1, ..., 4\}$.
%
\begin{equation}
    \mC(t) = \begin{pmatrix}
                \vn(t) &
                \vx_1(t) - \vx(t) &
                \cdots &
                \vx_4(t) - \vx(t)
             \end{pmatrix}\, \text{.}
\end{equation}
%
In the following, we omit the dependence of $\mC$, $\vn$, and $\vx_i$ on the
time $t$ and we assume that $\vx = \vNull$ wherever the meaning is clear from
context.
%

%
For the initial configuration at some time $t_0$, $\mC$ can be exactly
represented by a unit base configuration $\mC_0$ being transformed by a linear
transformation $\mT$, \ie,
%
\begin{equation*}
    \mC(t_0) = \mT \, \mC_0 \, \text{,} \quad  \text{with} \quad
    \mC_0 = \begin{pmatrix}
        \ve_3 & \ve_1 & -\ve_1 & \ve_2 & -\ve_2
    \end{pmatrix} \, \text{,}
\end{equation*}
%
where $\ve_i$ are the unit vectors in the $i$-th coordinate direction.
%
As the points track the surface over time, their relative position will change.
%
Their relation to $\mC_0$ might no longer be linear.
%
Now, $\mT$ is the linear transformation that best approximates the mapping
between $\mC_0$ and $\mC$, \ie, the solution to the least squares problem
%
\begin{equation*}
    e^2 = \Frob{\T\mC_0 \, \T\mT - \T\mC}^2 \rightarrow \text{min} \, \text{.}
\end{equation*}
%
Here, $\Frob{\cdot}$ denotes the Frobenius norm.
%

%
The residual error $e$ measures the deviation of the real mapping between
$\mC_0$ and $\mC$ from linear behavior.
%
However, its scale is dependent on the size of the micro-patch, \ie, the
magnitude of the last four columns of $\mC$, and it weights those columns
differently from the normal in the first column.
%
To get a more meaningful error, we normalize these columns by their average
magnitude and obtain a modified transformation $\mT_n$ and error $e_n$ by
%
\begin{equation}
\begin{gathered}
    \label{eqn:normalized_system}
    e_n^2 = \Frob{\T\mC_0 \, \T\mT_n
            - \T\mC_n}^2 \rightarrow \text{min} \, \text{,}\\
    \text{with}\quad
    \mC_n = \begin{pmatrix}
        \vn & \vx_1/c & \cdots & \vx_4/c
    \end{pmatrix}
    \quad  \text{and} \quad
    c = \frac{1}{4} \sum_i^4{\Norm{\vx_i}} \,\text{.}
\end{gathered}
\end{equation}
%

%
We separate this error into normal and tangential components
%
\begin{align}
    e_\perp =& \Frob{\vn \, \T\vn (\mT_n \, \mC_0 - \mC_n)} \\
    e_\shortparallel = & \Frob{(\mI - \vn \, \T\vn )
                                  (\mT_n \, \mC_0 - \mC_n)} \, \text{.}
\end{align}
%
$e_\perp$ is a measure for the deviation of the ghost particles $\vx_i$ from the
tangent plane.
%
As such, it measures the local surface curvature in relation to the patch size.
%
$e_\shortparallel$ measures how much the tangential deformation of the ghost
particles deviates from linear behavior.
%
As the micro-patch changes over time, we use these errors to decide when to
split or merge patches to ensure sufficient sampling of the surface geometry
and tangential deformation.
%
% subsection surface_patch_representation (end)
%
\subsection{Splitting and Merging Surface Patches} % (fold)
\label{sub:splitting_surface_patches}
%
We split a micro-patch into three new patches when one of the following
occurs:
%
\begin{itemize}
    \item The error $e_\perp$ exceeds a threshold $r_\perp$
    \item The error $e_\shortparallel$ exceeds a threshold $r_\shortparallel$
    \item The major axis exceeds a threshold $r_\textnormal{size}$
\end{itemize}
%
The first two are to ensure a sufficient sampling, the third is to ensure a
maximum patch size even on flat parts of the surface.
%
In a parallel distributed simulation, the size of a patch is limited by the
size of the block of the simulation domain it is contained in.
%

%
When splitting a micro-patch, we need to decide the positions of the points
forming the new patches.
%
In this context, it helps to think of the micro-patch as an ellipse being
defined by the positions of the ghost particles around the central point.
%
On initialization, the ghost particles are always placed along the principal
axes of the ellipse.
%
When the patch is transformed over time, the principal axes will generally not
stay aligned with the ghost particles.
%
The major and minor axes of the current configuration can be reconstructed from
the tangential component $\hat\mT$ of $\mT$, which is obtained similar to
\eqref{eqn:tangent_deformation_tensor}:
%
\begin{equation}
    \hat\mT = \T{\left( \ve_1 \quad \ve_2 \right)} \,
                \T\mU \, \mT \,
                \left( \ve_1 \quad \ve_2 \right) \, \text{,}
\end{equation}
%
where $\T\mU$ is the rotation obtained from the polar decomposition $\mT =
\mU\mP$.
%
The directions and extents of the principal axes of the micro-patch are
encoded in the singular value decomposition
%
\begin{equation*}
    \hat\mT = \hat\mV \, \hat\mSigma \, \T{\hat\mW} \, \text{.}
\end{equation*}
%
The singular values $\sigma_{1,2}$ on the diagonal of $\hat\mSigma$ are the
extents of the ellipse, while the rows of $\hat\mV$ are the directions of the
major and minor axis in tangent space.
%
The directions in \ac{3D} space are the columns of $\mU\left( \ve_1 \quad \ve_2
\right)\T{\hat\mV}$.
%

%
When splitting the micro-patch, we want the new patches to cover the whole
area captured by the old patch without overlapping, to prevent over- or
undersampling as patches are split multiple times.
%
We also want to leave the central point in place, so we can track its path for
as long as possible.
%
Consequently, we split the patch along its major axis into three identical new
ones (see \cref{fig:splitting_patch}).
%
The points of the new patches are again arranged in an orthogonal cross shape
with the axes parallel to the axes of the old patch.
%
The length of the new first axis is exactly $\sfrac{1}{3}$ of the length of the
old major axis.
%
The new second axis is as long as the old minor axis.
%
The new points are then projected onto the surface along the normal at the
central point.
%
From this point on, they are treated idependently again.
%
\begin{figure}[tb]
    \centering
    \setlength{\figurewidth}{0.9\linewidth}
    %
%
\begin{tikzpicture}[
  scale=\figurewidth/1cm*0.09,
  font=\small]
\tikzstyle{ghostparticle} = [fill=white]
\tikzstyle{patch} = [dashed]
\tikzstyle{central} = [fill]
\tikzstyle{vold} = [thick, gray!50]
\tikzstyle{old} = [thick, gray]
\tikzstyle{new} = [thick, black]
\tikzstyle{ax1} = [very thick, mycolor4]
\tikzstyle{ax2} = [very thick, mycolor1]

\begin{scope}[rotate=0]
% old patch
\draw[old, patch, fill=gray!20] (0, 0) circle [x radius =3, y radius=1.5];

% old ghost particles
\coordinate (oldgp1) at (2.4, 0.89);
\coordinate (oldgp2) at (-2.4, -0.89);
\coordinate (oldgp3) at (2.4, -0.89);
\coordinate (oldgp4) at (-2.4, 0.89);
\draw[old] (oldgp1) -- (oldgp2);
\draw[old] (oldgp3) -- (oldgp4);
\draw[old, ghostparticle] (oldgp1) circle [radius=0.1];
\draw[old, ghostparticle] (oldgp2) circle [radius=0.1];
\draw[old, ghostparticle] (oldgp3) circle [radius=0.1];
\draw[old, ghostparticle] (oldgp4) circle [radius=0.1];

% new patch

\draw[new, patch] (-2, 0) circle [x radius=1, y radius=1.5];
\draw[new, patch] (0, 0) circle [x radius=1, y radius=1.5];
\draw[new, patch] (2, 0) circle [x radius=1, y radius=1.5];

% main axis
\draw[new] (-3, 0) -- (3, 0);

% ghost particle lines
\draw[new] (0, -1.5) -- (0, 1.5);
\draw[new] (-2, -1.5) -- (-2, 1.5);
\draw[new] (2, -1.5) -- (2, 1.5);

% ellipse axes
\draw[ax1] (0, 0) -- (3, 0);
\draw[ax2] (0, 0) -- (0, 1.5);

% new ghost particles
\draw[new, ghostparticle] (-2, 1.5) circle [radius=0.1];
\draw[new, ghostparticle] (0, 1.5) circle [radius=0.1];
\draw[new, ghostparticle] (2, 1.5) circle [radius=0.1];
\draw[new, ghostparticle] (-2, -1.5) circle [radius=0.1];
\draw[new, ghostparticle] (0, -1.5) circle [radius=0.1];
\draw[new, ghostparticle] (2, -1.5) circle [radius=0.1];
\draw[new, ghostparticle] (-3, 0) circle [radius=0.1];
\draw[new, ghostparticle] (-1, 0) circle [radius=0.1];
\draw[new, ghostparticle] (1, 0) circle [radius=0.1];
\draw[new, ghostparticle] (3, 0) circle [radius=0.1];

% new central points
\draw[new, central] (-2, 0) circle [radius=0.1];
\draw[new, central] (0, 0) circle [radius=0.1];
\draw[new, central] (2, 0) circle [radius=0.1];

\node at (0, -2.2) {split};

\end{scope}

\begin{scope}[shift={(6, 0)}]
  \coordinate (v1) at (0.4, 0);
  \coordinate (v2) at (0.4, 1.5);
  \coordinate (center2) at (0, 0);
  \coordinate (ax2dir) at (-20:0.37);
  \coordinate (ax1dir) at (70:1.66);


  \draw[old, patch, fill=gray!20] (center2) circle [x radius=0.37, y radius =1.66, rotate=-20];
  \draw[vold, patch, fill=gray!10] ($(center2) - 2*(v1)$) circle [x radius=0.37, y radius =1.66, rotate=-20];
  \draw[vold, patch, fill=gray!10] ($(center2) + 2*(v1)$) circle [x radius=0.37, y radius =1.66, rotate=-20];

  \draw[old] ($(center2) - (v1)$) -- ($(center2) + (v1)$);
  \draw[old] ($(center2) - (v2)$) -- ($(center2) + (v2)$);

  \draw[vold] ($(center2) - 3*(v1)$) -- ($(center2) - (v1)$);
  \draw[vold] ($(center2) - 2*(v1) - (v2)$) -- ($(center2) - 2*(v1) + (v2)$);

  \draw[vold] ($(center2) + (v1)$) -- ($(center2) + 3*(v1)$);
  \draw[vold] ($(center2) + 2*(v1) - (v2)$) -- ($(center2) + 2*(v1) + (v2)$);

  \draw[new] ($(center2) + 3*(ax2dir)$) -- ($(center2) - 3*(ax2dir)$);
  \draw[ax2] (center2) -- (ax2dir);
  \draw[new] ($(center2) + (ax1dir)$) -- ($(center2) - (ax1dir)$);
  \draw[ax1] (center2) -- (ax1dir);

  \draw[new, patch] (center2) circle [x radius =1.11, y radius = 1.66, rotate=-20];

  \draw[old, ghostparticle] ($(center2) + (v2)$) circle [radius=0.1];
  \draw[old, ghostparticle] ($(center2) - (v2)$) circle [radius=0.1];
  \draw[old, ghostparticle] ($(center2) + (v1)$) circle [radius=0.1];
  \draw[old, ghostparticle] ($(center2) - (v1)$) circle [radius=0.1];

  \draw[vold, ghostparticle] ($(center2) + 2*(v1) + (v2)$) circle [radius=0.1];
  \draw[vold, ghostparticle] ($(center2) + 2*(v1) - (v2)$) circle [radius=0.1];
  \draw[vold, ghostparticle] ($(center2) - 2*(v1) + (v2)$) circle [radius=0.1];
  \draw[vold, ghostparticle] ($(center2) - 2*(v1) - (v2)$) circle [radius=0.1];
  \draw[vold, ghostparticle] ($(center2) + 3*(v1)$) circle [radius=0.1];
  \draw[vold, ghostparticle] ($(center2) - 3*(v1)$) circle [radius=0.1];

  \draw[vold, central] ($(center2) + 2*(v1)$) circle [radius=0.1];
  \draw[vold, central] ($(center2) - 2*(v1)$) circle [radius=0.1];

  \draw[new, central] (center2) circle [radius=0.1];
  \draw[new, ghostparticle] ($(center2) + 3*(ax2dir)$) circle [radius=0.1];
  \draw[new, ghostparticle] ($(center2) - 3*(ax2dir)$) circle [radius=0.1];
  \draw[new, ghostparticle] ($(center2) + (ax1dir)$) circle [radius=0.1];
  \draw[new, ghostparticle] ($(center2) - (ax1dir)$) circle [radius=0.1];

  \node at (0, -2.2) {merge};
\end{scope}

\end{tikzpicture}

    \tikzset{external/export=false}
    \caption{
        Splitting and merging micro-patches.
        Left: An old micro-patch
        (\protect\tikz[baseline=-0.5ex]
        \protect\draw [thick, dashed, gray, fill=gray!20]
        circle [x radius=2ex, y radius=1ex];)
        is split up along its major axis
        (\protect\tikz[baseline=-0.5ex]
        \protect\draw [very thick, mycolor4] (0, 0) -- (2ex, 0);)
        into three new identical patches
        (\protect\tikz[baseline=-0.5ex]
        \protect\draw [thick, dashed]
        circle [x radius=2ex, y radius=1ex];).
        Right: An old micro-patch
        (\protect\tikz[baseline=-0.5ex]
        \protect\draw [thick, dashed, gray, fill=gray!20]
        circle [x radius=2ex, y radius=1ex];)
        is extended to three times its size along its minor axis
        (\protect\tikz[baseline=-0.5ex]
        \protect\draw [very thick, mycolor1] (0, 0) -- (2ex, 0);).
        Neighboring patches
        (\protect\tikz[baseline=-0.5ex]
        \protect\draw [thick, dashed, gray!50, fill=gray!10]
        circle [x radius=2ex, y radius=1ex];) that were split off from the
        central one at some earlier time are deleted instead.
    }
    \label{fig:splitting_patch}
    \tikzset{external/export=true}
\end{figure}
%

%
The new central patch continues tracking the behavior of the neighborhood
around the same point as the old patch.
%
We therefore consider it to be the same entity, but with its ghost particles
reset to a more numerically stable configuration.
%
In contrast, we consider the new outer patches to be entirely new entities.
%
Consequently, if we say that a patch has been split or merged multiple times, we
mean that it has been the central patch in a number of these operations.
%

%
We consider a micro-patch for merging when
%
\begin{itemize}
    \item both errors $e_\perp$ and $e_\shortparallel$ decrease below
          lower thresholds $l_\perp$ and $l_\shortparallel$, respectively,
    \item or the minor axis decreases below a threshold $l_\textnormal{size}$
\end{itemize}
%

%
Because the micro-patches are completely independent, we cannot explicitly
merge multiple patches.
%
We therefore give every patch a counter, which is set to \num{0} when it is first
initialized and incremented each time it is split.
%
When the conditions for a merge are met, the behavior of a patch depends on its
counter.
%
If the counter is \num{0}, the patch is simply deleted.
%
Otherwise, the counter is decremented and the patch is extended by a factor of
\num{3} along its minor axis.
%
Assuming the behavior of neighboring patches is similar, this effectively means
that the new extended patch now covers the area of the neighboring patches,
which were deleted.
%
This assumption does not generally hold for all parts of the surface.
%
We therefore typically choose $l_\perp$, $l_\shortparallel$, and
$l_\textnormal{size}$ to be very small.
%
In this case, points will only be deleted in areas of extreme tangential
compression, where removing very small patches will only result in small errors,
and in areas of very simple deformation, where the assumption is unlikely to
be violated.
%

%
The central point of the extended patch stays in place, while the ghost
particles are again placed along the principal axes of the old patch, but with
the minor axis extended.
%
In this way, we ensure that the surface is not over-sampled where it is not
necessary and that no infeasibly small patches are tracked, \eg, in areas of
attracting tangential behavior.
%
Patches that were originally initialized at the start of the simulation are
never deleted, in order to prevent holes from forming.
%
Since the surface at the start of a simulation is generally very simple and can
be represented by a relatively small number of patches, this is not an issue
in practice.
%
% subsection splitting_surface_patches (end)

\subsection{Reconstructing Tangential Surface Deformation} % (fold)
\label{sub:reconstructing_tangential_surface_deformation}
%
Let $\hat\mF_{\,t_s}^{\,t_e}$ be the tangential deformation gradient for a time
interval $[t_s,\,t_e]$ at the central point $\vx(t_e)$.
%
To reconstruct it we need the corresponding deformation gradient
$\mF_{\,t_s}^{\,t_e}$, which describes the relative change in position of points in
a small neighborhood between times $t_s$ and $t_e$.
%
The behavior of the ghost particles of a micro-patch over time contains exactly
this information.
%
However, because all ghost particles are located on the surface, they do not
contain information about the behavior of $\mF$ in normal direction.
%
Fortunately, we only need the change in orientation of the surface normal to
reconstruct $\hat\mF$, as any other information is discarded when projecting
into the tangent plane.
%
It is therefore sufficient to determine a proxy transformation
$\mE_{\,t_s}^{\,t_e}$, which maps the micro-patch configuration $\mC(t_s)$ to
$\mC(t_e)$, and reconstruct $\hat\mF$ via
%
\begin{equation}
    \hat\mF_{\,t_s}^{\,t_e} = \T\mB \, \T\mU \, \mE_{\,t_s}^{\,t_e} \, \mB \, \text{,}
\end{equation}
%
with $\mB$ the basis vectors of the tangent plane at $t_s$ and $\mU$ the
rotation matrix from the polar decomposition of $\mE$.
%

%
Let us first assume the micro-patch was not split or merged in the time
interval, \ie, its ghost particles were not reset.
%
Because the real mapping between $\mC(t_s)$ and $\mC(t_e)$ will generally not be
linear, $\mE_{\,t_s}^{\,t_e}$ is the solution to a least squares problem very
similar to \eqref{eqn:normalized_system}.
%
To make the solution independent of the scale of the micro-patch, we scale the
last four columns of both $\mC(t_s)$ and $\mC(t_e)$ by the average norm of these
columns in $\mC(t_e)$.
%
The system for reconstructing $\mE_{\,t_s}^{\,t_e}$ is then
%
\begin{equation}
    \label{eqn:deformation_reconstruction}
    \Frob{\T\mC_n(t_s) \, \T{\mE_{\,t_s}^{\,t_e}} - \T\mC_n(t_e)}^2
        \rightarrow \text{min} \, \text{,}
\end{equation}
%
with $\mC_n(t_s)$ and $\mC_n(t_e)$ being the configurations with their last
columns scaled.
%

%
If the ghost particles were reset one or more times during the time interval,
$\mE_{\,t_s}^{\,t_e}$ is the concatenation of multiple transformations:
%
\begin{equation}
    \mE_{\,t_s}^{\,t_e} = \mE_{\,t_n}^{\,t_e} \cdot \mE_{\,t_{n-1}}^{\,t_n}
                      \cdot \; \cdots \; \cdot
                      \mE_{\,t_1}^{\,t_2} \cdot \mE_{\,t_s}^{\,t_1} \, \text{,}
\end{equation}
%
\begin{figure}[tb]
\centering
\setlength\figurewidth\linewidth
%
%
\begin{tikzpicture}[scale=\figurewidth/1cm*0.12,
					line join=round]
\tikzstyle{ghostparticle} = [fill=white]
\tikzstyle{patch} = [dashed]
\tikzstyle{central} = [fill]
\tikzstyle{vold} = [thick, gray!50]
\tikzstyle{old} = [thick, gray]
\tikzstyle{new} = [thick, black]
\tikzstyle{trarrow} = [very thick, black, arrows=-latex]

\definecolor{intv1color}{named}{mycolor1}
\definecolor{intv2color}{named}{mycolor2}

\coordinate (arrowoffset) at (0, 0.8);
\coordinate (arrowshorten) at (0.2, 0);

\begin{scope}
	\coordinate (center1) at (0, 0);
	\coordinate (x1) at (0.5, 0);
	\coordinate (x2) at (-0.4, 0);
	\coordinate (x3) at (0.08, 0.6);
	\coordinate (x4) at (-0.07, -0.5);

	\draw [new, patch] (center1) circle [x radius=0.43, y radius=0.56, rotate=-16];

	\draw (center1) -- (x1);
	\draw [ghostparticle] (x1) circle [radius=0.08cm];
	\draw (center1) -- (x2);
	\draw [ghostparticle] (x2) circle [radius=0.08cm];
	\draw (center1) -- (x3);
	\draw [ghostparticle] (x3) circle [radius=0.08cm];
	\draw (center1) -- (x4);
	\draw [ghostparticle] (x4) circle [radius=0.08cm];
	\draw [central] (center1) circle [radius=0.08cm];
\end{scope}


\begin{scope}[shift={(2, 0)}]
	\coordinate (center2) at (0, 0);
	\coordinate (ox1) at (0.8, -0.1);
	\coordinate (ox2) at (-0.7, 0.1);
	\coordinate (ox3) at (0.5, 0.6);
	\coordinate (ox4) at (-0.5, -0.6);

	\coordinate (x1) at (0.306, 0.121);
	\coordinate (x2) at (-0.306, -0.121);
	\coordinate (x3) at (-0.184, 0.465);
	\coordinate (x4) at (0.184, -0.465);

	\draw [old, patch, fill=gray!20] (center2) circle [x radius=1, y radius=0.5, rotate=21];

	\draw [old] (center2) -- (ox1);
	\draw [old, ghostparticle] (ox1) circle [radius=0.08cm];
	\draw [old] (center2) -- (ox2);
	\draw [old, ghostparticle] (ox2) circle [radius=0.08cm];
	\draw [old] (center2) -- (ox3);
	\draw [old, ghostparticle] (ox3) circle [radius=0.08cm];
	\draw [old] (center2) -- (ox4);
	\draw [old, ghostparticle] (ox4) circle [radius=0.08cm];

	\draw [new, patch] (center2) circle [x radius=0.333, y radius=0.5, rotate=21];

	\draw [new] (center2) -- (x1);
	\draw [new, ghostparticle] (x1) circle [radius=0.08cm];
	\draw [new] (center2) -- (x2);
	\draw [new, ghostparticle] (x2) circle [radius=0.08cm];
	\draw [new] (center2) -- (x3);
	\draw [new, ghostparticle] (x3) circle [radius=0.08cm];
	\draw [new] (center2) -- (x4);
	\draw [new, ghostparticle] (x4) circle [radius=0.08cm];
	\draw [central] (center2) circle [radius=0.08cm];
\end{scope}

\begin{scope}[shift={(3.6, 0)}]
	\draw [fill=black] (0, 0) circle [radius=0.04cm];
	\draw [fill=black] (0.3, 0) circle [radius=0.04cm];
	\draw [fill=black] (-0.3, 0) circle [radius=0.04cm];
\end{scope}

\begin{scope}[shift={(5, 0)}, rotate=-30]
	\coordinate (center3) at (0, 0);
	\coordinate (ox1) at (1, 0);
	\coordinate (ox2) at (-0.6, 0);
	\coordinate (ox3) at (0.15, 0.6);
	\coordinate (ox4) at (0.15, -0.6);

	\coordinate (x1) at (0.27, 0);
	\coordinate (x2) at (-0.27, 0);
	\coordinate (x3) at (0, 0.6);
	\coordinate (x4) at (0, -0.6);

	\draw [old, patch, fill=gray!20] (center3) circle [x radius=0.8, y radius=0.6, rotate=0];

	\draw [old] (center3) -- (ox1);
	\draw [old, ghostparticle] (ox1) circle [radius=0.08cm];
	\draw [old] (center3) -- (ox2);
	\draw [old, ghostparticle] (ox2) circle [radius=0.08cm];
	\draw [old] (center3) -- (ox3);
	\draw [old, ghostparticle] (ox3) circle [radius=0.08cm];
	\draw [old] (center3) -- (ox4);
	\draw [old, ghostparticle] (ox4) circle [radius=0.08cm];

	\draw [new, patch] (center3) circle [x radius=0.27, y radius=0.6, rotate=0];

	\draw [new] (center3) -- (x1);
	\draw [new, ghostparticle] (x1) circle [radius=0.08cm];
	\draw [new] (center3) -- (x2);
	\draw [new, ghostparticle] (x2) circle [radius=0.08cm];
	\draw [new] (center3) -- (x3);
	\draw [new, ghostparticle] (x3) circle [radius=0.08cm];
	\draw [new] (center3) -- (x4);
	\draw [new, ghostparticle] (x4) circle [radius=0.08cm];
	\draw [central] (center3) circle [radius=0.08cm];
\end{scope}

\begin{scope}[shift={(7, 0)}, rotate=-40]
	\coordinate (center4) at (0, 0);
	\coordinate (x1) at (0.35, 0);
	\coordinate (x2) at (-0.3, 0);
	\coordinate (x3) at (0.05, 0.6);
	\coordinate (x4) at (0.05, -0.6);

	\draw [new, patch] (center4) circle [x radius=0.325, y radius=0.6, rotate=0];

	\draw (center4) -- (x1);
	\draw [ghostparticle] (x1) circle [radius=0.08cm];
	\draw (center4) -- (x2);
	\draw [ghostparticle] (x2) circle [radius=0.08cm];
	\draw (center4) -- (x3);
	\draw [ghostparticle] (x3) circle [radius=0.08cm];
	\draw (center4) -- (x4);
	\draw [ghostparticle] (x4) circle [radius=0.08cm];
	\draw [central] (center4) circle [radius=0.08cm];
\end{scope}


\draw [trarrow, intv1color]
		($(center1) + (arrowoffset) + (arrowshorten)$) to[out=30, in=150]
			node [above, midway] {$\color{intv1color}\mE_{\,t_s}^{\,t_1}$}
		($(center2) + (arrowoffset) - (arrowshorten)$);

\draw [trarrow, intv2color]
		($(center3) + (arrowoffset) + (arrowshorten)$) to[out=30, in=150]
			node [above, midway] {\color{intv2color}$\mE_{\,t_n}^{\,t_e}$}
		($(center4) + (arrowoffset) - (arrowshorten)$);

\draw [trarrow]
		($(center1) - (arrowoffset) + (arrowshorten)$) to[out=-10, in=-170]
			node [above, midway] {
			$\mE_{\,t_s}^{\,t_e} =
				\textcolor{intv1color}{\mE_{\,t_n}^{\,t_e}} \cdot \;
				\cdots \;
				\cdot \textcolor{intv2color}{\mE_{\,t_s}^{\,t_1}}$
			}
		($(center4) - (arrowoffset) - (arrowshorten)$);

\end{tikzpicture}
\caption{Reconstructing the deformation for an arbitrary time interval by
         concatenating the deformations between split and merge events of the
         patch.}
\label{fig:transformation_concatenation}
\end{figure}
%
where $t_i$ are the discrete times between $t_s$ and $t_e$ when the ghost
particles were reset in the course of a split or merge operation.
%
Each sub-interval transformation is reconstructed from the new configuration of
the micro-patch just after a split/merge and the old configuration just before
the next one (see \cref{fig:transformation_concatenation}).
%

%
If we want to compute the tangential deformation a micro-patch at time $t_e$
has experienced since the start time $t_s$, it is possible that this patch has
not existed for the complete time interval, \ie, it was created at some time
$t_k > t_s$ during a split operation.
%
In this case, we estimate the transformation between $t_s$ and $t_k$ from its
parent patch by weighting the contributions of each ghost particle differently
depending on their distance from the center of the new patch.
%
The details of this algorithm are presented in
\cref{sec:transform_interpolation}. \Todo{move appendix into regular
chapter?}
%
With this, we can reconstruct the tangential deformation gradient of any
micro-patch for an arbitrary time interval.
%
% subsection reconstructing_tangential_surface_deformation (end)
%
\subsection{Initialization} % (fold)
\label{sub:initialization}
%
At the start of the simulation, the initial surface has to be seeded with
patches.
%
Most simulations start with a very simple surface, such as a plane or
sphere, for which this operation is trivial.
%
If the starting surface is more complex, any existing isosurface meshing
algorithm that produces near-equilateral triangles can be used, with patches
initialized to cover the area of the 1-ring of each vertex.
%
When initializing the micro-patches, care has to be taken not to leave any
holes, which might grow larger over time if the surface expands.
%
We ensure this by overlapping the initial patches, which will increase the
number of necessary patches by a constant factor, but is easy to implement.
%
More elaborate approaches which avoid covering the surface with more than one
layer of patches are conceivable.
%
% subsection initialization (end)
%
% section discretization (end)
