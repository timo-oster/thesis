
%
\section{Discussion} % (fold)
\label{sec:fst_discussion}
%
Due to its nature as an algorithm designed for in-situ execution in a highly
parallel environment, our method has some inherent limitations.
%
The most significant problem arises from the exponential nature of stretch
in a flow field.
%
If a constant flow field stretches a time surface by a factor of two in a
certain time interval, the surface will quadruple its area in double the time
and so on.
%
As a consequence, errors in estimating this stretch will accumulate
exponentially over time.
%

%
Because we are operating in-situ and can not go back in time to fix errors after
the fact, we must split micro-patches early enough to limit the amount of error
accumulation we get.
%
This is especially critical for keeping the surface sufficiently covered in
patches.
%
Even a tiny hole left at some point in time may eventually grow very large.
%
Splitting micro-patches early to limit error accumulation leads to very large
numbers of patches over time.
%
Therefore, the error thresholds $r_\perp$ and $r_\shortparallel$ have to
be carefully chosen, taking into account the acceptable error for the maximum
investigated time interval, and the overhead in terms of memory consumption and
computing time added to the simulation.
%
This is not a trivial task and requires some experience of the user, and it is
not clear how it could be simplified.
%
However, the tangential deformation will generally be investigated in statistics
with other simulated quantities, where the error can be analyzed and taken into
account.
%
In future work, we want to investigate a strategy for better controlling the
number of micro-patches over time.
%
This could be handled by periodically merging or redistributing patches on the
surface.
%
This is a global process that would not be very well suited for parallelization.
%

%
Because we track independent micro-patches, we do not produce a closed, manifold
mesh of the flame surface.
%
This is due to our strict parallelization requirements, which are not met by the
global nature of subdivision and join operations in meshes.
%
Direct rendering of the data produced by our method can be done via splatting of
the micro-patches.
%
If a closed mesh surface is required, it can be reconstructed using any
available meshing algorithm for point clouds.
%

%
Because our algorithm only tracks micro-patches initialized on the starting
surface, it does not handle the case of new disconnected surface parts
appearing during the course of the simulation.
%
This can be easily addressed by periodically checking for new parts of the
surface that are not yet covered, and initializing new patches.
%
For new surface parts streaming in from a non-periodic boundary, a more
sophisticated approach might be necessary, which is a subject for future work.
%

%
If we wanted to measure the tangential deformation of the surface only at single
points, we could simply compute the product integral of the instantaneous
Jacobian $\mJ(\vu)$ along the path of each point.
%
This would not require the tracking of ghost particles and would be cheaper in
terms of communication and memory overhead.
%
However, our goal is to track the whole surface.
%
By measuring deformation based on the relative movement of multiple points, we
get the average of a finite part of the surface.
%
This introduces a filtering effect and better represents the behavior of the
surface as a whole.
%
More importantly though, we need the information gained from tracking a group
of points to compute the error measures $e_\shortparallel$ and $e_\perp$, which
are based on the deviation from linear behavior.
%
Without this information, which is not included in $\mJ(\vu)$, we could only
use more inaccurate criteria for splitting and merging patches.
%
This would lead to larger errors in the measurement of deformation and larger
holes in the surface.
%

%
Our algorithm is the first to provide a viable way of tracking the whole flame
surface in-situ over the complete simulation time.
%
This opens new pathways to the investigation of flame behavior.
%
Single snapshots, which are still often the basis for the analysis of a
simulation, do not show detailed changes in surface shape over time, and the
correspondence between points on the surface in two different snapshots can not
be reconstructed.
%
Because we explicitly track single points on the surface over extended periods
of time, the evolution of simulation variables at the point positions over time
also becomes easily observable.
%
Obtaining this data for the complete flame surface enables visual and
statistical evaluation of direct numerical combustion simulations on a new
scale.
%
Our novel method of measuring tangential deformation provides combustion
researchers with a new quantity to study the effects of flame-turbulence
interactions.
%
This integration-based quantity is only made possible by the in-situ nature
of the algorithm.
%
Accurate path line integration is simply not possible as a post process, if the
simulation data can only be stored to disk in a massively reduced temporal or
spatial resolution.
%

%
The overhead in memory and computing time caused by our method is fairly large
compared to existing in-situ visualization approaches, which are generally
designed to require only a small fraction of the computing time of the
simulation \cite{Ma2009}.
%
In this context, it is important to note that our approach is not meant to be a
visualization method only, and it is not meant to be a general-purpose tool that
is activated for every simulation.
%
It is a specialized analysis tool that can be used in situations where
combustion researchers are specifically interested in the detailed behavior of
the flame surface over long time periods.
%

%
Such tools are necessary when a detailed analysis of the low-level behavior
of the flame is required.
%
For example, Scholtissek \etal{}~\cite{Scholtissek2017} report a four-fold
increase in computing time for their gradient trajectory tracking, which enabled
them to develop a more accurate flamelet model.
%
In this case, the analysis was employed in the context of a research project
that required accurate low-level information which can only be achieved with
high computational overhead.
%
We expect our algorithm to be used in a similar context.
%
It is of particular interest for the building of combustion models and the
investigation of local flame extinction and re-ignition mechanics.
%
Existing combustion literature, such as works by Sripakagorn
\etal{}~\cite{Sripakagorn2004}, observe the history of temperature and heat
release at single points on the surface that are tracked over time.
%
By providing such histories for a great number of points covering the whole
flame surface, we enable a statistical evaluation that could be the basis for
new models of unsteady flame behavior.
%
% section discussion (end)
