
%
The visual analysis of combustion processes is one of the challenges of modern
flow visualization.
%
In turbulent combustion research, the behavior of the flame surface contains
important information about the interactions between turbulence and chemistry.
%
The extraction and tracking of this surface is crucial for understanding
combustion processes.
% However, not much work has been done towards a complete analysis of such
% surfaces over the whole lifetime of a combustion simulation.
%
This is impossible to realize as a post-process because of the size of the
involved datasets, which are too large to be stored on disk.
%
% The major issue is that saving output data in a sufficient temporal resolution
% to enable an accurate tracking of a surface as a post process is prohibitively
% expensive in terms of storage space.
%
We present an in-situ method for tracking the flame surface directly during
simulation and computing the local tangential surface deformation for arbitrary
time intervals.
%
In a massively parallel simulation, the data is distributed over many processes
and only a single time step is in memory at any time.
%
To satisfy the demands on parallelism and accuracy posed by this situation, we
track the surface with independent micro-patches and adapt their distribution as
needed to maintain numerical stability.
%
% clarify that this is in contrast to going back in time and re-integrating with
% a higher resolution?
%
% The tangential deformation gradient is then combined from the displacements of
% the points between reseeding events.
%
With our method, we enable combustion researchers to observe the detailed
movement and deformation of the flame surface over extended periods of time and
thus gain novel insights into the mechanisms of turbulence-chemistry
interactions.
%
We validate our method on analytic ground truth data and show its applicability
on two real-world simulation.
