\section{Implementation} % (fold)
\label{sec:fst_implementation}
%
Our algorithm is implemented as an extension to the {DINO} direct numerical
simulation code \cite{Abdelsamie2016}.
%
The simulation domain is a rectangular box which is distributed to multiple
processors by a block decomposition along two of its three axes (pencil
decomposition).
%
This unusual decomposition is required because of the pressure solver, which
operates in Fourier space and needs the complete domain in memory in one
dimension.
%
The communication between the processors uses MPI \cite{MPIForum2015}.
%

%
Each tracked point gets a unique ID at its creation.
%
In each process, all points located in its domain block, as well as all points
in a halo region around the block, are kept in memory.
%
This halo region is large enough to contain all ghost particles of any patch
whose central point is inside the processor's block.
%

%
After each simulation step, the fluid velocity $\vv$ and derivatives of the
scalar variable $s$ defining the flame surface are interpolated from the
simulation grid at all surface points to compute their velocity $\vu$.
%
Because each simulation block only holds data for the grid cells inside its
boundaries, the values for the points in the halo region have to be obtained by
communicating with all neighbor processes.
%
The points are then advanced by one simulation time step.
%
For additional stability, we then perform a few steps of a Newton scheme to move
them back onto the isosurface we are tracking.
%
As the surface points move during the simulation, they will often migrate across
processor boundaries.
%
This is automatically handled by our implementation as part of the information
exchange with neighboring processors when updating the points in the halo
region.
%

%
Since our algorithm is running in lockstep with the simulation, only data for
the current time step is available in memory at any time and the time step
is controlled by the simulation.
%
This means that without generating additional memory overhead, only first order
schemes can be used to integrate the surface velocity $\vu$.
%
In our implementation, we therefore use an Euler scheme to advance the positions
of the points between simulation steps.
%

%
The simulation time step in \ac{DNS} is generally dominated by the chemistry time
scale, which is much smaller than the fluid velocity time scale.
%
Since the surface velocity typically follows the fluid velocity closely, and an
Euler scheme is adequate to track the flame surface almost everywhere, as
points move only a fraction of the size of a grid cell in each time step.
%
Exceptions to this rule only occur at very sharp creases in the surface, which
can occur when two parts of the surface fold into each other, possibly leading
to changes in topology.
%
The high surface velocities occurring here can lead to points migrating far
away from the surface in a single time step.
%
Since the surface at these locations is contracting rapidly anyway, we simply
delete any micro-patches that end up further away from the surface than one half
of their previous advection step, provided the distance is at least
$\sfrac{1}{5}$ the size of a grid cell.
%
In our tests, this strategy did not impact the accuracy of the surface tracking
in any significant way.
%

%
Depending on the values of the errors $e_\perp$ and $e_\shortparallel$, the
micro-patches are now split or merged and any patches that partially crossed
a non-periodic outflow boundary of the simulation domain are removed.
%
At this point, control is given back to the simulation, which performs the next
iteration.
%

%
The deformation gradients can either be computed in-situ or as a post-processing
step.
%
In the first case, the time intervals for the deformation need to be specified
before the simulation starts.
%
Each patch then remembers its configuration $\mC$ at the time it was last reset
and the transformations $\mE$ from the start of each time interval up until this
last reset time.
%
In the second case, the point positions and normals of all micro-patches are
stored to disk for each time step.
%
The deformation gradients can then be reconstructed for arbitrary time
intervals, but at the cost of increased hard disk storage demand.
%
% section implementation (end)