\section{Extracting Tensor Core Lines from Piecewise Linear Data} % (fold)
\label{sec:extracting_tensor_lines}
%
As we have mentioned in \cref{sub:tcl_definition}, the definition of tensor core
lines is almost a straightforward application of the \ac{PEV} operator, except
for the unknown $\vr$ in the second tensor field.
%
We can therefore not directly use the algorithm for extracting \ac{PEV} lines
we presented in \cref{sec:extracting_pev_lines}.
%
Instead we derive a different, more generic, algorithm that is based on the same
principles.
%
In fact, this algorithm can be used to extract \ac{PEV} lines and degenerate
lines as well, as we point out in \cref{sec:computing_pev_and_degenerate_lines}.
%
\Todo[inline]{Discuss with Christian about which parts he needs for his habil.}

%
Analogous to our \ac{PEV} extractor, we assume tetrahedral partition of the
three-dimensional domain and restrict our search to the boundaries between
tetrahedral cells.
%
The intersections of tensor core lines with these triangles are again isolated
points, which we extract using a root finding algorithm.
%

%
\subsection{General Algorithm}
%
A tensor core line consists of all locations $\vx$ where
$\mT(\vx)\, \vr \parallel \vr$ and $\nabla_{\vr}\mT(\vx)\, \vr \parallel \vr$,
see \eqref{eq:tensor_line}.
%
If two vectors are parallel, their outer product is zero.
%
We can therefore express the criterion as solutions to the system of equations
%
\begin{equation}\label{eq:system}
\begin{aligned}
  (\mT(\vx)\,\vr) \times \vr &= \vNull\\
  (\nabla_{\vr} \mT(\vx)\, \vr) \times \vr &= \vNull \,\text{.}
\end{aligned}
\end{equation}
%
This system consists of six polynomial equations in the unknown variables $\vx$
and $\vr\neq{}\vNull$.
%

%
The polynomials are of maximum degree 1 in $\vx$ and 3 in $\vr$.
%
Note that for a linear tensor field, $\nabla_{\vr}\mT(\vx)=\nabla_{\vr}\mT$ is
constant and thus independent of $\vx$.
%
Like for the \ac{PEV} algorithm, we parameterize the space of possible
eigenvector directions $\vr$ such that they are defined \wrt triangles covering
a hemisphere.
%
With the restriction of the local search space to triangles that bound
tetrahedra, the solution of the polynomial system is equivalent to finding
isolated (real) roots, \ie, points where all six polynomials simultaneously
become zero.
%

%
We again construct a bisection algorithm that uses the Bernstein-B\'ezier form
(\cref{sec:bb}) of the polynomials to exclude the presence of roots within
subtriangles.
%
Within the search space (\cref{sec:searchspace}) a recursive
subdivision (\cref{sec:subdiv}) approximates the locations of roots in
$\vx$-$\vr$-space with an arbitrary user-defined precision.
%
The local feature extraction is followed by a stage that clusters solutions and
then connects feature points to lines within cells (\cref{sec:cluster})
%
Finally, we use a filtering criterion for removing unstable solutions
(\cref{sec:filt}).
%
\subsection{Polynomial System in Bernstein-B\'ezier Form}
\label{sec:bb}
%
We consider bivariate polynomials $p:\,\RRSet^2\rightarrow\RRSet$ of degree
$n$, which are evaluated in a triangular domain $\Delta\subset\RRSet^2$.
%
Let indices $i,j,k\geq{}0$.
%
We write $p(\vu)$ in Bernstein-B\'ezier form as
%
\begin{equation}\label{eq:bb}
  p(\vu)=\Sum{B^n_{ijk}(\vu)\,b_{ijk}}{i+j+k=n}~,\quad%
  B^n_{ijk}(\vu)=\frac{n!}{i!j!k!}\,a_1^i\,a_2^j\,a_3^k
\end{equation}
%
with the bivariate Bernstein polynomials $B^n_{ijk}$ as basis and coefficients
(or B\'ezier control points) $b_{ijk}$.
%
The basis is defined w.r.t. the barycentric coordinates $a_\ell:=a_\ell(\vu)$,
the linear polynomials that satisfy
%
% \begin{equation}\label{eq:bc}
% \begin{aligned}
%   \sum _{\ell=1}^3 a_{\ell}(\vu)\,\vu_{\ell} = \vu\quad\text{and}\quad
%   \sum _{\ell=1}^3 a_{\ell}(\vu) = 1~.
% \end{aligned}
% \end{equation}
%
\begin{equation*}
a_1\vu_1+a_2\vu_2+a_3\vu_3 = \vu\quad\text{and}\quad
a_1+a_2+a_3 = 1
\end{equation*}
\wrt triangle $\Delta$ spanned by vertices $\vu_1,\vu_2,\vu_3$
\cite{Hoschek1993}.
%

%
The Bernstein-B\'ezier representation has a number of remarkable properties.
%
Important to our setting is the \emph{convex hull property}\/:
%
For any $\vu\in\Delta$ -- or equivalently all barycentric coordinates are
nonnegative -- the value $p(\vu)$ is bounded by the convex hull of the control
points $b_{ijk}$.
%
For scalar coefficients $b_{ijk}\in\RRSet$ this means that if either all
$b_{ijk} > \num{0}$ or all $b_{ijk} < \num{0}$ for $i+j+k=n$ then $p$
\emph{cannot} have a zero crossing (or real root) on $\Delta$.
%
We use this criterion for an iterative subdivision of a triangle $\Delta$ into
smaller and smaller triangles that either may contain or cannot contain a root.
%
\subsection{Parameterization of the Search Space}
\label{sec:searchspace}
%
The equations in the system~\eqref{eq:system} are polynomials in $\vx$ and
$\vr$.
%
This means the search space consists of two independent domains: position and
direction.
%

%
As pointed out before, positions $\vx$ are restricted to points on
triangles bounding tetrahedral cells.
%
For each triangle of a tetrahedron we represent positions $\vx$ in
barycentric coordinates \wrt this triangle.
%
Barycentric coordinates are defined in \emph{local coordinates} of the triangle
and therefore have two degrees of freedom.
%
After switching to barycentric coordinates the further steps, polynomial
evaluation and subdivision, are independent of the supporting triangle.
%

%
We represent a direction vector $\vr$ as a point on a hemisphere.
%
As not only its orientation (and thus sign) is irrelevant for the
system~\eqref{eq:system} but also its scale (as long as $\vr\neq{}\num{0}$), we
approximate the hemisphere by a triangulation.
%
This way, we use the same parameterization and the same representation
for $\vx$ and $\vr$.
%
We remark that there is no need for an ``accurate'' approximation of
the hemisphere.
%
We simply use a four-sided open pyramid (see \cref{fig:subdivision_scheme}).
%

%
\begin{figure}[t]
  \centering
  \setlength\figurewidth{\textwidth}
  %
\begin{tikzpicture}[scale=\figurewidth/1cm*0.18, line join=round]
\tikzstyle{currentface} = [fill=mycolor3]
\tikzstyle{axes} = [arrows=-latex]
\tikzstyle{trilines} = [thick, fill=white]
\tikzstyle{back} = [gray]

\coordinate (x1) at (90:1);
\coordinate (x2) at (-30:1);
\coordinate (x3) at (210:1);

\coordinate (x21) at ($0.5*(x1) + 0.5*(x2)$);
\coordinate (x22) at ($0.5*(x2) + 0.5*(x3)$);
\coordinate (x23) at ($0.5*(x3) + 0.5*(x1)$);

\coordinate (x31) at ($0.5*(x21) + 0.5*(x2)$);
\coordinate (x32) at ($0.5*(x2) + 0.5*(x22)$);
\coordinate (x33) at ($0.5*(x22) + 0.5*(x21)$);

\draw [trilines] (x1) -- (x2) -- (x3) -- cycle;
\draw [trilines] (x21) -- (x22) -- (x23) -- cycle;
\draw [trilines, currentface] (x31) -- (x32) -- (x33) -- cycle;

% \draw [thin] (1.8, 1.5) -- (5.1, 1.5) -- (5.1, -1) -- (1.8, -1) -- cycle;
% \draw [thin] (x33) -- (1.8, 1.5);
% \draw [thin] (x32) -- (1.8, -1);

% \node [above] at (x1) {$\vx_1, \mS_1, \mT_1$};
% \node [below] at (x2) {$\vx_2, \mS_2, \mT_2$};
% \node [below] at (x3) {$\vx_3, \mS_3, \mT_3$};
\node [below] at (x32) {$\Delta_{\vx}$};

\begin{scope}[shift={(2.8, -0.2)}]

    \coordinate (null) at (0, 0, 0);
    \coordinate (px) at (1, 0, 0);
    \coordinate (mx) at (-1, 0, 0);
    \coordinate (pz) at (0, 0, 1);
    \coordinate (mz) at (0, 0, -1);
    \coordinate (py) at (0, 1, 0);

    \coordinate (pxpz) at ($0.5*(px) + 0.5*(pz)$);
    \coordinate (pzmx) at ($0.5*(pz) + 0.5*(mx)$);
    \coordinate (mxmz) at ($0.5*(mx) + 0.5*(mz)$);
    \coordinate (mzpx) at ($0.5*(mz) + 0.5*(px)$);
    \coordinate (pxpy) at ($0.5*(px) + 0.5*(py)$);
    \coordinate (pzpy) at ($0.5*(pz) + 0.5*(py)$);
    \coordinate (mxpy) at ($0.5*(mx) + 0.5*(py)$);
    \coordinate (mzpy) at ($0.5*(mz) + 0.5*(py)$);

    \coordinate (x1) at ($0.5*(px) + 0.5*(pxpz)$);
    \coordinate (x2) at ($0.5*(pxpz) + 0.5*(pxpy)$);
    \coordinate (x3) at ($0.5*(px) + 0.5*(pxpy)$);

    \draw [axes] ($1.5*(mx)$) -- ($1.5*(px)$);
    \draw [axes] ($1.5*(mz)$) -- ($1.5*(pz)$);
    \draw [axes] (null) -- ($1.5*(py)$);

    \draw [trilines, back] (mx) -- (mz) -- (py) -- cycle;
    \draw [trilines, back] (mz) -- (px) -- (py) -- cycle;
    \draw [trilines] (px) -- (pz) -- (py) -- cycle;
    \draw [trilines] (pz) -- (mx) -- (py) -- cycle;

    \draw [trilines] (pxpz) -- (pzpy) -- (pxpy) -- cycle;
    \draw [trilines] (pzmx) -- (mxpy) -- (pzpy) -- cycle;

    \draw [trilines, currentface] (x1) -- (x2) -- (x3) -- cycle;

    \node [below] at (x1) {$\Delta_{\vr}$};

\end{scope}

\end{tikzpicture}
  \caption{The search space is parameterized on triangles $\Delta_{\vx}$ in position
  space and $\Delta_{\vr}$ on the hemisphere of possible eigenvector directions.
  The figure shows a pair of triangles after several subdivision steps.}
  \label{fig:subdivision_scheme}
\end{figure}
%

%
For a given triangle, we have to consider a tensor field $\mT$ that is linear in
$\vx$ and a direction vector $\vr$ that is linearly interpolated on a triangle
of the ``hemisphere''.
%
Then the left-hand-sides of the system~\eqref{eq:system} give three
polynomials that are linear in $\vx$ and quadratic in $\vr$ and three
polynomials that are cubic in $\vr$ as they don't depend on $\vx$
because $\nabla_{\vr}\mT$ is constant.
%

%
Each of these six polynomials can be written in the form
%
\begin{equation}\label{eq:polynomial}
  \begin{aligned}
    p(\vx,\vr)=%
    \sum_{\substack{i+j+k=1\\ \alpha+\beta+\gamma=3}}%
    B^1_{ijk}(\vx)\,B^3_{\alpha\beta\gamma}(\vr)\,%
    b_{ijk\,\alpha\beta\gamma}~.
  \end{aligned}
\end{equation}
%
This is the tensor product of the interpolation in $\vx$ and $\vr$.
%
It gives $\num{3}\times{}\num{10} = \num{30}$ coefficients
$b_{ijk\,\alpha\beta\gamma}$.
%
(All indices are nonnegative. Latin indices indicate position space,
Greek indices denote direction space.)
%
This form has degree \num{4} and can represent all polynomials in
system~\eqref{eq:system}.
%
We use this unified representation for didactic purposes.
%
Using the real number of degrees in $\vx$ and $\vr$ for each polynomial gives a
smaller number of coefficients (\num{18} or \num{10}).
%
Note that position and direction are expressed in \emph{local coordinates}\/ (or
barycentric coordinates) of the triangles, such that $p$ depends on only four
coordinates in total.
%
For the sake of a concise notation we write $p(\vx,\vr)$, $B^1_{ijk}(\vx)$
and $B^3_{\alpha\beta\gamma}(\vr)$.
%
\subsection{Root Finding by Subdivision}
\label{sec:subdiv}
%
Algorithms for finding roots of B\'ezier curves and surfaces are
typically based on the convex hull property and use a recursive
bisection~\cite{Rockwood1989,Hoschek1993}.
%
We adopt this technique.
%
% We adopt the standard bisection algorithm
% \cite{Rockwood:1989:RRT,Hoschek1993} for finding roots of
% polynomial functions $\RRSet\rightarrow\RRSet$ that is based on the
% convex hull property of the Bernstein-B\'ezier representation of
% polynomials.
%
The main differences in our setting are the fact that all six
equations in \eqref{eq:system} must be satisfied simultaneously and
that this is checked in two different two-dimensional domains:
position and space.
%
\subsubsection{Roots of One Single Polynomial}
%
The system \eqref{eq:system} defines six polynomials.
%
For the initialization of the algorithm, we need to determine the
Bernstein-B\'ezier form, \ie, the coefficients $b_{ijk\,\alpha\beta\gamma}$
in \eqref{eq:polynomial}, for each of these polynomials.
%
This can be done easily by sampling and interpolation:
%
There are $\num{3}\times{}\num{10}$ coefficients and two triangular parameter
domains.
%
As \eqref{eq:polynomial} lives in a tensor-product space,
interpolation of position and direction can be separated.
%
Chose \num{3} or \num{10} distinct sampling positions in $\Delta_{\vx}$ or
$\Delta_{\vr}$, respectively, and evaluate the values for the given
polynomial.
%
Then interpolate these values using the Bernstein-B\'ezier basis.
%
The interpolation conditions define a linear system that has a unique
solution.
%
We remark that the choice of sampling positions can be arbitrary as
long as they are distinct.
%
For polynomial degree $n$, we use the domain points
$\tfrac{1}{3}(i,j,k)$ with $i+j+k=n$ in barycentric coordinates.
%
This ensures a well-conditioned system matrix.
%
As the sampling positions are fixed, the system matrix is constant,
\ie, interpolation requires only inversion or factoring.
%
So after sampling, the conversion to Bernstein-B\'ezier form reduces
to a linear transform than can be expressed as a matrix-multiplication.
%

The outline of the subdivision algorithm is as follows.
%
We are given a pair $(\Delta_{\vx},\Delta_{\vr})$
of position-direction parameter triangles and a polynomial in
Bernstein-B\'ezier form.
%
The coefficients $b_{ijk\,\alpha\beta\gamma}$ indicate the absence of zero
crossing if they are either all positive or all negative.
%
In this case, no root is found and processing of
$(\Delta_{\vx},\Delta_{\vr})$ stops.
%
If there is any sign change or zeros in the coefficients, there
\emph{may} exist roots within the parameter space
$(\Delta_{\vx},\Delta_{\vr})$.
%
In this case we \emph{subdivide}\/ one of the parameter triangles.
%
We alternate between subdividing $\Delta_{\vx}$ in position-space and
$\Delta_{\vr}$ in direction-space.
%
For both, we use a regular 1:4-split (see \cref{fig:subdivision_scheme}).
%
Each of the four new subtriangles is processed recursively in the same
way.
%

%
\subsubsection{System of Polynomials}
%
We explained the basic algorithm for a single polynomial.
%
Solving system~\eqref{eq:system} means finding solutions $\vx,\vr$ where all six
polynomials become zero simultaneously.
%
We test each polynomial \emph{individually}.
%
Only if there is a sign change in the coefficients for \emph{all}
polynomials, a simultaneous root can exist.
%

%
Every level of subdivision restricts the parameter domain and thus
puts tighter bounds on the region that (potentially) contains a root.
%
The subdivision stops either if there cannot be any root in
$(\Delta_{\vx},\Delta_{\vr})$ or when the magnitude of all polynomials
drops below a threshold (see below).
%
In the latter case, we have found a root, and the barycenters of the
triangles define its location in parameter space.
%

Similar to the initial interpolation, the Bernstein-B\'ezier form of
the subdivided polynomial can be evaluated by a linear transformation:
%
Evaluate the polynomial at domain points in the new, smaller triangle
and apply interpolation.
%
Evaluation and interpolation can be combined to one transformation for each
of the four new triangles.
%

\subsubsection{Solution and Stopping Criterion}
%
All computations involving Bernstein-B\'ezier polynomials are done in
barycentric coordinates, which yields a concise implementation of the
algorithm.
%
However, the barycentric coordinates are relative to the current
triangle, and we still need to keep track of the absolute positions of
its vertices in parameter space for bounding the regions of roots.
%
This can be done with a small amount of bookkeeping by tracking the
subdivision steps such that any ``child'' triangle can be
reconstructed from its ``parent'' and ultimately from the initial
parameter triangle.
%

%
We stop the subdivision when we are close enough to a root.
%
We require the magnitude of all polynomials to drop below a threshold
simultaneously.
%
For each polynomial its magnitude is bounded in
$(\Delta_{\vx},\Delta_{\vr})$, and we test
%
\begin{equation}
  \begin{aligned}
    |p|~<~%
    \varepsilon_{\mathrm{t}}\,\cdot\,\invn{\mT}_{\infty}~,
  \end{aligned}
\end{equation}
%
where $|p|:=\max \{\, |b_{ijk\,\alpha\beta\gamma}| \,\}$ is an upper bound for
the magnitude of p in $(\Delta_{\vx},\Delta_{\vr})$.
%
The magnitude of the tensor field in $\Delta_{\vx}$ is given as
%
\begin{equation}
  \begin{aligned}
    \invn{\mT}_{\infty}=\sup_{\vx\in\Delta_{\vx}} \{\,\invn{\mT(\vx)}\,\}
    = \max_i\{\, \invn{\mT_i} \,\}~,
  \end{aligned}
\end{equation}
%
where $\mT_i$ denote the constant tensors at the triangle vertices
$i=1,2,3$, and $\invn{\mT_i}$ denotes their spectral norm.
%

%
The parameter $\varepsilon_{\mathrm{t}}$ defines a \emph{relative}\/ threshold,
which is independent of the local magnitude $\invn{\mT}_{\infty}$ of the tensor
field within $\Delta_{\vx}$.
%

\subsubsection{Breadth-First Search Modification}
%
%
As we hinted at in \cref{sec:tcl_theory}, the symmetries and regular shapes
inherent to data such as stress simulations of mechanical parts often lead to
\eqref{eq:tensor_line} being fulfilled or almost fulfilled on surface- or
volume-type structures.
%
Also there may be tensor core lines which do not intersect but which
are part of the domain triangle.
%
In these cases the roots are no longer isolated points but algebraic
structures.
%
As a consequence, the presented algorithm would not be efficient, as it would do
an exhaustive search for all ``points'' on the structure.
%
In cases where the higher-order structures are disturbed by noise and break down
to lines, the algorithm still has to do a large number of subdivisions before
reaching a termination criterion.
%

%
A simple modification of the algorithm enables detecting such cases:
%
we apply a breadth-first search when looking for roots.
%
In the implementation, we use a queue of pairs $(\Delta_{\vx},\Delta_{\vr})$ of
parameter regions with potential roots.
%
If the number of elements in the queue exceeds a threshold $M$, we
assume that the solution to \eqref{eq:system} forms an algebraic
structure and terminate the local search.
%
If the search is terminated for one of the initial triangles $\Delta_{\vr}$ that
tessellate the hemisphere, we still need to consider the other triangles,
because they may still contain isolated solutions.
%

\subsection{Clustering and Line Connection}
\label{sec:cluster}
%
The root finding returns a list of small parameter regions
$(\Delta_{\vx},\Delta_{\vr})$, which are assumed to contain a solution to
\eqref{eq:system}.
%
The size of these triangles is steered by the threshold
$\varepsilon_{\mathrm{t}}$.
%
Like for the \ac{PEV} algorithm described in
\cref{sub:final_numerical_algorithm}, the algorithm typically returns multiple
regions that all refer to the same solution due to numerical noise.
%
We apply the same post-process here for clustering such regions and selecting a
unique representative $(\vx,\vr)$ for each solution.
%
For each cluster, we select the triangle pair as a representative where $\max \{
|p_i| \}$ is smallest for all polynomials $i=1,\dots,6$.
%
We select the center points of both triangles in this pair as the solution
represented by this cluster.
%

%
For each tetrahedral cell of the dataset, we now connect the root points found
on its faces by a line segment.
%
Since in piecewise linear fields, tensor lines are always straight within a cell
(see \cref{sec:tcl_theory}), we greedily connect the two points with the
smallest difference in eigenvector direction until no more pairs are left.
%
Similar to the vortex core lines by Sujudi and Haimes \cite{Sujudi1995}, this
gives a set of discontinuous line segments.
%
%
\subsection{Filtering}
\label{sec:filt}
%
If the dataset contains not only lines, but also surface- or volume-type
structures where eigenvector trajectories are locally straight, our algorithm
might still find isolated solutions on these structures.
%
These occur if the unstable structures are disturbed by noise.
%
To filter these spurious points, we measure the numeric stability of a solution
given its representative $(\vx, \vr)$ with
%
%
\begin{equation}
  s(\vx, \vr) = \left|\det
      \begin{pmatrix}
          \frac{\nabla_{\vr_2}\mT(\vx)\,\vr}{\|\mT(\vx)\|_{\infty}} &
          \frac{\nabla_{\vr_1}\mT(\vx)\,\vr}{\|\mT(\vx)\|_{\infty}} &
          \vr
      \end{pmatrix}\right| \text{,}
\end{equation}
%
where $\vr, \vr_1, \vr_2$ are orthonormal.
%

%
The stability $s(\vx, \vr)$ measures the directional change that the eigenvector
$\vr$ of the tensor $\mT(\vx)$ experiences when moving orthogonal to the tensor
core line.
%
If this change is small, the magnitude of the determinant in $s$ will be small.
%
This is an indicator that the line is numerically unstable.
%
The normalization by $\|\mT(\vx)\|_{\infty}$ ensures the independence from the
scale of the tensor field.
%

%
Filtering out lines with small numeric stability $s$ is a post-process that must
be done manually by a user.
%
In practice, the distribution of $s$ over all found solutions shows an
exponential behavior.
%
In order to facilitate choosing a threshold, we visualize $s$ on a logarithmic
scale.
%