%
\section{Discussion} % (fold)
\label{sec:tcl_discussion}
%
We introduced tensor core lines as a new feature of second-order tensor fields.
%
It enables the quick detection of swirling behavior in tensor field lines.
%
Such behavior might not have a distinct physical meaning in all applications.
%
However, finding core lines helps to understand the structure of the tensor
field by breaking down a complex feature into a simple line structure that can
be easily visualized.
%
In this regard, our method fits in well with other feature-based visualization
methods.
%

%
Our method is a direct extension of the Sujudi/Haimes method for the extraction
of vortex core lines in vector fields.
%
As such, it shares many of its advantages and drawbacks.
%
The criterion is completely local and does not require integration.
%
As such, it is well parallelizable and not vulnerable to accumulating numerical
errors.
%
Still, we are hardly able to reach interactive run times, as we need to perform
an exhaustive search in a \ac{5D} space.
%
Like Sujudi/Haimes, we perform a search on piecewise linear data, which results
in straight lines within cells and discontinuities of the tensor core lines at
cell boundaries.
%
Using higher-order interpolation of the tensor field would help finding
continuous lines.
%

%
We have chosen to focus on piecewise linear tensor fields where each tensor
component is interpolated independently.
%
While alternative interpolation schemes have been proposed~\cite{Kindlmann2007},
component-wise interpolation is still widely used as a standard approach for
both tensor- and vector fields.
%

%
Unlike Sujudi/Haimes, we have no way of explicitly ensuring our solutions show
only swirling behavior by restricting them to regions where the derivative has
complex eigenvalues.
%
The derivative of the tensor field $\nabla \mT$ is a third-order tensor, for
which the definition of eigenvalues and eigenvectors is non-trivial
\cite{Zheng2007}.
%
This means that we also find structures similar to hyperbolic trajectories in
vector fields \cite{Machado2013,Machado2016}.
%
Further research is necessary in order to be able to distinguish these different
types of features.
%

%
We introduced a measure for the numeric stability of tensor core lines.
%
Unfortunately, filtering out numerically unstable solutions must be done as an
interactive post-processing step, as the threshold is different for each
dataset.
%
It is worth investigating if this process can be automated.
%
Nevertheless, the measure enables us to distinguish significant and
insignificant solutions, which is a very useful tool for assessing the result of
our algorithm.
%

%
Our algorithm is numerically very stable.
%
We have three free parameters, two of which can be chosen in a wide range
without significant influence on the results, as we show in
\cref{sub:performance}.
%
The parameter $M$, which influences run time the most, can be chosen the same
for most datasets and as such does not require fine-tuning either.
%

%
Our algorithm is only designed for extracting structurally stable line
features, but surfaces or regions where the zero curvature criterion is
almost fulfilled seem to be common in real-world stress tensor data.
%
This might be due to the common occurrence of symmetries and regular shapes in
human-made objects, which are most frequently the focus of structural analysis.
%
It would therefore be interesting to investigate if these structures can
explicitly be extracted, possibly by restricting the search space to the edges
of tetrahedral cells.
%

%
Finally, it is worth noting that neither the formal definition of tensor core
lines nor the extraction algorithm poses any restrictions on the tensor field,
except that it be differentiable.
%
As such, it might also be used on indefinite tensor data, such as the Jacobian
of a vector field.
%
Finding applications outside of stress tensor analysis is a subject for further
research.
%