
%
\section{Extracting Tensor Core Lines from Piecewise Linear Data} % (fold)
\label{sec:extracting_tensor_lines}
%
A piecewise linear tensor field is given by a tetrahedral mesh with a discrete
tensor attached to each mesh node.
%
The derivative $\nabla \mT$ is constant and can be determined on a per-cell
basis from the tensors at the four vertices.
%
\todo[inline]{discuss suitability of linear interpolation and motivate that
it is still commonly used in practice}
%
As we have shown, tensor core structures are generally lines.
%
It follows that the intersections of these lines with the triangular faces of
the mesh are generally isolated points.
%
We first find these isolated points on all faces and then connect them to line
segments in a second step.
%
This is consistent with established methods for finding line structures in
vector or tensor fields such as the vortex core line extractor by Sujudi and
Haimes \cite{Sujudi1995} and the extractor for degenerate lines in tensor fields
by Zheng and Pang \cite{Zheng2004}.
%

\subsection{General Algorithm} % (fold)
\label{sub:general_algorithm}
%
A tensor core line is defined by all locations $\vx$ where
$\lambda \mT(\vx) \, \vr = \mu \nabla_{\vr} \mT(\vx)\, \vr = \vr$
for a vector $\vr \neq \vNull$.
%
In a piecewise linear tensor field, we have a constant $\nabla_{\vr} \mT$
in each cell.
%
The solution to \eqref{eq:tensor_line} is then equivalent to the system
%
\begin{equation}\label{eq:targetfunctions}
\begin{aligned}
  (\mT(\vx)\,\vr) \times \vr &= \vNull\\
  (\nabla_{\vr} \mT\, \vr) \times \vr &= \vNull
\end{aligned}
\end{equation}
%
This system consists of six scalar functions $h_1(\vx, \vr), \dots, h_6(\vx,
\vr)$ that need to become zero at the same time.
%
Note that formally, the three scalar functions corresponding to $(\nabla_{\vr}
\mT\, \vr) \times \vr = \vNull$ are not dependent on $\vx$.
%
For the sake of uniformity in notation, we will pretend this dependency
exists in the following.
%

%
The functions $h_i$ are polynomials of degree 3.
% 
Finding a closed form solution for the intersections of the roots of these
functions is impossible. \todo{is it?}
% 
Instead, we employ a recursive search in the space of possible positions $\vx$
and eigenvector directions $\vr$.
% 
The search space for $\vx$ is the triangular cell face.
%
We subdivide the search space for $\vr$ into a set of triangles covering a
hemisphere.
% 
We then consider the functions $h_i(\vx, \vr)$ on each pair of triangles in
$\vx$ and $\vr$ as polynomials in Bernstein-B\'ezier form.
%
In this form, the functions are bounded by their set of coefficients.
%
If any $h_i$ on a triangle pair has coefficients that are all positive or all
negative, the triangle pair can not contain a solution and can be discarded.
%
Otherwise, we subdivide the triangles and recursively examine the coefficients
of each part again, using a breadth-first search strategy.
%
If the number of triangle pairs that were not yet discarded exceeds a threshold
$M$, we assume that we are not converging on isolated points and terminate the
search for this initial triangle in $\vr$.
% 
If we reach a subdivision level where all $h_i$ are smaller than a tolerance
everywhere on the triangle pair, we accept it as a solution.
%
We connect solutions we found on faces of the same tetrahedral cell into line
segments by matching points with similar eigenvector directions.
%
Finally, lines with low numeric stability that may occur due to noise are
filtered out with a user-controlled threshold.
%
We discuss the details of the algorithm in the following.
%
% subsection general_algorithm (end)

\subsection{Expressing the Target Functions in Bernstein B\'ezier Form} % (fold)
\label{sub:expressing_the_target_functions}
\todo{better title}
%
We express the position on a triangular face $\vx_\triangle = \{\vx_1, \vx_2,
\vx_3\}$ in barycentric coordinates $\vw = \T{(w_1, w_2, w_3)}$:
%
\begin{equation}
  \vx(\vw) = \sum_i w_i \vx_i
\end{equation}
%
with $w_1+w_2+w_3=1$ and $w_i > 0$.
%
The value of the tensor field $\mT$ on the triangle is similarly obtained
by linear interpolation between its three corner tensors $\mT_\triangle =
\{\mT_1, \mT_2, \mT_3\}$:
%
\begin{equation}
  \mT(\vx(\vw)) = \mT(\vw) = \sum_i w_i \mT_i
\end{equation}
%

%
We are not interested in the magnitude and orientation of $\vr$, only its
direction.
%
We can therefore safely restrict the search space for $\vr$ to a hemisphere
discretized by a number of triangles (see~\cref{fig:subdivision_scheme}).
%
In each triangle $\vr_\triangle = \{\vr_1, \vr_2, \vr_3\}$, $\vr$ is then also
expressed in barycentric coordinates $\vu = \T{(u_1, u_2, u_3)}$:
%
\begin{equation}
  \vr(\vu) = \sum_i u_i \vr_i
\end{equation}
%
If we substitute $\mT(\vx) = \mT(\vw)$ and $\vr = \vr(\vu)$ in
\eqref{eq:targetfunctions}, we get six polynomials $h_i(\vw, \vu)$
that are defined on the cartesian product of a pair of triangles
$(\vx_\triangle, \vr_\triangle)$.
%
The scale of these functions is dependent on the scale of $\mT$, which might
vary strongly within the same dataset.
%
This makes it hard to choose a universal tolerance for our root finding
algorithm.
%
We therefore normalize $h_i$ with the maximum operator norm (the largest
singular value) of the three corner tensors $\{\mT_1, \mT_2, \mT_3\}$.
%
This gives us our final functions
%
\begin{equation}
  \hat{h}_i(\vw, \vu) = \frac{h_i(\vw, \vu)}
      {\max(\{\|\mT_i\|_{\mathrm{op}} : \mT_i \in \mT_\triangle\})}
\end{equation}
%
Finding $\vw$ and $\vu$ that solve $\hat{h}_i(\vw, \vu) = 0$ gives us the
intersection of a tensor core line with the triangle at $\vx(\vw)$ and the
accompanying eigenvector direction $\vr(\vu)$.
%

%
Since we have now transfomed our functions into barycentric coordinates, we can
write them as the product of two Bernstein-B\'ezier triangles:
%
\begin{align}
\hat{h}(\vw, \vu) &= \sum_{\substack{i+j+k\\=\,n}} \mathrm{B}_{ijk}^n(\vw)\, c_{ijk}
                \sum_{\substack{o+p+q\\=\,m}} \mathrm{B}_{opq}^m(\vu)\, c_{opq} \\
            &= \sum_{\substack{i+j+k\\=\,n}} \; \sum_{\substack{o+p+q\\=\,m}}
              \mathrm{B}_{ijkopq}^{n, m}(\vw, \vu)\, c_{ijkopq} \\
% &\text{with}&& \\
\mathrm{B}_{ijk}^n(\vw) &= \frac{n!}{i!j!k!} w_1^i w_2^j w_3^k \\
\mathrm{B}_{ijkopq}^{n, m}(\vw, \vu) &= \mathrm{B}_{ijk}^n(\vw) \, \mathrm{B}_{opq}^m(\vu)
\end{align}
%
Here, $c_{ijkopq}$ are the coefficients of the combined polynomials.
%
In this form, the range of possible values for $\hat{h}(\vw, \vu)$ inside the
triangles is bounded by the smallest and largest coefficients $c_{ijkopq}$.
%
This means that if the function has any zero-crossings inside the triangles,
the coefficients can not be all positive or all negative.
%
We use this fact in our recursive subdivision scheme, which we detail in the
next section.
% 
% subsection expressing_the_target_functions (end)

\subsection{Recursive Subdivision Scheme} % (fold)
\label{sub:recursive_subdivision_scheme}
%
When we start our recursive search on a triangular face $\vx_\triangle$ of the
dataset, we first map it to a standard triangle
$\hat\vx_\triangle = \{\T{(1, 0, 0)}, \T{(0,1,0)}, \T{(0, 0, 1)}\}$.
%
In this way, we avoid accumulating numerical errors during subdivision for long
and thin triangles and we have a comparable scale of the starting configuration
that is independent of the mesh of the dataset.
% 
The real position of the tensor core points we find is later reconstructed via
the inverse transformation.
%

%
Our algorithm starts on a set of multiple triangle pairs in position and
direction space.
%
Each of these initial pairs consists of the standard triangle
$\hat\vx_\triangle$ and a triangle covering a portion of the hemisphere of
possible eigenvector directions.
% 
The choice of triangulation of the hemisphere is rather arbitrary.
% 
It is only important that it is complete and disjoint, and that each triangle
covers less than an angle of 90 degrees in any direction.
% 

% 
For a given pair $(\vx_\triangle, \vr_\triangle)$, we inspect the coefficients
of all six polynomials $\hat{h}_i(\vw, \vu)$.
%
If any one of them has coefficients that are all positive or all negative, the
triangle pair can not contain a solution and can be discarded.
%
Otherwise, a solution might be present.
%
In this case they are subdivided into four new pairs of triangles.
%
We alternate between subdividing $\vx_\triangle$ and $\vr_\triangle$ to limit
the number of new candidate pairs that have to be checked in each step.
%
Subdividing a triangle pair means computing the coefficients of $\hat{h}_i$ on
four new pairs, where one of the triangles remains the same, while the other is
split into four equally-sized parts.
\todo{Do we need a figure for this?}
%
This can be done by a fixed linear operator applied to the vector of
coefficients.
%
\Cref{fig:subdivision_scheme} shows a triangle pair after several subdivision
steps have been performed.
%



%
Subdivision terminates when the maximum magnitude of all $\hat{h}_i$ on the
current triangle pair is smaller than a threshold $\epsilon_{\mathrm{t}}$.
%
When a triangle pair is found that satisfies this condition, it is stored in a
list of solution candidates for further processing.
%

%
As we mentioned earlier, \todo{actually mention it earlier} for symmetric
tensors, there is a high probability of secondary eigenvectors forming a
surface where their curvature is also zero orthogonal to the tensor core line.
%
In this case, a recursive subdivision algorithm does not work, because it does
not converge on isolated points.
%
To detect this problem, we implement our subdivision scheme as a breadth-first
search and terminate the algorithm if the list of triangle pairs that have
not yet been discarded grows larger than a threshold $M$.
%
Since the eigenvectors of symmetric tensors are always orthogonal, this only
happens for some of the starting triangles covering the hemisphere of possible
eigenvector directions.
%
If the search is terminated in this way for one starting triangle in direction
space, we still continue searching in the other directions, as an isolated
solution might still be found for one of them.
%
\todo[inline]{Our method finds all intersection points if they exist
in constrast to the root finder for degenerate lines by Zheng, which is based
on an iterative Newton-Rhapson method that might get stuck in local minima.}
%
% subsection recursive_subdivision_scheme (end)

\subsection{Clustering and Filtering} % (fold)
\label{sub:clustering_and_filtering}
% 
The result of \cref{sub:recursive_subdivision_scheme} is a list of triangle
pairs $(\vx_{\triangle i}, \vr_{\triangle i})$ where $\hat{h}_i(\vw, \vu) <
\epsilon_{\mathrm{t}}$.
%
Since $\epsilon_{\mathrm{t}}$ is a small but finite number, it is possible that
multiple such solution candidates are found in close vicinity to a real solution.
%
Because of this, we employ a simple clustering to group similar triangle pairs
and select a representative as the solution.
%
We define the distance $d$ between two triangle pairs $\vp_1 = (\vx_{\triangle 1},
\vr_{\triangle 1})$ and $\vp_2 = (\vx_{\triangle 2}, \vr_{\triangle 2})$ as the
maximum Euclidean distance between the center points of the triangles in
position and direction space, \ie,
%
\begin{equation}
  d(\vp_1, \vp_2)
    = \max\left(\frac{1}{3}\sum \vx_{i1} - \vx_{i2},
      \; \frac{1}{3}\sum \vr_{i1} - \vr_{i2}\right) \text{.}
\end{equation}
%
We employ a simple single-linkage hierachical clustering algorithm:
%
We start with each triangle pair as a single cluster.
%
Two clusters are merged if the distance $d$ between any two elements from both
clusters is smaller than a clustering threshold $\epsilon_{\mathrm{c}}$.
%
We repeat this process until the number of clusters no longer changes.
%

%
This algorithm has a complexity of $\mathrm{O}(n^3)$ in the number of solution
candidates.
%
However, the number of solution candidates we find for each tensor core line
point is typically very small.
%
In the vast majority of cases, we find less than \num{200} candidates.
%
At this scale, the performance impact of the clustering algorithm is negligible.
%
For symmetric tensor fields, eigenvectors are always orthogonal.
%
For solutions which are isolated points and belong to different eigenvectors,
the separation between clusters in direction space is therefore rather large.
%
This means the choice of $\epsilon_{\mathrm{c}}$ is not critical as long as
it is not chosen extremely small, or so large that solutions belonging to
different eigenvectors are clustered together.
%
A choice of $\epsilon_{\mathrm{c}} = \num{1e-4}$ worked well in all our
examples.
%
For each cluster, we select the triangle pair as a representative where the
largest magnitude of all coefficients of $h_i$ is smallest.
%
We select the center points of both triangles in this pair as the solution
represented by this cluster.
%

%
For each cell in the dataset, we now connect the points found on its four faces
to line segments.
%
Since in piecewise linear tensor fields, tensor lines are always straight within
a cell, we simply greedily connect the two points with the smallest difference
in eigenvector direction until no more pairs are left.
%
Similar to the vortex core lines by Sujudi and Haimes \cite{Sujudi1995}, this
gives us a set of discontinuous line segments.
%

%
If the dataset contains not only lines, but also surfaces where hyperstreamlines
are locally straight, our algorithm might still find isolated solutions on those
surfaces.
%
These occur if the structurally unstable surface structures are disturbed by
noise.
%
To filter out these spurious points, we measure the numeric stability of
a solution $(\vx, \vr)$ with
%
\begin{equation}
  s(\vx, \vr) = \left|\det
      \begin{pmatrix}
          \frac{\nabla_{\vr_2}\mT(\vx)\,\vr}{\|\mT\|_{\mathrm{op}}} &
          \frac{\nabla_{\vr_1}\mT(\vx)\,\vr}{\|\mT\|_{\mathrm{op}}} &
          \vr
      \end{pmatrix}\right| \text{,}
\end{equation}
%
where $\vr, \vr_1, \vr_2$ are orthonormal.
%
The stability $s(\vx, \vr)$ measures the change in direction the eigenvector
$\vr$ of the tensor $\mT(\vx)$ experiences when moving orthogonal to the tensor
core line.
%
If this change is small, the magnitude of the determinant in $s$ will be small.
%
This is an indicator that the line is numerically unstable.
%
The normalization by $\|\mT\|_{\mathrm{op}}$ ensures the independence of the
scale of the tensor field.
%

%
We filter out all solutions where the line stability $s$ is small.
%
This is best done as a post-process where the user can interactively choose
a threshold $\epsilon_{\mathrm{s}}$.
%
In practice, the distribution of $s$ over all found solutions shows an
exponential behavior.
%
To make choosing a threshold easier for the user, we therefore apply it to the
logarithm of $s$.
%
% subsection clustering_and_filtering (end)
%
% section extracting_tensor_lines (end)