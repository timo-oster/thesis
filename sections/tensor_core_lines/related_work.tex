\section{Related Work} % (fold)
\label{sec:tcl_related_work}
%
The basis of this work is the extractor for centers of swirling flow in vector
fields first described by Sujudi and Haimes~\cite{Sujudi1995}.
%
In their original paper, they examine the Jacobian $\nabla \vv$ of a vector
field $\vv$.
%
They search for vortex core lines only in areas where $\nabla \vv$ has two
complex eigenvalues, \ie{}, where the flow shows locally swirling behavior.
%
The location of a vortex core line is then defined as the set of points where
the velocity is parallel to the single real eigenvector.
%
These are the locations where the swirling in the orthogonal plane vanishes
and only a motion along the eigenvector direction remains.
%
Note that vortex core lines defined in this manner are generally not streamlines
of the vector field.
%
When applying this method to a piecewise linear vector field, the result is a
set of straight line segments.
%
Since the derivative $\nabla \vv$ is constant within each cell, these segments
do not generally form closed lines.
%
Single line segments may appear due to noise, but if multiple line segments form
a visually coherent line, it is a strong indicator for vortical behavior in the
flow.
%
The advantage of this vortex core line extractor is its inherent locality, which
makes it well-suited for parallelization and avoids expensive line integration.
%
%
\subsection{Parallel Vectors} % (fold)
\label{sub:tcl_parallel_vectors}
%
Peikert and Roth~\cite{Peikert1999} showed that the vortex core lines described
by Sujudi and Haimes are locations where the acceleration $\nabla \vv \cdot \vv$
is parallel to the vector field, and identified this as one application of a
concept they called the \emph{parallel vectors operator}.
%
This concept had been applied in a lot of other contexts before, such as ridge
detection in scalar fields~\cite{Haralick1983} and extraction of
attachment/separation lines in flows~\cite{Kenwright1999}.
%

%
An overview of a number of algorithms for computing parallel vectors can be
found in Martin Roth's PhD thesis~\cite{Roth2000}.
%
A common approach is to first determine intersection points of the \ac{PV} lines
with the cell faces of a data set.
%
The resulting points are then connected to lines.
%
If there are exactly two intersections with the faces of a cell, they can be
trivially connected with a line.
%
In case of a higher number of intersections, different heuristics have been
employed, the simplest one being to subdivide the cell until there are again
only two intersections.
%

%
Another class of algorithms employs some form of line tracing to follow a
parallel vector line starting from a seed point that has been found by one of
the aforementioned face intersection methods.
%
Banks and Singer~\cite{Banks1995} and Miura and Kida~\cite{Miura1997} employed a
predictor-corrector scheme, following the lines in small steps and minimizing
the angle between the vector fields in an orthogonal plane after each step.
%
Sukharev et al.~\cite{Sukharev2006} presented a mix between both approaches by
first finding intersection points on a (potentially coarser) grid, and then
following the analytical tangent of the \ac{PV} line to connect them.
%
A similar approach is taken by Theisel et al.~\cite{Theisel2003a}, where the
\ac{PV} line is reformulated as a stream line in a {\em feature flow field} that is
derived from the original vector fields.
%
Given a starting point on the \ac{PV} line, it can be traced like a stream line with
any standard ODE solver.
%
The \emph{PVsolve} framework introduced by van Gelder and Pang~\cite{Gelder2009}
is a generalized approach for finding \ac{PV} lines in vector fields of different
dimensionality.
%
Their approach views the tracing of \ac{PV} lines as a root-finding problem and
presents an algorithm that does not suffer from accumulating errors.
%
Weinkauf et al.~\cite{Weinkauf2011} presented stable feature flow fields as an
extension to feature flow fields that similarly eliminates accumulating errors.
%

%
Most \ac{PV} algorithms assume they operate on piecewise linear or piecewise
bi-/trilinear data in two- or three-dimensional space.
%
However, there have been a number of publications dealing with higher-order data
or using higher-order methods in some form.
%
Roth and Peikert~\cite{Roth1998} introduced a method for finding strongly curved
vortex core lines by using higher-order derivatives of the flow field.
%
Bauer and Peikert~\cite{Bauer2002} used a scale-space technique to filter out
small-scale structures and irrelevant large-scale ones when computing vortex
cores.
%
Pagot et al.~\cite{Pagot2011} presented an algorithm for extracting \ac{PV} lines
from data represented by higher-order basis functions.
%
Due to its generality, the already mentioned \emph{PVsolve}
framework~\cite{Gelder2009} allows for extracting parallel vector surfaces in
time-dependent vector fields.
%
Similarly, Theisel at al.~\cite{Theisel2005} extracted \ac{PV} surfaces in unsteady
flows and applied it to vortex core line tracking.
%
All these methods work on vector fields, while our approach deals with
second-order tensor fields.
%
% subsection parallel_eigenvectors (end)

\subsection{Tensor Field Visualization} % (fold)
\label{sub:tcl_tensor_field_visualization}
%
Tensor fields (of second order) occur in a variety of different scientific
contexts.
%
Some examples are stress and strain tensors in mechanical engineering
applications, and diffusion tensors occuring in \ac{DTI}, a special \ac{MRI}
modality used to visualize fiber tracts, \eg, in the human brain.
%
Tensor field visualization methods can be roughly classified into three
different categories:
%
glyph-based, line-/surface-based and topology-based.
%

%
Glyph-based methods for tensor field visualization place small geometric objects
in space to represent certain characteristics of the local tensor.
%
Diffusion tensors, which are symmetric and positive definite, have been
visualized by Kindlmann~\cite{Kindlmann2004} using superquadric glyphs
aligned with the eigenvector directions.
%
By explicitly encoding eigenvalue signs in geometry, Schultz and
Kindlmann~\cite{Schultz2010a} extended this to indefinite tensors.
%
Gerrits \etal{}~\cite{Gerrits2017} additionally incorporated complex eigenvalue
information into the glyph design to visualize general second-order tensors
in \ac{2D} and \ac{3D}.
%
A specialized glyph for stress and strain tensors was introduced by Hashash
\etal{}~\cite{Hashash2003}.
%
Kindlmann and Westin~\cite{Kindlmann2006} addressed the problem of clutter by
optimizing glyph placement to increase information density while minimizing
occlusion.
%
Glyphs for comparative visualization of two different diffusion tensors were
used on medical data by Zhang \etal{}~\cite{Zhang2016}.
%

%
A simple extension of streamlines for visualizing symmetric tensor fields in \ac{3D}
are hyperstreamlines, first introduced by Delmarcelle and
Hesselink~\cite{Delmarcelle1993}.
%
These hyperstreamlines follow one of the eigenvectors of the tensor field, while
their cross-section is a cross shape or ellipse aligned with the other two
eigenvectors and scaled by the corresponding eigenvalues.
%
The problem of undefined integration direction in near-isotropic areas was
adressed by Weinstein \etal{} with a new concept called
tensorlines~\cite{Weinstein1999}.
%
These lines behave like hyperstreamlines, but try to preserve direction in areas
of equal eigenvalues by incorporating local context information.
%
As an extension to this concept, Jeremi{\'c} \etal{} introduced
hyperstreamsurfaces~\cite{Jeremic2002}, which are formed analogous to
hyperstreamlines by using a line instead of a point as the seed structure.
%

%
The topology of symmetric \ac{2D} and \ac{3D} tensor fields was studied by
Delmarcelle~\cite{Delmarcelle1994} and Hesselink~\cite{Hesselink1997}.
%
They characterized the topology of a tensor field by degenerate structures
where two or more eigenvalues are equal.
%
Zheng and Pang~\cite{Zheng2004} build on top of this work and provide
numerical algorithms for extracting the topological skeleton in practice.
%
Similar to the approach we take for extracting tensor core lines, their initial
algorithm is based on finding the roots of a number of discriminant functions on
the cell faces of a dataset using a bisection algorithm, and then connecting
them to lines.
%
Zheng \etal{} later introduced alternative approaches that are based on solving
a system of equations on each face, and on tracing the degenerate lines using
their analytical tangent~\cite{Zheng2005}.
%
As Schulz \etal{}~\cite{Schultz2007} showed, these features are very sensitive to
noise.
%
A more stable approach suitable to noisy data such as obtained from \ac{DTI}
scans was therefore proposed by Tricoche \etal{}~\cite{Tricoche2008}.
%
The notion of tensor topology was recently extended to surfaces of neutral and
traceless tensors by Palacios \etal{}~\cite{Palacios2016}, who also propose an
improved algorithm for extracting degenerate lines.
%
Assymetric tensors were studied in \ac{2D} by Zheng and Pang~\cite{Zheng2005a}
and in the context of flow visualization by Zhang \etal{}~\cite{Zhang2009}.
%
% - Gordon on derivatives of eigenvectors
% section related_work (end)