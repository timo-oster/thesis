
%
\section{Limitations and Future Research} % (fold)
\label{sec:tcl_limitations}
% 
Limitations can be discussed from two points of view:
%
the operator itself and the presented numerical extraction algorithm.
%
A limitation of the PEV operator is that it can only be applied to problems
where the norm of the tensors does not matter.
%
This limits the applicability but on the other hand focuses on features of the
tensor fields that are less covered by other methods.
%

%
The presented extractor works for piecewise linear tensor fields only.
%
An extension to hexahedral grids as well as higher order interpolations is
subject of future research.
%
The performance of the algorithm can be improved by parallelization.
%
In principle, the algorithm is parallelizable (each cell can be treated
independently).
%
However, even if this is carefully carried out, interactive frame rates (for
instance for comparing time-dependent tensor fields) are hardly achievable
because we still have to do a search in a 6D space.
%
Due to the possibility of many candidate solutions for each intersection of a
PEV line with a face, we can not give an upper limit on the error of the
PEV line position.
%
This might limit its applicability in cases where a highly accurate PEV line is
required.
%

%
Applications of the PEV operator are not limited to the examples shown in this
paper.
%
Further possible scenarios that are left to future research are the comparative
visualization of DT-MRI data or a comparative visualization of Jacobian fields
for flow visualization.
%