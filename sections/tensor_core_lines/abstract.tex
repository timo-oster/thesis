
%
Vortices are important features in vector fields that show a swirling behavior
around a common core.
%
The concept of a vortex core line describes the center of this swirling
behavior.
%
In this work, we examine the extension of this concept to 3D second-order
tensor fields.
%
Here, a behavior similar to vortices in vector fields can be observed for
trajectories of the eigenvectors.
%
Vortex core lines in vector fields were defined by Sujudi and Haimes to be the
locations where stream lines are parallel to an eigenvector of the Jacobian.
%
We show that a similar criterion applied to the eigenvector trajectories of a
tensor field yields structurally stable lines that we call \emph{tensor core
lines}.
%
We provide a formal definition of these structures and examine their
mathematical properties.
%
We also present a numerical algorithm for extracting tensor core lines in
piecewise linear tensor fields.
%
We find all intersections of tensor core lines with the faces of a dataset using
a simple and robust root finding algorithm.
%
Applying this algorithm to tensor fields obtained from structural mechanics
simulations shows that it is able to effectively detect and visualize regions of
rotational or hyperbolic behavior of eigenvector trajectories.
%