\chapter{Introduction} % (fold)
\label{cha:introduction}
%
\vspace{-\baselineskip}%
\lettrine[findent=-3.5pt, nindent=4pt, loversize=0.015,
lhang=0.08]{S}{imulations} are an integral part of modern engineering.
%
They use a set of models to make predictions about the behavior of a system
under given boundary conditions.
%
Compared to experiments, simulations are cheaper and quicker to set up, run, and
evaluate, and they often provide access to data that is hard or impossible to
measure in an experiment.
%

%
Simulation data -- just like data from experiments -- is often very large and
complex.
%
It is evaluated to validate the simulation and/or to derive new insight from
the data.
%
Visualization plays an important role in these tasks.
%
It translates the wealth of data into visual representations that are more
easily parsed by humans.
%

%
Visualization, and in particular scientific visualization, is a large field of
research that has produced numerous techniques to display and analyze the data
from engineering simulations and other sources.
%
For many applications, existing simple visualization techniques are sufficient
to derive the desired information from the data.
%
However, for more complex problems or technical requirements, more advanced
solutions are necessary.
%

%
Two active areas of research in visualization are concerned with large data and
data of some kind of higher order.
%
Large amounts of data mostly originate from large or very high-resolution
experiments or simulations.
%
Examples for higher-order data are tensor fields, time-dependent vector fields,
and ensemble or uncertain data.
%
The contributions of this thesis belong to these two areas of research.
%
We present visualization and analysis techniques for the flame surface in
large-scale turbulent combustion simulations, and we develop novel features
that help to understand the complexities of second-order tensor fields using
examples from solid mechanics simulations.
%
\section{Contributions} % (fold)
\label{sec:contributions}
%
The contributions of this thesis are separated into two main parts that at first
glance seem to be quite different.
%
The first is concerned with time-dependent scalar and vector fields, the second
with static tensor fields.
%
The first deals with very large data sizes while the second mostly handles
small- to medium-sized datasets.
%
The first focuses on a concrete application area while the second proposes more
general ideas.
%
However, both parts have in common that they propose techniques for the
extraction of interesting features from data produced by engineering
simulations.
%
In the following, we will give a short summary of our contributions and their
motivation.
%
\subsection{Analysis and Visualization of the Flame Surface in Turbulent
Combustion Simulations} % (fold)
\label{sub:contr_flame_vis}
%
Direct numerical simulations (\acs{DNS}\acused{DNS}) are the most accurate tool
for the simulation of turbulent combustion.
%
They are used as ``numerical experiments'' to gain data for development or
validation of new combustion models.
%
Because of their high spatial and temporal resolution, \ac{DNS} are run on
large, massively parallel supercomputers and produce huge amounts of data.
%
This raw data can not be stored completely in a reasonable amount of time and
has become a bottleneck for the analysis of simulation results.
%

%
A major focus of interest in the modeling of combustion processes is the flame
surface.
%
It is the area of the flame where most chemical reactions take place and most
heat is produced.
%
We present two approaches for the visualization and analysis of different
aspects of the flame surface in large-scale \ac{DNS} of turbulent combustion,
where raw data storage has become the bottleneck.
%

%
The first is a space-saving sparse representation for a certain type of flame.
%
Due to its smaller size, it allows storing and therefore analyzing a larger
number of simulation time steps.
%
This representation can directly be used for a statistical analysis of the flame
structure, and it is the basis for a new visualization technique that highlights
local differences in relation to the shape of the flame.
%
If required, the full data can be reconstructed on the original grid with some
loss of accuracy for the full flexibility of conventional post-processing.
%

%
The second approach we present is an algorithm for tracking the flame surface
in-situ during the simulation.
%
This circumvents the storage bottleneck by processing the data while it is still
in memory and only writing to disk the much smaller results.
%
We propose a massively parallel algorithm using independent micro-patches that
refine and coarsen independently to track the surface.
%
This allows tracking the surface over long time intervals and provides data
about the history of individual points attached to the surface as well as their
relative movement.
%
Using this data, which is impossible to obtain via traditional post-processing
for large simulation runs, combustion researchers can derive new combustion
models, especially incorporating unsteady phenomena.
%
% subsection contr_flame_vis
%
\subsection{Line Features in \ac{3D} Second-Order Tensor Fields} % (fold)
\label{sub:contr_tensor_fields}
%
Tensor fields occur in different scientific disciplines.
%
One important example are stress tensor fields, which are often the result of
solid mechanics simulations.
%
They describe the state and distribution of stresses in a solid object.
%

%
Tensor fields have more degrees of freedom than vector fields and are therefore
more challenging to visualize.
%
Almost all techniques for tensor field visualization have in common that they
represent tensors using their eigenvectors and eigenvalues.
%
Visualization techniques for vector fields can sometimes be applied to the
eigenvectors of a tensor field with some modifications.
%
Just as they help make sense of the structure of vector fields, they can make
sense of the structure of eigenvectors in tensor fields.
%

%
We make steps to translate a class of features from vector- to tensor fields
that has not been considered to date:
%
line features obtained by the \ac{PV} operator.
%
This operator yields all locations where two vector fields are parallel.
%
It can be used to compute ridge- and valley lines in scalar fields as well as
separation-, attachment-, and vortex core lines in vector fields.
%
We establish the \ac{PEV} operator as the equivalent of the \ac{PV} operator
for vector fields.
%
We then use it to translate the concept of vortex core lines to tensor fields by
defining \emph{tensor core lines} that mark the centers of ``swirling'' behavior
of the eigenvectors.
%
Using these new features, we demonstrate how to find locations of aligned
principal stress directions in objects under two different loads and how to find
areas in an object that are under twist.
%
% subsection contr_tensor_fields (end)
%
% section contributions (end)
%
\section{Thesis Structure} % (fold)
\label{sec:thesis_structure}
%
This thesis is separated into three main parts.
%
\Cref{part:background} provides background information that is relevant to
understand the context of this work.
%
\begin{itemize}
    \item \Cref{cha:sci_vis} gives an overview of the field of scientific
    visualization.
    %
    \item \Cref{cha:turbulent_combustion} introduces the field of turbulent
    combustion research that is relevant to the second part of this thesis.
    %
    It gives a phenomenological overview of combustion processes, outlines the
    methods involved in its modeling and simulation and discusses the state of
    the art in visualization of turbulent combustion.
\end{itemize}
%
%
\Cref{part:flame_vis} presents the contributions in visualization of the flame
surface in \ac{DNS} of turbulent combustion.
%
\begin{itemize}
    \item \Cref{cha:sparse_representation} presents a sparse representation
    for premixed flames that saves storage space and is the basis for a new
    flame visualization technique.
    %
    \item \Cref{cha:flame_surface_tracking} introduces an in-situ algorithm for
    tracking the flame surface during a massively parallel simulation run.
    %
    \item \Cref{cha:flame_vis_conclusions} concludes this part of the thesis.
\end{itemize}
%
%
\Cref{part:tensor_vis} presents new line features for second-order tensor
fields.
%
\begin{itemize}
    \item \Cref{cha:parallel_eigenvectors} introduces the \ac{PEV} operator,
    which yields all locations where two tensor fields have parallel real
    eigenvectors.
    %
    \item \Cref{cha:tensor_core_lines} applies the \ac{PEV} operator -- with
    some modifications -- to translate the concept of vortex core lines to the
    eigenvectors of a tensor field.
    %
    \item \Cref{cha:tensor_vis_conclusions} concludes the last part of the
    thesis.
\end{itemize}
%
% section thesis_structure (end)
%
\clearpage
\section{List of Publications} % (fold)
\label{sec:list_of_publications}
%
The following articles have been published in peer-reviewed journals and
conferences as results of this thesis:
%
% \begin{itemize}
%     \item \fullcite{Oster2014}
%     \item \fullcite{Abdelsamie2016}
%     \item \fullcite{Oster2018a}
%     \item \fullcite{Oster2018}
%     \item \fullcite{Oster2018b}
% \end{itemize}
%

%
\begin{itemize}[label={},leftmargin=0pt]
    \item T.~Oster, D.~J.~Lehmann, G.~Fru, H.~Theisel, and D.~Th\'evenin\\
        \textbf{Sparse Representation and Visualization for Direct Numerical
        Simulation Of Premixed Combustion}\\
        {\emph{Computer Graphics Forum} 33.3, pp. 321--330, 2014}

    \item A.~Abdelsamie, G.~Fru, T.~Oster, F.~Dietzsch, G.~Janiga,
        and D.~Thévenin\\
        \textbf{Towards Direct Numerical Simulations of Low-Mach Number
        Turbulent Reacting and Two-Phase Flows Using Immersed Boundaries}\\
        {\emph{Computers \& Fluids} 131, pp. 123--141, 2016}

    \item C.~Chi, A.~Abdelsamie, T.~Oster, and D.~Thévenin\\
        \textbf{Probability of Hotspot Ignition and Ignition Spot Tracking in
        Turbulent Hydrogen-Air Mixtures Using Direct Numerical Simulations}\\
        {\emph{\nth{8} European Combustion Meeting}, pp. 925--930, 2017}

    \item T.~Oster, A.~Abdelsamie, M.~Motejat, T.~Gerrits, C.~R\"ossl,
        D.~Thévenin, and H.~Theisel\\
        \textbf{On-The-Fly Tracking of Flame Surfaces for the Visual Analysis
        of Combustion Processes}\\
        {\emph{Computer Graphics Forum} 37.6, pp. 358--369, 2018}

    \item T.~Oster, C.~R\"ossl, and H.~Theisel\\
        \textbf{Core Lines in \ac{3D} Second-Order Tensor Fields}\\
        {\emph{Computer Graphics Forum} 37.3, pp. 327--337, 2018}

    \item T.~Oster, C.~R\"ossl, and H.~Theisel\\
        \textbf{The Parallel Eigenvectors Operator}\\
        {\emph{Vision, Modeling and Visualization}, 2018}
\end{itemize}
\clearpage
%
% section list_of_publications (end)

%
\section{Notation} % (fold)
\label{sec:notation}
%
We will use the following mathematical notation throughout the thesis.
%
Additional notation is introduced wherever it is needed.
%

\vspace*{\baselineskip}
%
\begin{tabularx}{\textwidth}{lX}
$s$, $a$, $\lambda$ & Scalars \\
$\vv$, $\vr$, $\vx$ & Column vectors \\
$v_i$, $r_i$, $x_i$ & Components of the respective vectors \\
$\vNull$ & The vector of all zeros \\
$\mA$, $\mS$, $\mSigma$ & Matrices \\
$\mI$ & The identity matrix \\
$\T{\mA}$, $\T{\vv}$ & Transpose of a matrix/vector \\
$\begin{pmatrix}\va & \vb & \vc\end{pmatrix}$ & Block matrix composed of several
    matrices/vectors \\
$\vv \cdot \vw$ & Inner (dot-) product of two vectors \\
$\vv \times \vw$ & Outer (cross-) product of two vectors \\
$\nabla$ & Nabla-Operator
    $\T{\left({\partial}/{\partial x_1}, \dots, {\partial}/{\partial x_n}\right)}$
\end{tabularx}
%
% section notation (end)
%
% chapter introduction (end)