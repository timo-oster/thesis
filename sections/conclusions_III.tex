\chapter{Conclusion} % (fold)
\label{cha:tensor_vis_conclusions}
%
\lettrine[lhang=0.06, loversize=-0.015, findent=-1pt]{I}{n the previous} two
chapters, we extended the idea of a class of feature-based visualization
techniques for vector- and scalar fields to the realm of tensor fields.
%
In \cref{cha:parallel_eigenvectors}, we established the \ac{PEV} operator as a
direct extension of the generic \ac{PV} operator.
%
It finds all locations where two tensor fields have parallel real eigenvectors.
%
Using this, we translated the concept of vortex core lines to their counterpart
in tensor fields in \cref{cha:tensor_core_lines}.
%
These tensor core lines mark the centers of twisting behavior of tensor field
lines.
%
Feature lines of this type can be extracted from piecewise linear data by
determining their intersections with the boundaries of tetrahedral cells.
%
The search for such intersections is a search for roots of higher-order
polynomials, which we solve using a recursive subdivision algorithm based on
their Bernstein-B\'ezier form.
%
Intersections we find in this way are then connected to lines afterwards.
%

%
The work presented in this part of the thesis is basic research into
higher-order features in tensor fields.
%
As an application area, we focus on the visualization of stress tensor fields
from structural mechanics simulations.
%
In-detail analysis of the topology and structure of stress tensor fields still
is not well established within the structural mechanics community.
%
One reason for this might be that tensor fields are even more complex than
vector fields, which are already challenging to visualize and understand.
%
The \ac{PEV} operator and tensor core lines are additions to the visualization
toolbox that help to better understand the complex ways in which forces act in
solid materials.
%
Such feature-based techniques break down complex behavior into simple geometric
primitives that are more easily parsed and understood.
%
Maybe the development of more such techniques can help to better establish
tensor field visualization with structural mechanics researchers.
%
% chapter tensor_vis_conclusions (end)