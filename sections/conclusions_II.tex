\chapter{Conclusion} % (fold)
\label{cha:flame_vis_conclusions}
%
\lettrine[loversize=-0.015, findent=-1pt]{W}{e have presented} two approaches
for the visualization and analysis of different aspects of the flame front in
turbulent combustion \ac{DNS}.
%
Both approaches are designed to deal with the huge amount of raw data that has
been become the bottleneck of large-scale simulations.
%

%
The sparse representation for premixed flames presented in
\cref{cha:sparse_representation} focuses on the analysis of flamelets, \ie, the
behavior orthogonal to the surface.
%
We sample the profiles of simulation variables at many locations distributed
over the surface and approximate them with simple models to get a space-saving
representation of the flame.
%
This representation can directly be used for flamelet-related analysis and
visualization, or the full scalar fields can be reconstructed for regular
post-processing.
%

%
The flame surface tracking algorithm presented in
\cref{cha:flame_surface_tracking} focuses on the tangential behavior of the
flame.
%
We track the surface using independent micro-patches that refine and coarsen
without using any neighbor information.
%
This gives us a complete picture of the behavior of the surface over time,
particularly about the history and relative movement of points attached to the
surface.
%
This information is crucial for understanding and modeling the unsteady behavior
of the flame.
%

%
Both approaches are contributions towards a visualization/analysis toolbox for
in-depth quantitative analysis of \ac{DNS} for the purpose of combustion
modeling.
%
In contrast to the many important contributions towards fast and effective
general-purpose visualization for large-scale visualizations that have been
developed in recent years, these approaches represent a more targeted class
of visualization tools that may be afforded more computational resources and
time in order to answer specific research questions.
%
More work is necessary to continue bridging the gap between the two and
develop a range of tools from general to specific that support the full analysis
process of combustion researchers.
%
An important area of research in this regard are visualization techniques that
support the analysis of unsteady behavior, such as transport and mixing.
%
The flame surface tracking algorithm we propose belongs to this category.
%
Compared to the numerous methods for visualizing single snapshots of the data,
these techniques are more challenging technically and conceptually.
%
As the frameworks for in-situ processing are improving and taking care of some
of the technical challenges, we will hopefully see increased activity in this
area.
%
% chapter flame_vis_conclusions (end)