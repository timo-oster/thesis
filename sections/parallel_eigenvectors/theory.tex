
%
\section{Theoretical Considerations} % (fold)
\label{sec:pev_theory}
% 
In this section we study which structurally stable structures the \ac{PEV} operator
yields.
%
Structurally stable here means stable in the presence of noise.
%
From the knowledge of the \ac{PV} operator for vector fields, one would expect curves
as \ac{PEV} solutions.
%
The case, however, is slightly more complicated because eigenvectors can
transition from real to imaginary, and they are not uniquely defined in
isotropic regions.
%
Even under consideration of these cases, we can formulate the main theorem
%
\begin{restatable}{theorem}{pevstablelines}
    \label{thm:pev_stable_lines}%
    The \ac{PEV} operator yields structurally stable curves that are either closed
    or end at the boundaries of the domain.
\end{restatable}
%
%\textcolor{red}{
%
We provide the proof for this theorem in
\cref{cha:proof_pev_stable_lines}.\Todo{Incorporate? Proof is from Holger. How
to handle?}
%
%}
%

\paragraph*{Bifurcation points}\Todo{Don't use named paragraphs like this. Transform into theorem?}
%
We define bifurcation points as locations where two or more \ac{PEV} lines intersect,
\ie, locations with more than one pair of parallel eigenvectors.
%
If $\mS, \mT$ are general second order tensors, bifurcation points are
structurally unstable, i.e., they disappear under small perturbations of $\mS,
\mT$.
%
To show this, we consider a \ac{PEV} line $l$ and observe the other eigenvectors (the
ones that do not define $l$) along its path.
%
Since they are not constrained by each other, more than one condition must be
fulfilled along $l$ for the other eigenvectors to become parallel.
%
This can be interpreted as having at least two independent scalar values that
must vanish at the same point along $l$.
%
If this happens, adding noise will split up the points on $l$ of common zero
crossings.
%

%
This situation is different if $\mS, \mT$ are symmetric.
%
In this case, there are structurally stable bifurcations points where all three
\ac{PEV} lines intersect, i.e., where $\mS, \mT$ have three pairs of parallel
eigenvectors.
%
This can be shown as follows: If $\mS, \mT$ have two pairs of parallel
eigenvectors, the third pair must be parallel as well, due to the orthogonality
of the eigenvectors.
%
Further, we consider again a \ac{PEV} line $l$ that is defined by the
vector $\ve$ along $l$ that is eigenvector of both $\mS$ and $\mT$.
%
Then every other eigenvector of $\mS$ and $\mT$ is perpendicular to $\ve$ and
can therefore be expressed by one number: the rotation angle around $\ve$.
%
This way, the conditions of further pairs of common eigenvectors can be
described as the roots of one scalar function.
%
This is structurally stable: adding noise will slightly change the location of
$l$ and slightly change the location of zero crossings on $l$, but does not make
them disappear.
%
Some \ac{PEV} lines with bifurcation points can be seen in~\cref{fig:rand_lines}.
%