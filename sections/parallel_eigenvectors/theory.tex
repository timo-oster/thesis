\section{Theoretical Considerations} % (fold)
\label{sec:pev_theory}
%
Having defined the \ac{PEV} operator in \cref{eq:pev_operator}, we use this
section to study its properties.
%
We show that like the \ac{PV} operator, the \ac{PEV} operator yields
structurally stable lines, \ie, they do not disappear when adding noise.
%
However, unlike the \ac{PV} operator, multiple \ac{PEV} lines may stably
intersect in a single point if the two tensor fields are symmetric.
% 

%
Given the similarity to the \ac{PV} operator for vector fields one would
already expect curves as \ac{PEV} solutions.
%
The case, however, is slightly more complicated because eigenvectors can
transition from real to imaginary, and they are not uniquely defined in
isotropic regions.
%
Even considering these cases we can formulate the main theorem
%
\begin{restatable}{theorem}{pevstablelines}
    \label{thm:pev_stable_lines}%
    The \ac{PEV} operator yields structurally stable curves that are either
    closed or end at the boundaries of the domain.
\end{restatable}
%
The proof for this theorem, which was provided by Holger Theisel, can be found
in \cref{cha:proof_pev_stable_lines}.
%

%
We now study the possibility of \ac{PEV} lines intersecting in a single point.
%
We call such points, where more than one pair of eigenvectors is parallel,
\emph{bifurcation points}.
%
\begin{theorem}\label{pev_bifurcation_unstable}
    For general (asymmetric) tensor fields, bifurcation points are structurally
    unstable, \ie, they disappear under small perturbations of the tensor
    fields.
\end{theorem}
%
To show this, we consider a \ac{PEV} line $l$ and observe the other eigenvectors
(the ones that do not define $l$) along its path.
%
Since they are not constrained by each other, more than one condition must be
fulfilled along $l$ for the other eigenvectors to become parallel.
%
This can be interpreted as having at least two independent scalar values that
must vanish at the same point along $l$.
%
If this happens, adding noise will split up the points on $l$ of common zero
crossings and the bifurcation point will disappear.
%

%
This situation is different if the tensor fields are symmetric.
%
\begin{theorem}\label{pev_bifurcation_stable}
    For symmetric tensor fields, structurally stable bifurcation points exist
    where both fields have three pairs of parallel eigenvectors.
\end{theorem}
%
This can be shown as follows:
%
If two symmetric tensor fields $\mS$, $\mT$ have two pairs of parallel
eigenvectors, the third pair must be parallel as well, due to the orthogonality
of the eigenvectors.
%
Further, we consider again a \ac{PEV} line $l$ that is defined by the
vector $\ve$ along $l$ that is eigenvector of both $\mS$ and $\mT$.
%
All other eigenvectors of $\mS$ and $\mT$ are perpendicular to $\ve$ and can
therefore be expressed by one number: the rotation angle around $\ve$.
%
The conditions of further pairs of common eigenvectors can then be described as
the roots of one scalar function: the difference in rotation angles.
%
Adding noise will slightly change the location of $l$ and slightly change the
location of zero crossings on $l$, but does not make them disappear.
%
Consequently, bifurcation points in symmetric tensor fields are structurally
stable.
%

%
\Cref{fig:rand_lines} shows some examples of \ac{PEV} lines in random linear
tensor fields.
%
The examples for symmetric tensor fields at the bottom show clear examples of
bifurcation points.
%