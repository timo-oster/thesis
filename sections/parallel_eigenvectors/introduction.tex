\vspace*{-2\baselineskip}\lettrine[lhang=0.05, loversize=0.047,
findent=-0.9pt]{F}{eature extraction} is one of the most successful types of
techniques for scientific visualization.
%
In this context, a feature is a geometric structure where the data fulfills
certain interesting criteria.
%
\Cref{cha:sci_vis} introduced several generic features, such as ridges in scalar
fields, critical points and vortex core lines in vector fields, and degenerate
structures in tensor fields.
%
Extracting and representing such features is an effective approach to
understanding even complex scientific datasets.
%

%
A number of line-type features for scalar- and vector fields can be expressed in
terms of a common operation: the \ac{PV} operator~\cite{Peikert1999}.
%
This operator yields all locations where two vector fields defined on the same
domain are parallel.
%
It delivers structurally stable lines which we call \ac{PV} lines.
%
Depending on the concrete vector fields it is applied to, the \ac{PV} operator
can be used to extract ridge and valley lines, extremum lines, vortex core
lines and separation- and attachment lines from scalar or vector data.
%

%
Some of these features are originally defined as the locations where a vector
is parallel to an eigenvector of a tensor field.
%
Such cases can be broken down to an application of the regular \ac{PV} operator.
%
However, this is not possible if a feature is defined by the locations where
two tensor fields have parallel eigenvectors.
%

%
In this chapter, we extend the concept of the \ac{PV} operator to tensor fields.
%
We define the \ac{PEV} operator on two (not necessarily symmetric) \ac{3D}
second-order tensor fields.
%
Let $\mS(\vx)$ and $\mT(\vx)$ be two such tensor fields.
%
Then the \ac{PEV} operator yields all locations $\vx$ where $\mS$ and $\mT$ have
parallel real eigenvectors.
%
This can be concisely expressed as
%
\begin{equation}\label{eq:pev_operator}
    \operatorname{PEV}(\mS,\mT) = \{ \vx \;|\; \exists \;\ve,\,
        \ve \parallel \mS(\vx)\,\ve \parallel \mT(\vx)\,\ve \}\,\text{.}
\end{equation}
%

%
In this chapter, we establish this operator by \ldots
%
\begin{itemize}
    \item
    \ldots{}\,studying its properties.
    %
    In particular, we show that the \ac{PEV} operator produces structurally
    stable line structures.
    %
    \item
    \ldots{}\,presenting a numerical algorithm to extract \ac{PEV} lines in
    piecewise linear tensor fields.
    %
    The main idea is to do a recursive search not only in \ac{3D} space but
    simultaneously in \ac{3D} space and the space of all possible eigenvectors.
    %
    \item
    \ldots{}\,applying it to compare pairs of stress tensor fields defined on
    the same domain.
    %
    % In this way, we are able to find locations where two
    % different loads produce principal stresses aligned in the same direction.
    %
\end{itemize}
%
At first glance, the extension of the \ac{PV} operator to eigenvectors seems
trivial:
%
given a tensor field, consider all eigenvector fields as vector fields and apply
the \ac{PV} operator to them.
%
However, this naive approach cannot give well-defined and stable results for
the following reasons:
%
\begin{itemize}
    \item Undefined length and orientation of eigenvectors:\\
    %
    Eigenvectors of a matrix are not unique but span linear subspaces.
    %
    To express the field of eigenvectors as a vector field, heuristic
    choices about the length and orientation of the vectors are necessary.
    %
    Applying such choices globally can not always give results that are free of
    discontinuities
    %
    \item Existence of multiple eigenvectors:\\
    %
    Regions with three real eigenvectors require a decision on which of them to
    use for the \ac{PV} operator -- a decision that is particularly non-unique
    in near-isotropic regions
    %
    (\ie{}, where the difference between two real eigenvalues is small).
    %
    \item Discontinuities in eigenvectors:\\
    %
    A small change of a tensor does not necessarily result in a small change of
    the eigenvectors.
    %
    In fact, in near-isotropic regions, a small change of the tensor may result
    in a large change of the eigenvector.
    %
    Moreover, in regions of transition between real and imaginary eigenvalues
    (\ie{}, in neighborhoods containing both tensors with all real eigenvalues and
    tensors with complex eigenvalues), a small change of the tensor can result
    in a sudden appearance or disappearance of real eigenvectors.
    %
\end{itemize}
%
All of these problems show that eigenvector fields are fundamentally different
from vector fields, for which the \ac{PV} operator is designed.
%
Extracting \ac{PEV} lines requires new algorithms that are explicitly designed
for tensor fields.
%

%
In the following, we first give an overview of related work.
%
We then give some theoretical properties of the \ac{PEV} operator in
\cref{sec:pev_theory}, before detailing our algorithm for finding \ac{PEV} lines
in piecewise linear tensor fields in \cref{sec:extracting_pev_lines}.
%
In \cref{sec:pev_results}, we apply our algorithm to mechanical stress tensor
data.
%
We close with a discussion and future work in \cref{sec:pev_discussion} and
\cref{sec:pev_limitations}.
%
\begin{figure}[t]
    \centering
    \begin{tikzpicture}[
        every node/.style={inner sep=0, outer sep=0,
            node distance=2mm and 4mm},
        label/.style={font=\small, anchor=north, rotate=90, yshift=8mm}
    ]
        \node (img1gen) {
            \includegraphics[width=0.33\textwidth-8mm*2/3,trim=90 0 55 0,clip]%
                {figures/Rand_general_1.png}
        };
        \node[label] at (img1gen.west) {general tensors};
        \node[right=of img1gen] (img2gen) {
            \includegraphics[width=0.33\textwidth-8mm*2/3,trim=90 0 55 0,clip]%
                {figures/Rand_general_3.png}
        };
        \node[right=of img2gen] (img3gen) {
            \includegraphics[width=0.33\textwidth-8mm*2/3,trim=90 0 55 0,clip]%
                {figures/Rand_general_4.png}
        };

        \node[below=of img1gen] (img1symm) {
            \includegraphics[width=0.33\textwidth-8mm*2/3,trim=90 0 55 0,clip]%
                {figures/Rand_symm_2.png}
        };
        \node[label] at (img1symm.west) {symmetric tensors};
        \node[right=of img1symm] (img2symm) {
            \includegraphics[width=0.33\textwidth-8mm*2/3,trim=90 0 55 0,clip]%
                {figures/Rand_symm_3.png}
        };
        \node[right=of img2symm] (img3symm) {
            \includegraphics[width=0.33\textwidth-8mm*2/3,trim=90 0 55 0,clip]%
                {figures/Rand_symm_4.png}
        };
    \end{tikzpicture}
    \caption{\ac{PEV} lines in pairs of random linear tensor fields.}
    \label{fig:rand_lines}
\end{figure}
%
%
% section introduction (end)