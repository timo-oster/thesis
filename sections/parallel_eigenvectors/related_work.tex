

\section{Related Work} % (fold)
\label{sec:pev_related_work}
%
The \ac{PEV} operator is a generalization of tensor core lines.
%
It is related to the \ac{PV} operator, by which it was motivated, as well as tensor
field visualization in general.
%
In this section, we give a more detailed explanation of tensor core lines, as
well as an overview of relevant previous work.
%
%\textcolor{red}{
\subsection*{Tensor Core Lines} % (fold)
\label{sub:tensor_core_lines}
%
Tensor core lines were introduced by Oster \etal\cite{Oster2018} as an
equivalent to vortex core lines for tensor fields.
%
Similar to the Sujudi/Haimes criterion for vortex core lines~\cite{Sujudi1995},
they define tensor core lines as the locations where eigenvector trajectories
have locally vanishing curvature.
%
For a tensor field $\mT$ and a vector $\vr$, a tensor core line passes
through location $\vx$ if
%
\begin{equation*}
    \vr \parallel \mT(\vx)\,\vr \parallel \nabla_{\vr}\mT(\vx)\,\vr\,\text{.}
\end{equation*}
%
To understand this criterion, it is important to note that $\vr \parallel
\mT\,\vr$ implies that $\vr$ is an eigenvector of $\mT$.
%
This is equivalent to the more commonly used formulation $\mT\vr = \lambda\vr$.
%
The tensor core line criterion consequentially states that $\vr$ is an
eigenvector of $\mT(\vx)$, and that $\vr$ is also an eigenvector of
$\nabla_{\vr}\mT(\vx)$, the directional derivative of $\mT$ along $\vr$.
%
In other words, $\mT(\vx)$ and $\nabla_{\vr}\mT(\vx)$ have a parallel
eigenvector $\vr$.
%
In this work, we relax this criterion and find locations where any two tensor
fields have parallel eigenvectors.
%
% subsection tensor_core_lines (end)
%}
%
\subsection*{Parallel Vectors} % (fold)
\label{sub:pev_parallel_vectors}
%
The parallel vectors operator was introduced by Peikert and Roth in 1999
\cite{Peikert1999} as a generalization of a concept that had been used with
slight variations in a lot of different contexts.
%
Among these are ridge detection in scalar fields~\cite{Haralick1983},
extraction of attachment/separation lines in flows~\cite{Kenwright1999},
and the identification of vortex core lines~\cite{Sujudi1995,Banks1995}.
%
% Haralick (1983) ridge detection,
% Sujudi-Haimes (1995),
% (Banks 1995) "A Predictor-Corrector Technique for Visualizing Unstready Flow"
% Kenwright (1999) attachment/separation lines
%

%
In his PhD thesis, Martin Roth~\cite{Roth2000} gives an overview of several
numerical algorithms for the parallel vectors operator.
%
Most of them are based on first finding intersections of \ac{PV} lines with the
cell faces of a dataset.
%
The resulting intersection points are then connected to lines using different
kinds of heuristics.
%

%
An alternative approach is to trace parallel vector lines starting from
a seed point.
%
Algorithms using this general approach have been proposed by Banks and
Singer~\cite{Banks1995}, Miura and Kida~\cite{Miura1997}, Sukharev et
al.~\cite{Sukharev2006} and Theisel et al.~\cite{Theisel2003a}.
%
Methods for avoiding the accumulation of errors when tracing \ac{PV} lines were
introduced by van Gelder and Pang~\cite{Gelder2009}, as well as Weinkauf et
al.~\cite{Weinkauf2011}.
%

%
While most \ac{PV} algorithms operate on piecewise linear data that is not
time-dependent, there are some publications that deal with higher-order data
or use higher-order methods.
%
This includes approaches for finding curved vortex core lines~\cite{Roth1998},
scale-space techniques~\cite{Bauer2002}, and computing the \ac{PV} operator on
time-dependent~\cite{Theisel2005,Fuchs2007} or piecewise analytic vector
fields~\cite{Pagot2011}.
%
% subsection parallel_eigenvectors (end)

\subsection*{Tensor Field Visualization} % (fold)
\label{sub:pev_tensor_field_visualization}
%
Tensor fields are generally visualized by Glyph-based, Integration-based or
Topology-based methods.
%

%
Tensor glyphs show the properties of a single tensor as a geometric object.
%
These glyphs vary in complexity depending on the properties of the tensor
that is visualized.
%
Glyphs have been designed for symmetric positive definite
tensors~\cite{Kindlmann2004}, indefinite symmetric
tensors~\cite{Schultz2010a}, and general tensors in \ac{2D} and
\ac{3D}~\cite{Gerrits2017}.
%
Specialized glyphs for structural mechanics applications~\cite{Hashash2003}
as well as comparative visualization of medical diffusion tensor
fields~\cite{Zhang2016} have also been proposed.
%
To visualize a tensor field using glyphs, they are usually placed on the grid
nodes of the dataset or on a regular grid superimposed on the data.
%
Kindlmann and Westin~\cite{Kindlmann2006} proposed a more sophisticated glyph
placement strategy that avoids occlusion.
%

%
Integration-based methods create geometric structures by following the field of
eigenvector directions starting from a seed structure.
%
The simplest example of this class is the hyperstreamline~\cite{Delmarcelle1993}.
%
It is commonly implemented as a tube following an eigenvector corresponding
to some ordered eigenvalue.
%
The cross-section of the tube is scaled and rotated according to the other
eigenvalues.
%
Weinstein \etal~\cite{Weinstein1999} introduced a similar concept that is more
stable in the vicinity of isotropic points.
%
An extension of hyperstreamlines are hyperstreamsurfaces~\cite{Jeremic2002},
which use a line instead of a point as the seed structure.
%

%
Topology-based methods focus on extracting a topological skeleton capturing
critical structures in tensor fields.
%
Degenerate structures where two or more eigenvectors are equal, were first
described for symmetric tensor fields by Delmarcelle~\cite{Delmarcelle1994} and
Hesselink~\cite{Hesselink1997}.
%
Zheng and Pang~\cite{Zheng2004,Zheng2005} introduced numerical algorithms for
extracting such structures.
%
A more stable approach for noisy data was proposed by Tricoche
\etal~\cite{Tricoche2008}
%
The topology of asymmetric tensor fields has also been
studied~\cite{Zheng2005a,Zhang2009}.
%
Recently, surfaces of neutral and traceless tensors were added to the
topological features of symmetric tensor fields~\cite{Palacios2016}.
%

% subsection tensor_field_visualization (end)
%
% section related_work (end)