\section{Related Work} % (fold)
\label{sec:pev_related_work}
%
The \ac{PEV} operator is related to the \ac{PV} operator as well as tensor field
visualization in general.
%
We have already given an overview of tensor field visualization in
\cref{sec:tensor_fields}, so we will focus on literature regarding the \ac{PV}
operator here.
%

%
The \ac{PV} operator was introduced by Peikert and Roth~\cite{Peikert1999} as a
generalization of a concept that had been used with slight variations in a lot
of different contexts.
%
Among these are ridge detection in scalar fields~\cite{Haralick1983},
extraction of attachment/separation lines in flows~\cite{Kenwright1999},
and the identification of vortex core lines~\cite{Sujudi1995,Banks1995}.
%

%
In his PhD thesis, Martin Roth~\cite{Roth2000} gives an overview of several
numerical algorithms for the \ac{PV} operator.
%
Most of them are based on first finding intersections of \ac{PV} lines with the
surface of cells of a dataset.
%
The resulting intersection points are then connected to lines using different
kinds of heuristics.
%

%
An alternative approach is to trace \ac{PV} lines starting from a seed point.
%
Algorithms using this general approach have been proposed by Banks and
Singer~\cite{Banks1995}, Miura and Kida~\cite{Miura1997}, Sukharev
\etal~\cite{Sukharev2006} and Theisel \etal~\cite{Theisel2003a}.
%
Methods for avoiding the accumulation of errors when tracing \ac{PV} lines were
introduced by van Gelder and Pang~\cite{Gelder2009}, as well as Weinkauf
\etal~\cite{Weinkauf2011}.
%

%
While most \ac{PV} algorithms operate on piecewise linear data that is not
time-dependent, there are some publications that deal with higher-order data
or use higher-order methods.
%
This includes approaches for finding curved vortex core lines~\cite{Roth1998},
scale-space techniques~\cite{Bauer2002}, and computing the \ac{PV} operator on
time-dependent~\cite{Theisel2005,Fuchs2007} or piecewise analytic vector
fields~\cite{Pagot2011}.
%
Recently, Gerrits \etal{}~\cite{Gerrits2018} proposed an approximate parallel
vectors operator for ensembles of more than two vector fields.
%
The \ac{PEV} operator we introduce here deals with higher-order data of a
different kind: It operates on tensor instead of vector data.
%
% section related_work (end)