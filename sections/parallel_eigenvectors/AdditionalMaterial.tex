\chapter[Proof that PEV Yields Structurally Stable Curves]
    {Proof that the Parallel Eigenvectors Operator Yields Structurally Stable
    Curves} % (fold)
\label{cha:proof_pev_stable_lines}

%
\lettrine[lhang=0.06, loversize=-0.015, findent=-1pt]{I}{n
\cref{sec:pev_theory}}, we formulated \cref{thm:pev_stable_lines}, which states
that the \ac{PEV} operator yields structurally stable curves that are either
closed or end at the domain boundary.
%
We give the proof for this theorem here.
%
This proof was derived and written by Holger Theisel.
%

%
The main idea to prove \cref{thm:pev_stable_lines} is to search for \ac{PEV}
lines not in \ac{3D} $(x,y,z)$ space but in a \ac{6D} $(x,y,z,u,v,w)$ space:
%
at every point $\vx=\T{(x,y,z)}$, all vector directions $\vr=\T{(u,v,w)}$ are
checked for being an eigenvector of $\mS$ and $\mT$.
%
This means that we search for all \ac{6D} points $\T{(\vx,\vr)}$ fulfilling
$\mS(\vx)\, \vr \times \vr = \vNull$ and $\mT(\vx)\, \vr \times \vr = \vNull$.
%
We formulate this to search for all \ac{6D} points $\T{(\vx,\vr)}$ where a
\ac{6D} vector field $\overline{\vh}$ vanishes:
%
\begin{equation}
    \label{eq:theory_1}
    \overline{\vh}(\vx,\vr) =
        \begin{pmatrix}
            \mS(\vx) \, \vr \times \vr \\
            \mT(\vx) \, \vr \times \vr
        \end{pmatrix}
    = \overline{\vNull}.
\end{equation}
%
Suppose a point $\T{(\vx_0,\vr_0)}$ is on a \ac{PEV} structure, \ie, fulfills
\cref{eq:theory_1}.
%
In order to study the \ac{PEV} structures in a linear neighborhood of
$\T{(\vx_0,\vr_0)}$, we search for all directions $\T{(d \vx,d \vr)}$ in which
$\overline{\vh}$ remains zero: $\nabla \overline{\vh} \cdot \T{(d \vx,d \vr)} =
\overline{\vNull}$.
%
In other words: we have to explore the null space of $\nabla \overline{\vh}$.
%
Applying elementary differentiation rules gives
%
\begin{equation} \label{eq:theory_2}
    \nabla \overline{\vh} =
        \begin{pmatrix}
            \mG_1 &   \mG_3 \\
            \mG_2 &   \mG_4
        \end{pmatrix}
        \,\text{,}
\end{equation}
%
with
%
\begin{equation} \label{eq:theory_3}
\begin{aligned}
    \mG_1 & =
        \begin{pmatrix}
            \mS_x \, \vr \times \vr &
            \mS_y \, \vr \times \vr &
            \mS_z \, \vr \times \vr
        \end{pmatrix}
        \,\text{,}
    \\
    \mG_2 & =
        \begin{pmatrix}
            \mT_x \, \vr \times \vr  &
            \mT_y \, \vr \times \vr  &
            \mT_z \, \vr \times \vr
        \end{pmatrix}
        \,\text{,}
    \\
    \mG_3 & =
        \begin{pmatrix}
        \mS \, \vj_1 \times \vr +  \mS \, \vr \times \vj_1 &
        \mS \, \vj_2 \times \vr +  \mS \, \vr \times \vj_2 &
        \mS \, \vj_3 \times \vr +  \mS \, \vr \times \vj_3
        \end{pmatrix}
        \,\text{,}
    \\
    \mG_4 & =
        \begin{pmatrix}
        \mT \, \vj_1 \times \vr +  \mT \, \vr \times \vj_1 &
        \mT \, \vj_2 \times \vr +  \mT \, \vr \times \vj_2 &
        \mT \, \vj_3 \times \vr +  \mT \, \vr \times \vj_3
        \end{pmatrix}
    \, \text{.}
\end{aligned}
\end{equation}
%
Then
%
\begin{equation} \label{eq:theory_4}
   \T{\mG_1} \, \vr \; =\;  \T{\mG_2} \, \vr\; =\; 0 \,\text{,}
\end{equation}
%
and from \cref{eq:theory_1} follows
%
\begin{equation} \label{eq:theory_5}
   \T{\mG_3} \, \vr \;=\; \T{\mG_4} \, \vr \;=\;  0 \, \text{,}
\end{equation}
and
\begin{equation} \label{eq:theory_5a}
   \mG_3 \, \vr \;=\; \mG_4 \, \vr \;=\; 0 \, \text{.}
\end{equation}
%
\Cref{eq:theory_4,eq:theory_5} give that
%
\begin{equation} \label{eq:theory_6}
   \Rank{\nabla \overline{\vh}} = 4
\end{equation}
%
in the structurally stable case.
%
This means that for $\Rank{\nabla \overline{\vh}} < 4$, adding noise to $\mS,
\mT$ brings $\Rank{\nabla \overline{\vh}}$ to $4$.
%
\Cref{eq:theory_6} means that the \ac{PEV} structure around $\T{(\vx_0,\vr_0)}$
is a 2-manifold in \ac{6D}.
%
To see \cref{eq:theory_6}, we consider a rotation of the underlying coordinate
system such that $\vr=(0,0,r_z)$.
%
Then \cref{eq:theory_4,eq:theory_5} give that the rotated tensors
$\mG_1,\mG_2,\mG_3,\mG_4$ have vanishing third columns.
%
This and \cref{eq:theory_1} gives that $\nabla \overline{\vh}$ has two colums,
which proves \cref{eq:theory_6}.
%

%
One vector in the null space of $\nabla \overline{\vh}$ is trivial and denotes a
simple scaling of $\vr$:
%
\Cref{eq:theory_4,eq:theory_5,eq:theory_5a}
give $\nabla \overline{\vh}
\cdot \T{(\vNull, \vr)} = \overline{\vNull}$.
%
This means that the projection of the null space of $\nabla \overline{\vh}$ into
the spatial subspace $\vx$ gives a one-manifold in \ac{3D}.
%
This shows that \ac{PEV} gives line structures in \ac{3D}.
%
To show that they are closed, we consider the 6 components of $\nabla
\overline{\vh}$ as scalar fields and interpret the \ac{PEV} structure as
intersection of their \ac{5D} iso-hypersurfaces.
%
Iso-hypersurfaces are always closed, which means their intersections are also
closed.
%

%
Note that the proof did not make any assumptions on the behavior of $\mS, \mT$
around $\T{(\vx_0,\vr_0)}$.
%
This means that it holds also in case of a transition from real to imaginary
eigenvectors of $\mS$ or $\mT$ as well as in regions of isotropic tensors.
%