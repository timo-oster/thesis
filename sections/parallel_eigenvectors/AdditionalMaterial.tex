\documentclass{egpubl}
\usepackage{EGVMV18}
\usepackage[utf8]{inputenc}
\WsPaper
\electronicVersion % can be used both for the printed and electronic version

% !! *please* don't change anything above
% !! unless you REALLY know what you are doing
% ------------------------------------------------------------------------

% for including postscript figures
% mind: package option 'draft' will replace PS figure by a filname within a frame
\ifpdf \usepackage[pdftex]{graphicx} \pdfcompresslevel=9
\else \usepackage[dvips]{graphicx} \fi

\PrintedOrElectronic

% prepare for electronic version of your document
\usepackage{t1enc,dfadobe}

\usepackage{egweblnk}
\usepackage{cite}

\usepackage{cool}
\usepackage{amsfonts}
\usepackage{mathtools}
\mathtoolsset{showonlyrefs}
\usepackage{bm}

%%%%%%%%%%%%%%%%%%%%%%%%%%%%%%%%%%%%%%%%%%%%%%%%%%%%%%%%%%%%%
%% LaTeX math macros
%%%%%%%%%%%%%%%%%%%%%%%%%%%%%%%%%%%%%%%%%%%%%%%%%%%%%%%%%%%%%

\newcommand{\mySetNotation}[1]{{\mathbb{#1}}}
\newcommand{\RRSet}{{\mySetNotation{R}}}
\newcommand{\EESet}{{\mySetNotation{E}}}
\newcommand{\NNSet}{{\mySetNotation{N}}}
\newcommand{\QQSet}{{\mySetNotation{Q}}}
\newcommand{\ZZSet}{{\mySetNotation{Z}}}


\newcommand{\myBoldNotation}[1]{{\mathbf #1}}
%\newcommand{\myBoldNotation}[1]{{\mathbold{#1}}} %for euler math
\newcommand{\va}   {{\myBoldNotation{a}}}
\newcommand{\vb}   {{\myBoldNotation{b}}}
\newcommand{\vc}   {{\myBoldNotation{c}}}
\newcommand{\vd}   {{\myBoldNotation{d}}}
\newcommand{\ve}   {{\myBoldNotation{e}}}
\newcommand{\vf}   {{\myBoldNotation{f}}}
\newcommand{\vg}   {{\myBoldNotation{g}}}
\newcommand{\vh}   {{\myBoldNotation{h}}}
\newcommand{\vi}   {{\myBoldNotation{i}}}
\newcommand{\vj}   {{\myBoldNotation{j}}}
\newcommand{\vk}   {{\myBoldNotation{k}}}
\newcommand{\vl}   {{\myBoldNotation{l}}}
\newcommand{\vm}   {{\myBoldNotation{m}}}
\newcommand{\vn}   {{\myBoldNotation{n}}}
\newcommand{\nn}   {{\myBoldNotation{n}}}
\newcommand{\vo}   {{\myBoldNotation{o}}}
\newcommand{\vp}   {{\myBoldNotation{p}}}
\newcommand{\pp}   {{\myBoldNotation{p}}}
\newcommand{\vq}   {{\myBoldNotation{q}}}
\newcommand{\vr}   {{\myBoldNotation{r}}}
\newcommand{\vs}   {{\myBoldNotation{s}}}
\newcommand{\vt}   {{\myBoldNotation{t}}}
\newcommand{\vu}   {{\myBoldNotation{u}}}
\newcommand{\vv}   {{\myBoldNotation{v}}}
\newcommand{\vw}   {{\myBoldNotation{w}}}
\newcommand{\vx}   {{\myBoldNotation{x}}}
\newcommand{\vy}   {{\myBoldNotation{y}}}
\newcommand{\vz}   {{\myBoldNotation{z}}}
\newcommand{\vNull}{{\myBoldNotation{0}}}
\newcommand{\vOne} {{\myBoldNotation{1}}}

\newcommand{\mA}   {{\myBoldNotation{A}}}
\newcommand{\mB}   {{\myBoldNotation{B}}}
\newcommand{\mC}   {{\myBoldNotation{C}}}
\newcommand{\mD}   {{\myBoldNotation{D}}}
\newcommand{\mE}   {{\myBoldNotation{E}}}
\newcommand{\mF}   {{\myBoldNotation{F}}}
\newcommand{\mG}   {{\myBoldNotation{G}}}
\newcommand{\mH}   {{\myBoldNotation{H}}}
\newcommand{\mI}   {{\myBoldNotation{I}}}
\newcommand{\mJ}   {{\myBoldNotation{J}}}
\newcommand{\mK}   {{\myBoldNotation{K}}}
\newcommand{\mL}   {{\myBoldNotation{L}}}
\newcommand{\mM}   {{\myBoldNotation{M}}}
\newcommand{\mN}   {{\myBoldNotation{N}}}
\newcommand{\mO}   {{\myBoldNotation{O}}}
\newcommand{\mP}   {{\myBoldNotation{P}}}
\newcommand{\mQ}   {{\myBoldNotation{Q}}}
\newcommand{\mR}   {{\myBoldNotation{R}}}
\newcommand{\mS}   {{\myBoldNotation{S}}}
\newcommand{\mT}   {{\myBoldNotation{T}}}
\newcommand{\mU}   {{\myBoldNotation{U}}}
\newcommand{\mV}   {{\myBoldNotation{V}}}
\newcommand{\mW}   {{\myBoldNotation{W}}}
\newcommand{\mX}   {{\myBoldNotation{X}}}
\newcommand{\mY}   {{\myBoldNotation{Y}}}
\newcommand{\mZ}   {{\myBoldNotation{Z}}}
\newcommand{\mNull}{{\myBoldNotation{0}}}

\newcommand{\myBoldGreekNotationLower}[1]{{\bm{#1}}}
\newcommand{\myBoldGreekNotationUpper}[1]{{\bm{#1}}}
\newcommand{\balpha}  {{\myBoldGreekNotationLower{\alpha}  }}
\newcommand{\bbeta}   {{\myBoldGreekNotationLower{\beta}   }}
\newcommand{\bgamma}  {{\myBoldGreekNotationLower{\gamma}  }}
\newcommand{\bdelta}  {{\myBoldGreekNotationLower{\delta}  }}
\newcommand{\bepsilon}{{\myBoldGreekNotationLower{\epsilon}}}
\newcommand{\bzeta}   {{\myBoldGreekNotationLower{\zeta}   }}
\newcommand{\btheta}  {{\myBoldGreekNotationLower{\theta}  }}
\newcommand{\biota}   {{\myBoldGreekNotationLower{\iota}   }}
\newcommand{\bkappa}  {{\myBoldGreekNotationLower{\kappa}  }}
\newcommand{\blambda} {{\myBoldGreekNotationLower{\lambda} }}
\newcommand{\bmu}     {{\myBoldGreekNotationLower{\mu}     }}
\newcommand{\bnu}     {{\myBoldGreekNotationLower{\nu}     }}
\newcommand{\bxi}     {{\myBoldGreekNotationLower{\xi}     }}
\newcommand{\bomicron}{{\myBoldGreekNotationLower{\omicron}}}
\newcommand{\bpi}     {{\myBoldGreekNotationLower{\pi}     }}
\newcommand{\brho}    {{\myBoldGreekNotationLower{\rho}    }}
\newcommand{\bsigma}  {{\myBoldGreekNotationLower{\sigma}  }}
\newcommand{\btau}    {{\myBoldGreekNotationLower{\tau}    }}
\newcommand{\bupsilon}{{\myBoldGreekNotationLower{\upsilon}}}
\newcommand{\bphi}    {{\myBoldGreekNotationLower{\phi}    }}
\newcommand{\bchi}    {{\myBoldGreekNotationLower{\chi}    }}
\newcommand{\bpsi}    {{\myBoldGreekNotationLower{\psi}    }}
\newcommand{\bomega}  {{\myBoldGreekNotationLower{\omega}  }}

\newcommand{\mLambda} {{\myBoldNotation{\Lambda}}}
\newcommand{\mSigma} {{\myBoldNotation{\Sigma}}}

\newcommand{\bGamma}  {{\myBoldGreekNotationUpper{\Gamma}  }}
\newcommand{\bDelta}  {{\myBoldGreekNotationUpper{\Delta}  }}
\newcommand{\bTheta}  {{\myBoldGreekNotationUpper{\Theta}  }}
\newcommand{\bLambda} {{\myBoldGreekNotationUpper{\Lambda} }}
\newcommand{\bXi}     {{\myBoldGreekNotationUpper{\Xi}     }}
\newcommand{\bPi}     {{\myBoldGreekNotationUpper{\Pi}     }}
\newcommand{\bSigma}  {{\myBoldGreekNotationUpper{\Sigma}  }}
\newcommand{\bUpsilon}{{\myBoldGreekNotationUpper{\Upsilon}}}
\newcommand{\bPhi}    {{\myBoldGreekNotationUpper{\Phi}    }}
\newcommand{\bPsi}    {{\myBoldGreekNotationUpper{\Psi}    }}
\newcommand{\bOmega}  {{\myBoldGreekNotationUpper{\Omega}  }}

\newcommand{\inn}[2][0cm]{\mathopen{}\left|{#2}\parbox[h][#1]{0cm}{}\right|}
\newcommand{\invn}[2][0cm]{\mathopen{}\left|\left|{#2}\parbox[h][#1]{0cm}{}\right|\right|}

%\newcommand{\Sa}{{\mathbf {Sa}}}
%\newcommand{\Topo}{{\mathbf {Topo}}}
%\newcommand{\Sep}{{\mathbf {Sep}}}
%\newcommand{\Str}{{\mathbf {Str}}}
%\newcommand{\Reg}{{\mathbf {Reg}}}

%\newcommand{\PPhi}{{\pmb \Phi}}
%\newcommand{\PPi}{{\pmb \Pi}}
%\newcommand{\PPhi}{{\bm \Phi}}
%\newcommand{\PPi}{{\bm \Pi}}
\newcommand{\PPhi}{{\Phi}}
\newcommand{\PPi}{{\Pi}}
\newcommand{\ppi}{{\pi}}
\newcommand{\ggam}{{\gamma}}

\newcommand{\cA}{{\mathcal A}}
\newcommand{\cB}{{\mathcal B}}
\newcommand{\cC}{{\mathcal C}}
\newcommand{\cD}{{\mathcal D}}
\newcommand{\cE}{{\mathcal E}}
\newcommand{\cF}{{\mathcal F}}
\newcommand{\cG}{{\mathcal G}}
\newcommand{\cH}{{\mathcal H}}
\newcommand{\cI}{{\mathcal I}}
\newcommand{\cJ}{{\mathcal J}}
\newcommand{\cK}{{\mathcal K}}
\newcommand{\cL}{{\mathcal L}}
\newcommand{\cM}{{\mathcal M}}
\newcommand{\cN}{{\mathcal N}}
\newcommand{\cO}{{\mathcal O}}
\newcommand{\cP}{{\mathcal P}}
\newcommand{\cQ}{{\mathcal Q}}
\newcommand{\cR}{{\mathcal R}}
\newcommand{\cS}{{\mathcal S}}
\newcommand{\cT}{{\mathcal T}}
\newcommand{\cU}{{\mathcal U}}
\newcommand{\cV}{{\mathcal V}}
\newcommand{\cW}{{\mathcal W}}
\newcommand{\cX}{{\mathcal X}}
\newcommand{\cY}{{\mathcal Y}}
\newcommand{\cZ}{{\mathcal Z}}

\newcommand{\ca}{{\mathcal a}}
\newcommand{\cb}{{\mathcal b}}
\newcommand{\cc}{{\mathcal c}}
\newcommand{\cd}{{\mathcal d}}
% \newcommand{\ce}{{\mathcal e}}
%\newcommand{\cf}{{\mathcal f}}
\newcommand{\cg}{{\mathcal g}}
\newcommand{\ch}{{\mathcal h}}
\newcommand{\ci}{{\mathcal i}}
\newcommand{\cj}{{\mathcal j}}
%\newcommand{\ck}{{\mathcal k}}
\newcommand{\cl}{{\mathcal l}}
\newcommand{\cm}{{\mathcal m}}
\newcommand{\cn}{{\mathcal n}}
\newcommand{\co}{{\mathcal o}}
\newcommand{\cp}{{\mathcal p}}
\newcommand{\cq}{{\mathcal q}}
%\newcommand{\cr}{{\mathcal r}}
\newcommand{\cs}{{\mathcal s}}
\newcommand{\ct}{{\mathcal t}}
\newcommand{\cu}{{\mathcal u}}
\newcommand{\cv}{{\mathcal v}}
\newcommand{\cw}{{\mathcal w}}
\newcommand{\cx}{{\mathcal x}}
\newcommand{\cy}{{\mathcal y}}
\newcommand{\cz}{{\mathcal z}}

%cool package Styles
\Style{IdentityMatrixSymb=\mI,%
       DSymb={\mathrm d},%
       IntegrateDifferentialDSymb={\mathrm d},%
       TrParen=p,%
       DetParen=p}
\Style{DDisplayFunc=outset}
\renewcommand{\Transpose}[1]{ {#1}^\Transp }
\newcommand{\TransposeInv}[1]{ {#1}^{-\Transp} }
\newcommand{\T}[1]{\Transpose{#1}}
\newcommand{\TInv}[1]{\TransposeInv{#1}}
\newcommand{\Inv}[1]{ {#1}^{-1} }
\newcommand{\Transp}{{{\mathrm T}}}
\newcommand{\Norm}[1]{ \left\|{#1}\right\| }
\newcommand{\Frob}[1]{ \Norm{#1}_F }

\newcommand{\Rank}[1]    {\operatorname{rank}\inp{#1}}
\newcommand{\Vol}[1]     {\operatorname{vol}\inp{#1}}
\renewcommand{\Vec}[1]   {\operatorname{vec}\inp{#1}}
\renewcommand{\Div}[1]   {\T{\nabla} \, #1}
\newcommand{\Diag}[1]    {\operatorname{diag}\inp{#1}}
\newcommand{\Dim}[1]     {\operatorname{dim}\inp{#1}}
\newcommand{\Diagof}[1]  {\operatorname{diag}^{-1}\inp{#1}}
\newcommand*{\diff}{\mathop{}\!\mathrm{d}} %e.g. $F(x) = \int\!f(x)\diff x$
\newcommand{\Argmin}[1]{\underset{#1}{\operatorname{argmin}} \;}
\newcommand{\Maxof}[1]{\underset{#1}{\operatorname{max}} \;}
\newcommand{\Minof}[1]{\underset{#1}{\operatorname{min}} \;}

\providecommand{\e}[1]{\ensuremath{\cdot 10^{#1}}}

%workaround for mathbold in cool, cf. http://tex.stackexchange.com/questions/97910
%\let\oldmb\mathbold
%\protected\def\mathbold{\oldmb}

\newcommand{\sgn}{\text{sgn}}
\newcommand{\var}{\text{Var}}

\newcommand{\CB} [1]{{\color{#1}     $\bullet$}}
\newcommand{\CBB}[2]{{\color[#1]{#2} $\bullet$}}

\newcommand{\ie}{i.e.}
\newcommand{\Ie}{I.e.}
\newcommand{\eg}{e.g.}
\newcommand{\Eg}{E.g.}
% \newcommand{\cf}{cf.\ }
% \newcommand{\Cf}{Cf.\ }
\newcommand{\etc}{etc.\ }
\newcommand{\wrt}{w.r.t.\ }
\newcommand{\Wrt}{W.r.t.\ }
\newcommand{\Wlog}{W.l.o.g.\ }
\newcommand{\wwlog}{w.l.o.g.\ }
\newcommand{\resp}{resp.\ }
\newcommand{\etal}{et al.\ }

% \newtheorem{theorem}{Theorem}
% \newtheorem{definition}{Definition}
% \newtheorem{expectation}{Expectation}
% \newtheorem{problem}{Problem}
% %\newtheorem{algorithm}{Algorithm}


% http://tex.stackexchange.com/questions/145716/underbrace-in-a-matrix
\newcommand{\underbracedmatrix}[2]{%
  \left(\;
  \smash[b]{\underbrace{
    \begin{matrix}#1\end{matrix}
  }_{#2}}
  \;\right)
  \vphantom{\underbrace{\begin{matrix}#1\end{matrix}}_{#2}}
}


\title{The Parallel Eigenvectors Operator -- Additional Material}
\author[T. Oster, C. R{\"o}ssl \& H. Theisel]
       {T. Oster, C. R{\"o}ssl and H. Theisel \\
        University of Magdeburg, Germany}

\begin{document}

% \section*{Additional Material} % (fold)
% \label{sec:additional_material}

\section*{Notation} % (fold)
\label{sec:pev_notation}
%
\noindent
\begin{tabular}{ll}
$\mM$ & Matrix in $\RRSet^{3 \times 3}$\\
$\vv$ & Column vector in $\RRSet^3$ \\
$v_i$ & Components of a vector $\vv$\\
$\begin{pmatrix} \va & \vb & \vc \end{pmatrix}$ & Block matrix of multiple matrices/vectors \\
$\overline{\mM}$, $\overline{\vr}$ & Matrix in $\RRSet^{6 \times 6}$/vector in $\RRSet^6$ \\
$\vj_i$ & Unit vector along i-th coordinate \\
$\mM_x, \vv_y,...$ & Partial derivatives of a Matrix/vector in $x, y, \dots$ \\
$\nabla$ & Nabla operator $\T{(\frac{\partial}{\partial x}, \frac{\partial}{\partial y}, \frac{\partial}{\partial z}, \dots)}$
\end{tabular}

\section*{Proof of Theorem 1} % (fold)
\label{sub:proof_of_theorem_1}

\begin{theorem}
    \label{thm:1}
    The PEV operator delivers structurally stable curves that are either closed
    or end at the boundaries of the domain.
\end{theorem}
%
The main idea to prove \autoref{thm:1} is to search for PEV lines not in 3D
$(x,y,z)$ space but in a 6D $(x,y,z,u,v,w)$ space:
%
at every point $\vx=\T{(x,y,z)}$, all vector directions $\vr=\T{(u,v,w)}$ are
checked for being an eigenvector of $\mS$ and $\mT$.
%
This means that we search for all 6D points $\T{(\vx,\vr)}$ fulfilling
$\mS(\vx)\, \vr \times \vr = \vNull$ and $\mT(\vx)\, \vr \times \vr = \vNull$.
%
We formulate this to search for all 6D points $\T{(\vx,\vr)}$ where a 6D vector
field $\overline{\vh}$ vanishes:
%
\begin{equation}
    \label{eq:theory_1}
    \overline{\vh}(\vx,\vr) =
        \begin{pmatrix}
            \mS(\vx) \, \vr \times \vr \\
            \mT(\vx) \, \vr \times \vr
        \end{pmatrix}
    = \overline{\vNull}.
\end{equation}
%
Suppose a point $\T{(\vx_0,\vr_0)}$ is on a PEV structure, \ie, fulfills
\eqref{eq:theory_1}.
%
In order to study the PEV structures in a linear neighborhood of
$\T{(\vx_0,\vr_0)}$, we search for all directions $\T{(d \vx,d \vr)}$ in which
$\overline{\vh}$ remains zero: $\nabla \overline{\vh} \cdot \T{(d \vx,d \vr)} =
\overline{\vNull}$.
%
In other words: we have to explore the null space of $\nabla \overline{\vh}$.
%
Applying elementary differentiation rules gives
%
\begin{equation}
    \label{eq:theory_2}
    \nabla \overline{\vh} =
        \begin{pmatrix}
            \mG_1 &   \mG_3 \\
            \mG_2 &   \mG_4
        \end{pmatrix}
\end{equation}
%
with
%
\begin{align}
    \label{eq:theory_3}
    \mG_1 & =
        \begin{pmatrix}
            \mS_x \, \vr \times \vr &
            \mS_y \, \vr \times \vr &
            \mS_z \, \vr \times \vr
        \end{pmatrix}
    \\
    \mG_2 & =
        \begin{pmatrix}
            \mT_x \, \vr \times \vr  &
            \mT_y \, \vr \times \vr  &
            \mT_z \, \vr \times \vr
        \end{pmatrix}
    \\
    \mG_3 & =
        \begin{pmatrix}
        \mS \, \vj_1 \times \vr +  \mS \, \vr \times \vj_1 &
        \mS \, \vj_2 \times \vr +  \mS \, \vr \times \vj_2 &
        \mS \, \vj_3 \times \vr +  \mS \, \vr \times \vj_3
        \end{pmatrix}
    \\
    \mG_4 & =
        \begin{pmatrix}
        \mT \, \vj_1 \times \vr +  \mT \, \vr \times \vj_1 &
        \mT \, \vj_2 \times \vr +  \mT \, \vr \times \vj_2 &
        \mT \, \vj_3 \times \vr +  \mT \, \vr \times \vj_3
        \end{pmatrix}
    \, \text{.}
\end{align}
%
Then
%
\begin{equation}
   \label{eq:theory_4}
   \T{\mG_1} \, \vr \; =\;  \T{\mG_2} \, \vr\; =\; 0
\end{equation}
%
and from \eqref{eq:theory_1} follows
%
\begin{equation}
   \label{eq:theory_5}
   \T{\mG_3} \, \vr \;=\; \T{\mG_4} \, \vr \;=\;  0 \, \text{.}
\end{equation}
and
\begin{equation}
   \label{eq:theory_5a}
   \mG_3 \, \vr \;=\; \mG_4 \, \vr \;=\; 0 \, \text{.}
\end{equation}
%
Equations \eqref{eq:theory_4} and \eqref{eq:theory_5} give that
%
\begin{equation}
   \label{eq:theory_6}
   \Rank{\nabla \overline{\vh}} = 4
\end{equation}
%
in the structurally stable case.
%
This means that for $\Rank{\nabla \overline{\vh}} < 4$, adding noise to $\mS,
\mT$ brings $\Rank{\nabla \overline{\vh}}$ to $4$.
%
Equation \eqref{eq:theory_6} means that the PEV structure around
$\T{(\vx_0,\vr_0)}$ is a 2-manifold in 6D.
%
To see Equation \eqref{eq:theory_6}, we consider a rotation of the underlying
coordinate system such that $\vr=(0,0,r_z)$.
%
Then Equations \eqref{eq:theory_4} and \eqref{eq:theory_5} give
that the rotated tensors $\mG_1,\mG_2,\mG_3,\mG_4$ have vanishing third columns.
%
This and Equation \eqref{eq:theory_1} gives that $\nabla \overline{\vh}$ has
two colums, which proves Equation \eqref{eq:theory_6}.
%

%
One vector in the null space of $\nabla \overline{\vh}$ is trivial and denotes a
simple scaling of $\vr$:
%
Equations \eqref{eq:theory_4} -- \eqref{eq:theory_5a}
give $\nabla \overline{\vh}
\cdot \T{(\vNull, \vr)} = \overline{\vNull}$.
%
This means that the projection of the null space of $\nabla \overline{\vh}$ into
the spatial subspace $\vx$ gives a one-manifold in 3D.
%
This shows that PEV gives line structures in 3D.
%
To show that they are closed, we consider the 6 components of $\nabla
\overline{\vh}$ as scalar fields and interpret the PEV structure as intersection
of their \ac{5D} iso-hypersurfaces.
%
Iso-hypersurfaces are always closed, which means their intersections are also
closed.
%

%
Note that the proof did not make any assumptions on
the behavior of $\mS, \mT$ around $\T{(\vx_0,\vr_0)}$.
%
This means that it holds also in case of a transition from real to imaginary
eigenvectors of $\mS$ or $\mT$ as well as in regions of isotropic tensors.
%
\end{document}