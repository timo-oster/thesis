

\section{Results} % (fold)
\label{sec:pev_results}
%
We applied our method to different stress tensor fields from structural
mechanics simulations.
%
Stress tensors are symmetric tensors that describe the local stress at a point
in a material under acting force.
%
Its eigenvectors are real and orthogonal and are aligned with the principal
stress directions acting on the point.
%
When comparing two different stress tensor fields, \ac{PEV} lines occur where two
principal stress directions align.
%
We show \ac{PEV} lines for three different stress tensor datasets:
%
Two different point loads applied to a uniform material, two different traction
forces applied to the end of a clamped beam, and two different load scenarios
applied to a flange.
%

%
We used the same parameters for all datasets:
%
$\epsilon_s = \num{1.d-3}$, $\epsilon_c = 5 \times \epsilon_s$, $\epsilon_d =
\num{1.d-9}$, $\epsilon_p = \num{1.d-3}$.
%
Our results were computed on a 4-core Intel Core i7 CPU at
\SI{3.4}{\giga\hertz}.
%
\subsection*{Point Loads} % (fold)
\label{ssub:point_loads}
\Todo{Don't use starred sectioning}
%
%
% 1.1 million cells
%
% 4.8 million faces
%
% Computing time: \SI{4.2}{\hour}
%
% Time per face: \SI{3.18}{\milli\second}
%
In this example, we compare two different point loads applied to a uniform
material with infinite extent.
%
The first load (red arrow in \autoref{fig:point_load}) is a compressive force,
the second load (blue arrow) is a tensile force of equal magnitude.
%
We compute the \ac{PEV} operator for the two resulting stress tensor fields.
%
The Point Load case has a closed analytic solution~\cite{Saada2013}, which we
sampled on a regular grid with $100 \times 50 \times 50$ points using the
\texttt{vtkPointLoad} source from the Visualization Toolkit~\cite{Schroeder2006}.
%
We then tetrahedralized the data, resulting in 1.1 million cells and 4.8 million
faces.
%
The computing time for this dataset was \SI{4.2}{\hour}, which means that \ac{PEV}
intersections on each face were found in \SI{3.2}{\milli\second} on average.
%

%
Since the point loads were applied in the same plane, this synthetic dataset
shows the rare case where eigenvectors are parallel on a plane instead of a
line.
%
This degenerate case also accounts for the long computing time, as for each face
intersected by the \ac{PEV} plane, the recursive subdivision can not be terminated
early.
%
Even though this structurally unstable case produces visual artifacts when
using our method, interesting \ac{PEV} line structures are still visible.
%
There is a bifurcation point exactly at both load points, extending into a
curved ring slightly below the surface.
%
At the second intersection of this ring with the central plane, another closed
\ac{PEV} structure embedded into the plane becomes visible.
%
Within a \ac{PEV} plane, structures where all three eigenvectors are parallel become
lines, instead of points.
%
A similar structure can be observed starting at the load points and leading
outwards.
%
In the center of the dataset, there is another \ac{PEV} line, orthogonal to the
plane and slightly curved downwards, separating the two load points.
%
For this dataset, we colored the \ac{PEV} lines by the absolute eigenvalue ratio of
the parallel eigenvectors.
%
Since both tensor fields result from forces of equal magnitude, this ratio makes
visible the directions in which forces propagate outwards from the load points.
%
% subsubsection point_loads (end)
%
\subsection*{Clamped Beam} % (fold)
\label{ssub:clamped_beam}
%
% 150k cells
%
% 600k faces
%
% Computing time: \SI{26}{\minute}
%
% Time per face: \SI{2.6}{\milli\second}
%
Next, we extracted \ac{PEV} lines for a beam that is fixed on one side.
%
We applied two different traction forces on the free end of the beam, whose
directions are shown in~\autoref{fig:beam_full}.
%
The Clamped Beam dataset consists of 150k cells and 600k faces.
%
The computing time was \SI{26}{\minute}, which means \SI{2.6}{\milli\second}
per face on average.
%

%
% The resulting \ac{PEV} lines are surprisingly complex for such a simple case.
% %
% However, the eigenvalues corresponding to the parallel eigenvectors are near
% zero in a lot of cases, which might mean they exist due to numerical noise.
% %
% Nevertheless, we did not filter them out for rendering, as they showcase nicely
% the high frequency of bifurcation points in symmetric tensor fields.
%

%
There are two regions of particular interest in the Clamped Beam.
%
The first is near the middle of the beam, where a curved structure has high
eigenvalues in both tensor fields (visible in red and blue
in~\autoref{fig:beam_full}).
%
This is where the beam experiences a lot of stress, and therefore the
tensor fields have a high magnitude.
%
The second is near the fixed end.
%
Here, the stress is high particularly for the more diagonal force indicated by
the red arrow in~\autoref{fig:beam_full}.
%
We find a high number of bifurcation points in this region.
%
Additionally, near the middle of the beam all three eigenvector directions are
parallel along a line structure near the center with considerable
length.
%
This area seems to be the most similar between the two scenarios in terms of
stress directions.
%
% subsubsection clamped_beam (end)
%
\subsection*{Flange} % (fold)
\label{ssub:flange}
%
\begin{figure*}
    \centering
    \begin{tabular}{lcr}
        \includegraphics[width=0.3\textwidth]{figures/PointLoad_total_sq} &
        \tikz{
            \node[anchor=south west, inner sep=0pt] (img) at (0,0) {
                \includegraphics[width=0.3\textwidth]{figures/PointLoad_detail1_sq}
            };
            \begin{scope}[x={(img.south east)},y={(img.north west)}]
                % \draw[help lines, opacity=0.5, xstep=.01,ystep=.01] (0,0) grid (1,1);
                % \draw[thin, xstep=.1,ystep=.1] (0,0) grid (1,1);
                % \foreach \x in {0,...,9} { \node [anchor=north] at (\x/10,0) {0.\x}; }
                % \foreach \y in {0,...,9} { \node [anchor=east] at (0,\y/10) {0.\y}; }
                % \draw[red, thick, rotate around={-45:(0.85, 0.27)}]
                %     (0.85, 0.27) ellipse (0.2 and 0.05);
                % \draw[red, thick, rotate around={15:(0.62, 0.47)}]
                %     (0.62, 0.47) ellipse (0.23 and 0.1);
                % \draw[red, thick, rotate around={7:(0.36, 0.64)}]
                %     (0.36, 0.64) ellipse (0.19 and 0.08);
                % \draw[red, thick, rotate around={20:(0.44, 0.51)}]
                %     (0.44, 0.51) ellipse (0.32 and 0.04);
                % \draw[red, thick] (0.45, 0.6) circle[radius=5pt];
                % \draw[red, thick] (0.5, 0.54) circle[radius=5pt];
                % \draw[red, thick] (0.55, 0.53) circle[radius=5pt];
            \end{scope}
        } &
        \tikz{
            \node[anchor=south west, inner sep=0pt] (img) at (0,0) {
                \includegraphics[width=0.3\textwidth]{figures/PointLoad_detail2_sq}
            };
            \begin{scope}[x={(img.south east)},y={(img.north west)}]
                % \draw[help lines, opacity=0.5, xstep=.01,ystep=.01] (0,0) grid (1,1);
                % \draw[thin, xstep=.1,ystep=.1] (0,0) grid (1,1);
                % \foreach \x in {0,...,9} { \node [anchor=north] at (\x/10,0) {0.\x}; }
                % \foreach \y in {0,...,9} { \node [anchor=east] at (0,\y/10) {0.\y}; }
                \draw[mycolor4, thick, rotate around={5:(0.1, 0.56)}]
                    (0.1, 0.56) ellipse (0.12 and 0.03);
                \draw[mycolor4, thick] (0.71, 0.57) circle[radius=5pt];
                \draw[mycolor4, thick] (0.91, 0.56) circle[radius=5pt];
            \end{scope}
        }
    \end{tabular}
    \caption{\ac{PEV} lines for the Point Load dataset. Lines are colored by
             absolute eigenvalue ratio. A compressive force (red arrow) is
             applied to obtain one stress tensor field, while an equivalent
             tensile force (blue arrow) is applied for the second stress tensor.
             Due to the parallel application of forces orthogonal to the
             surface, a plane of parallel eigenvectors forms between the load
             points.
             \Todo[inline]{Change color scale to be consistent with other figures.
                           Make images larger.}}
    \label{fig:point_load}
\end{figure*}
\begin{figure*}
    \begin{tabular}{lr}
        \includegraphics[width=0.45\textwidth]{figures/beam_full_total} &
        \tikz{
            \node[anchor=south west, inner sep=0pt] (img) at (0,0) {
                \includegraphics[width=0.45\textwidth]{figures/beam_full_detail1}
            };
            \begin{scope}[x={(img.south east)},y={(img.north west)}]
                % \draw[help lines, opacity=0.5, xstep=.01,ystep=.01] (0,0) grid (1,1);
                % \draw[thin, xstep=.1,ystep=.1] (0,0) grid (1,1);
                % \foreach \x in {0,...,9} { \node [anchor=north] at (\x/10,0) {0.\x}; }
                % \foreach \y in {0,...,9} { \node [anchor=east] at (0,\y/10) {0.\y}; }
                \draw[mycolor4, thick, rotate around={-25:(0.55, 0.5)}]
                    (0.55, 0.5) ellipse (0.35 and 0.25);
            \end{scope}
        } \\
        \tikz{
            \node[anchor=south west, inner sep=0pt] (img) at (0,0) {
                \includegraphics[width=0.45\textwidth]{figures/beam_full_detail3}
            };
            \begin{scope}[x={(img.south east)},y={(img.north west)}]
                % \draw[help lines, opacity=0.5, xstep=.01,ystep=.01] (0,0) grid (1,1);
                % \draw[thin, xstep=.1,ystep=.1] (0,0) grid (1,1);
                % \foreach \x in {0,...,9} { \node [anchor=north] at (\x/10,0) {0.\x}; }
                % \foreach \y in {0,...,9} { \node [anchor=east] at (0,\y/10) {0.\y}; }
                \draw[mycolor4, thick] (0.33, 0.5) circle[radius=5pt];
                \draw[mycolor4, thick] (0.27, 0.5) circle[radius=5pt];
                \draw[mycolor4, thick] (0.485, 0.575) circle[radius=5pt];
                \draw[mycolor4, thick] (0.67, 0.64) circle[radius=5pt];
                \draw[mycolor4, thick] (0.67, 0.53) circle[radius=5pt];
                \draw[mycolor4, thick] (0.72, 0.66) circle[radius=5pt];
                \draw[mycolor4, thick] (0.31, 0.4) circle[radius=5pt];
                \draw[mycolor4, thick] (0.31, 0.64) circle[radius=5pt];
                \draw[mycolor4, thick] (0.585, 0.58) circle[radius=5pt];
            \end{scope}
        } &
        \tikz{
            \node[anchor=south west, inner sep=0pt] (img) at (0,0) {
                \includegraphics[width=0.45\textwidth]{figures/beam_full_detail2}
            };
            \begin{scope}[x={(img.south east)},y={(img.north west)}]
                % \draw[help lines, opacity=0.5, xstep=.01,ystep=.01] (0,0) grid (1,1);
                % \draw[thin, xstep=.1,ystep=.1] (0,0) grid (1,1);
                % \foreach \x in {0,...,9} { \node [anchor=north] at (\x/10,0) {0.\x}; }
                % \foreach \y in {0,...,9} { \node [anchor=east] at (0,\y/10) {0.\y}; }
                \draw[mycolor4, thick] (0.75, 0.58) ellipse (0.2 and 0.05);
            \end{scope}
        }
    \end{tabular}
    \caption{\ac{PEV} lines for the Clamped Beam dataset. Two different traction
             forces are applied to the free end of the beam (red and blue
             arrows), while the other end is fixed on the wall. Lines are
             colored by the eigenvalue of the stress tensor corresponding to the
             red arrow (red is positive, blue is negative). Interesting
             structures mentioned in the text are highlighted.}
    \label{fig:beam_full}
\end{figure*}
\begin{figure*}
    \begin{tabular}{lcr}
        \includegraphics[width=0.3\textwidth]{figures/flange_full_total_sq} &
        \tikz{
            \node[anchor=south west, inner sep=0pt] (img) at (0,0) {
                \includegraphics[width=0.3\textwidth]{figures/flange_detail1_sq}
            };
            \begin{scope}[x={(img.south east)},y={(img.north west)}]
                % \draw[help lines, opacity=0.5, xstep=.01,ystep=.01] (0,0) grid (1,1);
                % \draw[thin, xstep=.1,ystep=.1] (0,0) grid (1,1);
                % \foreach \x in {0,...,9} { \node [anchor=north] at (\x/10,0) {0.\x}; }
                % \foreach \y in {0,...,9} { \node [anchor=east] at (0,\y/10) {0.\y}; }
                \draw[mycolor4, thick] (0.58, 0.68) circle[radius=5pt];
                \draw[mycolor4, thick] (0.47, 0.5) circle[radius=5pt];
            \end{scope}
        } &
        \tikz{
            \node[anchor=south west, inner sep=0pt] (img) at (0,0) {
                \includegraphics[width=0.3\textwidth]{figures/flange_detail2_sq}
            };
            \begin{scope}[x={(img.south east)},y={(img.north west)}]
                % \draw[help lines, opacity=0.5, xstep=.01,ystep=.01] (0,0) grid (1,1);
                % \draw[thin, xstep=.1,ystep=.1] (0,0) grid (1,1);
                % \foreach \x in {0,...,9} { \node [anchor=north] at (\x/10,0) {0.\x}; }
                % \foreach \y in {0,...,9} { \node [anchor=east] at (0,\y/10) {0.\y}; }
                \draw[mycolor4, thick] (0.75, 0.6) circle[radius=10pt];
                \draw[mycolor4, thick] (0.75, 0.4) circle[radius=10pt];
            \end{scope}
        }
    \end{tabular}
    \caption{\ac{PEV} lines for the Flange dataset. Two different load scenarios
             (indicated by blue and red arrows) are simulated. Lines are colored
             by the eigenvalue corresponding to the red arrows. To avoid visual
             clutter, we filtered the lines in the middle and right image by
             eigenvalue, removing lines where both eigenvalues are very small.
             Interesting structures mentioned in the text are highlighted.}
    \label{fig:flange_filtered}
\end{figure*}
%
% 1.2 million cells
%
% 5 million faces
%
% Computing time: \SI{36}{\minute}
%
% Time per face: \SI{0.45}{\milli\second}
%
Our final stress tensor dataset is a flange geometry from an
OpenFoam~\cite{OpenFoamWWW} tutorial.
%
We subjected the flange to two different loads, applied on the back wall and the
outlet tube (see red and blue arrows in \autoref{fig:flange_filtered}).
%
The original mesh uses polygonal cells, which is why we resampled the data,
resulting in 1.2 million cells and 5 million faces.
%
The computing time was \SI{36}{\minute}, \ie, \SI{0.5}{\milli\second} per face.
%

%
The dataset exhibits a lot of \ac{PEV} lines, which can be seen
in~\autoref{fig:flange_filtered} on the left.
%
Most of the \ac{PEV} lines correspond to eigenvectors with small eigenvalues in both
tensor fields.
%
We therefore filtered out all \ac{PEV} lines where both eigenvalues are very small
in the center and right images in~\autoref{fig:flange_filtered}.
%
Especially prominent are two bifurcation points with high eigenvalues between
the two outer screw holes and the central tube.
%
There are also \ac{PEV} lines leading outwards both above and below the screw holes.
%
In general, the most similar directions of significant stress are near the
screw holes and in the area where the large outlet tube meets the central block.
%
% subsubsection flange (end)
%
% subsection stress_tensor_data (end)
%
% section results (end)