
%
\section{Limitations and Future Research} % (fold)
\label{sec:pev_limitations}
% 
Limitations can be discussed from two points of view:
%
the operator itself and the presented numerical extraction algorithm.
%
A limitation of the \ac{PEV} operator is that it can only be applied to problems
where the norm of the tensors does not matter.
%
This limits the applicability but on the other hand focuses on features of the
tensor fields that are less covered by other methods.
%

%
The presented extractor works for piecewise linear tensor fields only.
%
An extension to hexahedral grids as well as higher order interpolations is
subject of future research.
%
The performance of the algorithm can be improved by parallelization.
%
In principle, the algorithm is parallelizable (each cell can be treated
independently).
%
However, even if this is carefully carried out, interactive frame rates (for
instance for comparing time-dependent tensor fields) are hardly achievable
because we still have to do a search in a \ac{5D} space.
%
Due to the possibility of many candidate solutions for each intersection of a
\ac{PEV} line with a face, we can not give an upper limit on the error of the
\ac{PEV} line position.
%
This might limit its applicability in cases where a highly accurate \ac{PEV} line is
required.
%

%
%\textcolor{red}{
%
We stated in~\autoref{thm:1} that \ac{PEV} lines are generally closed.
%
However, the results obtained from our algorithm sometimes exhibit gaps.
%
Because the extractor is a numerical algorithm, and not a combinatorial one, we
will sometimes not find solutions on faces where the presence of an intersection
point is numerically unstable.
%
This happens for example if the \ac{PEV} line is parallel and very close to a face,
or when the tensor field is almost zero.
%
%}
%

%
In this paper, we only show examples of the \ac{PEV} operator for (symmetric) stress
tensor fields.
%
Further possible scenarios that are left to future research are the comparative
visualization of DT-MRI data or a comparative visualization of Jacobian fields
for flow visualization, which are not necessarily symmetric.
%