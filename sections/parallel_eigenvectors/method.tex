\section[Extracting PEV Lines from Piecewise Linear Data]
        {Extracting \acs{PEV} Lines from Piecewise Linear Data} % (fold)
\label{sec:extracting_pev_lines}
%
We will now detail our algorithm for finding \ac{PEV} lines in piecewise linear
tensor fields.
%
We assume that both tensor fields are defined on the vertices of the same
tetrahedral mesh.
%
The general approach is to first find all intersections of \ac{PEV} lines with the
faces of the mesh, and then to connect those points to lines.
%

%
We showed that \ac{PEV} structures are lines in the structurally stable case.
%
It follows that their intersections with the triangular faces of a tetrahedral
mesh are isolated points.
%
Finding an analytic solution to the parallel eigenvectors problem is impossible,
as it involves the intersection of cubic polynomials.
%
Instead, we opt for a numerical approach that is based on recursive subdivision
both on the triangle and in the space of possible eigenvector directions.
%

%
Our algorithm can be summarized as follows:
%
We first find a direction $\vr$ which becomes an eigenvector of both $\mS$ and
$\mT$ at some (possibly different) points inside the triangle.
%
If such a direction is found, we subdivide the triangle and check the parts for
possible eigenvector directions again.
%
We do this until we converge on a single point where both $\mS$ and $\mT$
have parallel eigenvectors.
%
In order to find a valid direction $\vr$, we perform another recursive search in
the space of possible eigenvector directions, which we represent as some
triangulation of a hemisphere centered at the origin.
%
In the following, we describe the details of this algorithm.
%

\subsection{Mathematical Basis} % (fold)
\label{sub:mathematical_basis}
%
A linear tensor field on a triangle is defined by the tensors at its three
corners.
%
We denote the set of corner points as $\Delta_{\vx} = \{\vx_1, \vx_2,
\vx_3\}$, and the set of corner tensors as $\Delta_{\mS} = \{\mS_1, \mS_2,
\mS_3\}$ and $\Delta_{\mT} = \{\mT_1, \mT_2, \mT_3\}$.
%
We express the tensor fields in barycentric coordinates $\vw = \T{(w_1, w_2,
w_3)}$:
%
\begin{equation*}
    \mS(\vw) = \sum_i{w_i\mS_i}\,\text{,} \quad
    \mT(\vw) = \sum_i{w_i\mT_i}\,\text{,} \quad
    \text{with}\quad \sum_i{w_i} = 1 \, \text{.}
\end{equation*}
%
The position (in barycentric coordinates $\vw$) at which an arbitrary direction
$\vr$ becomes an eigenvector in $\mS$ is given by the solution to
%
\begin{equation*}
    \mS(\vw)\vr = \sum_i{w_i \mS_i \vr} = \lambda \vr \, \text{.}
\end{equation*}
%

%
Rather than needing the exact position, we want to know if the position is
inside the triangle, \ie{}, if $\vr$ is a valid eigenvector direction for $\mS$.
%
In barycentric coordinates, a point is inside the triangle if all $w_i > 0$.
%
Since the scaling factor $\lambda$ is arbitrary, we eliminate it:
%
\begin{gather*}
    \sum_i{\tilde{w}_i \mS_i \vr} =
    \mA(\vr)\tilde\vw =
    \vr \, \text{,} \\
    \text{with} \quad
    \mA(\vr) = \begin{pmatrix} \mS_1 \vr & \mS_2 \vr & \mS_3 \vr \end{pmatrix}
    \, \text{,} \quad \tilde\vw = \vw / \lambda \,\text{,}
\end{gather*}
%
and only require that all $\tilde{w}_i$ have the same sign.
%
Using Cramer's rule, the components of $\tilde\vw$ are
% 
\begin{equation*}
    \tilde{w}_i = \frac{\det \mA_i(\vr)}{\det \mA(\vr)}\,\text{.}
\end{equation*}
% 
Here, $\mA_i$ denotes the matrix $\mA$ with its $i$-th column replaced by $\vr$.
% 
Note that all $\tilde{w}_i$ are divided by the same factor $\det \mA(\vr)$.
%
Since this influences all signs of $\tilde{w}_i$ equally it can be ignored,
leading to
%
\begin{equation}
    \label{eq:coordinate_functions}
    \hat{w}_i(\vr) = \det \mA_i(\vr) \, \text{.}
\end{equation}
%

%
The equations for $\mT$ are analogous.
% 
In the following, we show all equations for $\mS$ only.
% 
The equivalent equations for $\mT$ can be obtained trivially by substituting
$\mT$ for $\mS$.
% 
We denote the solutions for $\mS$ and $\mT$ by $\hat\vw_\mS$ and $\hat\vw_\mT$
respectively, whenever it is necessary to discriminate them.
%
% subsection mathematical_basis (end)

\subsection{Subdivision in Direction Space} % (fold)
\label{sub:subdivision_in_direction_space}
%
The core of the algorithm is to find a direction $\vr$ for which all components
of $\hat\vw_\mS(\vr)$ have a common sign, and all components of
$\hat\vw_\mT(\vr)$ also have a common sign (that possibly differs from
$\hat\vw_\mS$).
%
If this is fulfilled, $\vr$ becomes an eigenvector somewhere inside the triangle
for both $\mS$ and $\mT$.
%

%
Note that the $\hat{w}_i(\vr)$ are cubic in $\vr$.
%
Finding an analytic solution for $\vr$ means analytically finding the
intersections of the roots of $\hat{w}_i(\vr)$, which is impossible.
%
Instead, we solve the problem numerically by applying another recursive search
in the space of all possible eigenvector directions.
%
Since we are looking for eigenvectors, the magnitude and orientation of $\vr$
are not significant.
%
We can therefore represent this space by some triangulation of a hemisphere
centered at the origin (\cref{fig:algorithm}, right).
%
We again express a direction in a triangle $\Delta_{\vr} = \{\vr_1, \vr_2,
\vr_3\}$ on this hemisphere in barycentric coordinates $u_j$ of its corner
vectors:
%
\begin{equation*}
    \vr(\vu) = \sum_j{u_j\vr_j} \, \text{.}
\end{equation*}
%
Substituting this in \cref{eq:coordinate_functions}, the barycentric coordinate
functions now become
% 
\begin{equation}
    \hat{w}_i(\vu) = \det \left( \sum_j{u_j\mA_i(\vr_j)} \right)\, \text{.}
\end{equation}
% 
% We express the relevant behavior of $\hat{w}_i(\vu)$ inside the triangle by
% the indicator function $s(\hat{w}_i(\vu))$:
% %
% \begin{equation}
%     s(\hat{w}_i(\vu)) =
%     \begin{cases}
%         \hfill 1 \text{,} &
%         \text{if} \quad  \hat{w}_i(\vu) > 0 \quad
%             \forall \{\vu\; |\; u_j \geqslant 0\} \\
%         \hfill -1 \text{,} &
%         \text{if} \quad  \hat{w}_i(\vu) < 0 \quad
%             \forall \{\vu\; |\; u_j \geqslant 0\} \\
%         \hfill 0 \text{,} & \text{else.}
%     \end{cases}
% \end{equation}
% %
% This indicator tells us if the sign of $\hat{w}_i$ is consistent everywhere
% inside $\Delta_{\vr}$, and if it is positive or negative.
% %
% Computing $s_i$ means deciding if there are roots of $\hat{w}_i$ anywhere inside
% the triangle.
%
We can now express the polynomials $\hat{w}_i$ in Bernstein-B\'ezier basis as
%
\begin{equation}
    \hat{w}_i(\vu) = \sum_{\substack{j,\, k,\, l \geqslant 0 \, \text{,} \\ j+k+l=3}}{
        \frac{3!}{j! \, k! \, l!} \, u_1^j \, u_2^k \, u_3^l \cdot b_{jkl}}
        \, \text{.}
\end{equation}
%
Here, $b_{jkl}$ are the \num{10} coefficients needed to express a trivariate
polynomial of degree \num{3}.
%
Note that because the barycentric coordinates $\vu$ are restricted to the
triangle $\Delta_\vr$, they really have only two degrees of freedom.
%

%
We use the property that a polynomial in Bernstein-B\'ezier form is bounded in
its domain by the convex hull of its coefficients \cite{Farin1997}.
%
This means that $\hat{w}_i$ is positive over the whole triangle if all $b_{jkl}
> \num{0}$, and negative over the whole triangle if all $b_{jkl} < \num{0}$.
%
If the $b_{jkl}$ have different signs, $\hat{w}_i$ might become \num{0}
somewhere inside the triangle.
%

%
We use this when recursively subdividing the triangle $\Delta_{\vr}$.
%
If any $\hat{w}_i$ might have roots within the triangle according to the
Bernstein-B\'ezier coefficients, then we can not make a decision.
%
We need to subdivide the triangle and check the different parts again.
%
If no $\hat{w}_i$ can have roots within the triangle as indicated by the
coefficients, then there are two possibilities:
%
\begin{enumerate}
  \item \label{i:accept} All $\hat{w}_i$ have the same sign everywhere on the
  triangle
  \item \label{i:reject} The $\hat{w}_i$ have different signs everywhere on the
  triangle
\end{enumerate}
%
In case~\ref{i:accept}, all directions within the triangle become
eigenvector directions somewhere in $\Delta_{\vx}$ for both $\mS$ and $\mT$.
%
If this happens, we can accept any direction within the current triangle as a
possible solution.
%
In case~\ref{i:reject}, no direction within the triangle can become an
eigenvector of both $\mS$ and $\mT$, and the triangle is discarded.
%
% %
% This results in an indicator functions $s_i$, which tells us if the sign of
% $\hat{w}_i$ is consistent everywhere inside $\Delta_{\vr}$, and if it is
% positive or negative.:
% %
% \begin{equation}
%     s_i =
%     \begin{cases}
%         \hfill 1 \text{,} &
%         \text{if} \quad  b_{jkl}(\hat{w}_i) > 0 \quad  \forall \ j+k+l=3 \\
%         \hfill -1 \text{,} &
%         \text{if} \quad  b_{jkl}(\hat{w}_i) < 0 \quad  \forall \ j+k+l=3 \\
%         \hfill 0 \text{,} & \text{else.}
%     \end{cases}
% \end{equation}
% %
% %
% If there are $i, j$ such that $s_i \neq 0$ and $s_j \neq 0$ with $s_i \neq s_j$,
% then the coordinate functions $\hat{w}_i(\vr)$ do not all have the same sign for
% any direction in $\Delta_{\vr}$.
% %
% This means that no direction in $\Delta_{\vr}$ will become an eigenvector in
% $\Delta_{\vx}$, and therefore $\Delta_{\vr}$ can be discarded.
% %
% If there are $i, j, k$ such that $s_i \neq 0$ and $s_i = s_j = s_k$ for both
% $\mS$ and $\mT$, then all directions in $\Delta_{\vr}$ will become eigenvectors
% somewhere in $\Delta_{\vx}$ for $\mS$ and $\mT$.
% %
% In this case, we have found valid eigenvector directions and can terminate the
% recursion.
% %
% If none of the above is the case, we have to subdivide $\Delta_{\vr}$ and
% investigate the parts again.
%
When the triangle becomes smaller than some subdivision threshold $\varepsilon_{\mathrm{r}}$,
and we still can not say for sure that there are no possible eigenvector
directions inside, we accept the central direction as a candidate.
%
\begin{figure}[t]
    \centering
    \setlength\figurewidth\linewidth
    %
\begin{tikzpicture}[scale=\figurewidth/1cm*0.14, line join=round]
\tikzstyle{currentface} = [fill=mycolor3]
\tikzstyle{axes} = [arrows=-latex]
\tikzstyle{trilines} = [thick, fill=white]
\tikzstyle{back} = [gray]

\coordinate (x1) at (90:1);
\coordinate (x2) at (-30:1);
\coordinate (x3) at (210:1);

\coordinate (x21) at ($0.5*(x1) + 0.5*(x2)$);
\coordinate (x22) at ($0.5*(x2) + 0.5*(x3)$);
\coordinate (x23) at ($0.5*(x3) + 0.5*(x1)$);

\coordinate (x31) at ($0.5*(x21) + 0.5*(x2)$);
\coordinate (x32) at ($0.5*(x2) + 0.5*(x22)$);
\coordinate (x33) at ($0.5*(x22) + 0.5*(x21)$);

\draw [trilines] (x1) -- (x2) -- (x3) -- cycle;
\draw [trilines] (x21) -- (x22) -- (x23) -- cycle;
\draw [trilines, currentface] (x31) -- (x32) -- (x33) -- cycle;

\draw [thin] (1.8, 1.5) -- (5.1, 1.5) -- (5.1, -1) -- (1.8, -1) -- cycle;
\draw [thin] (x33) -- (1.8, 1.5);
\draw [thin] (x32) -- (1.8, -1);

% \node [above] at (x1) {$\vx_1, \mS_1, \mT_1$};
% \node [below] at (x2) {$\vx_2, \mS_2, \mT_2$};
% \node [below] at (x3) {$\vx_3, \mS_3, \mT_3$};
\node [below] at (x32) {$\vx_\triangle$};

\begin{scope}[shift={(3.5, -0.2)}]

    \coordinate (null) at (0, 0, 0);
    \coordinate (px) at (1, 0, 0);
    \coordinate (mx) at (-1, 0, 0);
    \coordinate (pz) at (0, 0, 1);
    \coordinate (mz) at (0, 0, -1);
    \coordinate (py) at (0, 1, 0);

    \coordinate (pxpz) at ($0.5*(px) + 0.5*(pz)$);
    \coordinate (pzmx) at ($0.5*(pz) + 0.5*(mx)$);
    \coordinate (mxmz) at ($0.5*(mx) + 0.5*(mz)$);
    \coordinate (mzpx) at ($0.5*(mz) + 0.5*(px)$);
    \coordinate (pxpy) at ($0.5*(px) + 0.5*(py)$);
    \coordinate (pzpy) at ($0.5*(pz) + 0.5*(py)$);
    \coordinate (mxpy) at ($0.5*(mx) + 0.5*(py)$);
    \coordinate (mzpy) at ($0.5*(mz) + 0.5*(py)$);

    \coordinate (x1) at ($0.5*(px) + 0.5*(pxpz)$);
    \coordinate (x2) at ($0.5*(pxpz) + 0.5*(pxpy)$);
    \coordinate (x3) at ($0.5*(px) + 0.5*(pxpy)$);

    \draw [axes] ($1.5*(mx)$) -- ($1.5*(px)$);
    \draw [axes] ($1.5*(mz)$) -- ($1.5*(pz)$);
    \draw [axes] (null) -- ($1.5*(py)$);

    \draw [trilines, back] (mx) -- (mz) -- (py) -- cycle;
    \draw [trilines, back] (mz) -- (px) -- (py) -- cycle;
    \draw [trilines] (px) -- (pz) -- (py) -- cycle;
    \draw [trilines] (pz) -- (mx) -- (py) -- cycle;

    \draw [trilines] (pxpz) -- (pzpy) -- (pxpy) -- cycle;
    \draw [trilines] (pzmx) -- (mxpy) -- (pzpy) -- cycle;

    \draw [trilines, currentface] (x1) -- (x2) -- (x3) -- cycle;

    \node [below] at (x1) {$\vr_\triangle$};

\end{scope}

\end{tikzpicture}
    \caption{Two-level recursion scheme for finding intersections of \ac{PEV} lines
             with the faces of piecewise linear tensor fields. For each
             sub-triangle $\Delta_{\vx}$, a recursive search in the space of
             possible eigenvector directions is performed to find a direction
             $\vr$ that becomes an eigenvector of both $\mS$ and $\mT$ within
             $\Delta_{\vx}$.}
    \label{fig:algorithm}
\end{figure}
%
% subsection subdivision_in_direction_space (end)

\subsection{Final Numerical Algorithm} % (fold)
\label{sub:final_numerical_algorithm}
%
\begin{algorithm}[p]
    \SetKwProg{Fn}{Function}{}{end}
    \SetKwFunction{FindPEV}{FindPEV}
    \SetKwFunction{FindEigenDir}{FindEigenDir}
    \SetKwFunction{FindEigenDirRecursive}{FindEigenDirRecursive}
    \SetKwFunction{Split}{Split}
    \SetKw{KwNull}{null}
    \SetKw{KwOr}{or}

    \Fn{\FindPEV{$\Delta_{\mS}$, $\Delta_{\mT}$, $\Delta_{\vx}$}}{
        $\vr \gets$ \FindEigenDir{$\Delta_{\mS}$, $\Delta_{\mT}$}\;
        \uIf{$\vr$ is \KwNull}{
            \Return{$\{\}$}
            \tcp*{Discard triangle}
        }
        \ElseIf{size of $\Delta_{\vx} < \varepsilon_{\mathrm{s}}$}{
            \Return{$\{(\sfrac{1}{3}\sum{\vx_i}, \ \vr)\}$}
            \tcp*{Accept solution}
        }
        $l$ = \{\}\;
        \ForEach{$(\Delta'_{\mS}, \Delta'_{\mT}, \Delta'_{\vx}) \in
            (\text{%
                \Split{$\Delta_{\mS}$},
                \Split{$\Delta_{\mT}$},
                \Split{$\Delta_{\vx}$}})$}{
            $l \gets l \ \cup$
                \FindPEV{$\Delta'_{\mS}$,
                         $\Delta'_{\mT}$,
                         $\Delta'_{\vx}$}
            \tcp*{Recursive subdivision}
        }
        \Return{$l$}\;
    }
    % \BlankLine
    \Fn{\FindEigenDir{$\Delta_{\mS}$, $\Delta_{\mT}$}}{
        $R \gets$ set of triangles covering a hemisphere\;
        \ForEach{$\Delta_{\vr} \in R$}{
            $\vr \gets$ \FindEigenDirRecursive{$\Delta_{\mS}$,
                                               $\Delta_{\mT}$,
                                               $\Delta_{\vr}$}\;
            \If{$\vr$ is not \KwNull}{
                \Return{$\vr$}\;
            }
        }
        \Return{\KwNull}\;
    }
    % \BlankLine
    \Fn{\FindEigenDirRecursive{$\Delta_{\mS}$,
                               $\Delta_{\mT}$,
                               $\Delta_{\vr}$}}{
        Compute Bernstein-B\'ezier coefficients for $\hat{w}_i$\;
        \uIf{any $\hat{w}_i$ might have roots within $\Delta_{\vr}$}{
            \ForEach{$\Delta'_{\vr} \in$ \Split{$\Delta_{\vr}$}}{
               $\vr \gets$ %
                    \FindEigenDirRecursive{$\Delta_{\mS}$,
                                           $\Delta_{\mT}$,
                                           $\Delta'_{\vr}$}\;
                \If{$\vr$ is not \KwNull}{
                    \Return{$\vr$}\;
                }
            }
        }
        \ElseIf{all $\hat{w}_i$ have the same sign %
                \KwOr size of $\Delta_{\vr} < \varepsilon_{\mathrm{r}}$}{
            \Return{$\sfrac{1}{3}\sum{\vr_i}$}
            \tcp*{Accept solution}
        }
        \Return{\KwNull}
        \tcp*{Discard triangle}
    }
    \caption{Find intersections of \ac{PEV} lines with a triangle}
    \label{alg:find_pev_line}
\end{algorithm}
%
The complete algorithm for finding intersections of \ac{PEV} lines with a
triangle of the dataset now works as follows:
%
Start with the complete triangle as $\Delta_{\vx}$.
%
Then, search for a direction that becomes an eigenvector of both $\mS$ and $\mT$
somewhere inside the triangle by using the algorithm described in
\cref{sub:subdivision_in_direction_space}.
%
If such a direction is found, subdivide the triangle and process the parts
recursively.
%
If no direction is found, discard the triangle.
%
When a spatial sub-triangle becomes smaller than a subdivision threshold
$\varepsilon_{\mathrm{s}}$, we accept the center of the triangle and the
accompanying direction $\vr$ as a solution candidate.
%
\Cref{alg:find_pev_line} shows the procedure in pseudo code.
%

%
The result of the algorithm is a list of points on $\Delta_{\vx}$ with
corresponding eigenvector directions $\vr$.
%
This list of points has to be post-processed for two reasons:
%
\begin{enumerate}
    \item \label{i:clustering}
          %
          For each intersection of the \ac{PEV} line with the triangle, multiple
          adjacent candidate points may be found.
          %
          This happens if eigenvectors of $\mS$ and $\mT$ are closer than
          $\varepsilon_{\mathrm{r}}$ in a region larger than
          $\varepsilon_{\mathrm{s}}$, \eg, because the gradient of the tensor
          fields is very small, or because the \ac{PEV} line intersects the face
          at a very steep angle.
          %
          Choosing $\varepsilon_{\mathrm{r}}$ very small helps with this, but it
          can not be avoided in the presence of limited numerical precision on a
          computer.
          %
    \item \label{i:false_positives}
          %
          A candidate point might not be a \ac{PEV} point at all.
          %
          These false positives occur if there are directions $\vr$ that become
          eigenvectors of one of the tensor fields inside $\Delta_{\vx}$,
          while $\mA(\vr)$ has rank 1 for the other tensor field.
          %
          For this case, $\hat{\vw} = \vNull$, which means that a consistent
          sign of all components can never be determined, and subdivision can
          not be terminated early, even if the tensor field does not have any
          valid eigenvector directions inside $\Delta_{\vx}$.
          %
\end{enumerate}
%

% 
We deal with \cref{i:clustering} by clustering nearby solution candidates.
%
We employ a simple single-linkage hierarchical clustering
algorithm~\cite{Everitt2011}.
%
Given two candidate position-direction pairs $(\vx, \vr)$, we define the
distance as the maximum of their distances in position- and in direction space.
%
We start with each parameter region as a single cluster.
%
Two clusters are merged if the distance between any two elements from both
clusters is smaller than a clustering threshold $\varepsilon_{\mathrm{c}}$.
%
We repeat this process until the number of clusters no longer changes.
%

%
We then select the point in each cluster where the corresponding eigenvectors
are most parallel as the representative and discard the others.
%
Since we already have eigenvector directions for each point, we do not need to
explicitly compute them again.
%
Instead, we use the parallelism error
%
\begin{equation}
e_\mathrm{p} = {\left\| \frac{\mS(\vw) \vr}{\| \mS(\vw) \vr \|}
            \times \frac{\vr}{\| \vr \|} \right\|}
      + {\left\| \frac{\mT(\vw) \vr}{\| \mT(\vw) \vr \|}
            \times \frac{\vr}{\| \vr \|} \right\|} \, \text{,}
\end{equation}
%
which measures the deviation of $\vr$ from the true eigenvectors of both
$\mS(\vw)$ and $\mT(\vw)$.
%

This algorithm has a complexity of $\mathrm{O}(n^3)$ in the number of solution
candidates.
%%
Typically $n$ is small: less than \num{200} candidates are found in the vast
majority of cases.
%%
At this scale, the performance impact of the clustering algorithm is negligible.
%

%
In order to address \cref{i:false_positives}, we discard all candidate points
for which $e_\mathrm{p}$ is greater than some parallelism threshold
$\varepsilon_\mathrm{p}$.
%
This threshold can be chosen quite coarse (\eg~\num{0.01}), as $e_\mathrm{p}$ is
typically quite large for false positive candidate points.
%


%
In certain cases, the \ac{PEV} line might not intersect the triangle at a single
point.
%
This happens in the structurally unstable cases where eigenvectors are parallel
on a structure with a dimension larger than \num{1}, or where the \ac{PEV} line
is completely in the plane of the triangle.
%
In these cases, the recursive subdivision will not converge on isolated points and
slow down the algorithm considerably.
%
To mitigate this, we terminate the recursion if the number of triangle
subdivision operations exceeds a reasonable threshold.
%

%
Once we have clustered the candidate solutions and removed false positives, we
have a number of final \ac{PEV} points for each face of the mesh.
%
These \ac{PEV} points are now connected to lines on a cell-by-cell basis.
%
This problem is also faced when computing the \ac{PV} operator, where it has
been solved in a variety of ways using different heuristics, which can be
employed here as well.
%
In our implementation, we simply connect two points if they are the only
two intersections of a \ac{PEV} line with a grid cell.
%
In case of more than one intersection, we greedily connect pairs of points
that have the most similar parallel eigenvector directions, assuming that
\ac{PEV} lines are generally smooth relative to the grid resolution.
%
% subsection final_numerical_algorithm (end)
% section extracting_pev_lines (end)