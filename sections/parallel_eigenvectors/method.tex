
%
\section{Extracting \acs{PEV} Lines from Piecewise Linear Data} % (fold)
\label{sec:extracting_pev_lines}
%
We will now detail our algorithm for finding \ac{PEV} lines in piecewise linear
tensor fields.
%
We assume that both tensor fields are defined on the vertices of the same
tetrahedral mesh.
%
The general approach is to first find all intersections of \ac{PEV} lines with the
faces of the mesh, and then to connect those points to lines.
%

%
We showed that \ac{PEV} structures are lines in the structurally stable case.
%
It follows that their intersections with the triangular faces of a tetrahedral
mesh are isolated points.
%
Finding an analytic solution to the parallel eigenvector problem is impossible,
as it involves the intersection of cubic polynomials.
%
Instead, we opt for a numerical approach that is based on recursive subdivision
both on the triangle and in the space of possible eigenvector directions.
%

%
Our algorithm can be summarized as follows:
%
We first find a direction $\vr$ which becomes an eigenvector of both $\mS$ and
$\mT$ at some (possibly different) points inside the triangle.
%
If such a direction is found, we subdivide the triangle and check the parts for
possible eigenvector directions again.
%
We do this until we converge on a single point where both $\mS$ and $\mT$
have parallel eigenvectors.
%
% \begin{algorithm}
%     \caption{Find intersections of \ac{PEV} lines with a triangle}
%     \label{alg:find_pev_points}
%     \begin{algorithmic}[0]
%         \Function{FindPEV}{$\mS_\triangle$, $\mT_\triangle$, $\vx_\triangle$}
%             \State $\vr \gets $ \Call{FindEigenDir}{$\mS_\triangle$,
%                                                     $\mT_\triangle$}
%             \If{$\vr$ is \Null}
%                 \State \Return{\{\}}
%             \EndIf

%             \If{size of $\vx_\triangle < \epsilon_s$}
%                 \State \Return{$\{(\sfrac{1}{3}\sum{\vx_i}, \ \vr)\}$}
%                 \Comment{Location and eigenvector direction}
%             \EndIf

%             \State $l \gets \{\}$

%             \ForAll{$(\mS'_\triangle, \mT'_\triangle, \vx'_\triangle) \in$
%                    (\Call{Split}{$\mS_\triangle$},
%                     \Call{Split}{$\mT_\triangle$},
%                     \Call{Split}{$\vx_\triangle$})}
%                 \State $l \gets l \ \cup$
%                     \Call{FindPEV}{$\mS'_\triangle$,
%                                    $\mT'_\triangle$,
%                                    $\vx'_\triangle$}
%                 \Comment{Recursive subdivision}
%             \EndFor
%             \State \Return{$l$}
%         \EndFunction
%     \end{algorithmic}
% \end{algorithm}
%

%
In order to find a valid direction $\vr$, we perform another recursive search
in the space of possible eigenvector directions.
%
We represent this space as some triangulation of a hemisphere centered at the
origin.
%
For each triangle of directions, we have to decide whether it contains a valid
eigenvector direction, \ie, a direction that can become an eigenvector of both
$\mS$ and $\mT$ within the current sub-triangle in space.
%
If we are sure that there are no valid directions in the triangle, we can
discard it.
%
If we are sure that all directions within a triangle are valid directions, we
can terminate the recursion.
%
If we can not be sure whether the triangle contains valid directions, we
subdivide it and check the parts again.
%

%
Our algorithm is similar to the one described by Oster
\etal~\cite{Oster2018}.
%
Both are numerical algorithms that find singularities of polynomials of higher
degree.
%
However, the problem we solve in this work is different, as it does not involve
the derivatives of a tensor field, leading to a different algorithm.
%
In the following, we describe the details of this algorithm.
%

\subsection{Mathematical Basis} % (fold)
\label{sub:mathematical_basis}
%
A linear tensor field on a triangle is completely defined by the tensors at its
three corners.
%
We denote the set of corner points as \Todo{make notation consistent with tensor
core lines ($\vx_\triangle$ or $\triangle_{\vx}$)} $\vx_\triangle = \{\vx_1, \vx_2, \vx_3\}$, and the set of corner
tensors as $\mS_\triangle = \{\mS_1, \mS_2, \mS_3\}$ and $\mT_\triangle =
\{\mT_1, \mT_2, \mT_3\}$.
%
We express the tensor fields in barycentric coordinates $\vw = \T{(w_1, w_2,
w_3)}$:
%
\[
    \mS(\vw) = \sum_i{w_i\mS_i}\,\text{,} \quad
    \mT(\vw) = \sum_i{w_i\mT_i}\,\text{,} \quad
    \text{with}\quad \sum_i{w_i} = 1 \, \text{.}
\]
%
The position (in barycentric coordinates) at which an arbitrary direction $\vr$
becomes an eigenvector in $\mS$ is given by the solution to
%
\[
    \mS(\vw)\vr = \sum_i{w_i \mS_i \vr} = \lambda \vr \, \text{.}
\]
%
Rather than needing the exact position, we want to know if the position is
inside the triangle or not, \ie, if $\vr$ is a valid eigenvector direction for
$\mS$.
%
In barycentric coordinates, a point is inside the triangle if all $w_i > 0$.
%
Since the scaling factor $\lambda$ is arbitrary, we eliminate it:
%
\begin{gather}
    \sum_i{\tilde{w}_i \mS_i \vr} =
    \mA(\vr)\tilde\vw =
    \vr \, \text{,} \\
    \text{with} \quad
    \mA(\vr) = \begin{pmatrix} \mS_1 \vr & \mS_2 \vr & \mS_3 \vr \end{pmatrix}
    \, \text{,} \quad \tilde\vw = \vw / \lambda \,\text{,}
\end{gather}
%
and only require that all $\tilde{w}_i$ have the same sign.
%
Using Cramer's rule, we can give an analytic solution for the components of
$\tilde\vw$:
% 
\begin{equation}
    \tilde{w}_i = \frac{\det \mA_i(\vr)}{\det \mA(\vr)}
\end{equation}
% 
Here, $\mA_i$ denotes the matrix $\mA$ with its $i$-th column replaced by $\vr$.
% 
Note that all $\tilde{w}_i$ are divided by the same factor $\det \mA(\vr)$.
%
Since this influences all signs of $\tilde{w}_i$ equally it can be ignored,
leading to
%
\begin{equation}
    \label{eq:coordinate_functions}
    \hat{w}_i(\vr) = \det \mA_i(\vr) \, \text{.}
\end{equation}
%
The equations for $\mT$ are analogous.
% 
In the following, we show all equations for $\mS$ only.
% 
The equivalent equations for $\mT$ can be obtained trivially by substituting
$\mT$ for $\mS$.
% 
We denote the solutions for $\mS$ and $\mT$ by $\hat\vw_S$ and $\hat\vw_T$
respectively, whenever it is necessary to discriminate them.
%
% subsection mathematical_basis (end)

\subsection{Subdivision in Direction Space} % (fold)
\label{sub:subdivision_in_direction_space}
%
The core of the algorithm is to find a direction $\vr$ for which all components
of $\hat\vw_S(\vr)$ have the same sign, and all components of $\hat\vw_T(\vr)$
also have the same (but possibly different) sign.
%
This means that the direction $\vr$ becomes an eigenvector somewhere inside the
triangle for both $\mS$ and $\mT$.
%
Note that $\hat{w}_i(\vr)$ is cubic in $\vr$.
%
Finding an analytic solution for $\vr$ means analytically finding the
intersections of the roots of $\hat{w}_i(\vr)$, which is impossible.
%
%
Instead, we solve the problem numerically by applying another recursive search
in the space of all possible eigenvector directions.
%
The magnitude and orientation of $\vr$ is not significant.
%
We can therefore represent this space by some triangulation of a hemisphere
centered at the origin (\autoref{fig:algorithm}, right).
%
We again express a direction in a triangle $\vr_\triangle = \{\vr_1, \vr_2,
\vr_3\}$ on this hemisphere in barycentric coordinates $u_j$ of its corner
vectors:
%
\[
    \vr(\vu) = \sum_j{u_j\vr_j} \, \text{.}
\]
%
Substituting this in \eqref{eq:coordinate_functions}, the barycentric coordinate
functions now become
% 
\[
    \hat{w}_i(\vu) = \det \left( \sum_j{u_j\mA_i(\vr_j)} \right)\, \text{.}
\]
% 
% We express the relevant behavior of $\hat{w}_i(\vu)$ inside the triangle by
% the indicator function $s(\hat{w}_i(\vu))$:
% %
% \[
%     s(\hat{w}_i(\vu)) =
%     \begin{cases}
%         \hfill 1 \text{,} &
%         \text{if} \quad  \hat{w}_i(\vu) > 0 \quad
%             \forall \{\vu\; |\; u_j \geqslant 0\} \\
%         \hfill -1 \text{,} &
%         \text{if} \quad  \hat{w}_i(\vu) < 0 \quad
%             \forall \{\vu\; |\; u_j \geqslant 0\} \\
%         \hfill 0 \text{,} & \text{else.}
%     \end{cases}
% \]
% %
% This indicator tells us if the sign of $\hat{w}_i$ is consistent everywhere
% inside $\vr_\triangle$, and if it is positive or negative.
% %
% Computing $s_i$ means deciding if there are roots of $\hat{w}_i$ anywhere inside
% the triangle.
%
We can now express $\hat{w}_i$ in Bernstein-Bézier basis:
%
\[
    \hat{w}_i(\vu) = \sum_{\substack{j,\, k,\, l \geqslant 0 \, \text{,} \\ j+k+l=3}}{
        \frac{3!}{j! \, k! \, l!} \, u_1^j \, u_2^k \, u_3^l \cdot \alpha_{jkl}}
        \, \text{.}
\]
%
Here, $\alpha_{jkl}$ are the 10 coefficients needed to express a trivariate
polynomial of degree 3.
%
We use the property that a polynomial in Bernstein-Bézier form is bounded in its
domain by the convex hull of its coefficients \cite{Farin1997}.
%
This means that $\hat{w}_i$ is positive over the whole triangle if all
$\alpha_{jkl} > 0$, and negative over the whole triangle if all $\alpha_{jkl} <
0$.
%
If the $\alpha_{jkl}$ have different signs, $\hat{w}_i$ might become 0 somewhere
inside the triangle.
%

%
We use this when recursively subdividing the triangle $\vr_\triangle$.
%
If any $\hat{w}_i$ might have roots within the triangle according to the
Bernstein-Bézier coefficients, then we can not make a decision.
%
We need to subdivide the triangle and check the different parts again.
%
If no $\hat{w}_i$ can have roots within the triangle as indicated by the
coefficients, then there are two possibilities:
%
\begin{enumerate}
  \item \label{i:accept} All $\hat{w}_i$ have the same sign everywhere on the
  triangle
  \item \label{i:reject} The $\hat{w}_i$ have different signs everywhere on the
  triangle
\end{enumerate}
%
In case~\ref{i:accept}, all directions within the triangle become
eigenvector directions somewhere in $\vx_\triangle$ for both $\mS$ and $\mT$.
%
If this happens, we can accept any direction within the current triangle as a
possible solution.
%
In case~\ref{i:reject}, no direction within the triangle can become an
eigenvector of both $\mS$ and $\mT$, and the triangle is discarded.
%
% %
% This results in an indicator functions $s_i$, which tells us if the sign of
% $\hat{w}_i$ is consistent everywhere inside $\vr_\triangle$, and if it is
% positive or negative.:
% %
% \[
%     s_i =
%     \begin{cases}
%         \hfill 1 \text{,} &
%         \text{if} \quad  \alpha_{jkl}(\hat{w}_i) > 0 \quad  \forall \ j+k+l=3 \\
%         \hfill -1 \text{,} &
%         \text{if} \quad  \alpha_{jkl}(\hat{w}_i) < 0 \quad  \forall \ j+k+l=3 \\
%         \hfill 0 \text{,} & \text{else.}
%     \end{cases}
% \]
% %
% %
% If there are $i, j$ such that $s_i \neq 0$ and $s_j \neq 0$ with $s_i \neq s_j$,
% then the coordinate functions $\hat{w}_i(\vr)$ do not all have the same sign for
% any direction in $\vr_\triangle$.
% %
% This means that no direction in $\vr_\triangle$ will become an eigenvector in
% $\vx_\triangle$, and therefore $\vr_\triangle$ can be discarded.
% %
% If there are $i, j, k$ such that $s_i \neq 0$ and $s_i = s_j = s_k$ for both
% $\mS$ and $\mT$, then all directions in $\vr_\triangle$ will become eigenvectors
% somewhere in $\vx_\triangle$ for $\mS$ and $\mT$.
% %
% In this case, we have found valid eigenvector directions and can terminate the
% recursion.
% %
% If none of the above is the case, we have to subdivide $\vr_\triangle$ and
% investigate the parts again.
%
When the triangle becomes smaller than some subdivision threshold $\epsilon_r$,
and we still can not say for sure that there are no possible eigenvector
directions inside, we accept the central direction as a candidate.
%
% \begin{algorithm}
%     \caption{Find a direction that is an eigenvector in $\mS_\triangle$
%              and $\mT_\triangle$}
%     \label{alg:find_eigen_dir}
%     \begin{algorithmic}[0]
%         \Function{FindEigenDir}{$\mS_\triangle$, $\mT_\triangle$}
%             \State $R \gets $ set of triangles covering a hemisphere
%             \ForAll{$\vr_\triangle \in R$}
%                 \State $\vr \gets$ \Call{FindEigenDirRecursive}{$\mS_\triangle$,
%                                                                 $\mT_\triangle$,
%                                                                 $\vr_\triangle$}
%                 \If{$\vr$ is \KwNot \Null}
%                     \State \Return $\vr$
%                 \EndIf
%             \EndFor
%             \State \Return \Null
%         \EndFunction

%         \Function{FindEigenDirRecursive}{$\mS_\triangle$,
%                                          $\mT_\triangle$,
%                                          $\vr_\triangle$}
%             \State Compute $s_{S, i}$ and $s_{T, i}$
%             \If{$\max s_{S, i} \cdot \min s_{S, i} < 0$
%                     \KwOr $\max_i s_{T, i} \cdot \min s_{T, i} < 0$}
%                 \State \Return \Null
%             \ElsIf{($|\sum{s_{S, i}}| = 3$ \KwAnd $|\sum{s_{T, i}}| = 3$)
%                     \KwOr size of $\vr_\triangle < \epsilon_r$}
%                 \State \Return $\sfrac{1}{3}\sum{\vr_i}$
%             \Else
%                 \ForAll{$\vr'_\triangle \in$ \Call{Split}{$\vr_\triangle$}}
%                     \State $\vr \gets$ \Call{FindEigenDirRecursive}{$\mS_\triangle$,
%                                                                     $\mT_\triangle$,
%                                                                     $\vr'_\triangle$}
%                     \If{$\vr$ is \KwNot \Null}
%                         \State \Return $\vr$
%                     \EndIf
%                 \EndFor
%             \EndIf
%             \State \Return \Null
%         \EndFunction
%     \end{algorithmic}
% \end{algorithm}
%
\begin{figure}[tb]
    \centering
    \setlength\figurewidth\linewidth
    %
\begin{tikzpicture}[scale=\figurewidth/1cm*0.14, line join=round]
\tikzstyle{currentface} = [fill=mycolor3]
\tikzstyle{axes} = [arrows=-latex]
\tikzstyle{trilines} = [thick, fill=white]
\tikzstyle{back} = [gray]

\coordinate (x1) at (90:1);
\coordinate (x2) at (-30:1);
\coordinate (x3) at (210:1);

\coordinate (x21) at ($0.5*(x1) + 0.5*(x2)$);
\coordinate (x22) at ($0.5*(x2) + 0.5*(x3)$);
\coordinate (x23) at ($0.5*(x3) + 0.5*(x1)$);

\coordinate (x31) at ($0.5*(x21) + 0.5*(x2)$);
\coordinate (x32) at ($0.5*(x2) + 0.5*(x22)$);
\coordinate (x33) at ($0.5*(x22) + 0.5*(x21)$);

\draw [trilines] (x1) -- (x2) -- (x3) -- cycle;
\draw [trilines] (x21) -- (x22) -- (x23) -- cycle;
\draw [trilines, currentface] (x31) -- (x32) -- (x33) -- cycle;

\draw [thin] (1.8, 1.5) -- (5.1, 1.5) -- (5.1, -1) -- (1.8, -1) -- cycle;
\draw [thin] (x33) -- (1.8, 1.5);
\draw [thin] (x32) -- (1.8, -1);

% \node [above] at (x1) {$\vx_1, \mS_1, \mT_1$};
% \node [below] at (x2) {$\vx_2, \mS_2, \mT_2$};
% \node [below] at (x3) {$\vx_3, \mS_3, \mT_3$};
\node [below] at (x32) {$\vx_\triangle$};

\begin{scope}[shift={(3.5, -0.2)}]

    \coordinate (null) at (0, 0, 0);
    \coordinate (px) at (1, 0, 0);
    \coordinate (mx) at (-1, 0, 0);
    \coordinate (pz) at (0, 0, 1);
    \coordinate (mz) at (0, 0, -1);
    \coordinate (py) at (0, 1, 0);

    \coordinate (pxpz) at ($0.5*(px) + 0.5*(pz)$);
    \coordinate (pzmx) at ($0.5*(pz) + 0.5*(mx)$);
    \coordinate (mxmz) at ($0.5*(mx) + 0.5*(mz)$);
    \coordinate (mzpx) at ($0.5*(mz) + 0.5*(px)$);
    \coordinate (pxpy) at ($0.5*(px) + 0.5*(py)$);
    \coordinate (pzpy) at ($0.5*(pz) + 0.5*(py)$);
    \coordinate (mxpy) at ($0.5*(mx) + 0.5*(py)$);
    \coordinate (mzpy) at ($0.5*(mz) + 0.5*(py)$);

    \coordinate (x1) at ($0.5*(px) + 0.5*(pxpz)$);
    \coordinate (x2) at ($0.5*(pxpz) + 0.5*(pxpy)$);
    \coordinate (x3) at ($0.5*(px) + 0.5*(pxpy)$);

    \draw [axes] ($1.5*(mx)$) -- ($1.5*(px)$);
    \draw [axes] ($1.5*(mz)$) -- ($1.5*(pz)$);
    \draw [axes] (null) -- ($1.5*(py)$);

    \draw [trilines, back] (mx) -- (mz) -- (py) -- cycle;
    \draw [trilines, back] (mz) -- (px) -- (py) -- cycle;
    \draw [trilines] (px) -- (pz) -- (py) -- cycle;
    \draw [trilines] (pz) -- (mx) -- (py) -- cycle;

    \draw [trilines] (pxpz) -- (pzpy) -- (pxpy) -- cycle;
    \draw [trilines] (pzmx) -- (mxpy) -- (pzpy) -- cycle;

    \draw [trilines, currentface] (x1) -- (x2) -- (x3) -- cycle;

    \node [below] at (x1) {$\vr_\triangle$};

\end{scope}

\end{tikzpicture}
    \caption{Two-level recursion scheme for finding intersections of \ac{PEV} lines
             with the faces of piecewise linear tensor fields. For each
             sub-triangle $\vx_\triangle$, a recursive search in the space of
             possible eigenvector directions is performed to find a direction
             $\vr$ that becomes an eigenvector of both $\mS$ and $\mT$ within
             $\vx_\triangle$.}
    \label{fig:algorithm}
\end{figure}
%
% subsection subdivision_in_direction_space (end)

\subsection{Final Numerical Algorithm} % (fold)
\label{sub:final_numerical_algorithm}
%
The complete algorithm for finding intersections of \ac{PEV} lines with a triangle
of the dataset now works as follows:
%
Start with the complete triangle as $\vx_\triangle$.
%
Then, search for a direction that becomes an eigenvector of both $\mS$ and $\mT$
somewhere inside the triangle by using the algorithm described in
\autoref{sub:subdivision_in_direction_space}\Todo{looks weird}.
%
If such a direction is found, subdivide the triangle and process the parts
recursively.
%
If no direction is found, discard the triangle.
%
When a spatial sub-triangle becomes smaller than a subdivision threshold
$\epsilon_s$, we accept the center of the triangle as a solution candidate.
%

%
The result of the algorithm is a list of points on $\vx_\triangle$ with
corresponding eigenvector directions.
%
This list of points has to be post-processed for two reasons:
%
\begin{enumerate}
    \item \label{i:clustering}
          %
          For each intersection of the \ac{PEV} line with the triangle, multiple
          adjacent candidate points may be found.
          %
          This happens if eigenvectors of $\mS$ and $\mT$ are closer than
          $\epsilon_r$ in a region larger than $\epsilon_s$, \eg, because the
          gradient of the tensor fields is very small, or because the \ac{PEV} line
          intersects the face at a very steep angle.
          %
          Choosing $\epsilon_r$ very small helps with this, but it can not be
          avoided in the presence of limited numerical precision on a computer.
          %
    \item \label{i:false_positives}
          %
          A candidate point might not be a \ac{PEV} point at all.
          %
          These false positives occur if there are directions $\vr$ that become
          eigenvectors of one of the tensor fields inside $\vx_\triangle$, while
          $\mA(\vr)$ has rank 1 for the other tensor field.
          %
          For this case, $\hat{\vw} = \vNull$, which means that a consistent
          sign of all components can never be determined, and subdivision can
          not be terminated early, even if the tensor field does not have any
          valid eigenvector directions inside $\vx_\triangle$.
          %
\end{enumerate}
% 
We deal with \autoref{i:clustering} by clustering candidate points that are
closer than a certain distance threshold $\epsilon_c$.
%
We employ a simple single-linkage hierachical clustering
algorithm~\cite{Everitt2011}.
%
Each candidate point starts as a separate cluster.
%
Clusters are merged if the smallest distance between them is smaller than
$\epsilon_c$.
%
This is repeated until the number of clusters converges.
%
The clustering algorithm is the same as the one used by Oster
\etal~\cite{Oster2018} for a similar purpose.
%
We then select the point in each cluster where the corresponding eigenvectors
are most parallel as the representative and discard the others.
%
Since we already have eigenvector directions for each point, we do not need to
explicitly compute them again.
%
Instead, we use the parallelity error
%
\[
e_p = {\left\| \frac{\mS(\vw) \vr}{\| \mS(\vw) \vr \|}
            \times \frac{\vr}{\| \vr \|} \right\|}
      + {\left\| \frac{\mT(\vw) \vr}{\| \mT(\vw) \vr \|}
            \times \frac{\vr}{\| \vr \|} \right\|} \, \text{,}
\]
%
which measures the deviation of $\vr$ from the true eigenvectors of both
$\mS(\vw)$ and $\mT(\vw)$.
%

%
In order to address \autoref{i:false_positives}, we discard all candidate points
for which $e_p$ is greater than some parallelity threshold $\epsilon_p$.
%
This threshold can be chosen quite coarse (\eg~0.01), as $e_p$ is typically
quite large for false positive candidate points.
%


%
In certain cases, the \ac{PEV} line might not intersect the triangle at a single
point.
%
This happens in the structurally unstable cases where eigenvectors are parallel
on a structure with a dimension larger than 1, or where the \ac{PEV} line is
completely in the plane of the triangle.
%
In these cases, a recursive subdivision will not converge on isolated points and
slow down the algorithm considerably.
%
To mitigate this, we terminate the recursion if the number of triangle
subdivision operations exceeds a reasonable threshold.
%

%
Once we have clustered the candidate solutions and removed false positives, we
have a number of final \ac{PEV} points for each face of the mesh.
%
These \ac{PEV} points are now connected to lines on a cell-by-cell basis.
%
This problem is also faced when computing the \ac{PV} operator, where it has been
solved in a variety of ways using different heuristics, which can be employed
here as well.
%
In our implementation, we simply connect two points if they are the only
two intersections of a \ac{PEV} line with a grid cell.
%
In case of more than one intersection, we greedily connect pairs of points
that have the most similar parallel eigenvector directions, assuming that
\ac{PEV} lines are generally smooth relative to the grid resolution.
%
% subsection final_numerical_algorithm (end)
% section extracting_pev_lines (end)