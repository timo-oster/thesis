
%
\section{Discussion} % (fold)
\label{sec:pev_discussion}
%
The \ac{PEV} operator was introduced as a generic operator, its interpretation is
dependent on the application scenario.
%

%
For stress tensors in mechanical engineering, the \ac{PEV} operator gives insight
into the alignment of the tensors under different acting forces.
%
Areas with \ac{PEV} lines can be wanted or unwanted.
%
In areas with present \ac{PEV} lines, the stress tensor is similarly oriented for
different external forces.
%
This could be used \eg, for deciding the placement of structural reinforcements
or to guide the selection of materials.
%
In regions without \ac{PEV} lines, there is no stress in a preferred direction when
applying different outer forces and material with a more isotropic behavior
could be used.
%

%
Besides this particular interpretation, there are general interpretations that
are common to all applications of the \ac{PEV} operator.
%
The \ac{PEV} operator is agnostic to isotropic scaling of the tensors.
%
It gives information about the orientation of the tensors only.
%
In this way, the \ac{PEV} operator can be seen as an addition to many standard
measures for comparing tensors like norm, trace, or eigenvalues.
%

%
The presented algorithm for piecewise linear tensor fields does not use any
derivatives of the data.
%
It depends on a number of thresholds to guide subdivision levels and filtering.
%
The spatial subdivision threshold $\epsilon_d$ influences the accuracy of the
resulting \ac{PEV} lines.
%
A small threshold means more subdivisions and is one of the main factors
influencing performance.
%
Since subdivision converges to single points, the computing time increases
logarithmically when decreasing $\epsilon_d$.
%

%
The directional subdivision threshold $\epsilon_r$ guides the accuracy of the
obtained eigenvector direction.
%
Typically, the smaller the current spatial triangle $\vx_\triangle$ becomes,
the smaller the region of valid eigenvector directions.
%
For increasing subdivision level in space, the valid eigenvector directions
will converge on a point.
%
This means that for small $\epsilon_d$, the recursion in the space of directions
will generally proceed to the highest subdivision level.
%
The influence of $\epsilon_r$ on computation time is about the same as for
$\epsilon_d$.
%
However, an accurate determination of eigenvector direction is essential to
decrease the number of candidate \ac{PEV} solutions that have to be clustered.
%
This means that $\epsilon_r$ should be chosen very small.
%
We find $\epsilon_r = 1 \times 10^{-9}$ to be a choice that provides
consistently good results.
%

%
Because our algorithm can produce multiple candidate points for an intersection
of a \ac{PEV} line with a tetrahedral face, we need to cluster the results.
%
Theoretically, all candidate points should be in adjacent triangles, as
(unoriented) eigenvector directions in linear tensor fields do not oscillate
on small scales.
%
However, due to numerical noise and rounding errors on a computer, some
candidate triangles might not be exactly adjacent to each other.
%
To bridge this gap, the clustering threshold $\epsilon_c$ defines the radius in
which two candidate solutions are considered to belong to the same cluster.
%
Because the numerical noise influencing the size of gaps between candidate
solutions is random, we do not expect candidates to be more than two or three
lengths of $\epsilon_d$ from each other.
%
We recommend to set $\epsilon_c$ to some fixed multiple of $\epsilon_d$.
%
In our experiments, $\epsilon_c = 5 \epsilon_d$ proved sufficient for all
datasets.
%

%
The parallelism threshold $\epsilon_p$ is used to weed out false positive
candidates that are a byproduct of our algorithm.
%
It must be chosen carefully to separate false positive solutions from numerical
errors.
%
Because of this threshold, the spatial subdivision threshold $\epsilon_d$ can
not be chosen arbitrarily large.
%
The larger $\epsilon_d$, the larger the possible difference in eigenvector
direction between the real \ac{PEV} point and the tensor at the center of the
triangle, which is chosen as a representative.
%
In general, the choice of $\epsilon_p$ is dependent on $\epsilon_d$.
%
More spatial subdivision levels enable a smaller choice of the parallelism
threshold.
%
In our experiments, a choice of $\epsilon_p = 1 \times 10^{-2}$ for
$\epsilon_d = 1 \times 10^{-3}$, and $\epsilon_p = 1 \times 10^{-3}$ for
$\epsilon_d = 1 \times 10^{-6}$ worked very well.
%