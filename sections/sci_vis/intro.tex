\lettrine[lines=3, findent=-3pt, nindent=4pt, loversize=0.02]{S}{cientific
visualization} is the visualization of data from the natural sciences.
%
Although such data can come in a variety of forms, the term is most often used
to describe the techniques used for displaying spatial, possibly time-varying
data from physics, chemistry, engineering, or biomedical applications.
%
Examples for such data are volumetric scans of patients from medicine, wind
speed and pressure measurements from meteorology, and simulations of the air
flow around a car.
%

%
This thesis presents several new scientific visualization techniques.
%
The techniques are applied to simulation data from two different engineering
domains: turbulent combustion and solid mechanics.
%
To set the scene for these contributions, we will therefore use this chapter to
give an overview of the most important methods for visualizing scientific data.
%
Such data usually comes in the form of scalar-, vector-, or (second-order)
tensor fields.
%
We will take a look at the definition and basic visualization techniques for
each of these types of fields in the following sections.
%

%
This chapter is necessarily sparse on details and omits a lot of basic knowledge
on computer graphics, data representation, numerical algorithms, and
visualization in general.
%
A more thorough introduction to the field is given in Alexandru Telea's book
``Data Visualization''~\cite{Telea2014}.
%
Material for further reading can be found in the references cited throughout the
text.
%