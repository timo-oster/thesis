\documentclass[a4paper]{article}

\usepackage[cm]{fullpage}
\usepackage{fancyhdr}

\usepackage{mathptmx}
\usepackage{amsfonts}
\usepackage{graphicx}
\usepackage{times}

\usepackage{mathtools}

\usepackage{import}
\usepackage{pgfplots}
%\pgfplotsset{compat = 1.5}
\usepackage{pgfplotstable}

\usepackage[lofdepth, lotdepth]{subfig}

\usepackage[version=3]{mhchem}
\usepackage{caption}

\pagestyle{fancy}
\fancyhead[C]{EuroVis Online Submission 104: Additional Material}

\newlength\figureheight
\newlength\figurewidth

\title{EuroVis Online Submission 104: Additional Material}
\date{}
\author{}

\begin{document}
% \maketitle

% \section*{Details for Visualization Approach}

% Put here:
% \begin{itemize}
% 	\item More detailed explanation of diffusion process?
% 	\item Details on constructing the feature surface from $M$ and the shift values?
% 	\item Details on obtaining the difference values
% 	\item computation of the color mapping from difference values
% 	\item Computation of morphing with $u_2$
% \end{itemize}

\begin{figure}[t]
\centering
	\setlength\figureheight{4cm} 
	\setlength\figurewidth{0.5\textwidth}
	\input{figures/isovalue_variation.tikz}
\caption{Hausdorff distance of isosurface resulting from different isovalues
of the combustion progress variable to the 0.5-isosurface. Plot shows the mean
and standard deviation of data set \textsc{Hydrogen} over eight time steps of
the more complex stage of the simulation. The distance values were divided by
the length of the longest side of the simulation domain. It can be seen that the
Hausdorff distance is less than 1.5\% of the simulation domain size when varying
the isovalue by $\pm 0.1$, which is 20\% of the range of possible isovalues.
This shows that the choice of the isovalue for extracting the flame surface is
very robust.}
	\label{fig:figure6}
\end{figure}

\begin{figure}[t]
	\begin{center}
		\def\svgwidth{0.85\linewidth}
		\import{figures/}{Hydrogen_dataset_variables.pdf_tex}
	\end{center}
	\caption{Overview over all variables of dataset \textsc{Hydrogen} for the last
	and most complex time step.}
	\label{fig:hydrogen_variables}
\end{figure}

\begin{figure}[t]
	\begin{center}
		\includegraphics[width=0.85\textwidth]{figures/Additional_Material_I.pdf}
	\end{center}
	\caption*{Comparison of reconstruction quality for selected Variables in \textsc{Syngas I}}
	\label{fig:syngas1_variables}
\end{figure}

\begin{figure}[t]
	\begin{center}
		\includegraphics[width=0.85\textwidth]{figures/Additional_Material_II.pdf}
	\end{center}
	\caption*{Comparison of reconstruction quality for selected Variables in \textsc{Syngas II}}
	\label{fig:syngas2_variables}
\end{figure}

\begin{figure}[t]
	\begin{center}
		\includegraphics[width=0.85\textwidth]{figures/Additional_Material_III.pdf}
	\end{center}
	\caption*{Comparison of reconstruction quality for all Variables in \textsc{Syngas III}}
	\label{fig:syngas3_variables}
\end{figure}

\begin{figure}[t]
	\setlength\figureheight{19cm} 
	\setlength\figurewidth{0.33\textwidth}
\begin{tabular}{cc}
	\input{figures/msevcr_stack.tikz} &
	\input{figures/msevcr_stack_data3.tikz}
\end{tabular}
\caption{Error vs. reduction ratio for all variables.
	Left: \textsc{Syngas I}, right: \textsc{Syngas II}.
	The plots corresponding to each variable are shifted by a constant
	increment. The horizontal lines signify the zero-levels of the respective 
	plots.}
	\label{fig:figure4}
\end{figure}

% \begin{figure}[t]
% 	\begin{tabular}{cc}
% 		\includegraphics[width=0.45\textwidth]{figures/0-10} & 
% 		\includegraphics[width=0.45\textwidth]{figures/2-4}\\
% 		\includegraphics[width=0.45\textwidth]{figures/5-12} &
% 		\includegraphics[width=0.45\textwidth]{figures/10-12}
% 	\end{tabular}
% 	\caption*{Differences between different feature surfaces mapped
% 	on the flame surface. Top left: T (temperature) vs. \ce{O},
% 	top right: \ce{CO} vs. \ce{H},
% 	bottom left: \ce{H2} vs. \ce{OH},
% 	bottom right: \ce{O} vs. \ce{OH}}
% 	\label{fig:figure5}
% \end{figure}

\end{document}