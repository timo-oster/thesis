\section{A Sparse Representation for Premixed Flames}
\label{sec:compression}
%
As mentioned before, a flamelet lives on \ac{1D} lines orthogonal to the flame
surface.
%
If we extract the flame surface and store all flamelets as \ac{1D} profiles of
the simulation variables orthogonal to this surface, we have a full
representation of the flame front.
%
Since the variation of the simulation variables is typically small tangential to
the surface, we can store a sparse selection of flamelets and still represent
the flame with adequate accuracy.
%
This is the idea behind our sparse representation for premixed flames.
%

%
The transformation of raw \ac{DNS} data into our sparse representation consists
of three steps:
%
\begin{enumerate}
	\item Seed points on the flame surface.
	\item Sample the simulation variables along lines orthogonal to the flame
	surface, emanating from the seed points.
	\item Approximate the resulting profiles by models, reducing each profile to
	a set of model parameters.
\end{enumerate}
%

%
We extract the flame surface as an isosurface and distribute seed points on it
using random sampling adaptive to the surface curvature.
%
\ac{1D} profiles are then extracted orthogonal to the flame surface.
%
The profiles are centered on the surface points and have a limited length to
include the maximum flame thickness.
%
In this way, only the data of the flame front is captured, and the large, almost
constant areas containing fresh gases and hot products are discarded.
%
We only store the profiles of variables related to chemistry, namely the
temperature, heat release and species mass fractions.
%
We ignore the velocity and pressure fields, as they do not exhibit the same
structure as the chemical variables and can not be accurately represented by
\ac{1D} profiles in the flame front only.
%

%
Variables are grouped into three classes: \emph{Reactants} have high
values outside of the burnt region and low values inside,
\emph{products} are the opposite, and \emph{intermediates} occur near the flame
surface between unburnt and burnt regions but have low values on either side.
%
This behavior can be modeled with few degrees of freedom, reducing the amount of
data even further.
%
The information that has to be stored in the sparse representation consists of
the locations and directions of the profile lines, the model parameters for each
variable and profile line, and the full flame surface mesh for visualization.
%
% The compression $C$ then becomes a transformation from a set
% of simulation variables $(V_1, \dotsc, V_v)$ to a tuple consisting of the
% locations and directions of the profile lines, $(\vp_i, \vr_i)$, and
% the model parameters for each variable and line, $\Phi$:
% \begin{equation}
% 	\begin{aligned}
% 		C(V_1, \dotsc, V_v) &= ((\vp_i, ~\vr_i), ~\Phi^{V_1}, \dotsc, \Phi^{V_v}), ~ i = 1, \dotsc, n\\
% 		\text{with } \Phi^V &= \phi^V_{ij}, ~ j = 1, \dotsc, m \text{ .}\\
% 	\end{aligned}
% \end{equation}
% Here, the set of model parameters for all lines of variable $V$ is denoted as
% $\Phi^V \in \R^{n \times m}$, which is a matrix with entries $\phi^V_{ij}$,
% denoting the $j$-th model parameter of the $i$-th profile line.
%
We now describe how the lines are seeded, how the profiles are extracted from
those lines, and how these profiles are then approximated by simple models.
% 
\subsection{Strategy for Seeding Profile Lines} % (fold)
\label{sub:seeding}
%
\begin{figure}[t]
	\setlength\figurewidth\textwidth
	\centering
	\begin{tikzpicture}[
    thick,
    font=\small,
    flamesurface/.style={black},
    point/.style={fill=white, draw=black},
    max/.style={fill=mycolor4},
    min/.style={fill=mycolor3},
    infl/.style={fill=mycolor1},
    arrow/.style={-latex, very thick},
    legend line/.style={dash pattern=on 3pt off 3pt, shorten <=1mm},
    legend dark/.style={legend line, black},
    legend light/.style={legend line, dash phase=3pt, white},
]
    % width of the profile plots
    \newlength{\plotwidth}
    \setlength{\plotwidth}{0.4\figurewidth-0.4cm}

    % Scalar field
    \node[anchor=south west, inner sep=0,outer sep=0] (img) at (0, 0) {
        \includegraphics[trim=107 22 20 105,
                         clip,
                         width=0.6\figurewidth]{images/CO2_orig_rdoryl}
    };

    % flame surface and coordinates
    \begin{scope}[
        x={(img.south east)},
        y={(img.north west)},
        shift={(img.south west)}
    ]
        % \draw[help lines, opacity=0.5, xstep=.01,ystep=.01] (0,0) grid (1,1);
        % \draw[help lines,xstep=0.1,ystep=0.1] (0,0) grid (1,1);
        % \foreach \x in {0,...,9} { \node [anchor=north] at (\x/10,0) {0.\x}; }
        % \foreach \y in {0,...,9} { \node [anchor=east] at (0,\y/10) {0.\y}; }

        \draw[flamesurface] plot [smooth] coordinates {
            (0, 0.245)
            (0.04, 0.26)
            (0.06, 0.30)
            (0.08, 0.36)
            (0.1, 0.4)
            (0.14, 0.41)
            (0.19, 0.415)
            (0.23, 0.44)
            (0.27, 0.47)
            (0.31, 0.505)
            (0.36, 0.51)
            (0.4, 0.49)
            (0.425, 0.45)
            (0.435, 0.41)
            (0.435, 0.38)
            (0.445, 0.345)
            (0.49, 0.34)
            (0.535, 0.37)
            (0.575, 0.41)
            (0.61, 0.47)
            (0.63, 0.5)
            (0.67, 0.52)
            (0.71, 0.54)
            (0.73, 0.58)
            (0.725, 0.63)
            (0.7, 0.66)
            (0.62, 0.73)
            (0.585, 0.77)
            (0.565, 0.81)
            (0.56, 0.85)
            (0.575, 0.88)
            (0.61, 0.9)
            (0.67, 0.905)
            (0.715, 0.91)
            (0.76, 0.95)
            (0.79, 1)};


        \coordinate (p1) at (0.04, 0.26);
        \coordinate (p2) at (0.08, 0.36);
        \coordinate (p3) at (0.14, 0.41);
        \coordinate (p4) at (0.23, 0.44);
        \coordinate (p5) at (0.31, 0.505);
        \coordinate (p6) at (0.4, 0.49);
        \coordinate (p7) at (0.435, 0.41);
        \coordinate (p8) at (0.445, 0.345);
        \coordinate (p9) at (0.535, 0.37);
        \coordinate (p10) at (0.61, 0.47);
        \coordinate (p11) at (0.67, 0.52);
        \coordinate (p12) at (0.73, 0.58);
        \coordinate (p13) at (0.7, 0.66);
        \coordinate (p14) at (0.585, 0.77);
        \coordinate (p15) at (0.56, 0.85);
        \coordinate (p16) at (0.61, 0.9);
        \coordinate (p17) at (0.715, 0.91);

    \end{scope}

    % profile for point 1
    \begin{scope}[
            shift={($(img.south east)!0.66!(img.north east) + (0.3cm, 0.4cm)$)},
            x={\plotwidth},
            y={0.15\figurewidth}
        ]
        \coordinate (pl1) at (0, 0);
        \draw[arrow] (0, 0) -- (1, 0);
        \draw[point] (0.5, 0) circle (0.8mm) node [below] {$\vp_1$};
        \draw[-latex] (0, 0) -- (0, 1)
            node[right, yshift=1mm] {\ce{CO2}};
        \draw (0, 0.9) to[out=0, in=130]
              (0.5, 0.5) to[out=-50, in=180]
              (0.75, 0) --
              (0.9, 0) to[out=0, in=200]
              (1, 0.1);
        \draw[point, infl] (0.5, 0.5) circle (0.8mm);
        \draw[point, max] (0, 0.9) circle (0.8mm);
        \draw[point, min] (0.75, 0.0) circle (0.8mm);
    \end{scope}

    % profile for point 2
    \begin{scope}[
            shift={($(img.south east)!0.33!(img.north east) + (0.3cm, 0.4cm)$)},
            x={\plotwidth},
            y={0.15\figurewidth}
        ]
        \coordinate (pl2) at (0, 0);
        \draw[arrow] (0, 0) -- (1, 0);
        \draw[point] (0.5, 0) circle (0.8mm) node [below] {$\vp_2$};
        \draw[-latex] (0, 0) -- (0, 1)
            node[right, yshift=1mm] {\ce{CO2}};
        \draw (0, 0.75) to[out=-5, in=140]
              (0.5, 0.5) to[out=-40, in=180]
              (0.8, 0) --
              (1, 0);
        \draw[point, infl] (0.5, 0.5) circle (0.8mm);
        \draw[point, max] (0, 0.75) circle (0.8mm);
        \draw[point, min] (0.8, 0.0) circle (0.8mm);
    \end{scope}

    % profile for point 3
    \begin{scope}[
            shift={($(img.south east)!0.00!(img.north east) + (0.3cm, 0.4cm)$)},
            x={\plotwidth},
            y={0.15\figurewidth}
        ]
        \coordinate (pl3) at (0, 0);
        \draw[arrow] (0, 0) -- (1, 0);
        \draw[point] (0.5, 0) circle (0.8mm) node [below] {$\vp_3$};
        \draw[-latex] (0, 0) -- (0, 1)
            node[right, yshift=1mm] {\ce{CO2}};
        \draw (0, 0.75) to[out=10, in=180]
              (0.25, 0.85) to[out=0, in=130]
              (0.5, 0.5) to[out=-50, in=180]
              (0.8, 0) --
              (1, 0);
        \draw[point, infl] (0.5, 0.5) circle (0.8mm);
        \draw[point, max] (0.25, 0.85) circle (0.8mm);
        \draw[point, min] (0.8, 0.0) circle (0.8mm);
    \end{scope}

    % arrows
    \begin{scope}
        \clip (img.south east) rectangle (img.north west);

        \draw[arrow] (p1) -- +(130:1.2cm) -- +(310:1.2cm);
        \draw[arrow] (p2) -- +(155:1.2cm) -- +(335:1.2cm);
        \draw[arrow] (p3) -- +(93:1.2cm) -- +(273:1.2cm);
        \draw[arrow] (p4) -- +(125:1.2cm) -- +(305:1.2cm);
        \draw[arrow] (p5) -- +(105:1.2cm) -- +(285:1.2cm);
        \draw[arrow] (p6) -- +(55:1.2cm) -- +(235:1.2cm);
        \draw[arrow] (p7) -- +(7:1.2cm) -- +(187:1.2cm);
        \draw[arrow] (p8) -- +(60:1.2cm) -- +(240:1.2cm);
        \draw[arrow] (p9) -- +(130:1.2cm) -- +(310:1.2cm);
        \draw[arrow] (p10) -- +(145:1.2cm) -- +(325:1.2cm);
        \draw[arrow] (p11) -- +(115:1.2cm) -- +(295:1.2cm);
        \draw[arrow] (p12) -- +(175:1.2cm) -- +(355:1.2cm);
        \draw[arrow] (p13) -- +(230:1.2cm) -- +(50:1.2cm);
        \draw[arrow] (p14) -- +(215:1.2cm) -- +(35:1.2cm);
        \draw[arrow] (p15) -- +(165:1.2cm) -- +(345:1.2cm);
        \draw[arrow] (p16) -- +(100:1.2cm) -- +(280:1.2cm);
        \draw[arrow] (p17) -- +(125:1.2cm) -- +(305:1.2cm);
    \end{scope}

    % color bar
    \node[draw=black,
          inner sep=0.05mm,
          below right=0.3cm and 0.3cm of img.north west] (cscale) {
        \rotatebox{-90}{%
            \includegraphics[width=2.5cm, height=0.25cm]{figures/rdoryl.png}%
        }
    };
    \node[anchor=south west, inner sep=0, outer sep=0, xshift=1mm] at (cscale.south east) {
        \small min
    };
    \node[anchor=north west, inner sep=0, outer sep=0, xshift=1mm] at (cscale.north east) {
        \small max
    };
    \node[anchor=north] at (cscale.south) {\small \ce{CO2}};

    % points
    \foreach \i in {1,...,17} {
        \draw[point] (p\i) circle (0.8mm);
    }

    % marking circles
    \begin{scope}[
        x={(img.south east)},
        y={(img.north west)},
        shift={(img.south west)}
    ]
        \draw[mycolor1, ultra thick] (0.48, 0.385) circle (0.6cm);
        \draw[mycolor1, ultra thick, rotate around={-43: ($(p13)!0.5!(p14)$)}]
            ($(p13)!0.5!(p14)$) circle (0.78cm and 0.25cm);
    \end{scope}


    % profile plot connecting lines
    \draw[legend dark] (p14) |- (pl1);
    \draw[legend dark] (p12) |- (pl2);
    \draw[legend dark] (p9) |- (pl3);

    % \draw[legend light] (p14) |- (pl1);
    % \draw[legend light] (p12) |- (pl2);
    % \draw[legend light] (p9) |- (pl3);

    % point labels
    \node[above left=-0.1cm and 0.05cm of p14] {$\vp_1$};
    \node[above right=-0.05cm and -0.07cm of p12] {$\vp_2$};
    \node[right=0.1cm of p9] {$\vp_3$};
\end{tikzpicture}%
	\vspace*{-6mm}
	\tikzset{external/export=false}
	\caption{
		Cross section scheme of the flame surface with seeded points $\vp$.
		Profiles are sampled from profile lines $\vp(t)$ (arrows) at different
		locations on the flame surface. The minimum
		(\protect\tikz\protect\draw[thick, fill=mycolor3] circle
		[radius=0.8mm];), inflection (\protect\tikz\protect\draw[thick,
		fill=mycolor1] circle [radius=0.8mm];) , and maximum
		(\protect\tikz\protect\draw[thick, fill=mycolor4] circle
		[radius=0.8mm];) of the sigmoid shape are indicated on the profiles. }
	\label{fig:flamecrossing}
	\tikzset{external/export=true}
\end{figure}
%
Commonly, the flame surface in premixed combustion is defined as the
0.5-isosurface of a \emph{combustion progress variable}~\cite{Poinsot2012}.
%
This variable is defined as $({T-T_u})/({T_b-T_u})$, where $T_u$ and $T_b$ are
the temperature of the burnt and unburnt gases, and varies between 0 and 1.
%
The choice of the iso-value is very robust for premixed combustion in the
flamelet regime.
%
Our experiments show that variations of $\pm 0.1$ lead to isosurfaces with
Hausdorff distances less than 2\% of the domain size.
%

%
Anchor points $\vp_i$ for profile extraction are distributed on this surface,
which we extract using \ac{CGAL}~\cite{Boissonnat2005}.
%
This yields a mesh with approximately uniformly-spaced vertices and edges of
approximately the same length as the voxel size in the original data.
%
%We first place a point into every surface voxel. 
%
% Sub-voxel sampling would not yield more information due to the smoothness of \ac{DNS}
% data. 
%
Since this results in a large number of vertices, we select only some of them as
seed points, using a rejection sampling based on local surface curvature.
%
%This results in a large number of seed points. We therefore reduce the number
%of points by a random process based on the local surface curvature. 
%
%We use a method described by Kindlmann et al.~\cite{Kindlmann2003} to 
%
We estimate the principal curvatures $\kappa_1$ and $\kappa_2$ at each vertex
(see, e.g., \cite{Kindlmann2003}).
%
The curvatures are then transfomed into a seeding density by a logarithmic
function:
%
\begin{equation}
	\rho(\kappa_1, \kappa_2) =
		\frac{\ln(1 + \sqrt{\kappa_1^2+\kappa_2^2})}{q}\text{ .}
\end{equation}
%
Here, $q > 1$ steers the seeding density.
%
For each initial vertex, a uniformly distributed random number $r \in (0, 1)$ is
now generated, and the point is selected if $\rho < r$.
%
This results in few seed points in areas without curvature, and all vertices
being selected in areas where $\rho > 1$.
%
Areas of higher curvature get more seeds than less curved ones (see blue circles
in \cref{fig:flamecrossing}).
%
Hence, the storage size of the sparse representation depends on the surface
shape more than on the resolution of the data.
%
Adjusting $q$ changes the total number of seed points and balances size vs.
quality.
%
% The logarithmic shape of
% $\rho$ ensures that medium curvatures, which are most frequent, are not
% discarded as often, and sampling density is more even across areas of higher
% curvature. By allowing the density of the maximum curvature to drop below 1, it
% becomes possible to avoid overly dense seeding in high-curvature areas.
%
%\begin{figure}[tb]
%	\begin{center}
%		\includegraphics[width=\linewidth]{figures/lineseeding.pdf}
%	\end{center}
%	\caption{
%	2D example of flame surface with seeded points. More points are seeded at
%	locations of high curvature.
%	}
%	\label{fig:seeded_points}
%\end{figure}
%
%
% subsection seeding (end)
%
\subsection{Extracting Profile Lines} % (fold)
\label{sub:profile_extraction}
%
Once the points $\vp_i$ are seeded, profiles of all variables are sampled at
regular intervals along lines $\vp_i(t)=\vp_i + t \cdot \vn_i$, with $\Norm{\vn_i}
= 1$ (\cref{fig:flamecrossing}).
%
% According to Shannon's theorem the step size is $\sfrac{1}{2}$ of the voxel
% size.
% 
Due to the high resolution and accuracy of \ac{DNS} data, trilinear
interpolation is sufficient.
%
The length of the line is chosen sufficiently large to cover the whole flame
front.
%
The resulting profiles are approximated by simple models to further reduce the
required storage space.
%
% The resulting profiles are discretely sampled functions
%
% \begin{equation}
% 	P_{ij}^V = V(\vp_i(t_j)), ~ i=1, \dotsc, n, ~ j = 1, \dotsc, m,~
% 	t_j = -\frac{\tau}{2} + \frac{j-1}{m-1} \cdot \tau \text{ ,}
% \end{equation}
%
% where $V$ is the variable from which the profiles are extracted, $n$ is the
% number of lines, $m$ is the number of sample points along the line, and $\tau$
% is the length of the line. We are assuming a uniform grid with voxels of $1$
% unit edge length for simplicity.
%
% subsection profile_extraction (end)
\subsection{Model-Based Data Approximation} % (fold)
\label{sub:fitting}
%
% address a review comment
%----
% Even though \ac{DNS} data are simulated without any model assumptions,
% it is reasonable to describe the data outcome of them via models for
% two reasons: i) the model-based description of the data does not negatively effect
% the simulation benefits of \ac{DNS} data being that \ac{DNS} data have a low
% simulation error because the simulation still runs with the raw data per time step;
% and ii) the models facilitates the treatment and extraction of such
% characteristic features which the domain experts want to investigate for
% data analysis purposes. In the following, the compression scheme steps are described in detail.
%----
Although the main advantage of \ac{DNS} is its high precision due to the lack of
modeling assumptions, we use models to describe its outcome in a lower-
dimensional form.
%
These models are sufficient to facilitate the analysis of the scalar fields we
intend.
%
Additionally, they reduce the space needed to store the sparse representation.
%

%
Reactants, intermediates, and products each have very similar profiles that can
be locally approximated by simple models.
%
Reactants and products tend to exhibit profiles with a sigmoid shape,
transitioning from a constant high/low value in the unburnt region to a constant
low/high value in the burnt region, passing an inflection point in between.
%
This behavior can be expressed by a model based on a sigmoid function.
%
Intermediate species have a maximum near the flame surface, decreasing on both
sides.
%
They are approximated by a model based on a Gaussian bell curve.
%

%
Small fluctuations that deviate from these models may occur but are not relevant
for the general shape.
%
Greater deviations appear if a sample line crosses the flame surface multiple
times, which happens if multiple parts of the flame front come close to each
other.
%
In such cases, the characteristic behavior occurs multiple times across the
profile.
%
To handle this, the instance closest to the anchor point is identified and
isolated from the others.
%
\subsubsection{Model for Reactants and Products} % (fold)
\label{ssub:model_for_reactants_products}
%
Sigmoid shapes are commonly expressed with the logistic function $1/(1+e^{-t})$,
which we extend with parameters $\gamma$, adjusting the slope at the inflection
and $a$, determining the limits of the function at positive/negative infinity:
%
\begin{equation}
	\mathfrak{s}(t,a,\gamma)
		= \frac{2a}{ 1 + e^{ (-2\frac{\gamma t}{a}) } } - a \text{ .}
\end{equation}
%

%
While this function can roughly approximate the profiles of reactants and
products, it has too few degrees of freedom to reproduce the different
curvatures of the profiles at both sides of the inflection point.
%
For this reason, we use two pieces of this function, joining smoothly at the
inflection point (\cref{fig:models}, top).
%
\begin{equation}
%	\begin{aligned}
		\mathfrak{S}(t, a_l, a_r, \gamma, x_m, y_m) =
		\begin{cases}
			\mathfrak{s}(t, a_l, \gamma) + y_m, &\text{if }  t \leq x_m\\
			\mathfrak{s}(t, a_r, \gamma) + y_m, & \text{if }  t > x_m\\
		\end{cases}
%		&\text{with } \sgn(\gamma) = \sgn(a_r) = \sgn(a_l) \text{ ,}
%	\end{aligned}
\end{equation}
%
where $(x_m, y_m)$ is the location of the inflection point, $\gamma$ is the
slope at the inflection and $y_m-a_l$ and $y_m+a_r$ are the limits of the
function approaching negative and positive infinity.
%
% subsubsection model_for_reactants_products (end)
%
\subsubsection{Model for Intermediate Species} % (fold)
\label{ssub:model_for_intermediate_species}
%
The profiles of intermediates resemble Gaussian bell curves.
%
Since minimum and maximum of the profiles of intermediate species can vary, it
is necessary to extend the standard bell curve with additional parameters $y_m$,
the value at the maximum, and $y$, the limit at infinity.
%
\begin{equation}
	\mathfrak{g}(t, x_m, y_m, \sigma, y)
		= e^{-\frac{1}{2}(\frac{t-x_m}{\sigma})^2}\cdot(y_m-y)+y \text{ .}
\end{equation}
%

%
Since the profiles tend to have a steeper slope on the unburnt side and some of
them do not reach zero on the burnt side, a two-sided model is once again needed
to accurately capture this behavior.
%
We join the two bell curves smoothly at their maximum point, resulting in a
model with six parameters:
%
\begin{equation}
	\mathfrak{G}(t, x_m, y_m, \sigma_l, y_l, \sigma_r, y_r) =
	\begin{cases}
		\mathfrak{g}(t, x_m, y_m, \sigma_l, y_l), & \text{if }  t \leq x_m\\
		\mathfrak{g}(t, x_m, y_m, \sigma_r, y_r), & \text{if }  t > x_m\\
	\end{cases}
\end{equation}
%
where $(x_m, y_m)$ is the location of the maximum, $\sigma_l$ and $\sigma_r$
determine the slope of the left and right part of the function and $y_l$ and
$y_r$ are the limits of the function approaching negative and positive infinity
(\cref{fig:models}, bottom).
%
% \begin{figure}[ht]
% 	\centering
% 	\setlength\figureheight{3cm} 
% 	\setlength\figurewidth{7cm}
% 	\input{figures/gaussmodel.tikz}
% 	\caption{Gaussian model}
% 	\label{fig:gaussmodel}
% \end{figure}
%
% subsubsection model_for_intermediate_species (end)
%
\subsubsection{Fitting the Models to the Profiles} % (fold)
\label{ssub:fitting_the_models}
%
We now approximate the profiles by fitting the models.
%
For the sigmoid model, we need the position, value and first derivative at the
inflection point as well as the minimum and maximum values of the profile on
both sides of the flame front.
%
For the Gaussian model, the location and value of the maximum near the flame
front have to be known, as well as the minimum values $y_l$ and $y_r$.
%
To robustly determine these feature points, we have to account for the two types
of deviations that may occur in the profiles as described above:
%
small fluctuations, and the profile entering and leaving the reaction zone
multiple times.
%

%
To eliminate small fluctuations, we filter the profiles with a Gaussian kernel
\cite{Jaehne2005}.
%
The kernel size depends on the size of the fluctuations but can be chosen quite
small.
%
For the test data (see \cref{sec:reconstruction}) we used a size of $6$ voxels,
which translates to $\sigma = 1$.
%

%
Extrema are found at zero-crossings of the first derivative of the filtered
profile.
%
Possible shifts of the extrema due to the filtering are corrected by mapping
each maximum to the largest maximum, and each minimum to the smallest minimum in
the radius of the filter size, taking care that extrema do not switch sides.
%
For finding inflection points, we use the same approach but on the second
derivative.
%


%
Due to the possibility of crossing the flame surface multiple times (see $\vp_3$
in \cref{fig:flamecrossing}), we can find more feature points on the profile
than we need for our models.
%
We have to identify the ones closest to the anchor point.
%
For the sigmoid model, this is the inflection nearest to the center of the
profile.
%
The position and value of this point determine $x_m$ and $y_m$ of the sigmoid
model, while $\gamma$ is determined by the first derivative at the inflection.
%
The values of the first minimum and maximum left and right of the inflection
(depending on the sign of $\gamma$) determine $a_l$ and $a_r$.
%
The positions of these extrema, $x_l$ and $x_r$, are the boundaries of the
portion of the profile that the model is fitted to.
%
The rest of the profile, possibly containing other crossings of the flame
surface, is not considered, as those other crossings are already captured by
other profile lines seeded there.
%
\begin{figure}[t!]
	\centering
	\setlength\figurewidth{0.95\linewidth}
	\setlength\figureheight{2.2cm}
	%
\pgfplotstableread{
x y
-14.936708 9.439669e-06
-14.430379 9.790832e-06
-13.924050 1.036994e-05
-13.417721 1.103326e-05
-12.911392 1.176119e-05
-12.405063 1.083076e-05
-11.898734 9.568702e-06
-11.392405 7.851489e-06
-10.886075 5.901285e-06
-10.379746 4.116033e-06
-9.873417 3.019796e-06
-9.367088 2.312050e-06
-8.860759 1.873673e-06
-8.354430 1.571615e-06
-7.848101 1.350457e-06
-7.341772 9.620873e-07
-6.835443 7.691790e-07
-6.329113 7.234248e-07
-5.822784 4.464375e-07
-5.316455 6.137330e-07
-4.810126 1.168601e-06
-4.303797 2.883451e-06
-3.797468 8.607734e-06
-3.291139 1.538896e-05
-2.784810 4.085074e-05
-2.278481 7.368171e-05
-1.772151 9.791394e-05
-1.265822 0.000101
-0.759493 9.343289e-05
-0.253164 7.598200e-05
0.253164 6.136476e-05
0.759493 5.470099e-05
1.265822 5.128095e-05
1.772151 4.699871e-05
2.278481 4.260473e-05
2.784810 3.796287e-05
3.291139 3.453119e-05
3.797468 3.203549e-05
4.303797 3.045799e-05
4.810126 2.974594e-05
5.316455 2.935907e-05
%
7 		 3.195907e-05
10 		 3.55907e-05
12 		 3.735907e-05
15 		 3.145907e-05
%
% 5.822784 3.032043e-05
% 6.329113 3.190556e-05
% 6.835443 3.435575e-05
% 7.341772 3.740233e-05
% 7.848101 4.081712e-05
% 8.354430 4.489389e-05
% 8.860759 4.866396e-05
% 9.367088 5.131122e-05
% 9.873417 5.356448e-05
% 10.379746 5.864447e-05
% 10.886075 6.847978e-05
% 11.392405 8.069201e-05
% 11.898734 9.524663e-05
% 12.405063 0.000108
% 12.911392 9.813603e-05
% 13.417721 7.779592e-05
% 13.924050 5.396330e-05
% 14.430379 3.109881e-05
% 14.936708 1.150626e-05
}\dataradical%
%
\pgfplotstableread{
x y
-5.822784 4.761071e-07
-5.316455 6.099730e-07
-4.810126 1.183832e-06
-4.303797 3.166483e-06
-3.797468 8.654500e-06
-3.291139 2.070895e-05
-2.784810 4.136632e-05
-2.278481 6.804909e-05
-1.772151 9.181174e-05
-1.265822 0.000101
-0.759493 9.791479e-05
-0.253164 8.828693e-05
0.253164 7.515031e-05
0.759493 6.152742e-05
1.265822 4.978867e-05
1.772151 4.108846e-05
2.278481 3.544709e-05
2.784810 3.221574e-05
3.291139 3.057087e-05
3.797468 2.982378e-05
4.303797 2.952018e-05
4.810126 2.940956e-05
5.316455 2.937337e-05
}\modelradical%
%
\begin{tikzpicture}

\pgfplotsset{
	legend style={
		legend pos=north west,
		font=\small,
		legend cell align=left,
		draw=none},
	max/.style={mycolor4},
	infl/.style={mycolor1},
	min/.style={mycolor3},
	xtick={0},
	ytick={0},
	xmin=-15,
	xmax=15,
	every tick/.style={color=black, thin},
}

\begin{axis}[%
small,
name=sigmoid fit,
%at={(sigmoid model.below south west)}, yshift=-0.5cm,
anchor=north west,
clip=false,
width=\figurewidth,
height=\figureheight,
scale only axis,
scaled y ticks=base 10:0,
ymin=0,
ymax=2300,
% axis y line=left,
% axis x line=bottom,
xlabel=$t$,
% ylabel=$V$,
% every axis x label/.style={at={(current axis.right of origin)},anchor=north west},
]

\addplot [
color=mycolor4,
solid,
ultra thick
]
table[row sep=crcr]{
-4.303 316.491\\
-3.797 333.492\\
-3.291 352.480\\
-2.784 390.742\\
-2.278 462.220\\
-1.772 578.651\\
-1.265 731.693\\
-0.759 890.906\\
-0.253 1042.127\\
0.253 1178.895\\
0.759 1297.219\\
1.265 1395.718\\
1.772 1475.118\\
2.278 1537.480\\
2.784 1585.466\\
3.291 1621.811\\
3.797 1649.012\\
4.303 1669.186\\
4.810 1684.050\\
5.316 1694.947\\
5.822 1702.908\\
6.329 1708.707\\
6.835 1712.925\\
7.341 1724.045\\
};
\addlegendentry{Model $\mathfrak{S}(t)$}

\addplot [
color=black,
solid,
thick
]
table[row sep=crcr]{
-14.936 488.950\\
-14.430 487.997\\
-13.924 482.351\\
-13.417 474.442\\
-12.911 463.814\\
-12.405 450.874\\
-11.898 434.432\\
-11.392 419.604\\
-10.886 405.748\\
-10.379 390.223\\
-9.873 378.419\\
-9.367 367.060\\
-8.860 356.617\\
-8.354 347.594\\
-7.848 339.085\\
-7.341 332.486\\
-6.835 327.322\\
-6.329 322.787\\
-5.822 320.653\\
-5.316 318.845\\
-4.810 317.220\\
-4.303 316.491\\
-3.797 317.992\\
-3.291 329.504\\
-2.784 358.625\\
-2.278 429.448\\
-1.772 572.368\\
-1.265 731.693\\
-0.759 893.568\\
-0.253 1039.957\\
0.253 1175.875\\
0.759 1289.781\\
1.265 1379.821\\
1.772 1460.163\\
2.278 1519.044\\
2.784 1565.980\\
3.291 1607.133\\
3.797 1637.439\\
4.303 1662.985\\
4.810 1683.066\\
5.316 1698.107\\
5.822 1709.734\\
6.329 1715.562\\
6.835 1720.318\\
7.341 1724.045\\
7.848 1722.378\\
8.354 1721.440\\
8.860 1718.134\\
9.367 1710.456\\
9.873 1703.957\\
10.379 1693.400\\
10.886 1682.280\\
11.392 1672.077\\
11.898 1656.291\\
12.405 1642.015\\
12.911 1624.925\\
13.417 1604.251\\
13.924 1585.385\\
14.430 1561.301\\
14.936 1536.410\\
};
\addlegendentry{Data $V(t)$}

\draw[color=mycolor2, thick]
% -1.265, 731.693
	(axis cs: -3.53, 31.693) -- (axis cs: 3.265, 2131.693) 
	node[right, pos=0.95]{\small $\gamma=\mathfrak{S}'(x=x_m)$};

\draw[dashed]
	(axis cs: -6.303, 731.693) -- (axis cs: -2.303, 731.693);

\draw[dashed]
	(axis cs: 5.341, 731.693) -- (axis cs: 9.341, 731.693);

\addplot [
infl,
only marks,
mark=*,
mark options={draw=black},
forget plot
]
table[row sep=crcr]{
-1.265 731.693\\
};
\label{pgf:inflection_sig}
\draw[latex-latex] 
	(axis cs: -1.265, 731.693) -- (axis cs: -1.265, 0) 
	node[below, pos=1.0, anchor=north]{$x_m$}
	node[right, midway]{$y_m$};

\draw[dashed] (axis cs: -4.303, 316.491) -- (axis cs: -4.303, 0) node[below, pos=1.0, anchor=north]{$x_l$};
\addplot [
min,
only marks,
mark=*,
mark options={draw=black},
forget plot
]
table[row sep=crcr]{
-4.303 316.491\\
};
\label{pgf:min_sig}

\draw[dashed] (axis cs: 7.341, 731.693) -- (axis cs: 7.341, 0) node[below, pos=1.0, anchor=north]{$x_r$};
\addplot [
max,
only marks,
mark=*,
mark options={draw=black},
forget plot
]
table[row sep=crcr]{
7.341 1724.045\\
};
\label{pgf:max_sig}

\draw[latex-latex] 
	(axis cs: 7.341, 1724.045) -- (axis cs: 7.341, 731.693) 
	node[right, midway]{$a_r$};

\draw[latex-latex] 
	(axis cs: -4.303, 316.491) -- (axis cs: -4.303, 731.693) 
	node[left, midway]{$a_l$};
\end{axis}

\begin{axis}[%
small,
name=gauss fit,
at={(sigmoid fit.below south west)}, yshift=-0.5cm,
anchor=north west,
clip=false,
width=\figurewidth,
height=\figureheight,
scale only axis,
scaled y ticks=base 10:0,
ymin=0,
ymax=0.0001,
% axis y line=left,
% axis x line=bottom,
xlabel=$t$,
% ylabel=$V$,
% every axis x label/.style={at={(current axis.right of origin)},anchor=north west},
% every tick/.style={color=black, thin},
% xtick={-10, 0, 10},
]

\addplot [
color=mycolor4,
solid,
ultra thick
]
table[x=x, y expr=\thisrow{y}*0.6 + 0.2e-4]{\modelradical};
\addlegendentry{Model $\mathfrak{G}(t)$}

\addplot [
color=black,
solid,
thick
]
table[x=x, y expr=\thisrow{y}*0.6 + 0.2e-4]{\dataradical};
\addlegendentry{Data $V(t)$}

\addplot [
max,
only marks,
mark=*,
mark options={draw=black},
forget plot
]
table[row sep=crcr]{
-1.265822 8.06e-5\\
};
\label{pgf:max_gauss}
\draw[latex-latex] 
	(axis cs: -1.265822, 8.06e-5) -- (axis cs: -1.265822, 0)
	node[right, pos=0.7]{$y_m$} 
	node[below, pos=1.0, anchor=north]{$x_m$};

\addplot [
min,
only marks,
mark=*,
mark options={draw=black},
forget plot
]
table[row sep=crcr]{
-5.822784 2.0286e-5\\
};
\label{pgf:min_gauss}
\draw[latex-latex] 
	(axis cs: -5.822784, 2.0286e-5) -- (axis cs: -5.822784, 0) 
	node[left, midway]{$y_l$}
	node[below, pos=1.0, anchor=north]{$x_l$};

\addplot [
min,
only marks,
mark=*,
mark options={draw=black},
forget plot
]
table[row sep=crcr]{
5.316455 3.7624e-5\\
};
\draw[latex-latex] 
	(axis cs: 5.316455, 3.7624e-5) -- (axis cs: 5.316455, 0) 
	node[right, midway]{$y_r$}
	node[below, pos=1.0, anchor=north]{$x_r$};

\draw[latex-latex]
	(axis cs: -1.265822, 4.482e-5) -- (axis cs: -2.784810, 4.482e-5)
	node[pos=0.6, below]{$\sigma_l$};

\draw[latex-latex]
	(axis cs: -1.265822, 6.509e-5) -- (axis cs: 0.253164, 6.509e-5)
	node[pos=0.5, below]{$\sigma_r$};;

%-2.784810 4.136632e-05
%0.253164 7.515031e-05

\end{axis}
\end{tikzpicture}%
	\vspace*{-2mm}
	\caption{
	Sigmoid model (top) and Gaussian model (bottom) with examples of models
	fitted to a profile.}
	\label{fig:models}
\end{figure}
%
% \begin{figure}[t]
% 	\begin{center}
% 		\setlength\figureheight{2cm}
% 		\setlength\figurewidth{0.85\linewidth}	
% 		\input{figures/fit_sigmoid_5.tikz}\\
% 		\input{figures/fit_gauss_3.tikz}
% 	\end{center}
% 	\caption{
% 	Top: Fitting the sigmoid model to a profile of temperature.
% 	Bottom: Fitting the Gaussian model to a profile of \ce{H2O2}.
% 	The non-constant data values outside the fitted range show that the profile
% 	crosses the reaction zone multiple times.
% 	}
% 	\label{fig:fitting_models}
% \end{figure}
%

%
For the Gaussian model, the maximum nearest to the center determines $x_m$ and
$y_m$, while the values of the closest minima to both sides determine $x_l$ and
$x_r$, as well as $y_l$ and $y_r$.
%
Further extrema are ignored.
%
For both models, if there are no extrema on either side of $x_m$, the values at
the ends of the profiles are chosen instead.
%

% \begin{figure}[ht]
% 	\begin{center}
% 		\setlength\figureheight{2cm}
% 		\setlength\figurewidth{0.85\linewidth}
% 		\input{figures/fit_gauss_3.tikz}
% 	\end{center}
% 	\caption{
% 	Fitting the Gaussian model to a profile of \ce{H2O2}. The values outside of
% 	the fitted range show that the profile crosses the reaction zone three times.
% 	}
% 	\label{fig:fitting_gaussian}
% \end{figure}

%
While the feature points on the profile already determine all parameters for the
sigmoid model, the values for $\sigma_l$ and $\sigma_r$ of the Gaussian model
still have to be found.
%
We obtain an initial guess for $\sigma_{l/r}$ by transforming the intervals of
the profile between $x_l$ and $x_m$ and between $x_m$ and $x_r$ into the
interval $[0, 1]$ and regarding them as halves of two symmetric \acp{PDF}.
%
The variance of a discrete random variable $X$ with the \ac{PDF} $p(x)$ and
expected value $\mu$ is given by: $\var(X) = \int p(x) \cdot (x-\mu)^2 ~dx$.
%
Since for a normal distribution, the variance is $\sigma^2$, we can directly use
this equation on our transformed profiles.
%
We then refine this estimate with a simple optimization scheme such as Newton's
method to find the optimal fit.
%

% As an initial guess, we use the fact that for a normal
% distribution, $\sigma^2$ is equal to the variance. Since our model, like the
% normal distribution, is based on the Gaussian function, we can employ a method
% similar to estimating the variance of a probability density function (pdf) for
% determining an initial guess for $\sigma_l$ or $\sigma_r$. 
% By transforming each side of our Gaussian model into the value range $[0, 1]$
% and regarding it as one half of a symmetric pdf, we can directly employ the
% above equation to obtain an estimate for both parameters. We then refine this
% estimate with a simple optimization scheme such as Newton's method to find the
% optimal fit.

% Our profiles are not probability mass functions but by transforming them into
% something similar, we can use the same method to estimate a pseudo-variance and
% hence a good guess for $\sigma_{l/r}$. The values between $x_m$, which is our
% pseudo-$\mu$, and the nearest minimum to one side give us the basis for one half
% of a pseudo-pmf. By subtracting the minimum value and dividing the resulting
% values by twice their sum, we get something resembling one half of a pmf, which
% we can insert into equation \eqref{eqn:variance}, assuming it is symmetric
% in $x_m$. The resulting equation for estimating $\sigma_l$ is:
% \begin{equation}
% 	\begin{aligned}
% 	\sigma_li^2 =  &2 \cdot \sum_{j | x_li \leq x_ji \leq x_mi} \tilde{P}_ij^V \cdot (x_ji-x_mi)^2\\
% 	\tilde{P}_ij^V = &\frac{P_ij^V-y_li}{2 \cdot \sum_{j | x_li \leq x_ji \leq x_mi} P_ij^V-y_li}
% 	\end{aligned}
% \end{equation}
% Where $\tilde{P}_ij^V$ is the profile $P^V_ij$ between $x_li$ and $x_mi$,
% transformed into a half pmf like described. An estimate of $\sigma_ri$ can be
% determined in the same way.

% If the profile were perfectly similar to a normal distribution, this estimate of
% $\sigma$ would mean a perfect fit of the model to the data. If this is not the
% case, even small deviations can lead to an estimate that is not optimal. Finding
% the optimum in least squares sense is a nonlinear problem. Due to the good
% estimate we have, we can employ a simple optimization scheme such as Newton's
% method to find the solution with very few steps.

% For each profile
% line, we first determine the inflection point nearest to the center of the
% profile. We then find the nearest maximum or minimum at both sides of the
% inflection. The position, value and slope of the inflection as well as the
% positions and values of the extrema at both sides provide all necessary
% parameters of the model. 

% The parameter matrix $\Phi^V_{\textnormal{rp}}$ of the sigmoid model for reactants
% and products is
% \begin{equation}
% \Phi^V_{\textnormal{rp}} =
% 	(a_{li}, a_{ri}, m_i, x_{mi}, y_{mi}, x_{li}, x_{ri}),~
% 	i=1,\dotsc,n \text{ .}
% \end{equation}
%For intermediate species, the Gaussian model $G$ is fitted to the data. 
% Here,
% the maximum of the profile that is nearest to the anchor point at the center is
% first located. The minima at both sides of this maximum then give the boundaries
% of the relevant crossing of the flame surface, and the values for the limits
% $y_l$ and $y_r$ at both sides. To determine $\sigma_l$ and $\sigma_r$, we
% estimate an initial guess by regarding the profile between the left minimum and
% the central maximum, and the profile between the maximum and right minimum as
% halves of a transformed probability mass function. The optimal values are then
% found by minimizing the mean square error between data and model via Newton's
% method, starting at the initial guess estimated in the previous step. 
% The parameter matrix $\Phi^V_{\textnormal{int}}$ of the Gaussian model for
% intermediate species is 
% \begin{equation}
% \Phi^V_{\textnormal{int}} =
% 	(x_{mi}, y_{mi}, \sigma_{li}, y_{li}, \sigma_{ri}, y_{ri}, x_{li}, x_{ri}),~
% 	i=1,\dotsc,n \text{ .}
% \end{equation}

%
With this, the original data is now described by the flame surface mesh, the
$\vp_i$ and $\vn_i$ of the profile lines, and the model parameters for each
profile line and variable.
%
For the sigmoid model, this is $(a_{li},$ $a_{ri},$ $\gamma_i,$ $x_{mi},$ $y_{mi},$
$x_{li},$ $x_{ri})$ for each profile line.
%
For the Gaussian model, the parameters are $(x_{mi},$ $y_{mi},$
$\sigma_{li},$ $y_{li},$ $\sigma_{ri},$ $y_{ri},$ $x_{li},$ $x_{ri})$.
%
This comprises our sparse representation of the flame, with the flame surface
represented by the surface mesh and flamelets represented by the profile lines
orthogonal to the surface.
%
% subsubsection fitting_the_models (end)
%
% subsection fitting (end)
%
% section compression (end)
