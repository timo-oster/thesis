

\section{Results}
\label{sec:sr_results}

Limitation: Only works for chemical species and tightly coupled variables such
as temperature. Weakly coupled variables such as flow, pressure, density also
have significant gradients away from the flame surface and are not highly
aligned to the flame surface, making them unsuitable for this method. However,
especially for simulations involving practical fuels, the number of species
becomes so high that they are dominating.

Used error metric for comparing original and reconstructed data:
\todo[inline]{PSNR is used more often here. Is directly derivable from MSE and also accounts for range of data values}
\begin{align}
    E_a(R, O) &= \frac{1}{|H|}\int_{x \in H} \sqrt{\left(\frac{R(x) - O(x)}{max(O)-min(O)}\right)^2}\\
    E_s(R, O) &= \frac{1}{|H|}\int_{x \in H} \left(\frac{R(x) - O(x)}{max(O)-min(O)}\right)^2
\end{align}
Where $O$ is the original function, $R$ is the reconstructed function, $H$ is 
the convex hull of all sample lines (can not extrapolate data outside), 
and $|H|$ is the volume of H. $max(O)-min(O)$ is the range of the original data 
$O$, used to normalize the errors.

\begin{figure}
	\centering
	\setlength\figureheight{12cm} 
	\setlength\figurewidth{6cm}
	\input{figures/msevcr_stack.tikz}
	\caption{MSE vs. Compression rate (Dataset 1)}
	The error plots corresponding to each variable are shifted by a constant
	increment. The horizontal lines signify the zero-levels of the respective 
	plots.
	\label{fig:CRvMSE}
\end{figure}

\begin{figure}
	\centering
	\setlength\figureheight{6cm} 
	\setlength\figurewidth{6cm}
	\input{figures/msevcr_stack_data2.tikz}
	\caption{MSE vs. Compression rate (Dataset 2)}
	The error plots corresponding to each variable are shifted by a constant
	increment. The horizontal lines signify the zero-levels of the respective 
	plots.
	\label{fig:CRvMSE2}
\end{figure}

\begin{figure}
	\centering
	\setlength\figureheight{12cm} 
	\setlength\figurewidth{6cm}
	\input{figures/msevcr_stack_data3.tikz}
	\caption{MSE vs. Compression rate (Dataset 3)}
	The error plots corresponding to each variable are shifted by a constant
	increment. The horizontal lines signify the zero-levels of the respective 
	plots.
	\label{fig:CRvMSE3}
\end{figure}

\begin{figure}
	\centering
	\setlength\figureheight{4.5cm} 
	\setlength\figurewidth{6cm}
	\input{figures/mean_mse_fit.tikz}
	\caption{MSE of fitting (before reconstruction) for all variables}
	The error is normalized by the range of values in each variable.
	\label{fig:fitting_MSE}
\end{figure}

\begin{figure}
	\centering
	\setlength\figureheight{4.5cm} 
	\setlength\figurewidth{6cm}
	%
%
% This file was created by matlab2tikz v0.4.0.
% Copyright (c) 2008--2013, Nico Schlömer <nico.schloemer@gmail.com>
% All rights reserved.
% 
% The latest updates can be retrieved from
%   http://www.mathworks.com/matlabcentral/fileexchange/22022-matlab2tikz
% where you can also make suggestions and rate matlab2tikz.
% 
\begin{tikzpicture}

\begin{axis}[%
small,
width=\figurewidth,
height=\figureheight,
area legend,
legend style={at={(0.98, 1.15)}, anchor=north east, legend columns=1, font=\small},
scale only axis,
axis y line = left,
axis x line* = bottom,
xmin=0.5,
xmax=13.5,
ylabel={$E^V_{\text{fit}}$},
every axis y label/.style={at={(current axis.above origin)},anchor=south east},
scaled y ticks = false,
yticklabel style = {/pgf/number format/precision=3, /pgf/number format/fixed},
ybar,
ymin=0,
ymax=0.085,
xtick={1,2,3,4,5,6,7,8,9,10,11,12,13},
xticklabels={OH,\ce{O2},O,\ce{HO2},HCO,\ce{H2O2},\ce{H2O},\ce{H2},H,\ce{CO2},CO,\ce{CH2O},T},
every tick/.style={color=black, thin},
x tick label style = {rotate=90},
]
\addplot[ybar,mycolor1,bar width=0.01\figurewidth,fill=mycolor1] 
coordinates
{(1, 0.0160108368686018)
(2, 0.0179611197947222)
(3, 0.0512353285580825)
(4, 0.0280660639669209)
(5, 0.0216726758682414)
(6, 0.0468526577521646)
(7, 0.0159317115163465)
(8, 0.0185940722130954)
(9, 0.0297375593011707)
(10, 0.019667950937967)
(11, 0.020793157135868)
(12, 0.000950809666194)
(13, 0.030076831585527)};
\addlegendentry{\textsc{Syngas I}}

\addplot[ybar,mycolor2,bar width=0.01\figurewidth,fill=mycolor2] 
coordinates
{(1, 0.0235)
(2, 0.0165)
(3, 0.0632)
(4, 0.0312)
(5, 0.0125)
(6, 0.0492)
(7, 0.0148)
(8, 0.0174)
(9, 0.0322)
(10, 0.0245)
(11, 0.0247)
(12, 0.0008)
(13, 0.0342)};
\addlegendentry{\textsc{Syngas II}}

\addplot[ybar,mycolor3,bar width=0.01\figurewidth,fill=mycolor3]
coordinates
{
(10, 0.0226)
(6, 0.0136)
(5, 0.0068)	
};
\addlegendentry{\textsc{Syngas III}}

% \addplot[ybar,mycolor3,bar width=0.01\figurewidth,fill=mycolor3] 
% coordinates
% {
% (13, 0.036749)
% (14, 0.032756)
% (15, 0.032433)
% (9, 0.050241)
% (8, 0.019321)
% (7, 0.016394)
% (6, 0.037397)
% (4, 0.040298)
% (3, 0.029779)
% (2, 0.015316)
% (1, 0.085694)
% };
% \addlegendentry{\textsc{Hydrogen}}

\end{axis}
\end{tikzpicture}%
	\caption{Mean absolute error of fitting (before reconstruction) for all variables}
	The error is normalized by the range of values in each variable. 
	The average absolute error made by fitting the gauss and sigmoid models
	to the data is between 0.1\% and 5.1\%.
	\label{fig:fitting_abs_error}
\end{figure}


\begin{figure}[t]
\centering
	\begin{tabular}{ccc}
		&\subfloat[Original]{
			\includegraphics[width=0.27\linewidth]{images/H_orig}}&\\
		\subfloat[Linear Interpolation]{
			\includegraphics[width=0.27\linewidth]{images/H_lin_36}}&
		\subfloat[kNN Interpolation]{
			\includegraphics[width=0.27\linewidth]{images/H_knn_36}}&
		\subfloat[na\"{i}ve downsampling]{
			\includegraphics[width=0.27\linewidth]{images/H_down_36}}\\
		\subfloat[square error (linear)]{
			\includegraphics[width=0.27\linewidth]{images/H_diff_lin_36}}&
		\subfloat[square error (kNN)]{
			\includegraphics[width=0.27\linewidth]{images/H_diff_knn_36}}&
		\subfloat[square error (downsampling)]{
			\includegraphics[width=0.27\linewidth]{images/H_diff_down_36}}
	\end{tabular}
	\caption{Comparison of reconstruction results for \ce{H} with l=36 (ca. 5000 samples)}
	\label{fig:comp_H}
\end{figure}

\begin{figure}[t]
\centering
	\begin{tabular}{ccc}
		\subfloat[Original]{
			\includegraphics[width=0.27\linewidth]{images/O2_orig}}&
		\subfloat[l=1]{
			\includegraphics[width=0.27\linewidth]{images/O2_lin_1}}&
		\subfloat[l=12]{
			\includegraphics[width=0.27\linewidth]{images/O2_lin_12}}\\
		\subfloat[l=23]{
			\includegraphics[width=0.27\linewidth]{images/O2_lin_23}}&
		\subfloat[l=58]{
			\includegraphics[width=0.27\linewidth]{images/O2_lin_58}}&
		\subfloat[l=240]{
			\includegraphics[width=0.27\linewidth]{images/O2_lin_240}}\\
	\end{tabular}
	\caption{Reconstruction results of linear interpolation for different 
             sample densities for \ce{O2}}
	\label{fig:reconstruct_O2}
\end{figure}

\begin{figure}[t]
\centering
	\begin{tabular}{ccc}
		\subfloat[Original]{
			\includegraphics[width=0.27\linewidth]{images/H2O2_orig}}&
		\subfloat[l=1]{
			\includegraphics[width=0.27\linewidth]{images/H2O2_knn_1}}&
		\subfloat[l=12]{
			\includegraphics[width=0.27\linewidth]{images/H2O2_knn_12}}\\
		\subfloat[l=23]{
			\includegraphics[width=0.27\linewidth]{images/H2O2_knn_23}}&
		\subfloat[l=58]{
			\includegraphics[width=0.27\linewidth]{images/H2O2_knn_58}}&
		\subfloat[l=240]{
			\includegraphics[width=0.27\linewidth]{images/H2O2_knn_240}}\\
	\end{tabular}
	\caption{Reconstruction results of kNN interpolation for different 
             sample densities for \ce{H2O2}}
	\label{fig:reconstruct_H2O2}
\end{figure}

\begin{figure}[t]
\centering
	\begin{tabular}{cc}
		\multicolumn{2}{c}{
			\subfloat[Original]{
				\includegraphics[width=0.4\linewidth]{images/data3/H2O_orig}
				}
		}\\
		\subfloat[lin (l=1)]{
			\includegraphics[width=0.4\linewidth]{images/data3/H2O_lin_1}}&
		\subfloat[knn (l=1)]{
			\includegraphics[width=0.4\linewidth]{images/data3/H2O_knn_1}}\\
		\subfloat[lin (l=23)]{
			\includegraphics[width=0.4\linewidth]{images/data3/H2O_lin_23}}&
		\subfloat[knn (l=23)]{
			\includegraphics[width=0.4\linewidth]{images/data3/H2O_knn_23}}\\
		\subfloat[lin (l=58)]{
			\includegraphics[width=0.4\linewidth]{images/data3/H2O_lin_58}}&
		\subfloat[knn (l=58)]{
			\includegraphics[width=0.4\linewidth]{images/data3/H2O_knn_58}}\\
		\subfloat[lin (l=240)]{
			\includegraphics[width=0.4\linewidth]{images/data3/H2O_lin_240}}&
		\subfloat[knn (l=240)]{
			\includegraphics[width=0.4\linewidth]{images/data3/H2O_knn_240}}\\
	\end{tabular}
	\caption{Comparison of reconstruction quality by linear (left) and
			 knn (right) interpolation for \ce{H2O}}
	\label{fig:compare_lin_knn_H2O}
\end{figure}