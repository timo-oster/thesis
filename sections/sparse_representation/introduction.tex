
\section{Introduction}
\label{sec:intro}

\Todo[inline]{Condense intro and related work into one introductory section}

Direct numerical simulation (\acs{DNS}) \acused{DNS} is an extremely precise
method for simulating turbulent combustion processes. It resolves the underlying
equations on a high-resolution grid without any modelling assumptions. A single
\ac{DNS} run produces terabytes of multiple time-dependent \ac{3D} scalar fields
that have to be stored, analyzed, and visualized. Due to the sheer size of the
data, storing it in raw form is prohibitively expensive. This in turn prevents
effective analysis and visualization. Existing approaches only focus on a subset
of these problems, and fail to use the synergies an integrated approach has to
offer. We introduce a sparse data representation tailored to \ac{DNS} data of
premixed combustion that can directly be used for further analysis and
visualization, and that retains the possibility of reconstructing the full
scalar fields.

% A single
% \ac{DNS} run produces terabytes of multiple time-dependent 3D scalar fields for which
% -- due to the sheer size of the data -- efficient compression methods are
% necessary. Unfortunately, standard general-purpose compression techniques fail
% to give sufficient compression results here.
%
% We introduce a compression technique which is tailored to \ac{DNS} data of combustion
% simulations, giving significantly better compression results than standard
% techniques. 

Our approach is based on the observation that most of the relevant information
is located in narrow-band regions where the chemical reactions take place, the
so-called flame fronts. The different scalar fields behave similarly in these
regions. In the presence of turbulent flow, the flame front develops complex,
wrinkled shapes. The differences between scalar fields affected by turbulence,
and their development over time give meaningful insights into the chemical
process and its interaction with turbulent flow. An integrated approach for the
visual analysis of such \ac{DNS} data must solve three problems:
\begin{enumerate}
	\item \label{i:size} Handling the \emph{enormous size} of the simulation data
	\item \label{i:analyze} \emph{Analyzing} the behavior of the scalar fields
	\item \label{i:visualize} \emph{Visualizing} the differences between the scalar fields
\end{enumerate}
%
%----------------
% RW commend is addressed here
%Within such narrow-band regions a flame front is located:
%A flame front equates to a layer on which different
%chemicals mutual interact, which gives the flame front a clear physical interpretation.
%In 2D the flame front is given by a 1D curve, in 3D by a 2D surface.
%----------------
%Although there is no generally agreed-upon definition of the flame front's
%boundaries, there is a consensus about two main goals of a visual analysis of
%such \ac{DNS} data:
% \begin{itemize}
% 	\item Intra-field analysis: the development of the scalar fields inside the 
% 		flame front over time should be explored.
% 	\item Inter-field analysis: the properties of the different scalar fields
% 		inside the flame front should be compared.
% \end{itemize}
%
The sparse data representation presented here considers all these problems. We
fit domain-aware models to the data, exploiting its characteristics and
transforming it into a lower-dimensional parameter space that retains all
relevant information. In this space, the analysis and visualization can be
carried out without accessing the original data. If necessary, the original
scalar fields can still be reconstructed.

The paper is organized as follows: In \cref{sec:related}, we provide
background information and present related work. The sparse data representation
is introduced in \cref{sec:compression}. \Cref{sec:visualization}
presents our visual analysis approach, and \cref{sec:conclusion}
discusses our results and presents feedback from domain experts.
