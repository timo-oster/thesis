\vspace{-2\baselineskip}\lettrine[loversize=0.02, lhang=0.03,
findent=-0.7pt]{T}{he development} and validation of flamelet models is one of
the central applications of \ac{DNS}.
%
As already mentioned in \cref{sub:the_flamelet_assumption}, the flamelet
assumption is one of the most important simplifications applied for modeling
turbulent combustion.
%
It states that a turbulent flame behaves like a collection of strained laminar
flames (``flamelets'') located side-by-side on the flame surface.
%
This assumption is most often applicable to premixed flames and essentially
allows to treat the flame as a \ac{2D} manifold.
%

%
The analysis of \ac{DNS} data in the context of flamelet modeling can be for
two purposes: validation of existing flamelet models, or development of new
models.
%
In both cases, it is necessary to extract flamelet data from the \ac{DNS}.
%
This is often done by sampling the simulation variables along lines orthogonal
to the flame surface (see \eg, \cite{Zistl2009}).
%
The single flamelets obtained from this are then used for statistical analysis,
for example to check model assumptions or discover correlations that can be
incorporated into new models.
%

%
In this chapter we propose a sparse representation for premixed flames that
explicitly encodes flamelets and readily supports flamelet-related analysis
tasks.
%
This representation is significantly smaller than storing the full DNS data,
which enables the analysis of a larger number of time steps per simulation run.
%
We also propose a novel visualization based on this data that augments
traditional statistical analysis with a visual component.
%
If necessary, full scalar fields on the original grid can be reconstructed
from the sparse representation to retain full flexibility for post-processing.
%

%
In spirit, the approach we present is similar to existing works that propose
computing a smaller representation of the data that can later be used for
analysis.
%
Lakshminarasimhan \etal presented the ISABELA compression algorithm
\cite{Lakshminarasimhan2011} and a tool for querying the compressed data
\cite{Lakshminarasimhan2011a}.
%
Bremer and Landge~\cite{Bremer2009,Bremer2011,Bremer2010,Landge2014} presented
multiple works based on the computation of segmented merge trees that facilitate
a post-hoc exploration of thresholds on scalar fields.
%
Ye \etal~\cite{Ye2016} condensed high-resolution grid data into block-wise
probability density functions that can later be explored efficiently.
%

%
The rest of this chapter is organized as follows:
%
We first introduce our approach for constructing a sparse representation of the
flame in \cref{sec:compression}.
%
We then introduce our novel visualization techniques based on this
representation in \cref{sec:visualization}.
%
\Cref{sec:reconstruction} describes the reconstruction of full scalar fields and
evaluates the compression ratio and reconstruction quality we achieve.
%
Finally, we provide a discussion and conclusion in \cref{sec:conclusion}.
%