\section{Discussion and Conclusion}
\label{sec:conclusion}
%
We introduced a sparse representation for \ac{DNS} data of premixed combustion
that is tailored to flamelet-related analysis tasks.
%
The sparse representation enables storing the simulation results with far
smaller space requirements.
%
The space requirement is mainly dependent on the complexity of the flame shape,
not on the size or resolution of the data, i.e., the less complex the flame
shape, the less profile lines need to be seeded for an accurate representation.
%
Via fitting of models using feature points, the sparse representation directly
captures important characteristics of the scalar fields that can be analyzed in
different ways without the need for data reconstruction.
%
Feature surfaces derived from these models can directly be visualized and
facilitate the visual analysis of the data.
%

% Our approach is able to effectively compress simulation variables
% directly related to chemistry, which is the vast majority in simulations of
% practically relevant reactions. The compression quality is mainly dependent on
% the complexity of the flame shape, not on the size of the data, i.e., the less
% complex the flame shape, the less profile lines are required for compression.
% Thus, also very large data can be efficiently compressed by our technique.

%----------------
% RW commend is addressed here
Apart from the feature surfaces, further characteristics might be extracted from
the sparse representation, such as gradient fields and their topology.
%
Note that our approach is specifically tailored to \ac{DNS} data, but can be
used for other kinds of multi-field data where changes are only located in
narrow-band regions.
%
Such data might be geological data of the material within the earth's mantle, or
air field pressure measurements of the supersonic bang of a plane.
%----------------

%If the profiles of a variable do not conform with our model assumptions, fitting
%the models proves challenging and often produces extreme model parameters.
%
Despite its many advantages, our approach has some limitations.
%
Because we assume combustion in the flamelet regime, our approach has limited
applicability in scenarios where the flamelet assumption is not or only
partially fulfilled.
%
However, there is still a large number of practical scenarios for premixed
combustion where the flamelet assumption is valid.
%

%
Our sparse representation does not capture pressure or velocity information, as
these do not only vary in narrow-band regions and they do not necessarily
conform to the models we use to approximate variable profiles.
%
Many flamelet-related analysis tasks can do without this data.
%
If it is needed for flamelet analysis, one could store the raw unmodeled data
for these variables along the profile lines.
%
It would however not be possible to reconstruct this data on the original grid
with sufficient accuracy.
%
As an alternative, these variables could be stored in the original resolution.
%
Since pressure and velocity only represent a small fraction of the data in
combustion simulations, this will still result in a significant reduction in
storage size.
%

%
Our approach also has some technical limitations.
%
Seldom outliers lead to locally large distances between feature surface and
flame surface, which is visible as spots in the visualization.
%
These are however rare and indicate areas of unusual behavior on the flame
surface, which also provides meaningful information.
%
%Another possible problem is the reliance on a single isosurface when seeding the
%initial profile lines. 
%
Finally, relying on a random process to seed the profile lines might produce
insufficient numbers of samples in some regions.
%
This might be avoided by using a deterministic seeding approach and is left for
future work.
%

%
Four experts in combustion \ac{DNS} examined our approach, two of which were
also partly involved in its development.
%
They stated that the extraction of feature surfaces especially for variables
that have a maximum near the flame surface is a welcome addition to their set of
analysis tools.
%
Such surfaces are of particular interest in the comparison of combustion
processes in laminar vs. turbulent flows.
%
Previously, such surfaces were often approximated by iso- surfaces of other
variables, which were assumed to be close to the desired feature surface based
on the conditions in laminar flow.
%
The possibility of directly comparing feature surfaces with our approach opens
new possibilities in the investigation of the effects of turbulence on
combustion.
%

%
Another application proposed by the experts is the comparison of experimental
and simulation research.
%
In experiments, the flame surface is often determined by easily measurable
quantities, while more precise definitions are used in simulation research.
%
Our feature surfaces enable comparison of these definitions and deriving models
to make experiments and simulations more comparable.
%

%
This comparison of different flame surface definitions is also important when
comparing different simulations.
%
Since there is no universally agreed-upon definition of the flame surface,
different researchers often use different definitions, which could be
quantitatively compared with our method.
%

%
The sparse data representation, apart from much-needed space savings, opens
possibilities of statistical analysis of the relations of feature surfaces,
which could be used to improve combustion models for \ac{RANS} or \ac{LES}
methods.
%

%
Our approach is currently implemented as a post-processing step. By transforming
the original data into the sparse representation, disk space usage is reduced
considerably.
%
At the same time, the data is also brought into a form better suited for
flamelet analysis and feature surface visualization.
%
The implementation of the approach as an in-situ process brings additional
technical problems to be solved.
%
One of these problems is tracking the flame surface over time, such that the
correspondence between flamelets at different time steps can be maintained.
%
An approach to this surface tracking problem is presented in the next chapter.
%

% We have presented a sparse data representation for \ac{DNS} of premixed combustion
% that deals with the three core problems of \ac{DNS} data analysis: Storage
% requirements, analysis and visualization. The representation accurately
% approximates the original data while consuming dramatically less storage space.
% The extraction and visualization of feature surfaces we present as a possible
% application can be performed without reconstructing the data on the original
% grid. Experts confirm that our approach opens new and important possibilities
% of investigating the process of premixed turbulent combustion.

% We have presented an approach for compressing and visualizing data from \ac{DNS} of
% premixed combustion. Our approach explicitly preserves characteristics of the
% simulation variables and visualizes them. We achieve high compression ratios
% while maintaining good accuracy. The compressed data and the visualization of
% feature surfaces directly from this data provides novel ways of investigating
% important flame properties.
%
%\section{Conclusion} (end)