\lettrine[findent=-2pt, nindent=2pt, lhang=0.05, loversize=0.02]{C}{ombustion}
is one of the cornerstones of our civilization.
%
It provides light in a candle or carries objects into space in a rocket engine.
%
The majority of high-tech combustion processes occur under turbulent conditions.
%
The nature of turbulent flow, and therefore also turbulent combustion, is still
being actively studied by the scientific community.
%
Reducing the pollutant emissions and increasing the efficiency of combustion
processes is essential in todays world where we are confronted more and more
often with the realization that our resources are finite and the damage we do to
our environment can not easily be undone.
%

%
Simulations are an invaluable tool in the design of improved combustion
processes.
%
They allow to test different setups quickly and for a relatively low cost.
%
Efficient simulations need models of the relevant physical and chemical
processes.
%
As the demands on the accuracy of such models rises, more detailed insight into
the low-level phenomena of turbulent combustion is needed.
%
Obtaining this insight via experiments is challenging.
%
Often only a small number of variables can be observed at the same time and
observations are frequently limited to a \ac{2D} slice.
%

%
Another approach is \acf{DNS}, which is sometimes referred to as a ``numerical
experiment''.
%
In \ac{DNS}, the Navier-Stokes equation is directly solved on a very fine grid,
without using any higher-level modeling assumptions.
%
This is computationally very expensive, which is why \ac{DNS} is typically
performed on supercomputers using hundreds or thousands of cores.
%
In the simulation, all variables are available in full spatial and temporal
resolution.
%

%
Analyzing the data from such simulations poses a different challenge.
%
Due to the high spatial and temporal resolution, the raw data produced by a
single \ac{DNS} run can range from Terabytes to Petabytes.
%
Data of this size can not be written to disk or transferred over a network in a
reasonable amount of time, even if the enormous storage space that is required
was available.
%
This limits the post-analysis of the data to spatially or temporally downsampled
versions which lose a lot of important information.
%
In recent years, the subject of \emph{in-situ} analysis and visualization has
therefore gained popularity.
%
The idea is to process the data while it is still in memory during the
simulation.
%
The raw data is then discarded and only the results, which are typically much
smaller in size, are stored on disk.
%

%
This thesis contains two new in-situ focused approaches for analysis and
visualization of turbulent combustion \ac{DNS}.
%
To provide some important context, this chapter provides a short introduction
into the field of turbulent combustion research, the basics of turbulent
combustion modeling and simulation, and an overview of relevant research
concerning in-situ and post-processing of turbulent combustion data in
particular and large-scale simulations in general.
%