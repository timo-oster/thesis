\documentclass[oneside]{scrartcl}

\usepackage[utf8]{inputenc}

\usepackage[ngerman, english]{babel}

\usepackage{kpfonts}

% font encoding most suitable for european languages
\usepackage[T1]{fontenc}

\usepackage{microtype}

\usepackage{nth}

\usepackage{enumitem}
\setlist{
    topsep=1ex,
    parsep=0pt,
    listparindent=\parindent}

\addtokomafont{disposition}{\rmfamily}

\pagestyle{empty}


\author{Timo Oster}

\date{26. April 2019}

\makeatletter
\let\theauthor\@author
\let\thedate\@date
\makeatother

\begin{document}

\addsec*{Zusammenfassung}
%
Diese Arbeit pr\"asentiert neue Visualisierungsmethoden f\"ur Simulationen aus
zwei verschiedenen Ingenieurdisziplinen: Turbulente Verbrennung und
Festk\"orpermechanik.
%

%
Direkte numerische Simulationen turbulenter Verbrennung sind eine Basis f\"ur
die Entwicklung und Validierung h\"oherer Verbrenungsmodelle.
%
Ein besonderes Augenmerk bei der Analyse solcher Simulationen liegt auf der
Flammenoberfl\"ache, wo der Gro\ss{}teil aller chemischen Reaktionen
stattfindet.
%
Die Rechenleistung von Supercomputern w\"achst inzwischen wesentlich schneller
als die Leistung ihrer Speicherinfrastruktur.
%
Infolgedessen ist heute das Speichern der Ausgabedaten der Flaschenhals in
gro\ss{}en Simulationen.
%
Wir pr\"asentieren zwei neue Techniken f\"ur die Visualisierung und Analyse der
Flammenobefl\"ache in gro\ss{}en Simulationen turbulenter Verbrennungsvorg\"ange
vor dem Hintergrund dieses Flaschenhalses.
%
Die erste ist eine platzsparende, ausged\"unnte Darstellung f\"ur einen
bestimmten Typ von Flammen.
%
Diese erm\"oglicht die Analyse einer gr\"o\ss{}eren Anzahl von Zeitschritten der
Simulation und ist die Basis f\"ur eine neue Art von Flammenvisualisierung.
%
Die zweite ist ein Algorithmus zur Verfolgung der Flammenoberfl\"ache
\emph{in-situ} w\"ahrend der Simulation selbst.
%
Der Flaschenhals des Speichervorganges wird umgangen indem nur die wesentlich
kleineren Ergebnisse geschrieben werden.
%
Beide Verfahren tragen dazu bei, dass Verbrennungswissenschaftler auch in
Zukunft die Daten Analysieren k\"onnen, die ihre immer gr\"o\ss{}er werdenden
Simulationen produzieren.
%

%
Tensorfelder geh\"oren wegen ihrer vielen Freiheitsgrade zu den
herausforderndsten Daten f\"ur die Visualisierung.
%
Eine M\"oglichkeit, diese Komplexit\"at zu reduzieren ist die Extraktion von
Features.
%
Diese reduziert die Daten auf geometrische Primitive, die interessantes
Verhalten markieren.
%
Der \emph{parallel vectors operator}, der alle Orte bestimmt an denen zwei
Vektorfelder parallel sind, ist die Basis f\"ur eine Menge von Linien-Features
f\"ur Skalar- und Vektorfelder.
%
Wir übertragen diesen Operator auf Tensorfelder und definieren dort den
\emph{parallel eigenvectors operator}, der alle Orte bestimmt, an denen zwei
Tensorfelder parallele reelle Eigenvektoren haben.
%
Diese Idee nutzen wir anschlie\ss{}end zur Definition von \emph{tensor core
lines}, die die Zentren von ``wirbelndem'' Verhalten der Eigenvektoren markieren
und auf Wirbelkernlinien in Vektorfeldern basieren.
%
Mit diesem neuen Feature k\"onnen wir Verwindungen in Stresstensorfeldern aus
Strukturmechaniksimulationen erkennen.
%
\clearpage
%
\addsec*{Liste der Publikationen}
%
\begin{itemize}[label={},leftmargin=0pt]
    \item T.~Oster, D.~J.~Lehmann, G.~Fru, H.~Theisel, and D.~Th\'evenin\\
        \textbf{Sparse Representation and Visualization for Direct Numerical
        Simulation Of Premixed Combustion}\\
        {\emph{Computer Graphics Forum} 33.3, pp. 321--330, 2014}

    \item A.~Abdelsamie, G.~Fru, T.~Oster, F.~Dietzsch, G.~Janiga,
        and D.~Thévenin\\
        \textbf{Towards Direct Numerical Simulations of Low-Mach Number
        Turbulent Reacting and Two-Phase Flows Using Immersed Boundaries}\\
        {\emph{Computers \& Fluids} 131, pp. 123--141, 2016}

    \item C.~Chi, A.~Abdelsamie, T.~Oster, and D.~Thévenin\\
        \textbf{Probability of Hotspot Ignition and Ignition Spot Tracking in
        Turbulent Hydrogen-Air Mixtures Using Direct Numerical Simulations}\\
        {\emph{\nth{8} European Combustion Meeting}, pp. 925--930, 2017}

    \item T.~Oster\\
        \textbf{On-the-fly Visualization and Analysis for High-Resolution Combustion Simulations}\\
        \emph{Doktorandentag der Fakultät für Informatik}, Otto-von-Guericke-Universität Magdeburg, 12. September 2017

    \item T.~Oster, A.~Abdelsamie, M.~Motejat, T.~Gerrits, C.~R\"ossl,
        D.~Thévenin, and H.~Theisel\\
        \textbf{On-The-Fly Tracking of Flame Surfaces for the Visual Analysis
        of Combustion Processes}\\
        {\emph{Computer Graphics Forum} 37.6, pp. 358--369, 2018}

    \item T.~Oster, C.~R\"ossl, and H.~Theisel\\
        \textbf{Core Lines in 3D Second-Order Tensor Fields}\\
        {\emph{Computer Graphics Forum} 37.3, pp. 327--337, 2018}

    \item T.~Oster, C.~R\"ossl, and H.~Theisel\\
        \textbf{The Parallel Eigenvectors Operator}\\
        {\emph{Vision, Modeling and Visualization}, 2018}
\end{itemize}
%
\clearpage
%
\addsec*{Lebenslauf/wissenschaftlicher Werdegang}
%
\setlist{
    topsep=1ex,
    parsep=0pt,
    leftmargin=3.5cm}
%
\subsection*{Persönliche Daten}
%
\begin{enumerate}
    \item[Name:] Timo Oster
    \item[Geburtsdatum:] 13. Juni 1987
    \item[Geburtsort:] Wittlich
    \item[Anschrift:] Hegelstraße 19, 39104 Magdeburg
    \item[Email:] mail@timo-oster.net
\end{enumerate}
%
\subsection*{Ausbildung}
%
\begin{enumerate}
    \item[4/2012 -- 12/2011] Masterstudium Computervisualistik\\
                   Otto-von-Guericke-Universität Magdeburg\\
                   Master's Thesis: \emph{Globale Formanalyse von Flammen}\\
                   Abschluss: \textbf{Master of Science (Note: 1,0)}
    \item[10/2006 -- 4/2010] Bachelorstudium Computervisualistik\\
                             Otto-von-Guericke-Universität Magdeburg\\
                             Bachelor's Thesis: \emph{Erkennung von Nanofaseragglomeraten in REM-Bildern auf Basis von Texturinformationen}\\
                             Abschluss: \textbf{Bachelor of Science (Note: 1,3)}
    \item[4/2005] \textbf{Abitur (Note 1,3)}\\
                  Nikolaus-von-Kues Gymnasium, Bernkastel-Kues
\end{enumerate}
%
\subsection*{Berufserfahrung}
%
\begin{enumerate}
    \item[seit 3/2012] \textbf{Wissenschaftlicher Mitarbeiter} in der Visual Computing Group\\
                       Otto-von-Guericke-Universität Magdeburg
    \item[9/2010 -- 10/2012] \textbf{Softwareentwickler} (Nebentätigkeit)\\
                             Bundesanstalt für Arbeitsschutz und Arbeitssicherheit, Berlin
    \item[2/2008 -- 9/2009] \textbf{Java-Programmierer} (Studentische Hilfskraft)\\
                            Helmhotz-Zentrum für Umweltforschung, Magdeburg
\end{enumerate}
%
\subsection*{Besuchte Weiterbildungsveranstaltungen}
%
\begin{enumerate}
    \item[29.10.2018] Mini-Course: \textbf{Modelling Fluid Flow Using Kinetic Approaches}\\
                      IMPRS ProEng, MPI Magdeburg
    \item[25.--29.6.2012] \textbf{Fortran for Scientific Computing}\\
                          HLRS, Stuttgart
    \item[4.--7.2.2013] \textbf{Parallel Programming Workshop with MPI and OpenMP}\\
                  ZIH, TU Dresden
\end{enumerate}
%
\end{document}